\documentclass[12pt,]{article}
\usepackage{lmodern}
\usepackage{amssymb,amsmath}
\usepackage{ifxetex,ifluatex}
\usepackage{fixltx2e} % provides \textsubscript
\ifnum 0\ifxetex 1\fi\ifluatex 1\fi=0 % if pdftex
  \usepackage[T1]{fontenc}
  \usepackage[utf8]{inputenc}
\else % if luatex or xelatex
  \ifxetex
    \usepackage{mathspec}
  \else
    \usepackage{fontspec}
  \fi
  \defaultfontfeatures{Ligatures=TeX,Scale=MatchLowercase}
    \setsansfont[]{Times New Roman}
\fi
% use upquote if available, for straight quotes in verbatim environments
\IfFileExists{upquote.sty}{\usepackage{upquote}}{}
% use microtype if available
\IfFileExists{microtype.sty}{%
\usepackage{microtype}
\UseMicrotypeSet[protrusion]{basicmath} % disable protrusion for tt fonts
}{}
\usepackage[margin=1in]{geometry}
\usepackage{hyperref}
\hypersetup{unicode=true,
            pdftitle={Null model check},
            pdfborder={0 0 0},
            breaklinks=true}
\urlstyle{same}  % don't use monospace font for urls
\usepackage{longtable,booktabs}
\usepackage{graphicx,grffile}
\makeatletter
\def\maxwidth{\ifdim\Gin@nat@width>\linewidth\linewidth\else\Gin@nat@width\fi}
\def\maxheight{\ifdim\Gin@nat@height>\textheight\textheight\else\Gin@nat@height\fi}
\makeatother
% Scale images if necessary, so that they will not overflow the page
% margins by default, and it is still possible to overwrite the defaults
% using explicit options in \includegraphics[width, height, ...]{}
\setkeys{Gin}{width=\maxwidth,height=\maxheight,keepaspectratio}
\IfFileExists{parskip.sty}{%
\usepackage{parskip}
}{% else
\setlength{\parindent}{0pt}
\setlength{\parskip}{6pt plus 2pt minus 1pt}
}
\setlength{\emergencystretch}{3em}  % prevent overfull lines
\providecommand{\tightlist}{%
  \setlength{\itemsep}{0pt}\setlength{\parskip}{0pt}}
\setcounter{secnumdepth}{0}
% Redefines (sub)paragraphs to behave more like sections
\ifx\paragraph\undefined\else
\let\oldparagraph\paragraph
\renewcommand{\paragraph}[1]{\oldparagraph{#1}\mbox{}}
\fi
\ifx\subparagraph\undefined\else
\let\oldsubparagraph\subparagraph
\renewcommand{\subparagraph}[1]{\oldsubparagraph{#1}\mbox{}}
\fi

%%% Use protect on footnotes to avoid problems with footnotes in titles
\let\rmarkdownfootnote\footnote%
\def\footnote{\protect\rmarkdownfootnote}

%%% Change title format to be more compact
\usepackage{titling}

% Create subtitle command for use in maketitle
\newcommand{\subtitle}[1]{
  \posttitle{
    \begin{center}\large#1\end{center}
    }
}

\setlength{\droptitle}{-2em}
  \title{Null model check}
  \pretitle{\vspace{\droptitle}\centering\huge}
  \posttitle{\par}
  \author{}
  \preauthor{}\postauthor{}
  \date{}
  \predate{}\postdate{}

\usepackage{subfig}
\AtBeginDocument{%
\renewcommand*\figurename{Figure}
\renewcommand*\tablename{Table}
}
\AtBeginDocument{%
\renewcommand*\listfigurename{List of Figures}
\renewcommand*\listtablename{List of Tables}
}
\usepackage{float}
\floatstyle{ruled}
\makeatletter
\@ifundefined{c@chapter}{\newfloat{codelisting}{h}{lop}}{\newfloat{codelisting}{h}{lop}[chapter]}
\makeatother
\floatname{codelisting}{Listing}
\newcommand*\listoflistings{\listof{codelisting}{List of Listings}}

\begin{document}
\maketitle

I have tested two statistical models and two null models.

\begin{itemize}
\tightlist
\item
  Simple linear / Power-law
\item
  LMA shuffle / All shuffle
\end{itemize}

\subsection{Summary}\label{summary}

\begin{itemize}
\tightlist
\item
  For simple model, all shuffle shows lower
  r\textsuperscript{2}\textsubscript{rand} values.
\item
  For power-law model, the pattern is not that clear but both null
  models tend to produce lower r\textsuperscript{2}\textsubscript{rand}
  than simple model does.
\item
  Estimated parameters (the last two figures) suggest that Power-law
  model seems to be better (or more suitable).

  \begin{itemize}
  \tightlist
  \item
    Some of parameters in the simple model are not different from those
    estimated from randomized dataset, which seems to be
    problematic\ldots{}
  \end{itemize}
\end{itemize}

\subsection{Simple model}\label{simple-model}

\begin{align}
  &E[{\rm gross \; photosynthesis}]
  = E[A_{area_i} + R_{area_i}]
  = \alpha{\rm LMAp}_{i}
  = \alpha f_{i} {\rm LMA}_{i} \label{eq:E-AR}\\
  &E[{\rm LL}_{i}] = \beta_{1} {\rm LMAs}_{i}
  = \beta_{1} (1 - f_{i}) {\rm LMA}_{i} \label{eq:E-LL}\\
  &E[R_{areai}] = r_{p}{\rm LMAp}_{i} + r_{s} {\rm LMAs}_{i}
  = r_{p} f_{i} {\rm LMA}_{i} + r_{s} (1 - f_{i}) {\rm LMA}_{i}
  \label{eq:E-R}
\end{align}

where, E{[}\(\cdot\){]} is the expected value of the variable in
brackets, \emph{A}\textsubscript{area\emph{i}},
\emph{R}\textsubscript{area\emph{i}}, and LL\textsubscript{\emph{i}},
are, respectively, the net photosynthetic rate
(\emph{A}\textsubscript{max}) per unit area, dark respiration rate
(\emph{R}\textsubscript{dark}) per unit area, and leaf life span of leaf
i; \(\alpha\) is net photosynthetic rate per unit photosynthetic mass;
\(\beta_1\) is leaf lifespan per unit structural mass; and
r\textsubscript{p} and r\textsubscript{s}, are, respiration rates per
unit photosynthetic and structural leaf mass, respectively.

\subsection{Power-law model}\label{power-law-model}

\begin{align}
  &E[A_{area_i} + R_{area_i}]
  = \alpha_1{\rm LMAp}_{i}^{\alpha_2} \label{eq:E-AR2}\\
  &E[{\rm LL}_{i}] = \beta_{1} {\rm LMAs}_{i}^{\beta_2} \label{eq:E-LL2}\\
  &E[R_{areai}] = r{\rm LMAp}_{i}^{r_p}{\rm LMAs}_{i}^{r_s} \label{eq:E-R2}
\end{align}

\begin{align}
  &log[E[A_{area_i} + R_{area_i}]]
  = log\alpha_1 + {\alpha_2}log{\rm LMAp}_{i}\label{eq:E-AR3}\\
  &log[E[{\rm LL}_{i}]] =  log\beta_1 + {\beta_2}log{\rm LMAs}_{i} \label{eq:E-LL3}\\
  &log[E[R_{areai}]] =  logr + r_p log{\rm LMAp}_i + r_s log{\rm LMAs}_i \label{eq:E-R3}
\end{align}

\begin{itemize}
\tightlist
\item
  \(r_{p1} LMAp_{i}^{r_p2} + r_{s1} LMAs_{i}^{r_s2}\) did not work. Eq.
  (9) makes more sense.
\end{itemize}

\subsection{LMA shuffle}\label{lma-shuffle}

\begin{itemize}
\tightlist
\item
  We do not maintain covariance between LMA and other traits, but
  maintain covariances among A\textsubscript{area},
  R\textsubscript{area}, and LL.
\item
  This null model is suitable to test if the observed LMA is more
  meaningful than expected LMA from the observed covariance structure
  among other traits.
\item
  This null model shows that covariances structure among other traits
  alone can provide reasonable expected values of LMAp and LMAs but
  those LMAp and LMAs do not have to explain divergent pattens in the
  datasets than LMA does.
\item
  This null model is not suitable to test if covariance structure
  (underlying trade-offs) among A\textsubscript{area},
  R\textsubscript{area} and LL can produce observed patterns that can be
  explained by LMAp and LMAs.
\end{itemize}

\subsection{All shuffle}\label{all-shuffle}

\begin{itemize}
\tightlist
\item
  We do not maintain covariance structure (or trade-off) among traits.
\item
  This null model is suitable to test if covariance structure
  (underlying trade-offs) can generate patterns that can be explained by
  LMAp and LMAs.
\item
  For example, we could test an assumption that LMAp is proportional to
  (or increasing with) A\textsubscript{area} + R\textsubscript{area} and
  LL.
\end{itemize}


\end{document}

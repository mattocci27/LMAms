\documentclass[12pt,]{article}
\usepackage{lmodern}
\usepackage{amssymb,amsmath}
\usepackage{ifxetex,ifluatex}
\usepackage{fixltx2e} % provides \textsubscript
\ifnum 0\ifxetex 1\fi\ifluatex 1\fi=0 % if pdftex
  \usepackage[T1]{fontenc}
  \usepackage[utf8]{inputenc}
\else % if luatex or xelatex
  \ifxetex
    \usepackage{mathspec}
  \else
    \usepackage{fontspec}
  \fi
  \defaultfontfeatures{Ligatures=TeX,Scale=MatchLowercase}
    \setsansfont[]{Times New Roman}
\fi
% use upquote if available, for straight quotes in verbatim environments
\IfFileExists{upquote.sty}{\usepackage{upquote}}{}
% use microtype if available
\IfFileExists{microtype.sty}{%
\usepackage{microtype}
\UseMicrotypeSet[protrusion]{basicmath} % disable protrusion for tt fonts
}{}
\usepackage[margin=1in]{geometry}
\usepackage{hyperref}
\hypersetup{unicode=true,
            pdftitle={Null model check for Panama},
            pdfborder={0 0 0},
            breaklinks=true}
\urlstyle{same}  % don't use monospace font for urls
\usepackage{longtable,booktabs}
\usepackage{graphicx,grffile}
\makeatletter
\def\maxwidth{\ifdim\Gin@nat@width>\linewidth\linewidth\else\Gin@nat@width\fi}
\def\maxheight{\ifdim\Gin@nat@height>\textheight\textheight\else\Gin@nat@height\fi}
\makeatother
% Scale images if necessary, so that they will not overflow the page
% margins by default, and it is still possible to overwrite the defaults
% using explicit options in \includegraphics[width, height, ...]{}
\setkeys{Gin}{width=\maxwidth,height=\maxheight,keepaspectratio}
\IfFileExists{parskip.sty}{%
\usepackage{parskip}
}{% else
\setlength{\parindent}{0pt}
\setlength{\parskip}{6pt plus 2pt minus 1pt}
}
\setlength{\emergencystretch}{3em}  % prevent overfull lines
\providecommand{\tightlist}{%
  \setlength{\itemsep}{0pt}\setlength{\parskip}{0pt}}
\setcounter{secnumdepth}{0}
% Redefines (sub)paragraphs to behave more like sections
\ifx\paragraph\undefined\else
\let\oldparagraph\paragraph
\renewcommand{\paragraph}[1]{\oldparagraph{#1}\mbox{}}
\fi
\ifx\subparagraph\undefined\else
\let\oldsubparagraph\subparagraph
\renewcommand{\subparagraph}[1]{\oldsubparagraph{#1}\mbox{}}
\fi

%%% Use protect on footnotes to avoid problems with footnotes in titles
\let\rmarkdownfootnote\footnote%
\def\footnote{\protect\rmarkdownfootnote}

%%% Change title format to be more compact
\usepackage{titling}

% Create subtitle command for use in maketitle
\newcommand{\subtitle}[1]{
  \posttitle{
    \begin{center}\large#1\end{center}
    }
}

\setlength{\droptitle}{-2em}
  \title{Null model check for Panama}
  \pretitle{\vspace{\droptitle}\centering\huge}
  \posttitle{\par}
  \author{}
  \preauthor{}\postauthor{}
  \date{}
  \predate{}\postdate{}

\usepackage{subfig}
\AtBeginDocument{%
\renewcommand*\figurename{Figure}
\renewcommand*\tablename{Table}
}
\AtBeginDocument{%
\renewcommand*\listfigurename{List of Figures}
\renewcommand*\listtablename{List of Tables}
}
\usepackage{float}
\floatstyle{ruled}
\makeatletter
\@ifundefined{c@chapter}{\newfloat{codelisting}{h}{lop}}{\newfloat{codelisting}{h}{lop}[chapter]}
\makeatother
\floatname{codelisting}{Listing}
\newcommand*\listoflistings{\listof{codelisting}{List of Listings}}

\begin{document}
\maketitle

\subsection{Optimal LL}\label{optimal-ll}

According to optimal LL theory, the optimal leaf lifespan
(LL\textsubscript{opt}) maximizes a leaf's lifetime carbon gain per-unit
time, which is determined by potential LL, net photosynthetic rate, and
construction cost per unit leaf area:

\begin{align}
  &{\rm LL_{opt}} = \sqrt{2bC/(a-m)}\label{eq:kikuzawa}
\end{align}

where \emph{a} is the realized (i.e., light-dependent) gross
photosynthetic rate per unit leaf area, \emph{m} is the realized daily
respiration rate per unit leaf area, \emph{b} is the rate of decline in
photosynthetic capacity with leaf age (which determines potential LL in
the Kikuzawa model), and \emph{C} is the construction cost per unit leaf
area. To implement this Optimal LL Model, we assumed that (i) potential
LL is proportional to LMAs; and (ii) leaf construction cost per unit
area (g glucose per unit leaf area) is proportional to LMA. Assumption
(ii) is justified because leaf construction cost per unit mass (g
glucose per unit leaf mass) is strongly conserved. Thus, Eq.
\eqref{eq:kikuzawa} can be written in terms of the traits considered in
our analysis as:

\subsubsection{LMAsLMA (simple orginal
version)}\label{lmaslma-simple-orginal-version}

\begin{align}
  {\rm E[LL}_{i}] &=\beta_{2} \sqrt{{\rm LMAs}_{i}{\rm LMA}_{i}
  / (\theta_{\rm L} A_{{\rm area} \; i} - R_{{\rm area} \; i})}\label{eq:simple}
\end{align}

where \(\beta_{2}\) is a constant, and
\(\theta_{\rm L} (0 < \theta_{\rm L} < 1)\) is a scaling parameter that
accounts for the effects of light availability on the realized
photosynthetic rate. The expected value of the logarithm of LL for the
Optimal LL Model is:

\begin{itemize}
\tightlist
\item
  Potential LL is proportional to LMAs.
\end{itemize}

\subsubsection{\texorpdfstring{\(LMAs^{\beta_2}LMA\)
(power-law)}{LMAs\^{}\{\textbackslash{}beta\_2\}LMA (power-law)}}\label{lmasbeta_2lma-power-law}

\begin{align}
  {\rm E[LL}_{i}] &=\beta_{2} \sqrt{{\rm LMAs^{\beta_3}}_{i}{\rm LMA}_{i}
  / (\theta_{\rm L} A_{{\rm area} \; i} - R_{{\rm area} \; i})} \notag \\
  \label{eq:power}
\end{align}

\begin{itemize}
\tightlist
\item
  Potential LL and LMAs have a power-law relationship
\end{itemize}

\subsubsection{\texorpdfstring{\(LMAs^{\beta_3}(\beta_4 LMAp + (1 - \beta_4) LMAs)\)
(complicated)}{LMAs\^{}\{\textbackslash{}beta\_3\}(\textbackslash{}beta\_4 LMAp + (1 - \textbackslash{}beta\_4) LMAs) (complicated)}}\label{lmasbeta_3beta_4-lmap-1---beta_4-lmas-complicated}

\begin{align}
  {\rm E[LL}_{i}] &=\beta_{2} \sqrt{{\rm {LMAs}_i^{\beta_3}}_{i}{(\beta_4 \rm {LMAp}_i + (1 - \beta_4) \rm {LMAs_i})}
  / (\theta_{\rm L} A_{{\rm area} \; i} - R_{{\rm area} \; i})} \label{eq:comp}
\end{align}

\begin{itemize}
\tightlist
\item
  Potential LL and LMAs have a power-law relationship.
\item
  LMAp and LMAs have differenct construction cost.
\end{itemize}

\subsubsection{\texorpdfstring{\(LMAs^{\beta_2}\) (power-law,
simple)}{LMAs\^{}\{\textbackslash{}beta\_2\} (power-law, simple)}}\label{lmasbeta_2-power-law-simple}

\begin{align}
  {\rm E[LL}_{i}] &=\beta_{2} \sqrt{{\rm LMAs^{\beta_3}}_{i}
  / (\theta_{\rm L} A_{{\rm area} \; i} - R_{{\rm area} \; i})} \notag \\
  \label{eq:power-simple}
\end{align}

\begin{itemize}
\tightlist
\item
  Model looks simple but no clear relationship between Kikuzawa model
\end{itemize}

\newpage

\subsection{LMAsLMA}\label{lmaslma}

\begin{verbatim}
## Warning: 132 (100.0%) p_waic estimates greater than 0.4.
## We recommend trying loo() instead.
\end{verbatim}

\begin{verbatim}
## [1] 345.0468
\end{verbatim}

\begin{verbatim}
## Warning: Some Pareto k diagnostic values are too high. See help('pareto-k-
## diagnostic') for details.
\end{verbatim}

\begin{verbatim}
## Computed from 1500 by 132 log-likelihood matrix
## 
##          Estimate   SE
## elpd_loo   -205.1 11.6
## p_loo       132.6  2.6
## looic       410.1 23.2
## 
## Pareto k diagnostic values:
##                          Count  Pct 
## (-Inf, 0.5]   (good)       0    0.0%
##  (0.5, 0.7]   (ok)        11    8.3%
##    (0.7, 1]   (bad)      105   79.5%
##    (1, Inf)   (very bad)  16   12.1%
## See help('pareto-k-diagnostic') for details.
\end{verbatim}

\includegraphics{PA_check_files/figure-latex/unnamed-chunk-1-1.pdf}

\begin{verbatim}
##                  mean     se_mean         sd      X2.5.       X25.
## log_alpha  0.87720472 0.016306420 0.18917905  0.4876015  0.7592832
## log_beta  -0.39286197 0.004244811 0.06774749 -0.5199120 -0.4396547
## rp         0.58493267 0.006234821 0.07446665  0.4464344  0.5321613
## rs        -0.09355905 0.002949856 0.08318232 -0.2595270 -0.1490482
## alpha2     0.44542488 0.003871859 0.04732837  0.3592577  0.4149237
## log_r     -1.93723570 0.018702503 0.37376773 -2.6823689 -2.1801487
##                  X50.        X75.     X97.5.    n_eff      Rhat
## log_alpha  0.88950961  1.00480755  1.2165782 134.5950 1.0091938
## log_beta  -0.39314135 -0.34691364 -0.2541852 254.7239 1.0181798
## rp         0.58285150  0.63506750  0.7325707 142.6513 1.0171136
## rs        -0.09030317 -0.04312305  0.0748848 795.1706 0.9996336
## alpha2     0.44311363  0.47697229  0.5424852 149.4183 1.0088071
## log_r     -1.93209143 -1.70430896 -1.2092032 399.3964 1.0149758
\end{verbatim}

\begin{verbatim}
## [1] "LMAs^betaLMA; LMA all"
\end{verbatim}

\includegraphics{PA_check_files/figure-latex/unnamed-chunk-3-1.pdf}

\newpage

\subsection{\texorpdfstring{\[LMAs^{\beta_2}LMA\]}{LMAs\^{}\{\textbackslash{}beta\_2\}LMA}}\label{lmasbeta_2lma}

\begin{verbatim}
## Warning: 132 (100.0%) p_waic estimates greater than 0.4.
## We recommend trying loo() instead.
\end{verbatim}

\begin{verbatim}
## [1] 441.7458
\end{verbatim}

\begin{verbatim}
## Warning: Some Pareto k diagnostic values are too high. See help('pareto-k-
## diagnostic') for details.
\end{verbatim}

\begin{verbatim}
## Computed from 1500 by 132 log-likelihood matrix
## 
##          Estimate   SE
## elpd_loo   -244.0 15.9
## p_loo       140.2  8.7
## looic       488.0 31.8
## 
## Pareto k diagnostic values:
##                          Count  Pct 
## (-Inf, 0.5]   (good)      0     0.0%
##  (0.5, 0.7]   (ok)       23    17.4%
##    (0.7, 1]   (bad)      95    72.0%
##    (1, Inf)   (very bad) 14    10.6%
## See help('pareto-k-diagnostic') for details.
\end{verbatim}

\includegraphics{PA_check_files/figure-latex/unnamed-chunk-4-1.pdf}

\begin{verbatim}
##                  mean     se_mean         sd      X2.5.       X25.
## log_alpha  0.64082555 0.017274482 0.22335362  0.1908048  0.4927873
## log_beta   0.74606389 0.053814820 0.41461523  0.2137177  0.4685429
## rp         0.62437074 0.007351056 0.08388484  0.4625727  0.5702547
## rs        -0.04006785 0.009462842 0.09315505 -0.1944820 -0.1035263
## alpha2     0.50881237 0.004550767 0.05737463  0.4017228  0.4683242
## beta2      0.36255734 0.030776167 0.23374247 -0.2179552  0.2843073
##                 X50.        X75.    X97.5.     n_eff     Rhat
## log_alpha  0.6543551 0.799883146 1.0444118 167.17678 1.005119
## log_beta   0.6220374 0.897927771 1.7746737  59.35902 1.049469
## rp         0.6255789 0.680210222 0.7888601 130.21693 1.005300
## rs        -0.0519825 0.009180471 0.1879395  96.91022 1.022323
## alpha2     0.5067450 0.546237763 0.6267834 158.95370 1.007621
## beta2      0.4340739 0.520430374 0.6657765  57.68279 1.050797
\end{verbatim}

\begin{verbatim}
## \newpage
\end{verbatim}

\includegraphics{PA_check_files/figure-latex/unnamed-chunk-4-2.pdf}

\begin{verbatim}
## \newpage
\end{verbatim}

\begin{verbatim}
## [1] "LMAs^beta2 LMA; LMA all"
\end{verbatim}

\includegraphics{PA_check_files/figure-latex/unnamed-chunk-4-3.pdf}

\subsection{\texorpdfstring{\[LMAs^{\beta_2}(\beta_3 LMAp + (1 - \beta_3) LMAs)\]}{LMAs\^{}\{\textbackslash{}beta\_2\}(\textbackslash{}beta\_3 LMAp + (1 - \textbackslash{}beta\_3) LMAs)}}\label{lmasbeta_2beta_3-lmap-1---beta_3-lmas}

\begin{verbatim}
## Warning: 132 (100.0%) p_waic estimates greater than 0.4.
## We recommend trying loo() instead.
\end{verbatim}

\begin{verbatim}
## [1] 406.0067
\end{verbatim}

\begin{verbatim}
## Warning: Some Pareto k diagnostic values are too high. See help('pareto-k-
## diagnostic') for details.
\end{verbatim}

\begin{verbatim}
## Computed from 1500 by 132 log-likelihood matrix
## 
##          Estimate   SE
## elpd_loo   -238.7 17.2
## p_loo       141.6 10.9
## looic       477.3 34.4
## 
## Pareto k diagnostic values:
##                          Count  Pct 
## (-Inf, 0.5]   (good)      0     0.0%
##  (0.5, 0.7]   (ok)       18    13.6%
##    (0.7, 1]   (bad)      95    72.0%
##    (1, Inf)   (very bad) 19    14.4%
## See help('pareto-k-diagnostic') for details.
\end{verbatim}

\includegraphics{PA_check_files/figure-latex/unnamed-chunk-5-1.pdf}

\begin{verbatim}
##                  mean     se_mean         sd      X2.5.        X25.
## log_alpha  0.67611097 0.016613165 0.21592911  0.2490996  0.53530636
## log_beta   0.70798378 0.026308238 0.27505635  0.2758427  0.54436759
## rp         0.63999888 0.006476157 0.08880442  0.4816349  0.57820005
## rs        -0.04454638 0.006941428 0.08161392 -0.1899786 -0.09832092
## alpha2     0.49969627 0.004040009 0.05414438  0.3882623  0.46293585
## beta2      0.55579588 0.014794307 0.14795073  0.1255504  0.49694501
## log_r     -2.29924519 0.023895189 0.39124369 -3.1152153 -2.56714537
## log_site  -0.54836130 0.003350456 0.08600480 -0.7164323 -0.60543518
## q          0.39600457 0.004697088 0.07135057  0.2906915  0.34417442
## beta3      0.71854604 0.006696129 0.13513291  0.4319165  0.63046255
##                  X50.         X75.     X97.5.    n_eff     Rhat
## log_alpha  0.67087318  0.818203344  1.1192243 168.9342 1.006225
## log_beta   0.67808200  0.828212376  1.4800690 109.3100 1.020493
## rp         0.63323836  0.699816493  0.8198929 188.0331 1.013925
## rs        -0.05030242  0.002531049  0.1375743 138.2391 1.017191
## alpha2     0.49993072  0.535531977  0.6069317 179.6147 1.004406
## beta2      0.57048787  0.644565466  0.7916867 100.0104 1.023583
## log_r     -2.28381612 -2.015964004 -1.5935182 268.0858 1.002413
## log_site  -0.55057845 -0.491438308 -0.3759210 658.9273 1.007294
## q          0.38630346  0.437103362  0.5634283 230.7478 1.012214
## beta3      0.72645439  0.818704289  0.9599562 407.2627 1.005547
\end{verbatim}

\begin{verbatim}
## \newpage
\end{verbatim}

\begin{verbatim}
## \newpage
\end{verbatim}

\begin{verbatim}
## [1] "LMAs^beta2 beta3LMAp + (1-beta3)LMAs; LMA all"
\end{verbatim}

\includegraphics{PA_check_files/figure-latex/unnamed-chunk-5-2.pdf}

\subsection{\texorpdfstring{\[LMAs^{\beta_2}\]}{LMAs\^{}\{\textbackslash{}beta\_2\}}}\label{lmasbeta_2}

\begin{verbatim}
## Warning: 132 (100.0%) p_waic estimates greater than 0.4.
## We recommend trying loo() instead.
\end{verbatim}

\begin{verbatim}
## [1] 342.0478
\end{verbatim}

\begin{verbatim}
## Warning: Some Pareto k diagnostic values are too high. See help('pareto-k-
## diagnostic') for details.
\end{verbatim}

\begin{verbatim}
## Computed from 1500 by 132 log-likelihood matrix
## 
##          Estimate   SE
## elpd_loo   -203.7 11.7
## p_loo       135.7  3.3
## looic       407.4 23.3
## 
## Pareto k diagnostic values:
##                          Count  Pct 
## (-Inf, 0.5]   (good)       0    0.0%
##  (0.5, 0.7]   (ok)         9    6.8%
##    (0.7, 1]   (bad)      106   80.3%
##    (1, Inf)   (very bad)  17   12.9%
## See help('pareto-k-diagnostic') for details.
\end{verbatim}

\includegraphics{PA_check_files/figure-latex/unnamed-chunk-6-1.pdf}

\begin{verbatim}
##                  mean     se_mean         sd      X2.5.       X25.
## log_alpha  0.87178761 0.015137841 0.21225947  0.3981980  0.7406444
## log_beta   2.42578511 0.010547525 0.19035971  2.0316604  2.3052516
## rp         0.63082804 0.006114962 0.08475502  0.4767237  0.5722856
## rs        -0.05247914 0.004241408 0.07117964 -0.1896112 -0.1002804
## alpha2     0.44632099 0.003767548 0.05439912  0.3475137  0.4096910
## beta2      0.67241845 0.005367400 0.09790525  0.4842317  0.6072549
##                  X50.         X75.     X97.5.    n_eff     Rhat
## log_alpha  0.88193861  1.016008791 1.25386161 196.6103 1.021920
## log_beta   2.43060723  2.551373766 2.78620891 325.7234 1.000279
## rp         0.62734204  0.684882355 0.81282128 192.1071 1.012807
## rs        -0.05353729 -0.004479738 0.08964273 281.6381 1.005637
## alpha2     0.44430101  0.482149701 0.56465318 208.4809 1.020595
## beta2      0.67022611  0.732381868 0.87930606 332.7239 1.002385
\end{verbatim}


\end{document}

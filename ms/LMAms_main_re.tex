% Options for packages loaded elsewhere
\PassOptionsToPackage{unicode}{hyperref}
\PassOptionsToPackage{hyphens}{url}
%
\documentclass[
  12pt,
]{article}
\usepackage{lmodern}
\usepackage{amssymb,amsmath}
\usepackage{ifxetex,ifluatex}
\ifnum 0\ifxetex 1\fi\ifluatex 1\fi=0 % if pdftex
  \usepackage[T1]{fontenc}
  \usepackage[utf8]{inputenc}
  \usepackage{textcomp} % provide euro and other symbols
\else % if luatex or xetex
  \usepackage{unicode-math}
  \defaultfontfeatures{Scale=MatchLowercase}
  \defaultfontfeatures[\rmfamily]{Ligatures=TeX,Scale=1}
\fi
% Use upquote if available, for straight quotes in verbatim environments
\IfFileExists{upquote.sty}{\usepackage{upquote}}{}
\IfFileExists{microtype.sty}{% use microtype if available
  \usepackage[]{microtype}
  \UseMicrotypeSet[protrusion]{basicmath} % disable protrusion for tt fonts
}{}
\makeatletter
\@ifundefined{KOMAClassName}{% if non-KOMA class
  \IfFileExists{parskip.sty}{%
    \usepackage{parskip}
  }{% else
    \setlength{\parindent}{0pt}
    \setlength{\parskip}{6pt plus 2pt minus 1pt}}
}{% if KOMA class
  \KOMAoptions{parskip=half}}
\makeatother
\usepackage{xcolor}
\IfFileExists{xurl.sty}{\usepackage{xurl}}{} % add URL line breaks if available
\IfFileExists{bookmark.sty}{\usepackage{bookmark}}{\usepackage{hyperref}}
\hypersetup{
  hidelinks,
  pdfcreator={LaTeX via pandoc}}
\urlstyle{same} % disable monospaced font for URLs
\usepackage[margin=1in]{geometry}
\usepackage{longtable,booktabs}
% Correct order of tables after \paragraph or \subparagraph
\usepackage{etoolbox}
\makeatletter
\patchcmd\longtable{\par}{\if@noskipsec\mbox{}\fi\par}{}{}
\makeatother
% Allow footnotes in longtable head/foot
\IfFileExists{footnotehyper.sty}{\usepackage{footnotehyper}}{\usepackage{footnote}}
\makesavenoteenv{longtable}
\usepackage{graphicx}
\makeatletter
\def\maxwidth{\ifdim\Gin@nat@width>\linewidth\linewidth\else\Gin@nat@width\fi}
\def\maxheight{\ifdim\Gin@nat@height>\textheight\textheight\else\Gin@nat@height\fi}
\makeatother
% Scale images if necessary, so that they will not overflow the page
% margins by default, and it is still possible to overwrite the defaults
% using explicit options in \includegraphics[width, height, ...]{}
\setkeys{Gin}{width=\maxwidth,height=\maxheight,keepaspectratio}
% Set default figure placement to htbp
\makeatletter
\def\fps@figure{htbp}
\makeatother
\setlength{\emergencystretch}{3em} % prevent overfull lines
\providecommand{\tightlist}{%
  \setlength{\itemsep}{0pt}\setlength{\parskip}{0pt}}
\setcounter{secnumdepth}{-\maxdimen} % remove section numbering
\usepackage{times}
\usepackage{setspace}
\doublespacing
\usepackage{lineno}
\linenumbers
\usepackage{amsmath}

\usepackage{booktabs}
\usepackage{longtable}
\usepackage{array}
\usepackage{multirow}
\usepackage{wrapfig}
\usepackage{float}
\usepackage{colortbl}
\usepackage{pdflscape}
\usepackage{tabu}
\usepackage{threeparttable}
\usepackage{threeparttablex}
\usepackage[normalem]{ulem}
\usepackage{makecell}
\usepackage{xcolor}
% \usepackage{array}
\usepackage{booktabs}
\usepackage{longtable}
\usepackage{array}
\usepackage{multirow}
\usepackage{wrapfig}
\usepackage{float}
\usepackage{colortbl}
\usepackage{pdflscape}
\usepackage{tabu}
\usepackage{threeparttable}
\usepackage{threeparttablex}
\usepackage[normalem]{ulem}
\usepackage{makecell}
\ifluatex
  \usepackage{selnolig}  % disable illegal ligatures
\fi
\newlength{\cslhangindent}
\setlength{\cslhangindent}{1.5em}
\newenvironment{cslreferences}%
  {\setlength{\parindent}{0pt}%
  \everypar{\setlength{\hangindent}{\cslhangindent}}\ignorespaces}%
  {\par}

\author{}
\date{\vspace{-2.5em}}

\begin{document}

\textbf{Running title}: Metabolic and structural leaf mass

\textbf{Decomposing leaf mass into metabolic and structural components explains divergent patterns of trait variation within and among plant species}

Masatoshi Katabuchi\textsuperscript{1,2,3,7}, Kaoru Kitajima\textsuperscript{3,4,5}, S. Joseph Wright\textsuperscript{5}, Sunshine A. Van Bael\textsuperscript{5,6}, Jeanne L. D. Osnas\textsuperscript{3} and Jeremy W. Lichstein\textsuperscript{3}

\textsuperscript{1} Xishuangbanna Tropical Botanical Garden, Chinese Academy of Sciences, Menglun, Yunnan 666303 China

\textsuperscript{2} Center of Plant Ecology, Core Botanical Gardens, Chinese Academy of Sciences, Menglun, Yunnan 666303 China

\textsuperscript{3} Department of Biology, University of Florida, Gainesville, FL 32611, USA

\textsuperscript{4} Graduate School of Agriculture, Kyoto University, Kitashirakawa Oiwake-Cho, Kyoto 606-8502 Japan

\textsuperscript{5} Smithsonian Tropical Research Institute, 9100 Panama City Pl., Washington, DC 20521

\textsuperscript{6} Department of Ecology and Evolutionary Biology, Tulane University, New Orleans, LA 70118 USA

\textsuperscript{7} \textbf{Corresponding Author}: E-mail: \href{mailto:mattocci27@gmail.com}{\nolinkurl{mattocci27@gmail.com}}

\newpage

\hypertarget{abstract}{%
\section{Abstract}\label{abstract}}

\begin{itemize}
\item
  Across the global flora, photosynthetic and metabolic rates depend more strongly on leaf area than leaf mass.
  In contrast, intraspecific variation in these rates is strongly mass-dependent.
  These contrasting patterns suggest that the causes of variation in leaf mass per area (LMA) may be fundamentally different within vs.~among species.
\item
  We used statistical methods to decompose LMA into two conceptual components -- `metabolic LMAm (which determines photosynthetic capacity and metabolic rates, and also affects optimal leaf lifespan) and `structural' LMAs (which determines leaf toughness and potential leaf lifespan) using leaf trait data from tropical forest sites in Panama and a global leaf-trait database.
\item
  Statistically decomposing LMA into LMAm and LMAs provides improved predictions of trait variation (photosynthesis, respiration, and lifespan) across the global flora, and within and among tropical plant species in Panama.
  Our analysis shows that small scaling slope between metabolic leaf mass and photosynthetic rate, and similar variance in LMAm and in LMAs leads to area-proportionality of interspecific leaf traits.
  In contrast, intraspecific LMA variation is due to changes in LMAm, which explains why photosynthetic and metabolic traits are mass-dependent within species
\item
  Our results suggest that leaf trait variation is multi-dimensional and is not well-represented by the one-dimensional leaf economics spectrum.
\end{itemize}

\hypertarget{introduction}{%
\section{Introduction}\label{introduction}}

Leaf functional traits play an important role in ecological and physiological tradeoffs (Wright et al. \protect\hyperlink{ref-Wright2004a}{2004}, Onoda et al. \protect\hyperlink{ref-Onoda2017}{2017}, Reich \protect\hyperlink{ref-Reich2014}{2014}) and in carbon and nutrient cycling (e.g., Huntingford et al. \protect\hyperlink{ref-Huntingford2017}{2017}, Tcherkez et al. \protect\hyperlink{ref-Tcherkez2017}{2017}).
Thus, understanding the causes and consequences of leaf trait variation is an important goal in ecology, plant physiology, and biogeochemistry (references).
Different leaf assemblages exhibit markedly different patterns of trait variation.
For example, across global species, whole-leaf values of traits related to photosynthesis and metabolism (e.g., the photosynthetic capacity, respiration rate, or nutrient content of an entire leaf) tend to increase with leaf area, but not leaf mass per area (LMA; Osnas et al. (\protect\hyperlink{ref-Osnas2013}{2013})). In contrast, within species, these same whole-leaf trait values tend to increase with LMA from shade to full sunlight (Osada et al. \protect\hyperlink{ref-Osada2001}{2001}, Osnas et al. \protect\hyperlink{ref-Osnas2013}{2013}).
Functional groups (e.g., deciduous vs.~evergreen angiosperms) also differ from each other in terms of how interspecific trait variation depends on leaf mass and area (Osnas et al. \protect\hyperlink{ref-Osnas2018}{2018}).
These divergent patterns suggest the presence of multiple drivers of trait variation.

Strong interspecific correlations among LMA, leaf lifespan (LL), and mass-normalized leaf traits related to photosynthesis and metabolism have been interpreted as evidence for a single dominant axis of leaf functional variation (Wright et al. \protect\hyperlink{ref-Wright2004a}{2004}).
However, several studies suggest that these strong correlations result from mass-normalization of area-dependent traits (Lloyd et al. \protect\hyperlink{ref-Lloyd2013}{2013}, Osnas et al. \protect\hyperlink{ref-Osnas2013}{2013}).
Thus, although mass-normalization and the LES can be justified from economic principles (Westoby et al. \protect\hyperlink{ref-Westoby2013}{2013}), the evidence for a single dominant axis is controversial.

To understand why the same data can be interpreted as either supporting or opposing the existence of a single dominant axis of leaf trait variation, consider the conceptual model proposed by Osnas et al. (\protect\hyperlink{ref-Osnas2018}{2018}), in which LMA is comprised of two additive components: structural leaf mass per area (LMAs) and photosynthetic (hereafter `metabolic') leaf mass per area (LMAm).
Suppose that these two LMA components are independent axes of functional variation (Fig. \ref{fig:Hplt}a).
These two independent axes can be translated into either a two-dimensional trait space (if metabolic traits are area-normalized; Fig. \ref{fig:Hplt}b) or a one-dimensional trait space (if metabolic traits are mass-normalized; Fig. \ref{fig:Hplt}c).
Essentially, mass-normalization can collapse one of the functional dimensions, because leaves that differ substantially in their metabolic investment per-unit area can have similar values of mass-normalized metabolic traits (Fig. \ref{fig:Hplt}bc).

The hypothetical example in Fig. \ref{fig:Hplt} shows how mass-normalization can, in principle, make a two-dimensional trait space appear one-dimensional, but this example does not resolve the dimensionality of functional variation in real leaf assemblages.
One way to better understand the dimensionality of leaf trait variation is to compare models with different numbers of dimensions, and to ask if models with multiple dimensions provide improved statistical fits and conceptual insights compared to a single axis.
For example, the two-dimensional `LMAm + LMAs' model proposed by Osnas et al. (\protect\hyperlink{ref-Osnas2018}{2018}) could be compared to a one-dimensional LMA model in terms of their capacities to explain variation in other traits.
Doing so would require implementing the `LMAm + LMAs' model in a quantitative framework.
Implementing this model is challenging, because although certain leaf mass components can be neatly classified as `metabolic' or `structural' (Poorter et al. \protect\hyperlink{ref-Poorter2009}{2009}, Osnas et al. \protect\hyperlink{ref-Osnas2018}{2018}), other leaf mass components cannot.
For example, thick cell walls contribute to structural toughness (Onoda et al. \protect\hyperlink{ref-Onoda2015}{2015}), but at least some cell wall mass is required for the biomechanical support that enables photosynthesis.
One way forward is to use statistical methods to partition LMA into different components, rather than relying on direct measurements of morphological components (e.g., John et al. (\protect\hyperlink{ref-John2017}{2017})).

In this paper, we present a statistical modeling framework to partition LMA into metabolic and structural components: LMAm and LMAs.
We develop and test the model using leaf trait data from two tropical forest sites (sun and shade leaves from wet and dry sites in Panama) and the GLOPNET global leaf traits database (Wright et al. \protect\hyperlink{ref-Wright2004a}{2004}).
We use the model to address the following questions: (1) How does variation in LMA components related with mass and area proportionality of leaf traits? (2) Do LMAm and LMAs differ between evergreen and deciduous species, and between sun and shade leaves? and (3) How are measurable leaf photosynthetic and structural traits (e.g., concentrations of nitrogen and cellulose) related to LMAm and LMAs?

\hypertarget{material-and-methods}{%
\section{Material and Methods}\label{material-and-methods}}

\hypertarget{model-overview}{%
\subsection{Model overview}\label{model-overview}}

We developed a statistical modeling framework to partition LMA into additive LMAs and LMAm components (see below and Supplement SX).
For sample \emph{i} (e.g., sun or shade leaves of a given species; see Datasets below), we partition LMA\textsubscript{\emph{i}} into LMAm\textsubscript{\emph{i}} = \emph{f\textsubscript{i}} × LMA\textsubscript{\emph{i}} and LMAs\textsubscript{\emph{i}} = (1 -- \emph{f\textsubscript{i}}) × LMA\textsubscript{\emph{i}} by estimating a latent variable \emph{f\textsubscript{i}}.
The latent variables \emph{f\textsubscript{i}} are not directly observed, but they can be constrained by available data using Bayesian methods (Bishop \protect\hyperlink{ref-Bishop2006}{2006}, Gelman and Hill \protect\hyperlink{ref-Gelman2006}{2006}).
For example, we would expect high values of \emph{f\textsubscript{i}} (high LMAm\textsubscript{\emph{i}}) for leaves with high \emph{A}\textsubscript{area}, and low values of \emph{f\textsubscript{i}} (high LMAs\textsubscript{\emph{i}}) for leaves with high LL.
Given the large number of free parameters (i.e., one latent variable per leaf sample), it is possible for the model to over-fit the data, which could lead to spurious inferences.
Therefore, we performed tests with randomized data (see below) to evaluate model performance (Supplement SX).
Tests with randomized data suggest that our modeling approach reflect meaningful patterns in the observed trait data.

\hypertarget{modeling-leaf-lifespan-photosynthetic-capacity-and-dark-respiration-rate}{%
\subsection{Modeling leaf lifespan, photosynthetic capacity, and dark respiration rate}\label{modeling-leaf-lifespan-photosynthetic-capacity-and-dark-respiration-rate}}

We considered five sets of models, ranging from simple models with LMA as the sole predictor, to more complex models in which LMA was partitioned into metabolic and structural components (LMAm and LMAs).
The models are summarized in Table \ref{tab:modeltab}, and are explained in detail below.

First, we considered a simple set of models with LMA as the sole predictor for \emph{A}\textsubscript{area}, \emph{R}\textsubscript{area}, and LL.
In this model, the observed LMA for leaf sample i (where `sample' is a species in the GLOPNET dataset, or a species \(\times\) canopy position combination in the Panama dataset) are assumed to have scaling relationships with \emph{A}\textsubscript{area}, \emph{R}\textsubscript{area} and LL.

Next, we considered models in which LMA is partitioned into additive metabolic and structural functions; i.e., we assume that the sum of metabolic leaf mass per area (LMAm) and structural leaf mass per area (LMAs) is equal to total observed LMA for leaf sample \emph{i} (where a `sample' is a species in the GLOPNET dataset, or a species × canopy position combination in the Panama dataset):

\begin{align}
  &\mathrm{LMA}_{i} =\mathrm{LMAm}_{i} + \mathrm{LMAs}_{i} \label{eq:LMA}\\
  &\mathrm{LMAm}_{i} = f_{i} \mathrm{LMA}_{i} \label{eq:LMAm}\\
  &\mathrm{LMAs}_{i} = (1 - f_{i})  \mathrm{LMA}_{i}\label{eq:LMAs}
\end{align}

where, \emph{f\textsubscript{i}} is the fraction of LMA\textsubscript{\emph{i}} that is comprised of LMAm\textsubscript{\emph{i}}.
The \emph{f\textsubscript{i}} terms are not directly observed but can be estimated as latent variables in a Bayesian modeling framework (see details below).
In our model, net photosynthetic capacity, LL and total leaf dark respiration are determined by both metabolic and structural tissues:

In our model, net photosynthetic capacity (\emph{A}\textsubscript{area}) is determined by LMAm, LL is determined by LMAs, and total leaf dark respiration is determined by both metabolic and structural tissues:

\begin{align}
& \mathrm{E}[A_{\mathrm{area} \, i}]
= \alpha_0\mathrm{LMAm}_{i}^{\alpha_m}\mathrm{LMAs}_i^{\alpha_s}  =  \alpha_0 (f_i \mathrm{LMA}_{i})^{\alpha_m} \bigl\{(1-f_i) \mathrm{LMA}_{i}\bigr\}^{\alpha_s} \label{eq:E-A} \\
& \mathrm{E}[\mathrm{LL}_i] = \beta_0\mathrm{LMAm}_{i}^{\beta_m} \mathrm{LMAs}_{i}^{\beta_s}   = \beta_0 (f_i \mathrm{LMA}_{i})^{\beta_m} \bigl\{(1-f_i) \mathrm{LMA}_{i}\bigr\}^{\beta_s} \label{eq:E-LL} \\
& \mathrm{E}[R_{\mathrm{area} \, i}]
= \gamma_0\mathrm{LMAm}_{i}^{\gamma_m}ı \mathrm{LMAs}_{i}^{\gamma_s}
= \gamma_0 (f_i \mathrm{LMA}_{i})^{\gamma_m} \bigl\{(1-f_i)\mathrm{LMA}_{i}\bigr\}^{\gamma_s} \label{eq:E-R}
\end{align}

where, \emph{A}\textsubscript{area} \textsubscript{\emph{i}}, \emph{R}\textsubscript{area} \textsubscript{\emph{i}}, and LL\textsubscript{\emph{i}}, are, respectively, the net photosynthetic rate (\emph{A}\textsubscript{max}) per unit area, dark respiration rate (\emph{R}\textsubscript{dark}) per unit area, and leaf life span of leaf \emph{i}; \(\alpha_0\), \(\beta_0\), and \(\gamma_0\) are fitted constants; \(\alpha_m\), \(\beta_m\), and \(\gamma_m\) are fitted exponents that quantify the sensitivities of \emph{A}\textsubscript{area}, LL and \emph{R}\textsubscript{area}, respectively, to variation in LMAm; \(\alpha_s\), \(\beta_s\), and \(\gamma_s\) are fitted exponents that quantify the sensitivities of \emph{A}\textsubscript{area}, LL and \emph{R}\textsubscript{area}, respectively, to variation in LMAs.
The logarithms of \emph{A}\textsubscript{area}, LL, and \emph{R}\textsubscript{area} are assumed to have a multivariate normal distribution (see supplement).

We also considered models in which LMAs in Eq. \eqref{eq:E-LL} was replaced by leaf structural density, LMAs/LT, where LT is leaf thickness.
Leaf density has also been shown in some cases to be a good predictor for LL (Kitajima et al. \protect\hyperlink{ref-Kitajima2012}{2012}, \protect\hyperlink{ref-Kitajima2013}{2013}).
In our analysis, models with leaf density yielded qualitatively similar results as those based on LMAs, but did not perform as well in the model selection procedure (see below).
Therefore, we only present results of the models with LMAs.

Finally, we considered models that explicitly account for the effects of light availability on LL, consistent with optimall LL theory {[}Kikuzawa (\protect\hyperlink{ref-Kikuzawa1991}{1991}); Kikuzawa2004{]}.
Eqs. \eqref{eq:E-A}-\eqref{eq:E-R} implicitly account for light availability, because LMA tends to increase from shade to sun within species (Onoda et al. \protect\hyperlink{ref-Onoda2008}{2008}, Kenzo et al. \protect\hyperlink{ref-Kenzo2006}{2006}, Terashima et al. \protect\hyperlink{ref-Terashima2011}{2011}) that can be allocated to metabolic and/or structural functions.
However, LMA cannot account for both inter- and intraspecific variation in LL, because LL increases as LMA increases across species but decreases as LMA increases within species from shade to sun (Lusk et al. \protect\hyperlink{ref-Lusk2008}{2008}, Russo and Kitajima \protect\hyperlink{ref-Russo2016}{2016}) different light conditions.
Accounting for the details of leaf carbon balance -- e.g., diurnal and seasonal cycles (Xu et al. \protect\hyperlink{ref-Xu2017}{2017}) -- would complicate our modeling framework, but we can account for light effects in a simple way by -modifying Eq. \eqref{eq:E-LL} as follows:

\begin{align}
\mathrm{E[LL}_i] = \beta_0\mathrm{LMAm}_{i}^{\beta_m} \mathrm{LMAs}_{i}^{\beta_s} exp(\theta Light_i) \label{eq:E-LL2}
\end{align}

where \(\theta\) is the effect of light for sun leaves and the dummy variable \(Light_i\) is set to 1 for sun leaves and 0 for shade leaves.

\hypertarget{datasets}{%
\subsection{Datasets}\label{datasets}}

To fit the models described above, we used LMA (g m\textsuperscript{-2}), net photosynthetic capacity per unit leaf area (\emph{A}\textsubscript{area}; mol s\textsuperscript{-1} m\textsuperscript{-2}), dark respiration rate per unit leaf area (\emph{R}\textsubscript{area}; mol s\textsuperscript{-1} m\textsuperscript{-2}) and LL (months) from the GLOPNET global leaf traits database (Wright et al. \protect\hyperlink{ref-Wright2004a}{2004}) and from two tropical forest sites in Panama: Monumental Natural Metropolitano (MNM, ``dry site'') and Bosque Protector San Lorenzo (SL, ``wet site'')(Osnas et al. \protect\hyperlink{ref-Osnas2018}{2018}).
The GLOPNET data primarily represent interspecific variation, whereas the Panama data represent both inter- and intraspecific variation, including leaves sampled at two canopy positions (``sun'': full sun at the top of the canopy; and ``shade'': well shaded, sampled within 2 m of the forest floor) from trees within reach of canopy cranes.
The dry MNM site is a semi-deciduous coastal Pacific forest with a 5-month dry season from December-April and 1740 mm of annual rainfall (Wright et al. \protect\hyperlink{ref-Wright2003}{2003}).
The MNM crane is 40 m tall with a 51 m long boom. The wet SL site is an evergreen Caribbean coastal forest with 3100 mm of annual rainfall (Wright et al. \protect\hyperlink{ref-Wright2003}{2003}). The SL crane is 52 m tall with a 54 m long boom.

After deleting leaf samples (i.e., database records, which typically average over multiple individual leaves) that lacked one of the four traits (LMA, \emph{A}\textsubscript{area}, \emph{R}\textsubscript{area}, or LL), 198 samples for 198 unique species were available for GLOPNET, and 130 samples for 104 unique species were available for Panama (dry and wet sites combined; 26 species sampled in both sun and shade; no species with all four traits available at both sites).
For Panama, leaf thickness is available for 106 samples, which is used for model selection (See below).
Both datasets include additional traits that we used to interpret model results, but which were not used to fit models. These traits include nitrogen and phosphorus content per leaf unit area (\emph{N}\textsubscript{area} and \emph{P}\textsubscript{area}; g m\textsuperscript{-2}) in both datasets, leaf habit in GLOPNET (deciduous or evergreen), and cellulose content per unit area (\emph{CL}\textsubscript{area}; g m\textsuperscript{-2}) in Panama.

\hypertarget{model-estimation-and-selection}{%
\subsection{Model estimation and selection}\label{model-estimation-and-selection}}

We modeled \emph{A}\textsubscript{area} and \emph{R}\textsubscript{area} using Eqs. \eqref{eq:E-A} and \eqref{eq:E-R}, respectively, for both GLOPNET and Panam.
To model LL for GLOPNET, we used Eq. \eqref{eq:E-LL} (no light effects) for GL, because GLOPNET includes no information on light availability.
To model LL for Panama, we used Eq. \eqref{eq:E-LL2} (light effects model), which was motivated by the negative intraspecific LL-LMA relationship observed in Panama (Xu et al. \protect\hyperlink{ref-Xu2017}{2017}, Osnas et al. \protect\hyperlink{ref-Osnas2018}{2018}) and elsewhere (Lusk et al. \protect\hyperlink{ref-Lusk2008}{2008}, Russo and Kitajima \protect\hyperlink{ref-Russo2016}{2016}) and elsewhere.

Posterior distributions of all parameters were estimated using the Hamiltonian Monte Carlo algorithm (HMC) implemented in Stan (Carpenter et al. \protect\hyperlink{ref-Carpenter2017}{2017}).
Prior distributions for the latent variables \emph{f}\textsubscript{i} (which are used to partition LMA into LMAm and LMAs according to Eqs. \eqref{eq:E-A}-\eqref{eq:E-R}) were non-informative uniform priors beta(1, 1) (i.e., LMA was partitioned based on patterns in the data). See supplement X for more detail.
The Stan code use to fit models are also available from Github at: \href{https://github.com/mattocci27/LMAmLMAs}{https://github.com/mattocci27/XXX}.

Some parameters are obviously exchangeable in our model.
For example, \((f_i, \alpha_m, \alpha_s)\) = (0.2, 0.5, 1.0) and (0.8, 1.0, 0.5) yield the invariance of the likelihood.
To avoid this non-identifiability, we explore two methods based on our assumption that \emph{A}\textsubscript{area} depends more strongly on metabolic leaf mass (LMAm; parameter \(\alpha_m\) in Eq. \eqref{eq:E-A}) than structural leaf mass (LMAs; parameter \(\alpha_s\) in Eq. \eqref{eq:E-A}.
In the first method, we switched the \(\alpha_m\) and \(\alpha_s\) labels within a chain when the posterior means for \(\alpha_m\) was smaller than that for \(\alpha_s\) after the posterior distributions converged.
In the second method, we constrained \(\alpha_m\) \textgreater{} \(\alpha_s\) during MCMC.
Both methods yielded qualitatively similar results; we report results from the second method.
Convergence of the posterior distribution was assessed with the Gelman-Rubin statistic with a convergence threshold of 1.1 for all diagnostics (Gelman et al. \protect\hyperlink{ref-Gelman2013}{2013}).

Because our statistical approach includes one latent variable \emph{f\textsubscript{i}} to partition LMA into LMAm and LMAs for each leaf sample, one might expect a good match between predictions and observations simply due to the large number of free parameters, whether or not the model captured important biological mechanisms.
To evaluate model performance while controlling for the number of free parameters, we compared the fit of our model to observed and randomized datasets.
Specifically, we generated randomized datasets by randomizing all the trait values (LMA, \emph{A}\textsubscript{area}, \emph{R}\textsubscript{area} and LL) across species.
Thus, the randomized datasets had zero expected covariance among traits.
Model results obtained from the randomized datasets did not produce any significant patterns for the GLOPNET data or models did not converged for the Panama data (Supplement X).
Thus, the model is not guaranteed to work based on the assumptions alone.
We assume that estimates of LMAm and LMAs reflect meaningful patterns in the observed data.

To assess if decomposing LMA into photosynthetic and structural components lead to improved our model, we compared our model with models that only use LMA instead of LMAm and LMAs.
We assessed the model performance by using Pareto-smoothed importance sampling leave-one-out cross-validation (LOO; Vehtari et al. (\protect\hyperlink{ref-Vehtari2014}{2014})).
Similar to other information criteria (e.g.~Akaike Information Criterion (AIC); Burnham and Anderson (\protect\hyperlink{ref-Burnham2002}{2002})), a better model shows a smaller LOO Information Criterion (LOOIC).

\hypertarget{partitioning-lma-variance-into-metabolic-and-structural-components}{%
\subsection{Partitioning LMA variance into metabolic and structural components}\label{partitioning-lma-variance-into-metabolic-and-structural-components}}

We used the following identity to estimate the relative contributions of LMAm and LMAs to LMA variance, where again LMA = LMAm + LMAs:

\begin{align}
\mathrm{Var}(Y = X1 + X2) = \mathrm{Cov}(Y, X1+X2) = \mathrm{Cov}(Y,X1) + \mathrm{Cov}(Y,X2) \label{eq:var}
\end{align}

Thus, the fractions of total LMA variance due to variance in LMAm and LMAs were determined by the covariances Cov(LMA, LMAm) and Cov(LMA, LMAs), respectively, taken as proportions of the total variance Var(LMA).

\hypertarget{understanding-relationships-between-photosynthetic-capacity-and-lma}{%
\subsection{Understanding relationships between photosynthetic capacity and LMA}\label{understanding-relationships-between-photosynthetic-capacity-and-lma}}

We applied our LMAm+LMAs model (Eq. \eqref{eq:E-A}) to simulated data to better understand relationships between photosynthetic capacity (\emph{A}\textsubscript{max}) and LMA.
The causes and interpretation of \emph{A}\textsubscript{max} vs.~LMA relationships are controversial (Westoby et al. \protect\hyperlink{ref-Westoby2013}{2013}).
Although \emph{A}\textsubscript{max} is often mass-normalized (e.g., Wright et al. (\protect\hyperlink{ref-Wright2005}{2005}); Shipley et al. (\protect\hyperlink{ref-Shipley2006}{2006}); Blonder et al. (\protect\hyperlink{ref-Blonder2011}{2011})), Lloyd et al. (\protect\hyperlink{ref-Lloyd2013}{2013}) argued that photosynthesis is an area-based process, and therefore \emph{A}\textsubscript{max} should be area-normalized rather than mass-normalized when exploring trait relationships.
Consistent with this argument, Osnas et al. (\protect\hyperlink{ref-Osnas2013}{2013}) showed that across global species, \emph{A}\textsubscript{max} variation is strongly dependent on leaf area, but only weakly dependent on leaf mass (after controlling for interspecific variation in leaf area).

To better understand the factors affecting \emph{A}\textsubscript{max} vs.~LMA relationships, we created simulated datasets in which we varied the following factors: the sensitivity of \emph{A}\textsubscript{area} to variation in LMAm (parameter \(\alpha_m\) in Eq. \eqref{eq:E-A}); the sensitivity of \emph{A}\textsubscript{area} to LMAs (parameter \(\alpha_s\) in Eq. \eqref{eq:E-A}), and the degree to which LMAm and LMAs varied among leaves.
For each simulated dataset, we quantified the \emph{A}\textsubscript{max} vs.~LMA relationship following Osnas et al. (\protect\hyperlink{ref-Osnas2018}{2018}):

\begin{align}
A_{\mathrm{area} \, i} = a (LMA_i)^{b}\epsilon_i \label{eq:mass}
\end{align}

where LMA is the sum of LMAm and LMAs (Eq. \eqref{eq:LMA}), \emph{a} is a fitted constant, and \emph{b} is an index of mass-dependence as illustrated by the following cases (Osnas et al. \protect\hyperlink{ref-Osnas2013}{2013}, \protect\hyperlink{ref-Osnas2018}{2018}): if \emph{b} = 0, then \emph{A}\textsubscript{area} is independent of LMA, which implies that whole-leaf \emph{A}\textsubscript{max} is proportional to leaf area; conversely, if \emph{b} = 1, then \emph{A}\textsubscript{area} is proportional to LMA, which implies that the whole-leaf \emph{A}\textsubscript{max} is proportional to leaf mass.
Intermediate cases (0 \textless{} b \textless{} 1), as well as more extreme cases (b \textless{} 0 or b \textgreater{} 1), are also possible.
Note that although Eq. \eqref{eq:mass} uses area-normalized \emph{A}\textsubscript{max}, equivalent results can be obtained from mass-normalized \emph{A}\textsubscript{max} (Osnas et al. \protect\hyperlink{ref-Osnas2018}{2018}).

We created simulated datasets based on a range of assumptions for the LMA variance components: (i) LMAm variance \textless\textless{} LMAs variance, (ii) LMAm variance = LMAs variance, (iii) LMAm variance \textgreater\textgreater{} LMAs variance, (iv) our estimates (posterior means) of LMAm and LMAs variances from GLOPNET, (v) our estimates of LMAm and LMAs variances for Panama sun leaves, and (vi) our estimates of LMAm and LMAs variances for Panama shade leaves.
For each of the above cases, we considered a range of values for and (which determine the sensitivity of \emph{A}\textsubscript{area} to LMAm and LMAs in Eq \eqref{eq:E-A}), as well as the empirical estimates of these values based on GLOPNET and Panama datasets.
Each simulation had a sample size of 100 leaves, where the values of LMAm and LMAs for each leaf was a random draw from a lognormal distribution with median and variance as follows: for simulations (i)-(iii), median log(40) and variance log(1) or log(4) (for low and high variance, respectively); and for simulations (iv)-(vi), medians and variances were the posterior means from our GLOPNET and Panama analyses.

\hypertarget{results}{%
\section{Results}\label{results}}

\textbf{1. Nearly all leaf dark respiration is associated with metabolic leaf tissue mass.}
Metabolic leaf mass (LMAm) accounted for nearly all leaf dark respiration; i.e., estimated dark respiration rate per-unit structural mass (\(\gamma_s\)) was close to zero in analyses of both GLOPNET and Panama data (Table \ref{tab:paratab}).
Thus, although building costs are likely similar for different leaf chemical components and tissues (Williams et al. \protect\hyperlink{ref-Williams1989}{1989}, Villar and Merino \protect\hyperlink{ref-Villar2001}{2001}), our results suggest that leaf mass associated with photosynthetic function accounts for nearly all leaf maintenance respiration.

\textbf{2. Decomposing LMA into metabolic and structural components leads to improved predictions of \emph{A}\textsubscript{area}, \emph{R}\textsubscript{area} and LL.}
For both the GLOPNET global datasets and the Panama datasets, model including LMAm and LMAs showed better predictions than model including LMA based on approximate cross-validation (Table \ref{tab:lootab}), suggesting that decompose LMA into metabolic and structural components leads improved predictions of \emph{A}\textsubscript{area}, \emph{R}\textsubscript{area} and LL.

For the GLOPNET global dataset, \emph{A}\textsubscript{area} had a strong positive correlation with LMAm, a weak negative correlation with LMAs, and a non-significant correlation with total LMA (Figs. \ref{fig:GLplt}a-c; Table \ref{tab:lootab}).
\emph{R}\textsubscript{area} in GLOPNET also had a strong positive correlation with LMAm, which was stronger than the correlation between \emph{R}\textsubscript{area} and either LMA or LMAs (Figs. \ref{fig:GLplt}d-f).
Finally, LL in GLOPNET had a strong positive correlation with LMAs, which was stronger than the correlation between LL and either LMA or LMAm (Figs. \ref{fig:GLplt}g-i).

For the Panama dataset, we evaluated multiple models due to the availability of both sun and shade leaves.
The model including the light effects fit the data better than the model without the light effects (Table \ref{tab:lootab}).
All Panama results we report are for the model including the light effects unless stated otherwise.
\emph{A}\textsubscript{area} and \emph{R}\textsubscript{area} had stronger and more positive correlations with LMAm than with LMA or LMAs (Figs. \ref{fig:PAplt}a-f and Table \ref{tab:paratab}).
LL was not significantly correlated with LMA when all leaves were combined, but was strongly correlated with LMA for shade leaves at the dry site (Fig. \ref{fig:PAplt}g).
There was an apparent negative correlation between LMAm and LL (Fig. \ref{fig:PAplt}h) but the effect of LMAm on LL was not significant (Table \ref{tab:paratab}), implying that a weak covariance between LMAm and LMAs produced this negative correlation.
There was a positive correlation between LL and LMAs (Fig. \ref{fig:PAplt}i and Table \ref{tab:paratab}).
Because the LL vs.~LMAs relationship ignores important factors (i.e., effects of light on realized photosynthetic rates), the explained variance of LL from the model was higher (r\textsuperscript{2} = 0.45) than expected from the correlation between LMAs and LL (Fig. \ref{fig:PAplt}g, \ref{fig:LLplt}).

\textbf{3. Interspecific mass proportionality of \emph{A}\textsubscript{max} depended on both scaling slope and variance in LMA components.}
Percent of interspecific variation in LMA explained by structural components were 46\%, 96\% and 63\% for GLOPNET, shade leaves in Panama and sun leaves in Panama, respectively.
Large variance in LMAm and LMAs among leaves leads to mass and area proportionality of \emph{A}\textsubscript{max}, respectively (Fig. SX). Additionally, increasing the effects of LMAm on \emph{A}\textsubscript{area} (\(\alpha_m\)) leads to mass proportionality and increasing the effects of LMAs on \emph{A}\textsubscript{area} (\(\alpha_s\)) leads to area proportionality (Fig. \ref{fig:massplt}).

Interspecific variation in \emph{A}\textsubscript{max} among leaves in GLOPNET tended to be area proportional because of their negative \(\alpha_s\) and small \(\alpha_m\) values (Fig. \ref{fig:massplt}).
Interspecific variation in LMA in shade leaves in the Panama data was area depend because of their large variance in LMAs (Fig. \ref{fig:massplt}, Fig. SX).
Although sun leaves in the Panama data showed area proportionality, the
pattern is weaker than the rest of two because of the balance among \(\alpha_m\), \(\alpha_s\) and variance in LMAs (Fig. \ref{fig:massplt}).

\textbf{4. Evergreen leaves have both greater LMAm and LMAs than deciduous leaves, and sun leaves have both greater LMAm and LMAs than shade leaves.}
In the GLOPNET dataset, although evergreen leaves had higher both LMAm and LMAs than deciduous leaves, differences in LMAs between the two groups were more apparent (Fig. \ref{fig:boxplt}).
Thus, the higher total LMA in evergreen leaves in GLOPNET was primarily due to differences in LMAs.
Evergreen leaves have higher fraction of LMAs compared to deciduous leaves in the Panama dataset but that pattern was weak in the GLOPNET (Fig. SX).
In the Panama dataset, sun leaves have both greater LMAm and LMAs than shade leaves, but the differences in LMAm were more apparent (Fig. \ref{fig:boxplt}).
Thus, variation in LMAm plays a more important role within species than among species.

\textbf{5. Nitrogen and phosphorus per-unit leaf area are strongly correlated with LMAm, and cellulose per-unit leaf area is strongly correlated with LMAs.}
In the GLOPNET dataset, \emph{N}\textsubscript{area} and \emph{P}\textsubscript{area} had strong positive correlations with LMAm, but only weak correlations with LMAs (Fig. SX).
Similarly, in the Panama dataset, \emph{N}\textsubscript{area} and \emph{P}\textsubscript{area} had strong positive correlations with LMAm, but were not correlated with LMAs (Fig. \ref{fig:PA-NPC}).
In contrast, cellulose per-unit leaf area (\emph{CL}\textsubscript{area}), which was available for the Panama dataset but not for GLOPNET, had a strong positive correlation with LMAs, and a weak positive correlation with LMAm (Fig. \ref{fig:PA-NPC}).
\emph{CL}\textsubscript{area} was more strongly correlated with LMA than with LMAm or LMAs, but sun and shade leaves aligned along a common \emph{CL}\textsubscript{area}-LMAs relationship, as opposed to being offset for LMA and LMAm (Figs. \ref{fig:PA-NPC}g-i).

\hypertarget{discussion}{%
\section{Discussion}\label{discussion}}

Our analyses demonstrate that decomposing LMA variation into separate metabolic and structural components (LMAm and LMAs, respectively) leads to improved predictions of photosynthetic capacity (\emph{A}\textsubscript{max}), dark respiration rate (\emph{R}\textsubscript{dark}), and leaf lifespan (LL), as well as clear relationships with traits used for independent model evaluation (nitrogen, phosphorus, and cellulose concentrations).
The model with LMAm and LMAs showed better predictive accuracy than model with LMA alone for both intraspecific variation in relation to light and interspecific variation.
Below, we elaborate on the insights gained from our analysis and the implications of our results for the representation of leaf functional diversity in global ecosystem models.

Decomposing LMA into LMAm and LMAs provides insights into why interspecific variation in leaf traits related to photosynthesis and metabolism are primarily area-dependent (i.e., primarily independent of LMA when expressed per-unit area rather than mass-dependent (Osnas et al. \protect\hyperlink{ref-Osnas2013}{2013}, \protect\hyperlink{ref-Osnas2018}{2018})).
We expected that leaf metabolic traits tend to be area-depend when variance in LMAm is larger than variance in LMAs.
However, our results suggest that the scaling slopes (\(\alpha_m\) and \(\alpha_s\)) between LMAm, LMAs and \emph{A}\textsubscript{area} also affect trait area proportionality.
Leaf metabolic traits tend to be area-dependent when (i) variance in LMAm is small or similar to variance in LMAs, (ii) the effects of LMAm and LMAs on \emph{A}\textsubscript{area} (\(\alpha_m\) and \(\alpha_s\)) are small, (iii) or combinations of those (Fig. \ref{fig:massplt}).
Consistent with this explanation, the assemblage we examined where LMAs counted for the highest fraction of total LMA variation (96\% for Panama shade leaves) is also the assembled with the highest degree of trait area-dependence (Osnas et al. \protect\hyperlink{ref-Osnas2018}{2018}).
Our results also suggest that high degree of trait area-dependence in GLOPNET is due to the negative effect of LMAs on \emph{A}\textsubscript{area}.

The scaling slope \(\alpha_m\) less than 1 (Table \ref{tab:paratab}) suggest a diminishing return on addition LMAm (i.e., doubling LMAm does not simply result in doubling \emph{A}\textsubscript{max}).

\begin{enumerate}
\def\labelenumi{\arabic{enumi}.}
\tightlist
\item
  High LMAm might cause lower mesophyll conductance
\item
  High LMAm (thinck leaves) might cause chloroplast self-shading within a leaf
\item
  LMAm = (purely) metabolic active leaf mass + something we couldn't estimate in our analysis. All the leaf mass need to be allocated to LMAm or LMAs in our model. However, variation in non-structural carbohydrates in leaves, which is not estimated in our model, is relatively large (Martínez-Vilalta et al. \protect\hyperlink{ref-Martinez-Vilalta2016}{2016}). Simply partitioning leaf mass into two functions might underestimate how much \emph{A}\textsubscript{max} depending on the metabolic leaf mass .
\end{enumerate}

Decomposing LMA also provides insights as to why intraspecific patterns of trait variation differ from those observed across species.
In contrast to interspecific LMA variation, our analysis suggests that LMAm contributes half or more of the intraspecific increase in LMA from shade to sun (Figs. \ref{fig:massplt} andhSX).
The increase in LMAm from shade to sun -- which likely reflects an increase in the size and number of palisade mesophyll cells with increasing light availability (Onoda et al. \protect\hyperlink{ref-Onoda2008}{2008}, Terashima et al. \protect\hyperlink{ref-Terashima2011}{2011}) -- is also associated with an increase in LMAs from shade to sun (Fig. \ref{fig:boxplt}).
This positive covariance between LMAm and LMAs within species means that per-area values of LMAm-proportional traits (e.g., \emph{A}\textsubscript{area}) have a strong, positive relationship with total LMA, which implies trait mass-proportionality (Osnas et al. \protect\hyperlink{ref-Osnas2018}{2018}).

The improved predictions and understanding provided by decomposing LMA into photosynthetic and structural components challenge the view that leaf functional diversity can be accurately represented by a single leaf economics spectrum (LES) axis (Wright et al. \protect\hyperlink{ref-Wright2004a}{2004}).
Lloyd et al. (\protect\hyperlink{ref-Lloyd2013}{2013}) argued that the apparent dominance of a single LES axis is an artifact of expressing area-dependent leaf traits on a per-mass basis, and Osnas et al. (\protect\hyperlink{ref-Osnas2013}{2013}) demonstrated that across the global flora, traits related to photosynthesis and metabolism are indeed area-dependent.
Intraspecific patterns in trait variation, which contrast with interspecific patterns, pose additional challenges for a one-dimensional view of leaf functional diversity. Our analysis shows that considering two primary axes of leaf trait variation (photosynthesis and structure) provides improved quantitative predictions and insights compared to LES of a single dominant axis.
Our results point to a simple two-dimensional framework for representing leaf functional diversity in global ecosystem models: a LMAm axis that determines \emph{A}\textsubscript{max}, accounts for nearly all \emph{R}\textsubscript{dark} (see rp and rs estimates in Table S1) and a LMAs axis that determines potential LL through its effects on leaf toughness (Kleyer et al. \protect\hyperlink{ref-Kleyer2012}{2012}).
In the datasets we analyzed (the global flora and tropical trees in Panama), these two axes are only weakly correlated with each other (Fig. S5), which suggests that trait-based approaches to global ecosystem modeling (Scheiter et al. \protect\hyperlink{ref-Scheiter2013}{2013}, Wullschleger et al. \protect\hyperlink{ref-Wullschleger2014}{2014}) could consider these as independent axes.
For example, to simulate individual competing trees which from diverse community of growth strategies (Sakschewski et al. \protect\hyperlink{ref-Sakschewski2015}{2015}, \protect\hyperlink{ref-Sakschewski2016}{2016}), it would be possible to draw LMAm and LMAs values from distributions with realistic ranges instead of a single LMA distribution.

Our statistical decomposition of LMA into metabolic and structural components provides important insights, but additional insights and accuracy could be gained by a more mechanistic modeling approach.
For example, John et al. (\protect\hyperlink{ref-John2017}{2017}) decomposed interspecific LMA variation into anatomical components such as the size, number of layers, and mass density of cells in different leaf tissues.
If such detailed information became available for a large number of leaves, representing both intra- and interspecific variation, it should be possible to quantify how these anatomical traits scale up to leaf-level \emph{A}\textsubscript{max} , \emph{R}\textsubscript{dark}, and LL.
A simpler alternative would be to modify our model to account for variation in cell wall thickness (TCW): for a given cell size, increasing TCW would lead to an increase in lamina density, cellulose per volume, toughness, and LL (Kitajima et al. \protect\hyperlink{ref-Kitajima2012}{2012}, \protect\hyperlink{ref-Kitajima2016}{2016}, Kitajima and Poorter \protect\hyperlink{ref-Kitajima2010}{2010}), and a decrease in mesophyll conductance and \emph{A}\textsubscript{max} (Onoda et al. \protect\hyperlink{ref-Onoda2017}{2017}, Evans et al. \protect\hyperlink{ref-Evans2009}{2009}, Terashima et al. \protect\hyperlink{ref-Terashima2011}{2011}).
Indeed, our model might capture those patterns. The negative effects of LMAs on \emph{A}\textsubscript{area} in GLOPNET data indicate that high amount of photosynthetic proteins is not always translated into high photosynthetic capacity.
High LMAs that may be associated with a particular anatomical components might lead to lower mesophyll conductance.

\hypertarget{conclusions}{%
\section{Conclusions}\label{conclusions}}

It is widely recognized that LMA variation is associated with multiple tissues and functions, including metabolically active mesophyll that largely determines photosynthetic capacity, as well as structural and chemical components that contribute primarily to leaf toughness and defense (Roderick et al. \protect\hyperlink{ref-Roderick1999}{1999}, Shipley et al. \protect\hyperlink{ref-Shipley2006}{2006}, Lusk et al. \protect\hyperlink{ref-Lusk2010}{2010}).
It should not be surprising then, that partitioning LMA into metabolic and structural components yields enhanced predictions and improved understanding of patterns of leaf trait variation both within and among species.
Yet for over a decade, the vast literature on leaf traits has been strongly influenced by the view that leaf trait variation can usefully be represented by a single dominant axis of LES.
Our results provide quantitative evidence that this one-dimensional view of leaf trait variation is insufficient, and our model provides a biological explanation for previous statistical analyses that have demonstrated area-dependence of leaf traits across species (Osnas et al. \protect\hyperlink{ref-Osnas2013}{2013}), while also explaining mass-dependence within species. Our results suggest that small scaling slope between metabolic leaf mass and photosynthetic rate, and similar variance in LMAm and in LMAs leads to area-proportionality of interspecific leaf traits.
Thus, strong interspecific correlations between LMA and mass-normalized photosynthetic capacity (and related traits, such as respiration rate, and nitrogen and phosphorus concentrations) are likely driven by mass-normalization itself, rather than any functional dependence of these photosynthetic and metabolic traits on LMA (Lloyd et al. \protect\hyperlink{ref-Lloyd2013}{2013}, Osnas et al. \protect\hyperlink{ref-Osnas2013}{2013}).
In contrast, intraspecific variation in LMA is driven by coordinated changes in structural and metabolic mass components, which explains why mass-normalized photosynthetic and metabolic traits vary little from sun to shade within species (Poorter et al. \protect\hyperlink{ref-Poorter2006b}{2006}, Aranda et al. \protect\hyperlink{ref-Aranda2004}{2004}, Niinemets et al. \protect\hyperlink{ref-Niinemets2015}{2015}).

\hypertarget{acknowledgments}{%
\section{Acknowledgments}\label{acknowledgments}}

We thank Jonathan Dushoff for statistical advice and Martijn Slot for~helpful~comments that improved the paper.
Mirna Samaniego and Milton Garica provided indispensable assistance in data collection.
We thank the Smithsonian Tropical Research Institute (STRI), the Tropical~Canopy Biology Program at STRI and the Andrew W. Mellon Foundation for supporting this work. MK was supported by a Postdoctoral~Fellowship for~Research Abroad from the Japan Society for the Promotion of Science and the CAS President's International Fellowship for Young Staff.

\hypertarget{author-contributions}{%
\section{Author contributions}\label{author-contributions}}

MK, JWL and JLDO conceived of the study; KK, SJW and SAVB contributed data; MK devised the analytical approach and performed analyses; MK and JWL wrote the first draft of the manuscript, and all authors contributed to revisions.

\hypertarget{references}{%
\section{References}\label{references}}

\hypertarget{refs}{}
\begin{cslreferences}
\leavevmode\hypertarget{ref-Aranda2004}{}%
Aranda, I., F. Pardo, L. Gil, and J. A. Pardos. 2004. Anatomical basis of the change in leaf mass per area and nitrogen investment with relative irradiance within the canopy of eight temperate tree species. Acta Oecologica 25:187--195.

\leavevmode\hypertarget{ref-Bishop2006}{}%
Bishop, C. M. 2006. Pattern Recognition and Machine Learning. Springer, New York.

\leavevmode\hypertarget{ref-Blonder2011}{}%
Blonder, B., C. Violle, L. P. Bentley, and B. J. Enquist. 2011. Venation networks and the origin of the leaf economics spectrum. Ecol. Lett. 14:91--100.

\leavevmode\hypertarget{ref-Burnham2002}{}%
Burnham, K. P., and D. R. Anderson. 2002. Model Selection and Multimodel Inference: A Practical Information-Theoretic Approach. Second. Springer-Verlag, New York.

\leavevmode\hypertarget{ref-Carpenter2017}{}%
Carpenter, B., A. Gelman, M. D. Hoffman, D. Lee, B. Goodrich, M. Betancourt, M. Brubaker, J. Guo, P. Li, and A. Riddell. 2017. Stan : A Probabilistic Programming Language. J. Stat. Softw. 76:1--32.

\leavevmode\hypertarget{ref-Evans2009}{}%
Evans, J. R., R. Kaldenhoff, B. Genty, and I. Terashima. 2009. Resistances along the CO2 diffusion pathway inside leaves. J. Exp. Bot. 60:2235--2248.

\leavevmode\hypertarget{ref-Gelman2013}{}%
Gelman, A., J. B. Carlin, H. S. Stern, D. B. Dunson, A. Vehtari, and D. B. Rubin. 2013. Bayesian Data Analysis, Third Edition. Taylor \& Francis.

\leavevmode\hypertarget{ref-Gelman2006}{}%
Gelman, A., and J. Hill. 2006. Multilevel regression. Data Analysis Using Regression and Multilevel/Hierarchical Models.

\leavevmode\hypertarget{ref-Huntingford2017}{}%
Huntingford, C., O. K. Atkin, A. Martinez-de la Torre, L. M. Mercado, M. A. Heskel, A. B. Harper, K. J. Bloomfield, O. S. O'Sullivan, P. B. Reich, K. R. Wythers, E. E. Butler, M. Chen, K. L. Griffin, P. Meir, M. G. Tjoelker, M. H. Turnbull, S. Sitch, A. Wiltshire, and Y. Malhi. 2017. Implications of improved representations of plant respiration in a changing climate. Nat. Commun. 8:1602.

\leavevmode\hypertarget{ref-John2017}{}%
John, G. P., C. Scoffoni, T. N. Buckley, R. Villar, H. Poorter, and L. Sack. 2017. The anatomical and compositional basis of leaf mass per area. Ecol. Lett. 20:412--425.

\leavevmode\hypertarget{ref-Kenzo2006}{}%
Kenzo, T., T. Ichie, Y. Watanabe, R. Yoneda, I. Ninomiya, and T. Koike. 2006. Changes in photosynthesis and leaf characteristics with tree height in five dipterocarp species in a tropical rain forest. Tree Physiol. 26:865--873.

\leavevmode\hypertarget{ref-Kikuzawa1991}{}%
Kikuzawa, K. 1991. A Cost-Benefit Analysis of Leaf Habit and Leaf Longevity of Trees and Their Geographical. Am. Nat. 138:1250--1263.

\leavevmode\hypertarget{ref-Kitajima2013}{}%
Kitajima, K., R. A. Cordero, and S. J. Wright. 2013. Leaf life span spectrum of tropical woody seedlings: Effects of light and ontogeny and consequences for survival. Ann. Bot. 112:685--699.

\leavevmode\hypertarget{ref-Kitajima2012}{}%
Kitajima, K., A. M. Llorens, C. Stefanescu, M. V. Timchenko, P. W. Lucas, and S. J. Wright. 2012. How cellulose-based leaf toughness and lamina density contribute to long leaf lifespans of shade-tolerant species. New Phytol. 195:640--652.

\leavevmode\hypertarget{ref-Kitajima2010}{}%
Kitajima, K., and L. Poorter. 2010. Tissue-level leaf toughness, but not lamina thickness, predicts sapling leaf lifespan and shade tolerance of tropical tree species. New Phytol. 186:708--721.

\leavevmode\hypertarget{ref-Kitajima2016}{}%
Kitajima, K., S. J. Wright, and J. W. Westbrook. 2016. Leaf cellulose density as the key determinant of inter- and intra-specific variation in leaf fracture toughness in a species-rich tropical forest. Interface Focus 6:20150100.

\leavevmode\hypertarget{ref-Kleyer2012}{}%
Kleyer, M., S. Dray, F. Bello, J. Lepš, R. J. Pakeman, B. Strauss, W. Thuiller, and S. Lavorel. 2012. Assessing species and community functional responses to environmental gradients: Which multivariate methods? J. Veg. Sci. 23:805--821.

\leavevmode\hypertarget{ref-Lloyd2013}{}%
Lloyd, J., K. Bloomfield, T. F. Domingues, and G. D. Farquhar. 2013. Photosynthetically relevant foliar traits correlating better on a mass vs an area basis: Of ecophysiological relevance or just a case of mathematical imperatives and statistical quicksand? New Phytol. 199:311--321.

\leavevmode\hypertarget{ref-Lusk2010}{}%
Lusk, C. H., Y. Onoda, R. Kooyman, and A. Gutiérrez-Girón. 2010. Reconciling species-level vs plastic responses of evergreen leaf structure to light gradients: Shade leaves punch above their weight. New Phytol. 186:429--438.

\leavevmode\hypertarget{ref-Lusk2008}{}%
Lusk, C. H., P. B. Reich, R. A. Montgomery, D. D. Ackerly, and J. Cavender-Bares. 2008. Why are evergreen leaves so contrary about shade? Trends Ecol. Evol. 23:299--303.

\leavevmode\hypertarget{ref-Martinez-Vilalta2016}{}%
Martínez-Vilalta, J., A. Sala, D. Asensio, L. Galiano, G. Hoch, S. Palacio, F. I. Piper, and F. Lloret. 2016. Dynamics of non-structural carbohydrates in terrestrial plants: A global synthesis. Ecol. Monogr. 86:495--516.

\leavevmode\hypertarget{ref-Niinemets2015}{}%
Niinemets, Ü., T. F. Keenan, and L. Hallik. 2015. A worldwide analysis of within-canopy variations in leaf structural, chemical and physiological traits across plant functional types. New Phytol. 205:973--993.

\leavevmode\hypertarget{ref-Onoda2008}{}%
Onoda, Y., F. Schieving, and N. P. R. Anten. 2008. Effects of light and nutrient availability on leaf mechanical properties of Plantago major: A conceptual approach. Ann. Bot. 101:727--736.

\leavevmode\hypertarget{ref-Onoda2015}{}%
Onoda, Y., F. Schieving, and N. P. R. Anten. 2015. A novel method of measuring leaf epidermis and mesophyll stiffness shows the ubiquitous nature of the sandwich structure of leaf laminas in broad-leaved angiosperm species. J. Exp. Bot. 66:2487--2499.

\leavevmode\hypertarget{ref-Onoda2017}{}%
Onoda, Y., I. J. Wright, J. R. Evans, K. Hikosaka, K. Kitajima, Ü. Niinemets, H. Poorter, T. Tosens, and M. Westoby. 2017. Physiological and structural tradeoffs underlying the leaf economics spectrum. New Phytol. 214:1447--1463.

\leavevmode\hypertarget{ref-Osada2001}{}%
Osada, N., H. Takeda, A. Furukawa, and M. Awang. 2001. Leaf dynamics and maintenance of tree crowns in a malaysian rain forest stand. J. Ecol. 89:774--782.

\leavevmode\hypertarget{ref-Osnas2018}{}%
Osnas, J. L. D., M. Katabuchi, K. Kitajima, S. J. Wright, P. B. Reich, S. A. Van Bael, N. J. B. Kraft, M. J. Samaniego, S. W. Pacala, and J. W. Lichstein. 2018. Divergent drivers of leaf trait variation within species, among species, and among functional groups. Proc. Natl. Acad. Sci. U. S. A. 115:5480--5485.

\leavevmode\hypertarget{ref-Osnas2013}{}%
Osnas, J. L. D., J. W. Lichstein, P. B. Reich, and S. W. Pacala. 2013. Global leaf trait relationships: Mass, area, and the leaf economics spectrum. Science 340:741--744.

\leavevmode\hypertarget{ref-Poorter2009}{}%
Poorter, H., Ü. Niinemets, L. Poorter, I. J. Wright, and R. Villar. 2009. Causes and consequences of variation in leaf mass per area (LMA): A meta-analysis. New Phytol. 182:565--588.

\leavevmode\hypertarget{ref-Poorter2006b}{}%
Poorter, H., S. Pepin, T. Rijkers, Y. De Jong, J. R. Evans, and C. Körner. 2006. Construction costs, chemical composition and payback time of high- and low-irradiance leaves. J. Exp. Bot. 57:355--371.

\leavevmode\hypertarget{ref-Reich2014}{}%
Reich, P. B. 2014. The world-wide 'fast-slow' plant economics spectrum: A traits manifesto. J. Ecol. 102:275--301.

\leavevmode\hypertarget{ref-Roderick1999}{}%
Roderick, M. L., S. L. Berry, I. R. Noble, and G. D. Farquhar. 1999. A theoretical approach to linking the composition and morphology with the function of leaves. Funct. Ecol. 13:683--695.

\leavevmode\hypertarget{ref-Russo2016}{}%
Russo, S. E., and K. Kitajima. 2016. The Ecophysiology of Leaf Lifespan in Tropical Forests: Adaptive and Plastic Responses to Environmental Heterogeneity. Pages 357--383 \emph{in} G. Goldstein and S. L. Santiago, editors. Tropical Tree Physiology. Springer International Publishing, Cham.

\leavevmode\hypertarget{ref-Sakschewski2016}{}%
Sakschewski, B., W. von Bloh, A. Boit, L. Poorter, M. Peña-Claros, J. Heinke, J. Joshi, and K. Thonicke. 2016. Resilience of Amazon forests emerges from plant trait diversity. Nature Clim Change 6:1032--1036.

\leavevmode\hypertarget{ref-Sakschewski2015}{}%
Sakschewski, B., W. von Bloh, A. Boit, A. Rammig, J. Kattge, L. Poorter, J. Peñuelas, and K. Thonicke. 2015. Leaf and stem economics spectra drive diversity of functional plant traits in a dynamic global vegetation model. Glob. Change Biol. 21:2711--2725.

\leavevmode\hypertarget{ref-Scheiter2013}{}%
Scheiter, S., L. Langan, and S. I. Higgins. 2013. Next-generation dynamic global vegetation models: Learning from community ecology. New Phytol. 198:957--969.

\leavevmode\hypertarget{ref-Shipley2006}{}%
Shipley, B., M. J. Lechowicz, I. Wright, and P. B. Reich. 2006. Fundamental trade-offs generating the worldwide leaf economics spectrum. Ecology 87:535--541.

\leavevmode\hypertarget{ref-Tcherkez2017}{}%
Tcherkez, G., P. Gauthier, T. N. Buckley, F. A. Busch, M. M. Barbour, D. Bruhn, M. A. Heskel, X. Y. Gong, K. Y. Crous, K. Griffin, D. Way, M. Turnbull, M. A. Adams, O. K. Atkin, G. D. Farquhar, and G. Cornic. 2017. Leaf day respiration: Low CO2 flux but high significance for metabolism and carbon balance. New Phytol. 216:986--1001.

\leavevmode\hypertarget{ref-Terashima2011}{}%
Terashima, I., Y. T. Hanba, D. Tholen, and U. Niinemets. 2011. Leaf Functional Anatomy in Relation to Photosynthesis. Plant Physiol. 155:108--116.

\leavevmode\hypertarget{ref-Vehtari2014}{}%
Vehtari, A., T. Mononen, V. Tolvanen, T. Sivula, and O. Winther. 2014. Bayesian leave-one-out cross-validation approximations for Gaussian latent variable models. ArXiv Prepr. ArXiv14084050v2.

\leavevmode\hypertarget{ref-Villar2001}{}%
Villar, R., and J. Merino. 2001. Comparison of leaf construction costs in woody species with differing leaf life-spans in contrasting ecosystems. New Phytol. 151:213--226.

\leavevmode\hypertarget{ref-Westoby2013}{}%
Westoby, M., P. B. Reich, and I. J. Wright. 2013. Understanding ecological variation across species: Area-based vs mass-based expression of leaf traits. New Phytol. 199:322--323.

\leavevmode\hypertarget{ref-Williams1989}{}%
Williams, K., C. B. Field, and H. A. Mooney. 1989. Relationships Among Leaf Construction Cost, Leaf Longevity, and Light Environment in Rain-Forest Plants of the Genus Piper. Am. Nat. 133:198--211.

\leavevmode\hypertarget{ref-Wright2005}{}%
Wright, I. J., P. B. Reich, J. H. C. Cornelissen, D. S. Falster, P. K. Groom, K. Hikosaka, W. Lee, C. H. Lusk, Ü. Niinemets, J. Oleksyn, N. Osada, H. Poorter, D. I. Warton, and M. Westoby. 2005. Modulation of leaf economic traits and trait relationships by climate. Glob. Ecol. Biogeogr. 14:411--421.

\leavevmode\hypertarget{ref-Wright2004a}{}%
Wright, I. J., P. B. Reich, M. Westoby, D. D. Ackerly, Z. Baruch, F. Bongers, J. Cavender-Bares, T. Chapin, J. H. C. Cornellssen, M. Diemer, J. Flexas, E. Garnier, P. K. Groom, J. Gulias, K. Hikosaka, B. B. Lamont, T. Lee, W. Lee, C. Lusk, J. J. Midgley, M. L. Navas, Ü. Niinemets, J. Oleksyn, H. Osada, H. Poorter, P. Pool, L. Prior, V. I. Pyankov, C. Roumet, S. C. Thomas, M. G. Tjoelker, E. J. Veneklaas, and R. Villar. 2004. The worldwide leaf economics spectrum. Nature 428:821--827.

\leavevmode\hypertarget{ref-Wright2003}{}%
Wright, S. J., V. Horlyck, Y. Basset, H. Barrios, A. Bethancourt, S. Bohlman, G. Gilbert, G. Goldstein, E. A. Graham, K. Kitajima, M. T. Lerdau, F. C. Meinzer, F. Ødegaard, D. R. Reynolds, D. W. Roubik, S. Sakai, M. Samaniego, J. P. Sparks, S. Van Bael, K. Winter, and G. Zotz. 2003. Tropical Canopy Biology Program, Republic of Panama. Pages 137--155 \emph{in} Y. Basset, V. Horlyck, and S. J. Wright, editors. Studying Forest Canopies from Above: The International Canopy Crane Network. Panama.

\leavevmode\hypertarget{ref-Wullschleger2014}{}%
Wullschleger, S. D., H. E. Epstein, E. O. Box, E. S. Euskirchen, S. Goswami, C. M. Iversen, J. Kattge, R. J. Norby, P. M. Van Bodegom, and X. Xu. 2014. Plant functional types in Earth system models: Past experiences and future directions for application of dynamic vegetation models in high-latitude ecosystems. Ann. Bot. 114:1--16.

\leavevmode\hypertarget{ref-Xu2017}{}%
Xu, X., D. Medvigy, S. Joseph Wright, K. Kitajima, J. Wu, L. P. Albert, G. A. Martins, S. R. Saleska, and S. W. Pacala. 2017. Variations of leaf longevity in tropical moist forests predicted by a trait-driven carbon optimality model. Ecol. Lett. 20:1097--1106.
\end{cslreferences}

\newpage

\hypertarget{tables}{%
\section{Tables}\label{tables}}

\begin{longtable}[]{@{}ll@{}}
\caption{\label{tab:modeltab} Five model forms compared to predict \emph{A}\textsubscript{area}, LL and \emph{R}\textsubscript{area}.}\tabularnewline
\toprule
\begin{minipage}[b]{0.14\columnwidth}\raggedright
Model\strut
\end{minipage} & \begin{minipage}[b]{0.80\columnwidth}\raggedright
Expected trait values\strut
\end{minipage}\tabularnewline
\midrule
\endfirsthead
\toprule
\begin{minipage}[b]{0.14\columnwidth}\raggedright
Model\strut
\end{minipage} & \begin{minipage}[b]{0.80\columnwidth}\raggedright
Expected trait values\strut
\end{minipage}\tabularnewline
\midrule
\endhead
\begin{minipage}[t]{0.14\columnwidth}\raggedright
LMA\strut
\end{minipage} & \begin{minipage}[t]{0.80\columnwidth}\raggedright
\(\mathrm{E}[A_{\mathrm{area} \, i}] = \alpha_0\mathrm{LMA}_{i}^{\alpha}\)\strut
\end{minipage}\tabularnewline
\begin{minipage}[t]{0.14\columnwidth}\raggedright
\strut
\end{minipage} & \begin{minipage}[t]{0.80\columnwidth}\raggedright
\(\mathrm{E}[\mathrm{LL}_i] = \beta_0\mathrm{LMA}_{i}^{\beta}\)\strut
\end{minipage}\tabularnewline
\begin{minipage}[t]{0.14\columnwidth}\raggedright
\strut
\end{minipage} & \begin{minipage}[t]{0.80\columnwidth}\raggedright
\(\mathrm{E}[R_{\mathrm{area} \, i}] = \gamma_0\mathrm{LMA}_{i}^{\gamma}\)\strut
\end{minipage}\tabularnewline
\begin{minipage}[t]{0.14\columnwidth}\raggedright
LMA + light\strut
\end{minipage} & \begin{minipage}[t]{0.80\columnwidth}\raggedright
\(\mathrm{E}[A_{\mathrm{area} \, i}] = \alpha_0\mathrm{LMA}_{i}^{\alpha}\)\strut
\end{minipage}\tabularnewline
\begin{minipage}[t]{0.14\columnwidth}\raggedright
\strut
\end{minipage} & \begin{minipage}[t]{0.80\columnwidth}\raggedright
\(\mathrm{E}[\mathrm{LL}_i] = \beta_0\mathrm{LMA}_{i}^{\beta}exp(\theta Light_i)\)\strut
\end{minipage}\tabularnewline
\begin{minipage}[t]{0.14\columnwidth}\raggedright
\strut
\end{minipage} & \begin{minipage}[t]{0.80\columnwidth}\raggedright
\(\mathrm{E}[R_{\mathrm{area} \, i}] = \gamma_0\mathrm{LMA}_{i}^{\gamma}\)\strut
\end{minipage}\tabularnewline
\begin{minipage}[t]{0.14\columnwidth}\raggedright
LMAm + LMAs\strut
\end{minipage} & \begin{minipage}[t]{0.80\columnwidth}\raggedright
\(\mathrm{E}[A_{\mathrm{area} \, i}] = \alpha_0\mathrm{LMAm}_{i}^{\alpha_m} \mathrm{LMAs}_{i}^{\alpha_s}\)\strut
\end{minipage}\tabularnewline
\begin{minipage}[t]{0.14\columnwidth}\raggedright
\strut
\end{minipage} & \begin{minipage}[t]{0.80\columnwidth}\raggedright
\(\mathrm{E}[\mathrm{LL}_i] = \beta_0\mathrm{LMAm}_{i}^{\beta_m} \mathrm{LMAs}_{i}^{\beta_s}\)\strut
\end{minipage}\tabularnewline
\begin{minipage}[t]{0.14\columnwidth}\raggedright
\strut
\end{minipage} & \begin{minipage}[t]{0.80\columnwidth}\raggedright
\(\mathrm{E}[R_{\mathrm{area} \, i}] = \gamma_0\mathrm{LMAm}_{i}^{\gamma_m} \mathrm{LMAs}_{i}^{\gamma_s}\)\strut
\end{minipage}\tabularnewline
\begin{minipage}[t]{0.14\columnwidth}\raggedright
LMAm + LMAs + light\strut
\end{minipage} & \begin{minipage}[t]{0.80\columnwidth}\raggedright
\(\mathrm{E}[A_{\mathrm{area} \, i}]= \alpha_0\mathrm{LMAm}_{i}^{\alpha_m} \mathrm{LMAs}_{i}^{\alpha_s}\)\strut
\end{minipage}\tabularnewline
\begin{minipage}[t]{0.14\columnwidth}\raggedright
\strut
\end{minipage} & \begin{minipage}[t]{0.80\columnwidth}\raggedright
\(\mathrm{E}[\mathrm{LL}_i] = \beta_0\mathrm{LMAm}_{i}^{\beta_m} \mathrm{LMAs}_{i}^{\beta_s}exp(\theta Light_i)\)\strut
\end{minipage}\tabularnewline
\begin{minipage}[t]{0.14\columnwidth}\raggedright
\strut
\end{minipage} & \begin{minipage}[t]{0.80\columnwidth}\raggedright
\(\mathrm{E}[R_{\mathrm{area} \, i}] = \gamma_0\mathrm{LMAm}_{i}^{\gamma_m} \mathrm{LMAs}_{i}^{\gamma_s}\)\strut
\end{minipage}\tabularnewline
\begin{minipage}[t]{0.14\columnwidth}\raggedright
LMAm + LMA/LT + light\strut
\end{minipage} & \begin{minipage}[t]{0.80\columnwidth}\raggedright
\(\mathrm{E}[A_{\mathrm{area} \, i}]= \alpha_0\mathrm{LMAm}_{i}^{\alpha_m} (\mathrm{LMAs}_{i}/\mathrm{LT}_{i})^{\alpha_s}\)\strut
\end{minipage}\tabularnewline
\begin{minipage}[t]{0.14\columnwidth}\raggedright
\strut
\end{minipage} & \begin{minipage}[t]{0.80\columnwidth}\raggedright
\(\mathrm{E}[\mathrm{LL}_i] = \beta_0\mathrm{LMAm}_{i}^{\beta_m} (\mathrm{LMAs}_{i}/\mathrm{LT}_{i})^{\beta_s}exp(\theta Light_i)\)\strut
\end{minipage}\tabularnewline
\begin{minipage}[t]{0.14\columnwidth}\raggedright
\strut
\end{minipage} & \begin{minipage}[t]{0.80\columnwidth}\raggedright
\(\mathrm{E}[R_{\mathrm{area} \, i}] = \gamma_0\mathrm{LMAm}_{i}^{\gamma_m} \mathrm{LMAs}_{i}^{\gamma_s}\)\strut
\end{minipage}\tabularnewline
\bottomrule
\end{longtable}

\newpage

\hypertarget{section}{%
\section{}\label{section}}

\begin{longtable}[]{@{}lll@{}}
\caption{\label{tab:lootab} Summary of the model comparison. Lower approximate leave-one-out cross-validation (LOOIC) indicates better predictive accuracy. N; number of samples.}\tabularnewline
\toprule
& LOOIC & N\tabularnewline
\midrule
\endfirsthead
\toprule
& LOOIC & N\tabularnewline
\midrule
\endhead
\textbf{GLOPNET} & &\tabularnewline
LMAm and LMAs & 746 & 198\tabularnewline
LMA & 766.2 & 198\tabularnewline
\textbf{Panama} & &\tabularnewline
LMAm + LMAs + light & 501.1 & 106\tabularnewline
LMA + light & 503 & 106\tabularnewline
LMAm + LMAs/LT + light & 503.4 & 106\tabularnewline
LMAm + LMAs & 518.2 & 106\tabularnewline
LMA & 528.1 & 106\tabularnewline
\bottomrule
\end{longtable}

\newpage

\hypertarget{section-1}{%
\section{}\label{section-1}}

\begin{longtable}[]{@{}lll@{}}
\caption{\label{tab:paratab} Posterior medians {[}95\% credible interval{]} of parameters for the best models. `LMAm + LMAs model' and `LMAm + LMAs + light model' are shown for GLOPNET and Panama data, respectively. Bold values are significantly different from zero based on the 95\% credible interval.}\tabularnewline
\toprule
\begin{minipage}[b]{0.39\columnwidth}\raggedright
Parameters\strut
\end{minipage} & \begin{minipage}[b]{0.26\columnwidth}\raggedright
GLOPNET\strut
\end{minipage} & \begin{minipage}[b]{0.26\columnwidth}\raggedright
Panama\strut
\end{minipage}\tabularnewline
\midrule
\endfirsthead
\toprule
\begin{minipage}[b]{0.39\columnwidth}\raggedright
Parameters\strut
\end{minipage} & \begin{minipage}[b]{0.26\columnwidth}\raggedright
GLOPNET\strut
\end{minipage} & \begin{minipage}[b]{0.26\columnwidth}\raggedright
Panama\strut
\end{minipage}\tabularnewline
\midrule
\endhead
\begin{minipage}[t]{0.39\columnwidth}\raggedright
Effect of LMAm on \emph{A}\textsubscript{area} (\(\alpha_m\))\strut
\end{minipage} & \begin{minipage}[t]{0.26\columnwidth}\raggedright
\textbf{0.206 {[}0.108, 0.294{]}}\strut
\end{minipage} & \begin{minipage}[t]{0.26\columnwidth}\raggedright
\textbf{0.442 {[}0.338, 0.574{]}}\strut
\end{minipage}\tabularnewline
\begin{minipage}[t]{0.39\columnwidth}\raggedright
Effect of LMAs on \emph{A}\textsubscript{area} (\(\alpha_s\))\strut
\end{minipage} & \begin{minipage}[t]{0.26\columnwidth}\raggedright
\textbf{-0.111 {[}-0.179, -0.037{]}}\strut
\end{minipage} & \begin{minipage}[t]{0.26\columnwidth}\raggedright
\textbf{0.29 {[}0.16, 0.391{]}}\strut
\end{minipage}\tabularnewline
\begin{minipage}[t]{0.39\columnwidth}\raggedright
Effect of LMAm on LL (\(\beta_m\))\strut
\end{minipage} & \begin{minipage}[t]{0.26\columnwidth}\raggedright
\textbf{0.261 {[}0.097, 0.466{]}}\strut
\end{minipage} & \begin{minipage}[t]{0.26\columnwidth}\raggedright
0.086 {[}-0.119, 0.314{]}\strut
\end{minipage}\tabularnewline
\begin{minipage}[t]{0.39\columnwidth}\raggedright
Effect of LMAs on LL (\(\beta_s\))\strut
\end{minipage} & \begin{minipage}[t]{0.26\columnwidth}\raggedright
\textbf{0.785 {[}0.627, 0.93{]}}\strut
\end{minipage} & \begin{minipage}[t]{0.26\columnwidth}\raggedright
\textbf{0.376 {[}0.172, 0.567{]}}\strut
\end{minipage}\tabularnewline
\begin{minipage}[t]{0.39\columnwidth}\raggedright
Effect of LMAm on \emph{R}\textsubscript{area} (\(\gamma_m\))\strut
\end{minipage} & \begin{minipage}[t]{0.26\columnwidth}\raggedright
\textbf{0.218 {[}0.096, 0.316{]}}\strut
\end{minipage} & \begin{minipage}[t]{0.26\columnwidth}\raggedright
\textbf{0.6 {[}0.345, 0.822{]}}\strut
\end{minipage}\tabularnewline
\begin{minipage}[t]{0.39\columnwidth}\raggedright
Effect of LMAs on \emph{R}\textsubscript{area} (\(\gamma_s\))\strut
\end{minipage} & \begin{minipage}[t]{0.26\columnwidth}\raggedright
0.049 {[}-0.033, 0.14{]}\strut
\end{minipage} & \begin{minipage}[t]{0.26\columnwidth}\raggedright
0.204 {[}-0.059, 0.469{]}\strut
\end{minipage}\tabularnewline
\begin{minipage}[t]{0.39\columnwidth}\raggedright
Effect of light on LL (\(\theta\))\strut
\end{minipage} & \begin{minipage}[t]{0.26\columnwidth}\raggedright
-\strut
\end{minipage} & \begin{minipage}[t]{0.26\columnwidth}\raggedright
\textbf{-1.046 {[}-1.471, -0.617{]}}\strut
\end{minipage}\tabularnewline
\bottomrule
\end{longtable}

\newpage

\hypertarget{figures}{%
\section{Figures}\label{figures}}

\begin{figure}
\centering
\includegraphics{LMAms_main_re_files/figure-latex/Hplt-1.pdf}
\caption{\label{fig:Hplt}Example of a two-dimensional functional space that can result in either a two- or one-dimensional trait space, depending on how metabolic traits are normalized. (A) Hypothetical indpendent variation in two leaf mass per area (LMA) components: metabolic leaf mass per area (LMAm, which largely determines per-area values of photosynthesis, respiration, and nutrient concentrations) and structural leaf mass per area (LMAs, which determines leaf toughness but has little effect on metabolic traits). (B) Two-dimensional relationship between photosynthetic capacity (\emph{A}\textsubscript{max}) per-unit leaf area and total LMA (equal to LMAm + LMAs). (C) One-dimensional relationship between \emph{A}\textsubscript{max} per-unit leaf mass and total LMA. Variation in LMAm and LMAs and \emph{A}\textsubscript{max} values in panels B and C are derived from our analysis of the GLOPNET database. \(A_{\mathrm{area} \, i}=1.9\mathrm{LMAm}_i^{0.36}\mathrm{LMAs}^{-0.25}\epsilon_i\), where \(\epsilon_i\) is log-normally distributed with log-mean with 0, log-scale parameter with 0.3. (see Methods and Results).}
\end{figure}

\newpage

\hypertarget{section-2}{%
\section{}\label{section-2}}

\begin{figure}
\centering
\includegraphics{LMAms_main_re_files/figure-latex/GLplt-1.pdf}
\caption{\label{fig:GLplt}Observed and estimated leaf-trait relationships in the global GLOPNET dataset. (A) Leaf life span (LL), net photosynthetic rate per unit leaf area (\emph{A}\textsubscript{area}), and dark respiration rate per unit leaf area (\emph{R}\textsubscript{area}) are plotted against observed LMA. (B) Residuals of LL, \emph{A}\textsubscript{area}, and \emph{R}\textsubscript{area} regressed on posteior estimates of photosynthetic and structural LMA components (LMAm and LMAs, respectively) are plotted against LMAm and LMAs (only means are shown). Pearson correlation coefficients are for LMA (left column) and posterior distributions of LMAm (middle column) and LMAs (right column).}
\end{figure}

\newpage

\hypertarget{section-3}{%
\section{}\label{section-3}}

\begin{figure}
\centering
\includegraphics{LMAms_main_re_files/figure-latex/PAplt-1.pdf}
\caption{\label{fig:PAplt}Observed and estimated leaf-trait relationships in the Panama dataset. Estimates are from the Optimal LL Model (Eq. 8). Details as in Fig. 2. Results for other LL models are summarized in Table SX.}
\end{figure}

\newpage

\hypertarget{section-4}{%
\section{}\label{section-4}}

\begin{figure}
\centering
\includegraphics{LMAms_main_re_files/figure-latex/LLplt-1.pdf}
\caption{\label{fig:LLplt}Observed vs.~predicted leaf lifespan (LL) in the Optimal LL Model (Eq. 8). Predicted LL values are posterior medians. The dashed line indicates the 1:1 relationship. The \emph{r}\textsuperscript{2} value is the posterior median of Bayesian \emph{r}\textsuperscript{2} (Gelman et al.~2019).}
\end{figure}

\newpage

\hypertarget{section-5}{%
\section{}\label{section-5}}

\begin{figure}
\centering
\includegraphics{LMAms_main_re_files/figure-latex/massplt-1.pdf}
\caption{\label{fig:massplt}The relationships among mass proportionality (\emph{b\textsubscript{i}} in Eq. \eqref{eq:mass}), variance in LMAm and LMAs, and scaling slopes between LMAm and \emph{A}\textsubscript{area} (\(\alpha_m\)) and LMAs and \emph{A}\textsubscript{area} (\(\alpha_s\)) for the global GLOPNET dataset, sun leaves in Panama, and shade leaves in Panama. The photosynthetic rate (\emph{A}\textsubscript{max}) is primarily mass-dependent (\emph{b\textsubscript{i}} \textgreater{} 0.5, white to blue), primarily area-dependent (0.5 \textgreater{} \emph{b\textsubscript{i}} \textgreater{} 0, white to orange) and purely area-proportional (\emph{b\textsubscript{i}} = 0, white) (Osnas et al.~2018). Note that when \emph{b\textsubscript{i}} is negative, \emph{A}\textsubscript{area} decreases with LMA, which is not realistic.}
\end{figure}

\newpage

\hypertarget{section-6}{%
\section{}\label{section-6}}

\begin{figure}
\centering
\includegraphics{LMAms_main_re_files/figure-latex/boxplt-1.pdf}
\caption{\label{fig:boxplt}Boxplots comparing leaf mass per area (LMA) and posterior medians of photosynthetic and structural LMA components (LMAm and LMAs, respectively) across deciduous (D) and evergreen (E) leaves in the GLOPNET dataset (top), and across sites (wet and dry) and canopy strata (sun and shade) in the Panama dataset (bottom). The center line in each box indicates the median, upper and lower box edges indicate the interquartile range, whiskers show 1.5 times the interquartile range, and points are outliers. Groups sharing the same letters are not significantly different (P \textgreater{} 0.05; t-tests). To isolate the effects of intraspecific variation (i.e., plastic responses to light), the Panama results shown here only include species for which both sun and shade leaves were available. Qualitatively similar results were obtained when all Panama species were included (Fig. SX). Note that the vertical axis is on a log\textsubscript{10} scale.}
\end{figure}

\newpage

\hypertarget{section-7}{%
\section{}\label{section-7}}

\begin{figure}
\centering
\includegraphics{LMAms_main_re_files/figure-latex/PA-NPC-1.pdf}
\caption{\label{fig:PA-NPC}Measured traits related to photosynthesis and metabolism traits (nitrogen and phosphorus per-unit leaf area; \emph{N}\textsubscript{area} and \emph{P}\textsubscript{area}) are strongly correlated with estimates (posterior medians) of the photosynthetic LMA component (LMAm), and a measured structural trait (cellulose per-unit leaf area; \emph{CL}\textsubscript{area}) is strongly correlated with estimates of the structural LMA component (LMAs) for the Panama dataset. Note that sun and shade leaves align along a single relationship for \emph{CL}\textsubscript{area} vs.~LMAs, but not for \emph{CL}\textsubscript{area} vs.~LMA or LMAm. \emph{N}\textsubscript{area}, \emph{P}\textsubscript{area}, and \emph{CL}\textsubscript{area} data were not used to fit the models, and are presented here as independent support for the model results. Analogous results were obtained for \emph{N}\textsubscript{area} and \emph{P}\textsubscript{area} vs.~LMAm for GLOPNET (Fig. S3). Others details as in Fig. GL. Results for other LL models are reported in Table SX.}
\end{figure}

\end{document}

%%
%DIF LATEXDIFF DIFFERENCE FILE
%DIF DEL /var/folders/1v/_ypjfm2d71bf8nf01_2jk2t00000gn/T/pXnnmRX57h/latexdiff-vc-ce3856c6/ms/LMAms_main.tex   Mon Sep  5 18:45:08 2022
%DIF ADD ms/LMAms_main.tex                                                                                     Mon Oct 10 13:47:03 2022
% Copyright (c) 2017 - 2020, Pascal Wagler;
% Copyright (c) 2014 - 2020, John MacFarlane
%
% All rights reserved.
%
% Redistribution and use in source and binary forms, with or without
% modification, are permitted provided that the following conditions
% are met:
%
% - Redistributions of source code must retain the above copyright
% notice, this list of conditions and the following disclaimer.
%
% - Redistributions in binary form must reproduce the above copyright
% notice, this list of conditions and the following disclaimer in the
% documentation and/or other materials provided with the distribution.
%
% - Neither the name of John MacFarlane nor the names of other
% contributors may be used to endorse or promote products derived
% from this software without specific prior written permission.
%
% THIS SOFTWARE IS PROVIDED BY THE COPYRIGHT HOLDERS AND CONTRIBUTORS
% "AS IS" AND ANY EXPRESS OR IMPLIED WARRANTIES, INCLUDING, BUT NOT
% LIMITED TO, THE IMPLIED WARRANTIES OF MERCHANTABILITY AND FITNESS
% FOR A PARTICULAR PURPOSE ARE DISCLAIMED. IN NO EVENT SHALL THE
% COPYRIGHT OWNER OR CONTRIBUTORS BE LIABLE FOR ANY DIRECT, INDIRECT,
% INCIDENTAL, SPECIAL, EXEMPLARY, OR CONSEQUENTIAL DAMAGES (INCLUDING,
% BUT NOT LIMITED TO, PROCUREMENT OF SUBSTITUTE GOODS OR SERVICES;
% LOSS OF USE, DATA, OR PROFITS; OR BUSINESS INTERRUPTION) HOWEVER
% CAUSED AND ON ANY THEORY OF LIABILITY, WHETHER IN CONTRACT, STRICT
% LIABILITY, OR TORT (INCLUDING NEGLIGENCE OR OTHERWISE) ARISING IN
% ANY WAY OUT OF THE USE OF THIS SOFTWARE, EVEN IF ADVISED OF THE
% POSSIBILITY OF SUCH DAMAGE.
%%

%%
% This is the Eisvogel pandoc LaTeX template.
%
% For usage information and examples visit the official GitHub page:
% https://github.com/Wandmalfarbe/pandoc-latex-template
%%

\DeclareUnicodeCharacter{2212}{-}

% Options for packages loaded elsewhere
\PassOptionsToPackage{unicode}{hyperref}
\PassOptionsToPackage{hyphens}{url}
\PassOptionsToPackage{dvipsnames,svgnames*,x11names*,table}{xcolor}
%
\documentclass[
  12pt,
  a4paper,
,tablecaptionabove
]{scrartcl}
\usepackage{lmodern}
\usepackage{setspace}
\setstretch{1.2}
\usepackage{amssymb,amsmath}
\usepackage{ifxetex,ifluatex}
\ifnum 0\ifxetex 1\fi\ifluatex 1\fi=0 % if pdftex
  \usepackage[T1]{fontenc}
  \usepackage[utf8]{inputenc}
  \usepackage{textcomp} % provide euro and other symbols
\else % if luatex or xetex
  \usepackage{unicode-math}
  \defaultfontfeatures{Scale=MatchLowercase}
  \defaultfontfeatures[\rmfamily]{Ligatures=TeX,Scale=1}
\fi
% Use upquote if available, for straight quotes in verbatim environments
\IfFileExists{upquote.sty}{\usepackage{upquote}}{}
\IfFileExists{microtype.sty}{% use microtype if available
  \usepackage[]{microtype}
  \UseMicrotypeSet[protrusion]{basicmath} % disable protrusion for tt fonts
}{}
\makeatletter
\@ifundefined{KOMAClassName}{% if non-KOMA class
  \IfFileExists{parskip.sty}{%
    \usepackage{parskip}
  }{% else
    \setlength{\parindent}{0pt}
    \setlength{\parskip}{6pt plus 2pt minus 1pt}}
}{% if KOMA class
  \KOMAoptions{parskip=half}}
\makeatother
\usepackage{xcolor}
\definecolor{default-linkcolor}{HTML}{A50000}
\definecolor{default-filecolor}{HTML}{A50000}
\definecolor{default-citecolor}{HTML}{4077C0}
\definecolor{default-urlcolor}{HTML}{4077C0}
\IfFileExists{xurl.sty}{\usepackage{xurl}}{} % add URL line breaks if available
\IfFileExists{bookmark.sty}{\usepackage{bookmark}}{\usepackage{hyperref}}
\hypersetup{
  colorlinks=true,
  linkcolor=blue,
  filecolor=default-filecolor,
  citecolor=default-citecolor,
  urlcolor=default-urlcolor,
  breaklinks=true,
  pdfcreator={LaTeX via pandoc with the Eisvogel template}}
\urlstyle{same} % disable monospaced font for URLs
\usepackage[margin=1in]{geometry}
\usepackage{longtable,booktabs}
% Correct order of tables after \paragraph or \subparagraph
\usepackage{etoolbox}
\makeatletter
\patchcmd\longtable{\par}{\if@noskipsec\mbox{}\fi\par}{}{}
\makeatother
% Allow footnotes in longtable head/foot
\IfFileExists{footnotehyper.sty}{\usepackage{footnotehyper}}{\usepackage{footnote}}
\makesavenoteenv{longtable}
% add backlinks to footnote references, cf. https://tex.stackexchange.com/questions/302266/make-footnote-clickable-both-ways
\usepackage{footnotebackref}
\usepackage{graphicx}
\makeatletter
\def\maxwidth{\ifdim\Gin@nat@width>\linewidth\linewidth\else\Gin@nat@width\fi}
\def\maxheight{\ifdim\Gin@nat@height>\textheight\textheight\else\Gin@nat@height\fi}
\makeatother
% Scale images if necessary, so that they will not overflow the page
% margins by default, and it is still possible to overwrite the defaults
% using explicit options in \includegraphics[width, height, ...]{}
\setkeys{Gin}{width=\maxwidth,height=\maxheight,keepaspectratio}
\setlength{\emergencystretch}{3em}  % prevent overfull lines
\providecommand{\tightlist}{%
  \setlength{\itemsep}{0pt}\setlength{\parskip}{0pt}}
\setcounter{secnumdepth}{-\maxdimen} % remove section numbering
% Make \paragraph and \subparagraph free-standing
\ifx\paragraph\undefined\else
  \let\oldparagraph\paragraph
  \renewcommand{\paragraph}[1]{\oldparagraph{#1}\mbox{}}
\fi
\ifx\subparagraph\undefined\else
  \let\oldsubparagraph\subparagraph
  \renewcommand{\subparagraph}[1]{\oldsubparagraph{#1}\mbox{}}
\fi

% Make use of float-package and set default placement for figures to H.
% The option H means 'PUT IT HERE' (as  opposed to the standard h option which means 'You may put it here if you like').
\usepackage{float}
\floatplacement{figure}{H}

\usepackage{booktabs}
\usepackage{longtable}
\usepackage{array}
\usepackage{multirow}
\usepackage{wrapfig}
\usepackage{float}
\usepackage{colortbl}
\usepackage{pdflscape}
\usepackage{tabu}
\usepackage{threeparttable}
\usepackage{threeparttablex}
\usepackage[normalem]{ulem}
\usepackage{makecell}
\usepackage{xcolor}
\usepackage{xr}
\externaldocument{LMAms_SI}
\usepackage{lineno}
\linenumbers
\makeatletter
\makeatother
\makeatletter
\makeatother
\makeatletter
\@ifpackageloaded{caption}{}{\usepackage{caption}}
\AtBeginDocument{%
\ifdefined\contentsname
  \renewcommand*\contentsname{Table of contents}
\else
  \newcommand\contentsname{Table of contents}
\fi
\ifdefined\listfigurename
  \renewcommand*\listfigurename{List of Figures}
\else
  \newcommand\listfigurename{List of Figures}
\fi
\ifdefined\listtablename
  \renewcommand*\listtablename{List of Tables}
\else
  \newcommand\listtablename{List of Tables}
\fi
\ifdefined\figurename
  \renewcommand*\figurename{Fig.}
\else
  \newcommand\figurename{Fig.}
\fi
\ifdefined\tablename
  \renewcommand*\tablename{Table}
\else
  \newcommand\tablename{Table}
\fi
}
\@ifpackageloaded{float}{}{\usepackage{float}}
\floatstyle{ruled}
\@ifundefined{c@chapter}{\newfloat{codelisting}{h}{lop}}{\newfloat{codelisting}{h}{lop}[chapter]}
\floatname{codelisting}{Listing}
\newcommand*\listoflistings{\listof{codelisting}{List of Listings}}
\makeatother
\makeatletter
\@ifpackageloaded{caption}{}{\usepackage{caption}}
\@ifpackageloaded{subcaption}{}{\usepackage{subcaption}}
\makeatother
\makeatletter
\@ifpackageloaded{tcolorbox}{}{\usepackage[many]{tcolorbox}}
\makeatother
\makeatletter
\@ifundefined{shadecolor}{\definecolor{shadecolor}{rgb}{.97, .97, .97}}
\makeatother
\makeatletter
\makeatother

\newlength{\cslhangindent}
\setlength{\cslhangindent}{1.5em}
\newlength{\csllabelwidth}
\setlength{\csllabelwidth}{3em}
\newenvironment{CSLReferences}[2] % #1 hanging-ident, #2 entry spacing
 {% don't indent paragraphs
  \setlength{\parindent}{0pt}
  % turn on hanging indent if param 1 is 1
  \ifodd #1 \everypar{\setlength{\hangindent}{\cslhangindent}}\ignorespaces\fi
  % set entry spacing
  \ifnum #2 > 0
  \setlength{\parskip}{#2\baselineskip}
  \fi
 }%
 {}
\usepackage{calc}
\newcommand{\CSLBlock}[1]{#1\hfill\break}
\newcommand{\CSLLeftMargin}[1]{\parbox[t]{\csllabelwidth}{#1}}
\newcommand{\CSLRightInline}[1]{\parbox[t]{\linewidth - \csllabelwidth}{#1}\break}
\newcommand{\CSLIndent}[1]{\hspace{\cslhangindent}#1}

\date{}


%%
%% added
%%

%
% language specification
%
% If no language is specified, use English as the default main document language.
%

\ifnum 0\ifxetex 1\fi\ifluatex 1\fi=0 % if pdftex
  \usepackage[shorthands=off,main=english]{babel}
\else
    % Workaround for bug in Polyglossia that breaks `\familydefault` when `\setmainlanguage` is used.
  % See https://github.com/Wandmalfarbe/pandoc-latex-template/issues/8
  % See https://github.com/reutenauer/polyglossia/issues/186
  % See https://github.com/reutenauer/polyglossia/issues/127
  \renewcommand*\familydefault{\sfdefault}
    % load polyglossia as late as possible as it *could* call bidi if RTL lang (e.g. Hebrew or Arabic)
  \usepackage{polyglossia}
  \setmainlanguage[]{english}
\fi



%
% for the background color of the title page
%

%
% break urls
%
\PassOptionsToPackage{hyphens}{url}

%
% When using babel or polyglossia with biblatex, loading csquotes is recommended
% to ensure that quoted texts are typeset according to the rules of your main language.
%
\usepackage{csquotes}

%
% captions
%
%\definecolor{caption-color}{HTML}{777777}
\definecolor{caption-color}{HTML}{37474F}
%\usepackage[font={stretch=1.2}, textfont={color=caption-color}, position=top, skip=4mm, labelfont=bf, singlelinecheck=false, justification=raggedright]{caption}
\usepackage[font={stretch=1}, textfont={color=caption-color}, position=top, skip=2mm, labelfont=bf, singlelinecheck=false, justification=raggedright]{caption}
\setcapindent{0em}

%
% blockquote
%
\definecolor{blockquote-border}{RGB}{221,221,221}
\definecolor{blockquote-text}{RGB}{119,119,119}
\usepackage{mdframed}
\newmdenv[rightline=false,bottomline=false,topline=false,linewidth=3pt,linecolor=blockquote-border,skipabove=\parskip]{customblockquote}
\renewenvironment{quote}{\begin{customblockquote}\list{}{\rightmargin=0em\leftmargin=0em}%
\item\relax\color{blockquote-text}\ignorespaces}{\unskip\unskip\endlist\end{customblockquote}}

%
% Source Sans Pro as the de­fault font fam­ily
% Source Code Pro for monospace text
%
% 'default' option sets the default
% font family to Source Sans Pro, not \sfdefault.
%
\ifnum 0\ifxetex 1\fi\ifluatex 1\fi=0 % if pdftex
    \usepackage[default]{sourcesanspro}
  \usepackage{sourcecodepro}
  %\usepackage{}
  \else % if not pdftex
    \usepackage[default]{sourcesanspro}
  \usepackage{sourcecodepro}
  %\usepackage{}

  % XeLaTeX specific adjustments for straight quotes: https://tex.stackexchange.com/a/354887
  % This issue is already fixed (see https://github.com/silkeh/latex-sourcecodepro/pull/5) but the
  % fix is still unreleased.
  % TODO: Remove this workaround when the new version of sourcecodepro is released on CTAN.
  \ifxetex
    \makeatletter
    \defaultfontfeatures[\ttfamily]
      { Numbers   = \sourcecodepro@figurestyle,
        Scale     = \SourceCodePro@scale,
        Extension = .otf }
    \setmonofont
      [ UprightFont    = *-\sourcecodepro@regstyle,
        ItalicFont     = *-\sourcecodepro@regstyle It,
        BoldFont       = *-\sourcecodepro@boldstyle,
        BoldItalicFont = *-\sourcecodepro@boldstyle It ]
      {SourceCodePro}
    \makeatother
  \fi
  \fi

%
% heading color
%
\definecolor{heading-color}{RGB}{40,40,40}
\addtokomafont{section}{\color{heading-color}}
% When using the classes report, scrreprt, book,
% scrbook or memoir, uncomment the following line.
%\addtokomafont{chapter}{\color{heading-color}}

%
% variables for title and author
%
\usepackage{titling}
\title{}
\author{}

%
% tables
%

\definecolor{table-row-color}{HTML}{F5F5F5}
\definecolor{table-rule-color}{HTML}{999999}

%\arrayrulecolor{black!40}
\arrayrulecolor{table-rule-color}     % color of \toprule, \midrule, \bottomrule
\setlength\heavyrulewidth{0.3ex}      % thickness of \toprule, \bottomrule
\renewcommand{\arraystretch}{1.3}     % spacing (padding)


%
% remove paragraph indention
%
\setlength{\parindent}{0pt}
\setlength{\parskip}{6pt plus 2pt minus 1pt}
\setlength{\emergencystretch}{3em}  % prevent overfull lines

%
%
% Listings
%
%


%
% header and footer
%
\usepackage{fancyhdr}

\fancypagestyle{eisvogel-header-footer}{
  \fancyhead{}
  \fancyfoot{}
  \lhead[]{}
  \chead[]{}
  \rhead[]{}
  %\lfoot[\thepage]{}
  \cfoot[]{}
  \cfoot[]{\thepage}
  \renewcommand{\headrulewidth}{0.0pt}
 % \renewcommand{\footrulewidth}{0.0pt}
 % \renewcommand{\headrulewidth}{0.4pt}
 % \renewcommand{\footrulewidth}{0.4pt}
}
\pagestyle{eisvogel-header-footer}

%%
%% end added
%%
%DIF PREAMBLE EXTENSION ADDED BY LATEXDIFF
%DIF UNDERLINE PREAMBLE %DIF PREAMBLE
\RequirePackage[normalem]{ulem} %DIF PREAMBLE
\RequirePackage{color}\definecolor{RED}{rgb}{1,0,0}\definecolor{BLUE}{rgb}{0,0,1} %DIF PREAMBLE
\providecommand{\DIFaddtex}[1]{{\protect\color{blue}\uwave{#1}}} %DIF PREAMBLE
\providecommand{\DIFdeltex}[1]{{\protect\color{red}\sout{#1}}}                      %DIF PREAMBLE
%DIF SAFE PREAMBLE %DIF PREAMBLE
\providecommand{\DIFaddbegin}{} %DIF PREAMBLE
\providecommand{\DIFaddend}{} %DIF PREAMBLE
\providecommand{\DIFdelbegin}{} %DIF PREAMBLE
\providecommand{\DIFdelend}{} %DIF PREAMBLE
\providecommand{\DIFmodbegin}{} %DIF PREAMBLE
\providecommand{\DIFmodend}{} %DIF PREAMBLE
%DIF FLOATSAFE PREAMBLE %DIF PREAMBLE
\providecommand{\DIFaddFL}[1]{\DIFadd{#1}} %DIF PREAMBLE
\providecommand{\DIFdelFL}[1]{\DIFdel{#1}} %DIF PREAMBLE
\providecommand{\DIFaddbeginFL}{} %DIF PREAMBLE
\providecommand{\DIFaddendFL}{} %DIF PREAMBLE
\providecommand{\DIFdelbeginFL}{} %DIF PREAMBLE
\providecommand{\DIFdelendFL}{} %DIF PREAMBLE
%DIF HYPERREF PREAMBLE %DIF PREAMBLE
\providecommand{\DIFadd}[1]{\texorpdfstring{\DIFaddtex{#1}}{#1}} %DIF PREAMBLE
\providecommand{\DIFdel}[1]{\texorpdfstring{\DIFdeltex{#1}}{}} %DIF PREAMBLE
\newcommand{\DIFscaledelfig}{0.5}
%DIF HIGHLIGHTGRAPHICS PREAMBLE %DIF PREAMBLE
\RequirePackage{settobox} %DIF PREAMBLE
\RequirePackage{letltxmacro} %DIF PREAMBLE
\newsavebox{\DIFdelgraphicsbox} %DIF PREAMBLE
\newlength{\DIFdelgraphicswidth} %DIF PREAMBLE
\newlength{\DIFdelgraphicsheight} %DIF PREAMBLE
% store original definition of \includegraphics %DIF PREAMBLE
\LetLtxMacro{\DIFOincludegraphics}{\includegraphics} %DIF PREAMBLE
\newcommand{\DIFaddincludegraphics}[2][]{{\color{blue}\fbox{\DIFOincludegraphics[#1]{#2}}}} %DIF PREAMBLE
\newcommand{\DIFdelincludegraphics}[2][]{% %DIF PREAMBLE
\sbox{\DIFdelgraphicsbox}{\DIFOincludegraphics[#1]{#2}}% %DIF PREAMBLE
\settoboxwidth{\DIFdelgraphicswidth}{\DIFdelgraphicsbox} %DIF PREAMBLE
\settoboxtotalheight{\DIFdelgraphicsheight}{\DIFdelgraphicsbox} %DIF PREAMBLE
\scalebox{\DIFscaledelfig}{% %DIF PREAMBLE
\parbox[b]{\DIFdelgraphicswidth}{\usebox{\DIFdelgraphicsbox}\\[-\baselineskip] \rule{\DIFdelgraphicswidth}{0em}}\llap{\resizebox{\DIFdelgraphicswidth}{\DIFdelgraphicsheight}{% %DIF PREAMBLE
\setlength{\unitlength}{\DIFdelgraphicswidth}% %DIF PREAMBLE
\begin{picture}(1,1)% %DIF PREAMBLE
\thicklines\linethickness{2pt} %DIF PREAMBLE
{\color[rgb]{1,0,0}\put(0,0){\framebox(1,1){}}}% %DIF PREAMBLE
{\color[rgb]{1,0,0}\put(0,0){\line( 1,1){1}}}% %DIF PREAMBLE
{\color[rgb]{1,0,0}\put(0,1){\line(1,-1){1}}}% %DIF PREAMBLE
\end{picture}% %DIF PREAMBLE
}\hspace*{3pt}}} %DIF PREAMBLE
} %DIF PREAMBLE
\LetLtxMacro{\DIFOaddbegin}{\DIFaddbegin} %DIF PREAMBLE
\LetLtxMacro{\DIFOaddend}{\DIFaddend} %DIF PREAMBLE
\LetLtxMacro{\DIFOdelbegin}{\DIFdelbegin} %DIF PREAMBLE
\LetLtxMacro{\DIFOdelend}{\DIFdelend} %DIF PREAMBLE
\DeclareRobustCommand{\DIFaddbegin}{\DIFOaddbegin \let\includegraphics\DIFaddincludegraphics} %DIF PREAMBLE
\DeclareRobustCommand{\DIFaddend}{\DIFOaddend \let\includegraphics\DIFOincludegraphics} %DIF PREAMBLE
\DeclareRobustCommand{\DIFdelbegin}{\DIFOdelbegin \let\includegraphics\DIFdelincludegraphics} %DIF PREAMBLE
\DeclareRobustCommand{\DIFdelend}{\DIFOaddend \let\includegraphics\DIFOincludegraphics} %DIF PREAMBLE
\LetLtxMacro{\DIFOaddbeginFL}{\DIFaddbeginFL} %DIF PREAMBLE
\LetLtxMacro{\DIFOaddendFL}{\DIFaddendFL} %DIF PREAMBLE
\LetLtxMacro{\DIFOdelbeginFL}{\DIFdelbeginFL} %DIF PREAMBLE
\LetLtxMacro{\DIFOdelendFL}{\DIFdelendFL} %DIF PREAMBLE
\DeclareRobustCommand{\DIFaddbeginFL}{\DIFOaddbeginFL \let\includegraphics\DIFaddincludegraphics} %DIF PREAMBLE
\DeclareRobustCommand{\DIFaddendFL}{\DIFOaddendFL \let\includegraphics\DIFOincludegraphics} %DIF PREAMBLE
\DeclareRobustCommand{\DIFdelbeginFL}{\DIFOdelbeginFL \let\includegraphics\DIFdelincludegraphics} %DIF PREAMBLE
\DeclareRobustCommand{\DIFdelendFL}{\DIFOaddendFL \let\includegraphics\DIFOincludegraphics} %DIF PREAMBLE
%DIF COLORLISTINGS PREAMBLE %DIF PREAMBLE
\RequirePackage{listings} %DIF PREAMBLE
\RequirePackage{color} %DIF PREAMBLE
\lstdefinelanguage{DIFcode}{ %DIF PREAMBLE
%DIF DIFCODE_UNDERLINE %DIF PREAMBLE
  moredelim=[il][\color{red}\sout]{\%DIF\ <\ }, %DIF PREAMBLE
  moredelim=[il][\color{blue}\uwave]{\%DIF\ >\ } %DIF PREAMBLE
} %DIF PREAMBLE
\lstdefinestyle{DIFverbatimstyle}{ %DIF PREAMBLE
	language=DIFcode, %DIF PREAMBLE
	basicstyle=\ttfamily, %DIF PREAMBLE
	columns=fullflexible, %DIF PREAMBLE
	keepspaces=true %DIF PREAMBLE
} %DIF PREAMBLE
\lstnewenvironment{DIFverbatim}{\lstset{style=DIFverbatimstyle}}{} %DIF PREAMBLE
\lstnewenvironment{DIFverbatim*}{\lstset{style=DIFverbatimstyle,showspaces=true}}{} %DIF PREAMBLE
%DIF END PREAMBLE EXTENSION ADDED BY LATEXDIFF

\begin{document}

%%
%% begin titlepage
%%

%%
%% end titlepage
%%



\ifdefined\Shaded\DIFdelbegin %DIFDELCMD < \renewenvironment{Shaded}{\begin{tcolorbox}[boxrule=0pt, interior hidden, sharp corners, frame hidden, enhanced, borderline west={3pt}{0pt}{shadecolor}, breakable]}{\end{tcolorbox}}%%%
\DIFdelend \DIFaddbegin \renewenvironment{Shaded}{\begin{tcolorbox}[frame hidden, sharp corners, breakable, enhanced, boxrule=0pt, borderline west={3pt}{0pt}{shadecolor}, interior hidden]}{\end{tcolorbox}}\DIFaddend \fi

\textbf{Running title}: Metabolic and structural leaf mass

\textbf{Decomposing leaf mass into metabolic and structural components
explains divergent patterns of trait variation within and among plant
species}

Masatoshi Katabuchi\textsuperscript{1,2,6}, Kaoru
Kitajima\textsuperscript{2,3,4}, S. Joseph Wright\textsuperscript{4},
Sunshine A. Van Bael\textsuperscript{4,5}, Jeanne L. D.
Osnas\textsuperscript{2} and Jeremy W. Lichstein\textsuperscript{2}

\textsuperscript{1} Xishuangbanna Tropical Botanical Garden, Chinese
Academy of Sciences, Menglun, Yunnan 666303 China

\textsuperscript{2} Department of Biology, University of Florida,
Gainesville, FL 32611, USA

\textsuperscript{3} Graduate School of Agriculture, Kyoto University,
Kitashirakawa Oiwake-Cho, Kyoto 606-8502 Japan

\textsuperscript{4} Smithsonian Tropical Research Institute, 9100 Panama
City Pl., Washington, DC 20521

\textsuperscript{5} Department of Ecology and Evolutionary Biology,
Tulane University, New Orleans, LA 70118 USA

\textsuperscript{6} \textbf{Corresponding Author}: E-mail:
mattocci27@gmail.com

\DIFdelbegin %DIFDELCMD < \newpage
%DIFDELCMD < 

%DIFDELCMD < \hypertarget{abstract}{%
%DIFDELCMD < \section{Abstract}\label{abstract}}
%DIFDELCMD < 

%DIFDELCMD < %%%
\DIFdelend \begin{itemize}
\item
  Across the global flora, photosynthetic and metabolic rates depend
  more strongly on leaf area than leaf mass. In contrast, intraspecific
  variation in these rates is strongly mass-dependent. These contrasting
  patterns suggest that the causes of variation in leaf mass per area
  (LMA) may be fundamentally different within vs.~among species.
\item
  We \DIFdelbegin \DIFdel{used statistical methods }\DIFdelend \DIFaddbegin \DIFadd{developed statistical modeling framework }\DIFaddend to decompose LMA into two
  conceptual components -- \DIFdelbegin \DIFdel{`metabolic LMAp }\DIFdelend \DIFaddbegin \DIFadd{metabolic LMAm }\DIFaddend (which determines
  photosynthetic capacity and \DIFdelbegin \DIFdel{metabolic rates, and also affects optimal leaf lifespan) and `structural ' }\DIFdelend \DIFaddbegin \DIFadd{dark respiration) and structural }\DIFaddend LMAs
  (which determines leaf toughness and potential leaf lifespan) \DIFaddbegin \DIFadd{- }\DIFaddend using
  leaf trait data from tropical forest sites in Panama and a global
  leaf-trait database.
\item
  \DIFdelbegin \DIFdel{Statistically decomposing LMA into LMAp }\DIFdelend \DIFaddbegin \DIFadd{Decomposing LMA into LMAm }\DIFaddend and LMAs provides improved predictions of
  \DIFaddbegin \DIFadd{leaf }\DIFaddend trait variation (photosynthesis, respiration, and lifespan)\DIFdelbegin \DIFdel{across the global flora, and within and among tropical plant
  species in Panama. Our analysis shows that small scaling slope between
  metabolic leaf mass and photosynthetic rate, and similar variance
  in
  LMAp andin LMAs leads to area-proportionality of interspecific leaf
  traits}\DIFdelend . \DIFaddbegin \DIFadd{We
  show that strong area-dependence of metabolic traits across species
  can result from multiple of factors, including high LMAs variance
  and/or a slow increase in photosynthetic capacity with increasing
  LMAm. }\DIFaddend In contrast, \DIFdelbegin \DIFdel{intraspecific LMA variation is due to changes in
  LMAp, which explains why photosynthetic and metabolic traits are
  mass-dependent within
  species }\DIFdelend \DIFaddbegin \DIFadd{strong mass-dependence of metabolic traits within
  species across light levels results from LMAm increasing from sun to
  shade. LMAm and LMAs were nearly independent of each other in both
  global and Panama datasets.
}\DIFaddend \item
  Our results suggest that leaf \DIFdelbegin \DIFdel{trait }\DIFdelend \DIFaddbegin \DIFadd{functional }\DIFaddend variation is
  multi-dimensional and \DIFdelbegin \DIFdel{is not well-represented by the one-dimensional leaf economics
  spectrum}\DIFdelend \DIFaddbegin \DIFadd{that biogeochemical models should treat
  metabolic and structural leaf components separately}\DIFaddend .
\end{itemize}

\hypertarget{introduction}{%
\section{Introduction}\label{introduction}}

Leaf functional traits play an important role in ecological and
physiological tradeoffs (\protect\DIFdelbegin %DIFDELCMD < \hyperlink{ref-Wright2004a}{Wright et
%DIFDELCMD < al. 2004}%%%
\DIFdel{, }\DIFdelend \DIFaddbegin \hyperlink{ref-Onoda2017}{Onoda et al.,
2017}\DIFadd{; }\DIFaddend \protect\DIFdelbegin %DIFDELCMD < \hyperlink{ref-Reich2014}{Reich 2014}%%%
\DIFdel{,
}\DIFdelend \DIFaddbegin \hyperlink{ref-Reich2014}{Reich, 2014}\DIFadd{;
}\DIFaddend \protect\DIFdelbegin %DIFDELCMD < \hyperlink{ref-Onoda2017}{Onoda et al. 2017}%%%
\DIFdelend \DIFaddbegin \hyperlink{ref-Wright2004a}{I. J. Wright et al., 2004}\DIFaddend ) and in
carbon and nutrient cycling (\DIFdelbegin %DIFDELCMD < \protect\hyperlink{ref-Tcherkez2017}{Tcherkez et al.
%DIFDELCMD < 2017}%%%
\DIFdel{, }\DIFdelend e.g.,
\protect\DIFdelbegin %DIFDELCMD < \hyperlink{ref-Huntingford2017}{Huntingford et al.
%DIFDELCMD < 2017}%%%
\DIFdelend \DIFaddbegin \hyperlink{ref-Huntingford2017}{Huntingford et al., 2017}\DIFadd{;
}\protect\hyperlink{ref-Tcherkez2017}{Tcherkez et al., 2017}\DIFaddend ). Thus,
understanding the causes and consequences of leaf trait variation is an
important goal in ecology, plant physiology, and biogeochemistry
(\DIFdelbegin \DIFdel{references}\DIFdelend \DIFaddbegin \protect\hyperlink{ref-Bonan2002}{Bonan et al., 2002}\DIFadd{;
}\protect\hyperlink{ref-Poorter2009}{Poorter et al., 2009}\DIFaddend ). Different
leaf assemblages exhibit markedly different patterns of trait variation.
For example, across global species, whole-leaf values of traits related
to photosynthesis and metabolism (e.g., the photosynthetic capacity,
respiration rate, or nutrient content of \DIFdelbegin \DIFdel{an entire leaf}\DIFdelend \DIFaddbegin \DIFadd{entire leaves}\DIFaddend ) tend to increase
\DIFaddbegin \DIFadd{more strongly }\DIFaddend with leaf area \DIFdelbegin \DIFdel{, but
not leaf mass per area (LMA; }\DIFdelend \DIFaddbegin \DIFadd{than leaf mass (}\DIFaddend Osnas et al.
(\protect\hyperlink{ref-Osnas2013}{2013}))\DIFdelbegin \DIFdel{. In contrast, within species,
these same whole-leaf trait values tend to increase with LMA from shade
to full sunlight }\DIFdelend \DIFaddbegin \DIFadd{, whereas the opposite pattern
(strong mass-dependence of these same traits) is observed within species
across canopy light gradients }\DIFaddend (\protect\DIFdelbegin %DIFDELCMD < \hyperlink{ref-Osada2001}{Osada et al. 2001}%%%
\DIFdel{,
}\DIFdelend \DIFaddbegin \hyperlink{ref-Osada2001}{Osada
et al., 2001}\DIFadd{; }\DIFaddend \protect\DIFdelbegin %DIFDELCMD < \hyperlink{ref-Osnas2018}{Osnas et al. 2018}%%%
\DIFdelend \DIFaddbegin \hyperlink{ref-Osnas2018}{Osnas et al., 2018}\DIFaddend ).
Functional groups (e.g., deciduous vs.~evergreen angiosperms) also
differ from each other in terms of how photosynthetic and metabolic
traits variation depend on leaf mass and area
(\protect\DIFdelbegin %DIFDELCMD < \hyperlink{ref-Osnas2018}{Osnas et al.
%DIFDELCMD < 2018}%%%
\DIFdelend \DIFaddbegin \hyperlink{ref-Osnas2018}{Osnas et al., 2018}\DIFaddend ). These divergent
patterns suggest the presence of multiple drivers of trait variation.

Strong interspecific correlations among \DIFdelbegin \DIFdel{LMA}\DIFdelend \DIFaddbegin \DIFadd{leaf mass per area (LMA)}\DIFaddend , leaf
lifespan (LL), and mass-normalized leaf traits related to photosynthesis
and metabolism \DIFaddbegin \DIFadd{(hereafter, `metabolic traits') }\DIFaddend have been interpreted as
evidence for a single dominant axis of leaf functional variation
(\protect\DIFdelbegin %DIFDELCMD < \hyperlink{ref-Wright2004a}{Wright et al.
%DIFDELCMD < 2004}%%%
\DIFdelend \DIFaddbegin \hyperlink{ref-Wright2004a}{I. J. Wright et al., 2004}\DIFaddend ).
However, the interpretation of these strong correlations, which result
from mass-normalization of area-dependent traits
(\protect\DIFdelbegin %DIFDELCMD < \hyperlink{ref-Lloyd2013}{Lloyd et al. 2013}%%%
\DIFdel{,
}\DIFdelend \DIFaddbegin \hyperlink{ref-Lloyd2013}{Lloyd et al., 2013}\DIFadd{;
}\DIFaddend \protect\DIFdelbegin %DIFDELCMD < \hyperlink{ref-Osnas2013}{Osnas et al. 2013}%%%
\DIFdelend \DIFaddbegin \hyperlink{ref-Osnas2013}{Osnas et al., 2013}\DIFaddend ), is
controversial. On the one hand, mass-normalization has been justified
based on economic principles alone
(\protect\DIFdelbegin %DIFDELCMD < \hyperlink{ref-Westoby2013}{Westoby et al.
%DIFDELCMD < 2013}%%%
\DIFdelend \DIFaddbegin \hyperlink{ref-Westoby2013}{Westoby et al., 2013}\DIFaddend ), because
leaf mass provides a simple index of investment that underlies the leaf
economics spectrum (LES), ranging from cheap, low-LMA leaves with a fast
rate of return per-unit investment to expensive, high-LMA leaves with a
slow rate of return (\protect\DIFdelbegin %DIFDELCMD < \hyperlink{ref-Wright2004a}{Wright et al. 2004}%%%
\DIFdelend \DIFaddbegin \hyperlink{ref-Wright2004a}{I. J. Wright et
al., 2004}\DIFaddend ). On the other hand, because the lifetime return on
investment depends on LL (\protect\DIFdelbegin %DIFDELCMD < \hyperlink{ref-Westoby2000}{Westoby et al. 2000}%%%
\DIFdel{,
}\DIFdelend \DIFaddbegin \hyperlink{ref-Falster2012}{Falster et
al., 2012}\DIFadd{; }\DIFaddend \protect\DIFdelbegin %DIFDELCMD < \hyperlink{ref-Falster2012}{Falster et al. 2012}%%%
\DIFdel{),
it has been
argued that }\DIFdelend \DIFaddbegin \hyperlink{ref-Westoby2000}{Westoby et al., 2000}\DIFadd{),
and normalizing traits by their }\DIFaddend annualized construction costs (or
mass/LL) should be used as a normalizer in leaf economics analyses
(\protect\DIFdelbegin %DIFDELCMD < \hyperlink{ref-Osnas2013}{Osnas et al. 2013}%%%
\DIFdelend \DIFaddbegin \hyperlink{ref-Osnas2013}{Osnas et al., 2013}\DIFaddend ). In global
analyses, normalizing leaf traits by mass/LL yields similarly weak
correlations as area-normalization, because leaf area is roughly
proportional to mass/LL across global species
(\protect\DIFdelbegin %DIFDELCMD < \hyperlink{ref-Osnas2013}{Osnas et al. 2013}%%%
\DIFdelend \DIFaddbegin \hyperlink{ref-Osnas2013}{Osnas et al., 2013}\DIFaddend ). Area-dependence
of metabolic traits and the inconsistent correlation strengths obtained
from different normalizers (\protect\DIFdelbegin %DIFDELCMD < \hyperlink{ref-Lloyd2013}{Lloyd et
%DIFDELCMD < al. 2013}%%%
\DIFdel{, }\DIFdelend \DIFaddbegin \hyperlink{ref-Lloyd2013}{Lloyd et
al., 2013}\DIFadd{; }\DIFaddend \protect\DIFdelbegin %DIFDELCMD < \hyperlink{ref-Osnas2013}{Osnas et al. 2013}%%%
\DIFdelend \DIFaddbegin \hyperlink{ref-Osnas2013}{Osnas et al., 2013}\DIFaddend ) do
not invalidate mass-normalization or the LES
(\protect\DIFdelbegin %DIFDELCMD < \hyperlink{ref-Westoby2013}{Westoby et al. 2013}%%%
\DIFdelend \DIFaddbegin \hyperlink{ref-Westoby2013}{Westoby et al., 2013}\DIFaddend ) These
considerations do, however, lead us to question the evidence for a
single dominant axis of leaf functional diversity, which has important
implications for how trait variation is represented in \DIFdelbegin \DIFdel{vegetation }\DIFdelend \DIFaddbegin \DIFadd{ecosystem }\DIFaddend models
(\protect\DIFdelbegin %DIFDELCMD < \hyperlink{ref-Bonan2002}{Bonan et al. 2002}%%%
\DIFdel{,
}\DIFdelend \DIFaddbegin \hyperlink{ref-Bonan2002}{Bonan et al., 2002}\DIFadd{;
}\DIFaddend \protect\DIFdelbegin %DIFDELCMD < \hyperlink{ref-Sakschewski2016}{Sakschewski et al. 2016}%%%
\DIFdelend \DIFaddbegin \hyperlink{ref-Sakschewski2016}{Sakschewski et al., 2016}\DIFaddend ).

To understand why the same data can be interpreted as either supporting
or opposing the existence of a single dominant axis of leaf trait
variation, consider the conceptual model proposed by Osnas et al.
(\protect\hyperlink{ref-Osnas2018}{2018}), in which LMA is comprised of
two additive components: photosynthetic LMA (\DIFdelbegin \DIFdel{LMAp}\DIFdelend \DIFaddbegin \DIFadd{LMAm}\DIFaddend ) -- the mass per area
of chloroplasts and other metabolically active leaf components that
contribute directly to photosynthesis \DIFaddbegin \DIFadd{and respiration }\DIFaddend -- and structural
LMA (LMAs) -- the mass per area of structural leaf components that
contribute to toughness and durability
(\protect\DIFdelbegin %DIFDELCMD < \hyperlink{ref-Kitajima2012}{Kitajima
%DIFDELCMD < et al. 2012}%%%
\DIFdelend \DIFaddbegin \hyperlink{ref-Kitajima2016}{Kitajima et al., 2016}\DIFaddend ,
\protect\DIFdelbegin %DIFDELCMD < \hyperlink{ref-Kitajima2016}{2016}%%%
\DIFdel{,
}\DIFdelend \DIFaddbegin \hyperlink{ref-Kitajima2012}{2012}\DIFadd{;
}\DIFaddend \protect\DIFdelbegin %DIFDELCMD < \hyperlink{ref-Onoda2017}{Onoda et al. 2017}%%%
\DIFdel{). Consider the
scenario where }\DIFdelend \DIFaddbegin \hyperlink{ref-Onoda2017}{Onoda et al., 2017}\DIFadd{). Suppose }\DIFaddend these
two LMA components are independent axes of functional variation, so that
\DIFdelbegin \DIFdel{LMAp }\DIFdelend \DIFaddbegin \DIFadd{LMAm }\DIFaddend and LMAs are uncorrelated across species (Fig.~\ref{fig-Hplt}a).
\DIFdelbegin %DIFDELCMD < 

%DIFDELCMD < %%%
\DIFdelend These two independent axes can be translated into either a
two-dimensional trait space (if metabolic traits are area-normalized;
Fig.~\ref{fig-Hplt}b) or a one-dimensional trait space (if metabolic
traits are mass-normalized; Fig.~\ref{fig-Hplt}c). While both
Fig.~\ref{fig-Hplt}b and Fig.~\ref{fig-Hplt}c are both `correct'
representations of the same data, they lead to different perceptions
about the dimensionality of leaf functional variation. If \DIFdelbegin \DIFdel{LMAp}\DIFdelend \DIFaddbegin \DIFadd{LMAm}\DIFaddend , LMAs,
and/or other axes are largely independent and have distinct functional
consequences, then it would not be possible to accurately represent
functional variation with a single axis.

The hypothetical example in Fig.~\ref{fig-Hplt} shows how
mass-normalization can, in principle, make a two-dimensional trait space
appear one-dimensional, but \DIFdelbegin \DIFdel{this example does not resolve }\DIFdelend the dimensionality of functional variation
in real leaf assemblages \DIFaddbegin \DIFadd{remains an open question}\DIFaddend . One way to better
understand the dimensionality of leaf trait variation is to compare
models with different numbers of dimensions, and to ask if models with
multiple dimensions provide improved statistical fits and conceptual
insights compared to a single axis. For example, the two-dimensional
`\DIFdelbegin \DIFdel{LMAp }\DIFdelend \DIFaddbegin \DIFadd{LMAm }\DIFaddend + LMAs' model proposed by Osnas et al.
(\protect\hyperlink{ref-Osnas2018}{2018}) could be compared to a
one-dimensional LMA model in terms of their capacities to explain
variation in other traits. However, implementing the `\DIFdelbegin \DIFdel{LMAp }\DIFdelend \DIFaddbegin \DIFadd{LMAm }\DIFaddend + LMAs' model
is challenging, because although certain leaf mass components can be
neatly classified as `metabolic' or `structural'
(\protect\DIFdelbegin %DIFDELCMD < \hyperlink{ref-Poorter2009}{Poorter et al. 2009}%%%
\DIFdel{,
}\DIFdelend \DIFaddbegin \hyperlink{ref-Osnas2018}{Osnas et al., 2018}\DIFadd{;
}\DIFaddend \protect\DIFdelbegin %DIFDELCMD < \hyperlink{ref-Osnas2018}{Osnas et al. 2018}%%%
\DIFdelend \DIFaddbegin \hyperlink{ref-Poorter2009}{Poorter et al., 2009}\DIFaddend ), other leaf
mass components cannot. For example, thick cell walls contribute to
structural toughness (\protect\DIFdelbegin %DIFDELCMD < \hyperlink{ref-Onoda2015}{Onoda et al.
%DIFDELCMD < 2015}%%%
\DIFdelend \DIFaddbegin \hyperlink{ref-Onoda2015}{Onoda et al.,
2015}\DIFaddend ), but at least some cell wall mass is required for the
biomechanical support that enables photosynthesis. Thus, partitioning
LMA into different functional components requires novel empirical or
modeling approaches.

In this paper, we present a statistical modeling framework to partition
LMA into metabolic and structural components: \DIFdelbegin \DIFdel{LMAp }\DIFdelend \DIFaddbegin \DIFadd{LMAm }\DIFaddend and LMAs. We develop
and test this two-dimensional model using leaf trait data from two
tropical forest sites (sun and shade leaves from wet and dry sites in
Panama) and the GLOPNET global leaf traits database
(\protect\DIFdelbegin %DIFDELCMD < \hyperlink{ref-Wright2004a}{Wright et al. 2004}%%%
\DIFdelend \DIFaddbegin \hyperlink{ref-Wright2004a}{I. J. Wright et al., 2004}\DIFaddend ). We use
the model to address the following questions: (1) \DIFdelbegin \DIFdel{How does variation in LMA
components related with mass and area proportionality of leaf
traits }\DIFdelend \DIFaddbegin \DIFadd{Are measured leaf
traits (including photosynthetic capacity, dark respiration rate, LL,
and concentrations of nutrients and cellulose) better predicted by a
one-dimensional (total LMA) or two-dimensional (LMAm-LMAs) model}\DIFaddend ? (2)
\DIFdelbegin \DIFdel{Do LMAp }\DIFdelend \DIFaddbegin \DIFadd{What are the relative contributions of LMAm and LMAs to total LMA
variance in different leaf assemblages (Panama sun leaves, Panama shade
leaves, and the global flora)? (3) Do LMAm }\DIFaddend and LMAs differ between
evergreen and deciduous species, and between sun and shade leaves? \DIFdelbegin \DIFdel{and }\DIFdelend \DIFaddbegin \DIFadd{If
so, how? and (4) How do the answers to the preceding questions inform
our understanding of empirical patterns of trait variation }\DIFaddend (\DIFdelbegin \DIFdel{3) How are measurable leaf
photosynthetic and structural traits (}\DIFdelend e.g.,
\DIFdelbegin \DIFdel{concentrations of nitrogen
and cellulose)related to LMAp and LMAs}\DIFdelend \DIFaddbegin \DIFadd{relationships among measured leaf traits)}\DIFaddend ?

\hypertarget{material-and-methods}{%
\section{Material and Methods}\label{material-and-methods}}

\hypertarget{overview}{%
\subsection{Overview}\label{overview}}

We considered multiple approaches to modeling datasets that include
observations of leaf mass per area (LMA), leaf lifespan (LL), net
photosynthetic capacity (A), and dark respiration rate (R). These traits
comprise four of the six traits in the global leaf economics spectrum
(LES) analysis of \DIFaddbegin \DIFadd{I. J. }\DIFaddend Wright et al.
\DIFdelbegin \DIFdel{~(2004}\DIFdelend \DIFaddbegin \DIFadd{(}\protect\hyperlink{ref-Wright2004a}{2004}\DIFaddend ). For simplicity, we did not
include the other two LES traits -- leaf nitrogen (N) and phosphorus (P)
concentrations -- in our modeling framework\DIFdelbegin \DIFdel{, but we did explore
relationships between our model predictions and observed }\DIFdelend \DIFaddbegin \DIFadd{. Instead, we reserved
observations of }\DIFaddend leaf N and P \DIFaddbegin \DIFadd{concentrations (and cellulose content, when
available) for independent model tests}\DIFaddend . We considered a simple
one-dimensional model that predicts LL, A, and R from LMA alone, as well
as two-dimensional models that predict LL, A, and R from two additive
LMA components: structural and metabolic \DIFaddbegin \DIFadd{and structural }\DIFaddend leaf mass per
area (\DIFdelbegin \DIFdel{LMAs and LMAp}\DIFdelend \DIFaddbegin \DIFadd{LMAm and LMAs}\DIFaddend , respectively). We formulate the models in terms of
area-normalized A and R (\emph{A}\textsubscript{area} and
\emph{R}\textsubscript{area}, respectively), as these trait forms \DIFdelbegin \DIFdel{are
naturally related to the LMA components (particularly LMAp) in our
model.
}\DIFdelend \DIFaddbegin \DIFadd{share
the same denominator as LMA and its components.
}

\DIFaddend To implement the two-dimensional models, we developed a statistical
modeling framework to partition LMA into additive \DIFdelbegin \DIFdel{LMAs and LMAp }\DIFdelend \DIFaddbegin \DIFadd{LMAm and LMAs
}\DIFaddend components. We fit the models to two datasets: the GLOPNET global leaf
traits dataset (\protect\DIFdelbegin %DIFDELCMD < \hyperlink{ref-Wright2004a}{Wright et al.
%DIFDELCMD < 2004}%%%
\DIFdelend \DIFaddbegin \hyperlink{ref-Wright2004a}{I. J. Wright et al.,
2004}\DIFaddend ), which primarily represents interspecific variation; and the
Panama dataset described by Osnas et al.
\DIFdelbegin \DIFdel{~(2018}\DIFdelend \DIFaddbegin \DIFadd{(}\protect\hyperlink{ref-Osnas2018}{2018}\DIFaddend ), which includes traits for
both sun and shade leaves at wet and dry tropical forest sites. Because
\DIFdelbegin \DIFdel{LMAs and LMAp }\DIFdelend \DIFaddbegin \DIFadd{LMAm and LMAs }\DIFaddend are modeled (rather than observed), the two-dimensional
models require one parameter per analysis unit (a species in GLOPNET or
a species \(\times\) canopy-position in the Panama dataset) to partition
observed LMA into \DIFdelbegin \DIFdel{LMAs and LMAp}\DIFdelend \DIFaddbegin \DIFadd{LMAm and LMAs}\DIFaddend . Given the large number of parameters in
these models, we performed tests with randomized data to evaluate if our
two-dimensional models were prone to overfitting, which could lead to
spurious conclusions. These tests suggested that our two-dimensional
modeling approach revealed meaningful patterns in the observed trait
data.
\DIFdelbegin \DIFdel{We compared predictions from the
one- and two-dimensional models to address the questions listed at the
end of the Introduction.
}\DIFdelend 

\hypertarget{modeling-leaf-lifespan-photosynthetic-capacity-and-dark-respiration-rate}{%
\subsection{Modeling leaf lifespan, photosynthetic capacity, and dark
respiration
rate}\label{modeling-leaf-lifespan-photosynthetic-capacity-and-dark-respiration-rate}}

We considered five types of models, ranging from simple models with LMA
as the sole predictor, to more complex models in which LMA was
partitioned into \DIFdelbegin \DIFdel{metabolic and structural components (LMAp and LMAs)}\DIFdelend \DIFaddbegin \DIFadd{LMAm and LMAs}\DIFaddend . In all models, the unit of analysis is a
`leaf sample', defined as a species in the GLOPNET dataset or a species
\(\times\) canopy position in the Panama dataset (\DIFdelbegin \DIFdel{see Datasets }\DIFdelend \DIFaddbegin \DIFadd{the datasets are
described }\DIFaddend below).

First, we considered a simple set of models with LMA as the sole
predictor for \emph{A}\textsubscript{area},
\emph{R}\textsubscript{area}, and LL \DIFdelbegin \DIFdel{. In this model, LMA is assumed to have }\DIFdelend \DIFaddbegin \DIFadd{according to }\DIFaddend power-law
relationships\DIFdelbegin \DIFdel{with }\emph{\DIFdel{A}}%DIFAUXCMD
\DIFdel{\textsubscript{area},
}\emph{\DIFdel{R}}%DIFAUXCMD
\DIFdel{\textsubscript{area} and LL}\DIFdelend .

Next, we considered \DIFdelbegin \DIFdel{several }\DIFdelend models in which LMA is partitioned into additive
metabolic and structural \DIFdelbegin \DIFdel{functions; i.e., we assume that the
sum of metabolic leaf mass per area (LMAp) and structural leaf mass per
area (LMAs) is equal to total observed LMA }\DIFdelend \DIFaddbegin \DIFadd{components }\DIFaddend for leaf sample \emph{i}:

\begin{equation}\protect\DIFdelbegin %DIFDELCMD < \hypertarget{eq-LMA}{}{
%DIFDELCMD < \begin{aligned}
%DIFDELCMD <   &\mathrm{LMA}_{i} =\mathrm{LMAp}_{i} + \mathrm{LMAs}_{i} \\
%DIFDELCMD <   &\mathrm{LMAp}_{i} = f_{i} \mathrm{LMA}_{i} \\
%DIFDELCMD <   &\mathrm{LMAs}_{i} = (1 - f_{i})  \mathrm{LMA}_{i}
%DIFDELCMD < \end{aligned}
%DIFDELCMD < }%%%
\DIFdelend \DIFaddbegin \hypertarget{eq-LMA}{}{
\begin{aligned}
  &\mathrm{LMA}_{i} =\mathrm{LMAm}_{i} + \mathrm{LMAs}_{i} \\
  &\mathrm{LMAm}_{i} = f_{i} \mathrm{LMA}_{i} \\
  &\mathrm{LMAs}_{i} = (1 - f_{i})  \mathrm{LMA}_{i}
\end{aligned}
}\DIFaddend \label{eq-LMA}\end{equation}

where the \emph{f\textsubscript{i}} values -- the fractions of LMA
comprised of \DIFdelbegin \DIFdel{LMAp }\DIFdelend \DIFaddbegin \DIFadd{LMAm }\DIFaddend for each sample \emph{i} -- are estimated as latent
variables in our modeling framework (see details below). \DIFdelbegin \DIFdel{The models we
considered combine Eq.~\ref{eq-LMA} with functional forms that
predict
observed values Amax,
Rdark, and LL from }\DIFdelend \DIFaddbegin \DIFadd{We assumed that
the observed values of }\emph{\DIFadd{A}}\DIFadd{\textsubscript{area},
}\emph{\DIFadd{R}}\DIFadd{\textsubscript{area} and LL were related to }\DIFaddend the unobserved
values \DIFdelbegin \DIFdel{LMAs and LMAp.
}%DIFDELCMD < 

%DIFDELCMD < %%%
\DIFdel{The simplest of these forms is a multivariate power-law}\DIFdelend \DIFaddbegin \DIFadd{LMAm and LMAs according to multivariate power-laws}\DIFaddend :

\begin{align}
& \mathrm{E}[A_{\mathrm{area} \, i}]
= \alpha_0\DIFdelbegin \DIFdel{\mathrm{LMAp}}\DIFdelend \DIFaddbegin \DIFadd{\mathrm{LMAm}}\DIFaddend _{i}\DIFdelbegin \DIFdel{^{\alpha_p}}\DIFdelend \DIFaddbegin \DIFadd{^{\alpha_m}}\DIFaddend \mathrm{LMAs}_i^{\alpha_s}  =  \alpha_0 (f_i \mathrm{LMA}_{i})\DIFdelbegin \DIFdel{^{\alpha_p} }\DIFdelend \DIFaddbegin \DIFadd{^{\alpha_m} }\DIFaddend \bigl\{(1-f_i) \mathrm{LMA}_{i}\bigr\}^{\alpha_s} \label{eq:Aarea} \\
& \mathrm{E}[R_{\mathrm{area} \, i}]
= \gamma_0\DIFdelbegin \DIFdel{\mathrm{LMAp}}\DIFdelend \DIFaddbegin \DIFadd{\mathrm{LMAm}}\DIFaddend _{i}^{\gamma_p} \mathrm{LMAs}_{i}^{\gamma_s}
= \gamma_0 (f_i \mathrm{LMA}_{i})^{\gamma_p} \bigl\{(1-f_i)\mathrm{LMA}_{i}\bigr\}^{\gamma_s} \label{eq:Rarea} \\
& \mathrm{E}[\mathrm{LL}_i] = \beta_0\DIFdelbegin \DIFdel{\mathrm{LMAp}}\DIFdelend \DIFaddbegin \DIFadd{\mathrm{LMAm}}\DIFaddend _{i}\DIFdelbegin \DIFdel{^{\beta_p} }\DIFdelend \DIFaddbegin \DIFadd{^{\beta_m} }\DIFaddend \mathrm{LMAs}_{i}^{\beta_s}  = \beta_0 (f_i \mathrm{LMA}_{i})\DIFdelbegin \DIFdel{^{\beta_p} }\DIFdelend \DIFaddbegin \DIFadd{^{\beta_m} }\DIFaddend \bigl\{(1-f_i) \mathrm{LMA}_{i}\bigr\}^{\beta_s} \label{eq:LL_pot} \tag{4a}  \\
& \mathrm{E[LL}_i] = \beta_0\DIFdelbegin \DIFdel{\mathrm{LMAp}}\DIFdelend \DIFaddbegin \DIFadd{\mathrm{LMAm}}\DIFaddend _{i}\DIFdelbegin \DIFdel{^{\beta_p} }\DIFdelend \DIFaddbegin \DIFadd{^{\beta_m} }\DIFaddend \mathrm{LMAs}_{i}^{\beta_s} exp(\theta Light_i)  \stepcounter{equation} \label{eq:LL_opt} \tag{4b}
\end{align}

where E{[}\(\cdot\){]} indicates expected value; the \(\alpha\),
\(\beta\), and \(\gamma\) terms are fitted parameters; and the
logarithms of \emph{A}\textsubscript{area}, LL, and
\emph{R}\textsubscript{area} are assumed to have a multivariate normal
distribution (Appendix S1). \DIFdelbegin \DIFdel{We assumed that }\DIFdelend \DIFaddbegin \DIFadd{). In preliminary analyses, we also
considered alternative (non-power-law) forms, but none of these
performed better than the power-law forms and are not discussed further.
}

\DIFadd{To ensure that the model was identifiable, we imposed two broad
assumptions}\DIFaddend : (i) \emph{A}\textsubscript{area} depends more strongly on
metabolic leaf mass (\DIFdelbegin \DIFdel{LMAp: parameters \(\alpha_p\) in Eq. }\DIFdelend \DIFaddbegin \DIFadd{LMAm: parameter \(\alpha_m\) in }\DIFaddend \ref{eq:Aarea})
than structural leaf mass (LMAs: parameter \(\alpha_s\)), and (ii) LL
depends more strongly on LMAs (\(\beta_s\) in
Eqs.\ref{eq:LL_pot}-\ref{eq:LL_opt}) than LMAm (\DIFdelbegin \DIFdel{\(\beta_p\)). In the first
assumption , we set }\DIFdelend \DIFaddbegin \DIFadd{\(\beta_m\)). The first
assumption was implemented in different model versions either by setting
}\DIFaddend \(\alpha_s\) \DIFdelbegin \DIFdel{to zero or we tested if \(\alpha_p\) is
greater than }\DIFdelend \DIFaddbegin \DIFadd{= 0 or by imposing the constraint \(\alpha_m\)
\textgreater{} }\DIFaddend \(\alpha_s\). \DIFdelbegin \DIFdel{In }\DIFdelend \DIFaddbegin \DIFadd{Similarly, }\DIFaddend the second assumption \DIFdelbegin \DIFdel{, we set \(\beta_p\)
to zero or we tested if }\DIFdelend \DIFaddbegin \DIFadd{was
implemented either by setting \(\beta_m\) = 0 or by imposing the
constraint }\DIFaddend \(\beta_s\) \DIFdelbegin \DIFdel{is greater than \(\beta_p\). When we
used all of
the four scaling parameters (i.e., \(\alpha_p\),
}\DIFdelend \DIFaddbegin \DIFadd{\textgreater{} \(\beta_m\). The weaker form of
these assumptions (\(\alpha_m\) \textgreater{} }\DIFaddend \(\alpha_s\) \DIFdelbegin \DIFdel{, \(\beta_p\), }\DIFdelend and
\(\beta_s\) \DIFdelbegin \DIFdel{), we constrained \(\alpha_p\)
\textgreater{} }\DIFdelend \DIFaddbegin \DIFadd{\textgreater{} \(\beta_m\)) is primarily a labeling
constraint and only weakly constrains the possible biological model
outcomes by excluding the possibility that a single LMA component could
be the primary determinant of both }\emph{\DIFadd{A}}\DIFadd{\textsubscript{area} and LL.
The stronger form of the assumptions (}\DIFaddend \(\alpha_s\) \DIFaddbegin \DIFadd{= 0 and \(\beta_m\) =
0) leads to a more parsimonious model (fewer parameters). We considered
different combinations of the strong and weak forms of the assumptions
for }\emph{\DIFadd{A}}\DIFadd{\textsubscript{area} (\(\alpha_m\) }\DIFaddend and \DIFaddbegin \DIFadd{\(\alpha_s\)) and LL
(}\DIFaddend \(\beta_s\) \DIFdelbegin \DIFdel{\textgreater{} \(\beta_p\)
to avoid the invariance of the likelihood. }\DIFdelend \DIFaddbegin \DIFadd{and \(\beta_m\)) using cross-validation (see details
below). We did not impose any constraints on
}\emph{\DIFadd{R}}\DIFadd{\textsubscript{area} (\(\gamma_s\) and \(\gamma_m\)).
}\DIFaddend 

We also considered models in which LMAs in \DIFdelbegin \DIFdel{Eq}\DIFdelend \DIFaddbegin \DIFadd{Eqs}\DIFaddend .
\ref{eq:LL_pot}-\ref{eq:LL_opt} was replaced by leaf structural density
(LMAs/LT, where LT is leaf thickness), which \DIFdelbegin \DIFdel{has been shown in some
cases to be }\DIFdelend \DIFaddbegin \DIFadd{is based on the observation
that cellulose density is }\DIFaddend a good predictor for LL \DIFaddbegin \DIFadd{in some species
assemblages }\DIFaddend (\protect\DIFdelbegin %DIFDELCMD < \hyperlink{ref-Kitajima2012}{Kitajima et al. 2012}%%%
\DIFdelend \DIFaddbegin \hyperlink{ref-Kitajima2013}{Kitajima et al.,
2013}\DIFaddend , \protect\DIFdelbegin %DIFDELCMD < \hyperlink{ref-Kitajima2013}{2013}%%%
\DIFdel{). In our analysis, models with
leaf density }\DIFdelend \DIFaddbegin \hyperlink{ref-Kitajima2012}{2012}\DIFadd{). These models with
LMAs/LT }\DIFaddend yielded qualitatively similar results as those based on LMAs,
but did not perform as well in the model evaluation (see below).
Therefore, we only present results \DIFdelbegin \DIFdel{of }\DIFdelend \DIFaddbegin \DIFadd{from }\DIFaddend the models with LMAs.

Finally, we considered a functional form that accounts for the effects
of light availability on LL. This light-dependent model is motivated by
optimal LL theory, which predicts decreasing LL with increasing light
availability (refs), and also by the often-observed `LMA
counter-gradient', whereby LL and LMA positively covary across species
but negatively covary within species across light gradients
(\protect\DIFdelbegin %DIFDELCMD < \hyperlink{ref-Lusk2008}{Lusk et al. 2008}%%%
\DIFdel{,
}\DIFdelend \DIFaddbegin \hyperlink{ref-Lusk2008}{Lusk et al., 2008}\DIFadd{;
}\DIFaddend \protect\DIFdelbegin %DIFDELCMD < \hyperlink{ref-Russo2016}{Russo and Kitajima 2016}%%%
\DIFdel{,
}\DIFdelend \DIFaddbegin \hyperlink{ref-Osnas2018}{Osnas et al., 2018}\DIFadd{;
}\DIFaddend \protect\DIFdelbegin %DIFDELCMD < \hyperlink{ref-Osnas2018}{Osnas et al. 2018}%%%
\DIFdelend \DIFaddbegin \hyperlink{ref-Russo2016}{Russo \& Kitajima, 2016}\DIFaddend ).
Mechanistically modeling how light affects LL via leaf carbon balance
(\protect\DIFdelbegin %DIFDELCMD < \hyperlink{ref-Xu2017}{Xu et al. 2017}%%%
\DIFdelend \DIFaddbegin \hyperlink{ref-Xu2017}{Xu et al., 2017}\DIFaddend ) is beyond the scope of
our study. Instead, we introduced light effects in a simple way by
modifying Eq. \ref{eq:LL_pot} to Eq. \ref{eq:LL_opt} where the dummy
variable \(Light_i\) is set to 1 for sun leaves and 0 for shade leaves,
and \(exp(\theta)\) is the sun:shade LL ratio.

\DIFdelbegin \DIFdel{In summary, we have five model forms representing LL: (1) LMA, (2) LMAp
and LMAs, (3) LMA and light, (4) LMAp, LMAs and light, and (5) LMAp,
LMAs/LT and light, with four types of parameter restrictions: (a)
\(\alpha_s = 0\) and \(\beta_p = 0\), (b) \(\beta_p = 0\), (c)
\(\alpha_s = 0\), and (d) \(\alpha_p > \alpha_s\) and
\(\beta_s > \beta_p\) (Table 1).
}%DIFDELCMD < 

%DIFDELCMD < %%%
\DIFdelend \hypertarget{datasets}{%
\subsection{Datasets}\label{datasets}}

We fit the models described above using observations of LMA (g
m\textsuperscript{-2}), \emph{A}\textsubscript{area} (mol
s\textsuperscript{-1} m\textsuperscript{-2}),
\emph{R}\textsubscript{area} (mol s\textsuperscript{-1}
m\textsuperscript{-2}) and LL (months) from the GLOPNET global leaf
traits database (\protect\DIFdelbegin %DIFDELCMD < \hyperlink{ref-Wright2004a}{Wright et al.
%DIFDELCMD < 2004}%%%
\DIFdelend \DIFaddbegin \hyperlink{ref-Wright2004a}{I. J. Wright et
al., 2004}\DIFaddend ) and from two tropical forest sites in Panama: Monumental
Natural Metropolitano (MNM, ``dry site'') and Bosque Protector San
Lorenzo (SL, ``wet site''). The GLOPNET data primarily \DIFdelbegin \DIFdel{represent interspecific variation }\DIFdelend \DIFaddbegin \DIFadd{represents
interspecific variation (the dataset reports 2,548 species \(\times\)
site combinations, with 2}\DIFaddend ,\DIFdelbegin \DIFdel{whereas the Panama data represent }\DIFdelend \DIFaddbegin \DIFadd{021 unique species and only 350 species
occurring at more than one site) and only reports data for sun leaves if
data for both sun and shade leaves are available
(}\protect\hyperlink{ref-Write2004a}{\textbf{Write2004a?}}\DIFadd{). In contrast,
the Panama dataset represents }\DIFaddend both inter- and intraspecific variation,
including leaves sampled at two canopy positions (``sun'': full sun at
the top of the canopy; and ``shade'': well shaded, sampled within 2 m of
the forest floor) from trees within reach of canopy cranes. The dry MNM
site is a semi-deciduous coastal Pacific forest with a 5-month dry
season from December-April and 1740 mm of annual rainfall
(\protect\DIFdelbegin %DIFDELCMD < \hyperlink{ref-Wright2003}{Wright et al.
%DIFDELCMD < 2003}%%%
\DIFdelend \DIFaddbegin \hyperlink{ref-Wright2003}{S. J. Wright et al., 2003}\DIFaddend ). The MNM
crane is 40 m tall with a 51 m long boom. The wet SL site is an
evergreen Caribbean coastal forest with 3100 mm of annual rainfall
(\protect\DIFdelbegin %DIFDELCMD < \hyperlink{ref-Wright2003}{Wright et al. 2003}%%%
\DIFdelend \DIFaddbegin \hyperlink{ref-Wright2003}{S. J. Wright et al., 2003}\DIFaddend ). The SL
crane is 52 m tall with a 54 m long boom. Additional details of the
Panama dataset are described in Osnas et al.
(\protect\hyperlink{ref-Osnas2018}{2018}).

We restricted our analysis to database records for which all four traits
(LMA, \emph{A}\textsubscript{area}, \emph{R}\textsubscript{area}, and
LL; each typically averaged over multiple leaves) were available. Each
database record corresponds to an analysis unit (or `leaf sample') \DIFaddbegin \DIFadd{as
}\DIFaddend described above; i.e., a species in GLOPNET, or a species \(\times\)
canopy position in the Panama dataset. After excluding database records
that lacked one or more of the four traits, 198 samples for 198 unique
species were available for GLOPNET, and 130 samples for 104 unique
species were available for Panama (dry and wet sites combined; 26
species sampled in both sun and shade; no species with all four traits
available at both sites). In addition to \DIFdelbegin \DIFdel{the four primary traits,
Model
5 (Table 1) }\DIFdelend \DIFaddbegin \DIFadd{LMA,
}\emph{\DIFadd{A}}\DIFadd{\textsubscript{area}, }\emph{\DIFadd{R}}\DIFadd{\textsubscript{area}, and LL, the
model based on structural leaf density }\DIFaddend also requires observations of
leaf thickness, which was not available in GLOPNET but was available for
106/130 Panama samples. Both \DIFaddbegin \DIFadd{the GLOPNET and Panama }\DIFaddend datasets include
additional traits that we used to interpret model results, but which
were not used to fit models. These traits include nitrogen and
phosphorus content per leaf unit area (\emph{N}\textsubscript{area} and
\emph{P}\textsubscript{area}; g m\textsuperscript{-2}) \DIFdelbegin \DIFdel{in both datasets, leaf
habitin GLOPNET
}\DIFdelend \DIFaddbegin \DIFadd{and leaf
habit}\DIFaddend (deciduous or evergreen), and cellulose content per unit area
(\emph{CL}\textsubscript{area}; g m\textsuperscript{-2}) in \DIFdelbegin \DIFdel{Panama}\DIFdelend \DIFaddbegin \DIFadd{the Panama
dataset}\DIFaddend .

\hypertarget{model-estimation-and-evaluation}{%
\subsection{Model estimation and
evaluation}\label{model-estimation-and-evaluation}}

We modeled \emph{A}\textsubscript{area} and \emph{R}\textsubscript{area}
using Eqs. \ref{eq:Aarea} and \ref{eq:Rarea}, respectively, for both
GLOPNET and Panama. To model LL for GLOPNET, we used Eq. \ref{eq:LL_pot}
(no light effects), because GLOPNET does not report canopy position and
thus primarily represents interspecific variation (see above). To model
LL for Panama, we used Eq. \ref{eq:LL_opt} (light effects model), which
was motivated by the negative intraspecific LL-LMA relationship observed
in Panama (\protect\DIFdelbegin %DIFDELCMD < \hyperlink{ref-Xu2017}{Xu et al. 2017}%%%
\DIFdel{,
}\DIFdelend \DIFaddbegin \hyperlink{ref-Osnas2018}{Osnas et al., 2018}\DIFadd{;
}\DIFaddend \protect\DIFdelbegin %DIFDELCMD < \hyperlink{ref-Osnas2018}{Osnas et al. 2018}%%%
\DIFdelend \DIFaddbegin \hyperlink{ref-Xu2017}{Xu et al., 2017}\DIFaddend ) and elsewhere
(\protect\DIFdelbegin %DIFDELCMD < \hyperlink{ref-Lusk2008}{Lusk et al. 2008}%%%
\DIFdel{,
}\DIFdelend \DIFaddbegin \hyperlink{ref-Lusk2008}{Lusk et al., 2008}\DIFadd{;
}\DIFaddend \protect\DIFdelbegin %DIFDELCMD < \hyperlink{ref-Russo2016}{Russo and Kitajima 2016}%%%
\DIFdelend \DIFaddbegin \hyperlink{ref-Russo2016}{Russo \& Kitajima, 2016}\DIFaddend ).

Posterior distributions of all parameters were estimated using the
Hamiltonian Monte Carlo algorithm (HMC) implemented in Stan
(\protect\DIFdelbegin %DIFDELCMD < \hyperlink{ref-Carpenter2017}{Carpenter et al. 2017}%%%
\DIFdelend \DIFaddbegin \hyperlink{ref-Carpenter2017}{Carpenter et al., 2017}\DIFaddend ). We used
non-informative or weakly informative prior distributions
(\protect\DIFdelbegin %DIFDELCMD < \hyperlink{ref-Lemoine2019}{Lemoine 2019}%%%
\DIFdelend \DIFaddbegin \hyperlink{ref-Lemoine2019}{Lemoine, 2019}\DIFaddend ). Prior
distributions for the latent variables \emph{f}\textsubscript{i} (which
are used to partition LMA into \DIFdelbegin \DIFdel{LMAp }\DIFdelend \DIFaddbegin \DIFadd{LMAm }\DIFaddend and LMAs according to Eqs. 4-6 were
non-informative uniform(0, 1) distributions, (i.e., LMA was partitioned
based on patterns in the data). See Appendix S1 for more detail. The
Stan code use to fit models is available from Github at:
\href{https://github.com/mattocci27/xxx}{https://github.com/mattocci27/XXX}.
Convergence of the posterior distribution was assessed with the
Gelman-Rubin statistic with a convergence threshold of 1.05 for all
parameters (\protect\DIFdelbegin %DIFDELCMD < \hyperlink{ref-Gelman2013}{Gelman et al. 2013}%%%
\DIFdelend \DIFaddbegin \hyperlink{ref-Gelman2013}{Gelman et al., 2013}\DIFaddend ).

Because our statistical approach includes many parameters (one latent
variable \emph{f\textsubscript{i}} to partition LMA into \DIFdelbegin \DIFdel{LMAp }\DIFdelend \DIFaddbegin \DIFadd{LMAm }\DIFaddend and LMAs
for each leaf sample), we implemented tests with randomized data
(\emph{n} = 10) to assess potential overfitting. We generated randomized
datasets by randomizing all the trait values (LMA,
\emph{A}\textsubscript{area}, \emph{R}\textsubscript{area} and LL)
across species. Thus, the randomized datasets had zero expected
covariance among traits. Model results obtained from the randomized
datasets did not convergent\DIFdelbegin \DIFdel{or }\DIFdelend \DIFaddbegin \DIFadd{, }\DIFaddend showed divergent transitions \DIFaddbegin \DIFadd{or converged
but did not show significant patterns in the scaling parameters
}\DIFaddend (Appendix S2). Divergent transitions suggest that estimates are based on
the incomplete exploration of the simulated Hamiltonian trajectory and
not trusted (\protect\DIFdelbegin %DIFDELCMD < \hyperlink{ref-Betancourt2016}{Betancourt 2016}%%%
\DIFdel{).
(in a
simple sentence: Model did not work). }\DIFdelend \DIFaddbegin \hyperlink{ref-Betancourt2016}{Betancourt, 2016}\DIFadd{).
}\DIFaddend In contrast, models fit to observed data converged without divergent
transitions and produced significant results for both GLOPNET and Panama
(see Results). Thus, the tests with randomized data indicate that our
model is not inherently prone to overfitting or to producing patterns
from noise. We therefore assume that estimates of \DIFdelbegin \DIFdel{LMAp }\DIFdelend \DIFaddbegin \DIFadd{LMAm }\DIFaddend and LMAs obtained
from the GLOPNET and Panama datasets reflect meaningful patterns in the
observations and allow for meaningful tests of our hypotheses.

Instead of using exact leave-one-out cross-validation (LOO), which is a
robust way to compare models with different numbers of parameters but
computationally impractical, we compared the performance of different
models (Table 1) using Pareto-smoothed importance sampling leave-one-out
cross-validation (PSIS-LOO; Vehtari et al.
(\protect\hyperlink{ref-Vehtari2014}{2014}); Vehtari et al.
(\protect\hyperlink{ref-Vehtari2017}{2017})), an approximation to LOO.
The PSIS-LOO gives the accurate and reliable results
(\protect\DIFdelbegin %DIFDELCMD < \hyperlink{ref-Vehtari2014}{Vehtari et al. 2014}%%%
\DIFdelend \DIFaddbegin \hyperlink{ref-Vehtari2014}{Vehtari et al., 2014}\DIFaddend ,
\protect\hyperlink{ref-Vehtari2017}{2017}). Similar to other information
criteria (e.g.~Akaike Information Criterion (AIC); Burnham \DIFdelbegin \DIFdel{and }\DIFdelend \DIFaddbegin \DIFadd{\& }\DIFaddend Anderson
(\protect\hyperlink{ref-Burnham2002}{2002})), a better model in terms of
predictive accuracy shows a smaller LOO Information Criterion (LOOIC).

\hypertarget{partitioning-lma-variance-into-metabolic-and-structural-components}{%
\subsection{Partitioning LMA variance into metabolic and structural
components}\label{partitioning-lma-variance-into-metabolic-and-structural-components}}

We \DIFdelbegin \DIFdel{used the following identity to estimate the relative contributions of
LMAp and LMAs to LMA variance, where again LMA = LMAp + LMAs:
}%DIFDELCMD < 

%DIFDELCMD < %%%
\begin{displaymath}\DIFdel{\protect\hypertarget{eq-var}{}{
\mathrm{Var}(Y = X1 + X2) = \mathrm{Cov}(Y, X1+X2) = \mathrm{Cov}(Y,X1) + \mathrm{Cov}(Y,X2)
}%DIFDELCMD < \label{eq-var}%%%
}\end{displaymath}%DIFAUXCMD
%DIFDELCMD < 

%DIFDELCMD < %%%
\DIFdel{where Var(\(\cdot\)) is variance and Cov(\(\cdot\)) is covariance. Thus,
the fractions of total LMA variance due to variance in LMAp and LMAs are
Cov(LMA, LMAp)/Var(LMA) and Cov(LMA, LMAs)/Var(LMA), respectively.
}%DIFDELCMD < 

%DIFDELCMD < %%%
\DIFdel{We also }\DIFdelend partitioned variance of the posterior means of \DIFdelbegin \DIFdel{LMAp }\DIFdelend \DIFaddbegin \DIFadd{LMAm }\DIFaddend and LMAs into
between group (evergreen vs.~deciduous for GLOPNET \DIFaddbegin \DIFadd{and Panama}\DIFaddend , and dry
vs.~wet sites and sun vs.~shade leaves for Panama) and within group
components using ANOVA.

\hypertarget{understanding-relationships-between-photosynthetic-capacity-and-lma}{%
\subsection{Understanding relationships between photosynthetic capacity
and
LMA}\label{understanding-relationships-between-photosynthetic-capacity-and-lma}}

We applied our \DIFdelbegin \DIFdel{LMAp}\DIFdelend \DIFaddbegin \DIFadd{LMAm}\DIFaddend +LMAs model to simulated data to better understand
relationships between photosynthetic capacity
(\emph{A}\textsubscript{max}) and LMA (Eq. \ref{eq:Aarea}). The causes
and interpretation of \emph{A}\textsubscript{max} vs.~LMA relationships
are controversial (\protect\DIFdelbegin %DIFDELCMD < \hyperlink{ref-Westoby2013}{Westoby et al.
%DIFDELCMD < 2013}%%%
\DIFdelend \DIFaddbegin \hyperlink{ref-Westoby2013}{Westoby et al.,
2013}\DIFaddend ). Although \emph{A}\textsubscript{max} is often mass-normalized
(e.g., \DIFaddbegin \DIFadd{I. J. }\DIFaddend Wright et al. (\protect\hyperlink{ref-Wright2005}{2005});
Shipley et al. (\protect\hyperlink{ref-Shipley2006}{2006}); Blonder et
al. (\protect\hyperlink{ref-Blonder2011}{2011})), Lloyd et al.
(\protect\hyperlink{ref-Lloyd2013}{2013}) argued that photosynthesis is
an area-based process, and therefore \emph{A}\textsubscript{max} should
be area-normalized when exploring trait relationships. Consistent with
this argument, Osnas et al. (\protect\hyperlink{ref-Osnas2013}{2013})
showed that across global species, variation in
\emph{A}\textsubscript{max} is strongly dependent on leaf area, but only
weakly dependent on leaf mass (after controlling for interspecific
variation in leaf area). Osnas et al.
(\protect\hyperlink{ref-Osnas2018}{2018}) further showed that this
result (strong area-dependence but weak mass-dependence of A across
species) is most apparent within species groups in which LL is highly
variable (`Area-dependence and `mass-dependence' refer to the degree to
which whole-leaf trait values depend on leaf area and mass,
respectively; see Osnas et al.
(\protect\hyperlink{ref-Osnas2013}{2013}), Osnas et al.
(\protect\hyperlink{ref-Osnas2018}{2018})).

To better understand the factors affecting \emph{A}\textsubscript{max}
vs.~LMA relationships, we created simulated datasets in which we varied
the following factors: the sensitivity of \emph{A}\textsubscript{area}
to variation in \DIFdelbegin \DIFdel{LMAp (parameter \(\alpha_p\) }\DIFdelend \DIFaddbegin \DIFadd{LMAm (parameter \(\alpha_m\) }\DIFaddend in Eq. \ref{eq:Aarea}); the
sensitivity of \emph{A}\textsubscript{area} to LMAs (parameter
\(\alpha_s\) in Eq. \ref{eq:Aarea}), and the total LMA variance due to
variance in LMAs. For each simulated dataset, we quantified the
\emph{A}\textsubscript{max} vs.~LMA relationship following Osnas et al.
(\protect\hyperlink{ref-Osnas2018}{2018}):

\begin{equation}\protect\hypertarget{eq-mass}{}{
A_{\mathrm{area} \, i} = a (LMA_i)^{b}\epsilon_i
}\label{eq-mass}\end{equation}

where LMA is the sum of \DIFdelbegin \DIFdel{LMAp }\DIFdelend \DIFaddbegin \DIFadd{LMAm }\DIFaddend and LMAs (Eq.~\ref{eq-LMA}), \emph{a} is a
fitted constant, and \emph{b} is an index of mass-dependence as
illustrated by the following cases
(\protect\DIFdelbegin %DIFDELCMD < \hyperlink{ref-Osnas2013}{Osnas et al. 2013}%%%
\DIFdelend \DIFaddbegin \hyperlink{ref-Osnas2018}{Osnas et al., 2018}\DIFaddend ,
\protect\DIFdelbegin %DIFDELCMD < \hyperlink{ref-Osnas2018}{2018}%%%
\DIFdelend \DIFaddbegin \hyperlink{ref-Osnas2013}{2013}\DIFaddend ): if \emph{b} = 0, then
\emph{A}\textsubscript{area} is independent of LMA, which implies that
whole-leaf \emph{A}\textsubscript{max} is proportional to leaf area;
conversely, if \emph{b} = 1, then \emph{A}\textsubscript{area} is
proportional to LMA, which implies that whole-leaf
\emph{A}\textsubscript{max} is proportional to leaf mass. Intermediate
cases (0 \textless{} \emph{b} \textless{} 1), as well as more extreme
cases (\emph{b} \(\leq\) 0 or \emph{b} \(\geq\) 1), are also possible.
Note that although Eq.~\ref{eq-mass} uses area-normalized
\emph{A}\textsubscript{max}, equivalent results are obtained from
mass-normalized \emph{A}\textsubscript{max}
(\protect\DIFdelbegin %DIFDELCMD < \hyperlink{ref-Osnas2018}{Osnas et al. 2018}%%%
\DIFdelend \DIFaddbegin \hyperlink{ref-Osnas2018}{Osnas et al., 2018}\DIFaddend ).

We prepared \DIFdelbegin \DIFdel{LMAp }\DIFdelend \DIFaddbegin \DIFadd{LMAm }\DIFaddend and LMAs values based on the empirical estimates of
these values from our GLOPNET and Panama analyses, with a sample size of
100. For simulations of GLOPNET and Panama sun leaves, \DIFdelbegin \DIFdel{LMAp }\DIFdelend \DIFaddbegin \DIFadd{LMAm }\DIFaddend values were
generated by a random draw from a lognormal distribution with the
empirical estimates of mean log(\DIFdelbegin \DIFdel{LMAp}\DIFdelend \DIFaddbegin \DIFadd{LMAm}\DIFaddend ) and standard deviations of
log(\DIFdelbegin \DIFdel{LMAp}\DIFdelend \DIFaddbegin \DIFadd{LMAm}\DIFaddend ). LMAs values were generated by a random draw from a lognormal
distribution with the empirical estimates of mean log(LMAs), while with
changing the standard deviation of log(LMAs) between log(1.01) and
log(10).

Because there was a negative covariance between \DIFdelbegin \DIFdel{LMAp }\DIFdelend \DIFaddbegin \DIFadd{LMAm }\DIFaddend and LMAs in the
Panama shade leaves dataset (Fig. S\DIFdelbegin \DIFdel{\ref{fig-LMAp_LMAs}}\DIFdelend \DIFaddbegin \DIFadd{\ref{fig-LMAm_LMAs}}\DIFaddend ), we used a
multivariate normal distribution to generated \DIFdelbegin \DIFdel{LMAp }\DIFdelend \DIFaddbegin \DIFadd{LMAm }\DIFaddend and LMAs for the
simulated Panama shade leaves dataset. Similar to the GLOPNET and Panama
sun leaves datasets, we fixed the empirical estimates of mean log(\DIFdelbegin \DIFdel{LMAp}\DIFdelend \DIFaddbegin \DIFadd{LMAm}\DIFaddend )
and log(LMAs), and standard deviation of log(\DIFdelbegin \DIFdel{LMAp}\DIFdelend \DIFaddbegin \DIFadd{LMAm}\DIFaddend ), while with changing
the standard deviation of log(LMAs) between log(1.01) and log(10). The
correlation coefficient was set to -0.4.

For simulations of Panama shade leaves, \DIFdelbegin \DIFdel{LMAp }\DIFdelend \DIFaddbegin \DIFadd{LMAm }\DIFaddend and LMAs values were
generated by a random draw from a multivariate normal distribution with
the empirical estimates of mean log(\DIFdelbegin \DIFdel{LMAp}\DIFdelend \DIFaddbegin \DIFadd{LMAm}\DIFaddend ) and standard deviations of
log(\DIFdelbegin \DIFdel{LMAp}\DIFdelend \DIFaddbegin \DIFadd{LMAm}\DIFaddend ), the empirical estimates of mean log(LMAs), and the estimated
correlation value with log(\DIFdelbegin \DIFdel{LMAp}\DIFdelend \DIFaddbegin \DIFadd{LMAm}\DIFaddend ) and log(LMAs), while with changing the
standard deviation of log(LMAs) between log(1.01) and log(10).
\emph{A}\textsubscript{area} was then generated based on \DIFdelbegin \DIFdel{LMAp}\DIFdelend \DIFaddbegin \DIFadd{LMAm}\DIFaddend , LMAs and
estimated \DIFdelbegin \DIFdel{\(\alpha_p\) }\DIFdelend \DIFaddbegin \DIFadd{\(\alpha_m\) }\DIFaddend and \(\alpha_s\) values (Eq.~\ref{eq-mass}) for
each dataset. Using the above values, we calculated \emph{b} for
GLOPNET, Panama sun leaves and Panama shade leaves. We repeated these
steps 1000 times.
\DIFdelbegin \DIFdel{We }\DIFdelend \DIFaddbegin 

\DIFadd{Finally, we }\DIFaddend compared the simulated results with the fractions of total
LMA variance due to variance in \DIFdelbegin \DIFdel{LMAp }\DIFdelend \DIFaddbegin \DIFadd{LMAm }\DIFaddend and LMAs of the empirical data\DIFdelbegin \DIFdel{(see above) .
}\DIFdelend \DIFaddbegin \DIFadd{. We
used the following identity to estimate the relative contributions of
LMAm and LMAs to LMA variance, where again LMA = LMAm + LMAs:
}\DIFaddend 

\DIFaddbegin \begin{equation}\DIFadd{\protect\hypertarget{eq-var}{}{
\mathrm{Var}(Y = X1 + X2) = \mathrm{Cov}(Y, X1+X2) = \mathrm{Cov}(Y,X1) + \mathrm{Cov}(Y,X2)
}\label{eq-var}}\end{equation}

\DIFadd{where Var(\(\cdot\)) is variance and Cov(\(\cdot\)) is covariance. Thus,
the fractions of total LMA variance due to variance in LMAm and LMAs are
Cov(LMA, LMAm)/Var(LMA) and Cov(LMA, LMAs)/Var(LMA), respectively.
}

\DIFaddend \hypertarget{results}{%
\section{Results}\label{results}}

\textbf{1. Decomposing LMA into metabolic and structural components
leads to improved predictions of \emph{A}\textsubscript{area},
\emph{R}\textsubscript{area} and LL.} For both the GLOPNET global
datasets and the Panama datasets, two-dimensional model (\DIFdelbegin \DIFdel{LMAp }\DIFdelend \DIFaddbegin \DIFadd{LMAm }\DIFaddend and LMAs)
performed better in cross validation than one-dimensional (LMA) models
(Table 1), suggesting that decomposing LMA into metabolic and structural
components leads to improved predictions of
\emph{A}\textsubscript{area}, \emph{R}\textsubscript{area} and LL.
Additionally, the effects of \DIFdelbegin \DIFdel{LMAp }\DIFdelend \DIFaddbegin \DIFadd{LMAm }\DIFaddend on LL for the GLOPNET and the Panama
dataset and the effect of LMAs on the Panama dataset were not selected
in the best models (Table 1).

For the GLOPNET global dataset, \emph{A}\textsubscript{area} had a
positive correlation with \DIFdelbegin \DIFdel{LMAp}\DIFdelend \DIFaddbegin \DIFadd{LMAm}\DIFaddend , a negative correlation with LMAs, and a
non-significant correlation with total LMA (Fig.~\ref{fig-GLplt}a-c;
Table 2). \emph{R}\textsubscript{area} in GLOPNET also had a positive
correlation with \DIFdelbegin \DIFdel{LMAp}\DIFdelend \DIFaddbegin \DIFadd{LMAm}\DIFaddend , which was stronger than the correlation between
\emph{R}\textsubscript{area} and either LMA or LMAs
(Fig.~\ref{fig-GLplt}d-f). Finally, LL in GLOPNET had a strong positive
correlation with LMAs, which was stronger than the correlation between
LL and LMA (Fig.~\ref{fig-GLplt}g-i).

For the Panama dataset, including the effect of light (sun vs.~shade)
imporved model performance (Table 1). All Panama results we report are
for models including light effects unless stated otherwise.
\emph{A}\textsubscript{area} and \emph{R}\textsubscript{area} had
stronger and more positive correlations with \DIFdelbegin \DIFdel{LMAp }\DIFdelend \DIFaddbegin \DIFadd{LMAm }\DIFaddend than with LMA
(Fig.~\ref{fig-PAplt}a-f and Table 2). LL was not significantly
correlated with LMA when all leaves were combined, but was strongly
correlated with LMA for shade leaves at the dry site
(Fig.~\ref{fig-PAplt}g). There was a positive correlation between LL and
LMAs (Fig.~\ref{fig-PAplt}i), and this relationship was strengthened by
accounting the effects of LL (Fig.~\ref{fig-LLplt}).

\textbf{2. Nearly all leaf dark respiration is associated with metabolic
leaf tissue mass.} According to our model results, metabolic leaf mass
(\DIFdelbegin \DIFdel{LMAp}\DIFdelend \DIFaddbegin \DIFadd{LMAm}\DIFaddend ) accounted for nearly all leaf dark respiration; i.e., estimated
dark respiration rate per-unit structural mass (\(\gamma_s\)) was close
to zero in analyses of both GLOPNET and Panama data (Table 1). Thus,
although building costs are likely similar for different leaf chemical
components and tissues (\protect\DIFdelbegin %DIFDELCMD < \hyperlink{ref-Williams1989}{Williams et
%DIFDELCMD < al. 1989}%%%
\DIFdel{, }\DIFdelend \DIFaddbegin \hyperlink{ref-Villar2001}{Villar \&
Merino, 2001}\DIFadd{; }\DIFaddend \protect\DIFdelbegin %DIFDELCMD < \hyperlink{ref-Villar2001}{Villar and Merino 2001}%%%
\DIFdelend \DIFaddbegin \hyperlink{ref-Williams1989}{Williams et al.,
1989}\DIFaddend ), our results suggest that leaf mass associated with
photosynthetic function accounts for nearly all leaf maintenance
respiration.

\textbf{3. Evergreen leaves have greater LMAs than deciduous leaves, and
sun leaves have both greater \DIFdelbegin \DIFdel{LMAp }\DIFdelend \DIFaddbegin \DIFadd{LMAm }\DIFaddend and LMAs than shade leaves.} Evergreen
vs.~decidous explains the \DIFdelbegin \DIFdel{majority of variance }\DIFdelend \DIFaddbegin \DIFadd{large variance (39.0\%) }\DIFaddend in LMAs but not \DIFdelbegin \DIFdel{LMAp }\DIFdelend \DIFaddbegin \DIFadd{LMAm
}\DIFaddend in GLOPNET (Fig.~\ref{fig-vpart}). Thus, the higher total LMA in
evergreen leaves in GLOPNET (Fig.~\ref{fig-boxplt_de}) was primarily due
to differences in LMAs. In the Panama dataset, light (sun vs.~shade)
explains the majority of variance in \DIFdelbegin \DIFdel{LMAp }\DIFdelend \DIFaddbegin \DIFadd{LMAm (75.6\%) }\DIFaddend but not LMAs \DIFaddbegin \DIFadd{(18.3\%)
}\DIFaddend , and site (wet/dry) explains \DIFdelbegin \DIFdel{33}\DIFdelend \DIFaddbegin \DIFadd{25.9}\DIFaddend \% of variance in LMAs
(Fig.~\ref{fig-vpart}). Sun leaves have both greater \DIFdelbegin \DIFdel{LMAp }\DIFdelend \DIFaddbegin \DIFadd{LMAm }\DIFaddend and LMAs than
shade leaves, but the differences in \DIFdelbegin \DIFdel{LMAp }\DIFdelend \DIFaddbegin \DIFadd{LMAm }\DIFaddend were more apparent
(Fig.~\ref{fig-boxplt_pa}). Thus, variation in \DIFdelbegin \DIFdel{LMAp }\DIFdelend \DIFaddbegin \DIFadd{LMAm }\DIFaddend plays a more
important role within species than among species. Evergreen leaves have
higher fraction of LMAs compared to deciduous leaves both in the Panama
and the GLOPNET dataset (Fig. S\ref{fig-box_frac}).

\textbf{4. LMA variance components determine the area-
vs.~mass-dependence of leaf photosynthetic capacity
(\emph{A}\textsubscript{max}).} Analysis of simulated data showed that
the percent of LMA variation due to \DIFdelbegin \DIFdel{LMAp }\DIFdelend \DIFaddbegin \DIFadd{LMAm }\DIFaddend vs.~LMAs variation determined
the area- vs.~mass-dependence of \emph{A}\textsubscript{max} (i.e., the
degree to which whole-leaf \emph{A}\textsubscript{max} depends on leaf
area or mass; Osnas et al. (\protect\hyperlink{ref-Osnas2013}{2013});
Osnas et al. (\protect\hyperlink{ref-Osnas2018}{2018})). Generally,
mass-dependence of \emph{A}\textsubscript{max} decreases with LMAs
variance (Fig.~\ref{fig-massplt}). Other factors, including the
senstivity of \emph{A}\textsubscript{max} to variation in \DIFdelbegin \DIFdel{LMAp }\DIFdelend \DIFaddbegin \DIFadd{LMAm }\DIFaddend and LMAs
(\DIFdelbegin \DIFdel{\(\alpha_p\) }\DIFdelend \DIFaddbegin \DIFadd{\(\alpha_m\) }\DIFaddend and \(\alpha_s\)) and covariance between \DIFdelbegin \DIFdel{LMAp }\DIFdelend \DIFaddbegin \DIFadd{LMAm }\DIFaddend and LMAs,
also affect the relationship between mass-dependence of
\emph{A}\textsubscript{max} and LMAs variance (Fig.~\ref{fig-massplt},
Fig. S\ref{fig-mass_prop_sim}-\ref{fig-mass_prop_comp}). Although
relative variance of LMAs was intermediate in GLOPNET (51.6\%),
interspecific variation in \emph{A}\textsubscript{max} among leaves in
GLOPNET tended to be area-dependent because of their lower \DIFdelbegin \DIFdel{\(\alpha_p\)
}\DIFdelend \DIFaddbegin \DIFadd{\(\alpha_m\)
}\DIFaddend and negative \(\alpha_s\). Interspecific variation in
\emph{A}\textsubscript{max} in sun and shade leaves in the Panama data
was area-dependent because of their large variance in LMAs
(Fig.~\ref{fig-massplt}).

\textbf{5. Nitrogen and phosphorus per-unit leaf area are strongly
correlated with \DIFdelbegin \DIFdel{LMAp}\DIFdelend \DIFaddbegin \DIFadd{LMAm}\DIFaddend , and cellulose per-unit leaf area is strongly
correlated with LMAs.} In the GLOPNET dataset,
\emph{N}\textsubscript{area} and \emph{P}\textsubscript{area} had strong
positive correlations with \DIFdelbegin \DIFdel{LMAp}\DIFdelend \DIFaddbegin \DIFadd{LMAm}\DIFaddend , but only weak correlations with LMAs
(Fig. S\DIFdelbegin \DIFdel{\ref{fig-glnp}}\DIFdelend \DIFaddbegin \DIFadd{\ref{fig-gl_point_np2}}\DIFaddend ). Similarly, in the Panama dataset,
\emph{N}\textsubscript{area} and \emph{P}\textsubscript{area} had strong
positive correlations with \DIFdelbegin \DIFdel{LMAp}\DIFdelend \DIFaddbegin \DIFadd{LMAm}\DIFaddend , but were not correlated with LMAs
(Fig.~\ref{fig-PA-NPC}). In contrast, cellulose per-unit leaf area
(\emph{CL}\textsubscript{area}), which was available for the Panama
dataset but not for GLOPNET, had a strong positive correlation with
LMAs, and a weak positive correlation with \DIFdelbegin \DIFdel{LMAp }\DIFdelend \DIFaddbegin \DIFadd{LMAm }\DIFaddend (Fig.~\ref{fig-PA-NPC}).
\emph{CL}\textsubscript{area} was more strongly correlated with LMA than
with \DIFdelbegin \DIFdel{LMAp }\DIFdelend \DIFaddbegin \DIFadd{LMAm }\DIFaddend or LMAs, but sun and shade leaves aligned along a common
\emph{CL}\textsubscript{area}-LMAs relationship, as opposed to being
offset for LMA and \DIFdelbegin \DIFdel{LMAp }\DIFdelend \DIFaddbegin \DIFadd{LMAm }\DIFaddend (Fig.~\ref{fig-PA-NPC}g-i).

\hypertarget{discussion}{%
\section{Discussion}\label{discussion}}

Our analyses demonstrate that decomposing LMA variation into separate
metabolic and structural components (\DIFdelbegin \DIFdel{LMAp }\DIFdelend \DIFaddbegin \DIFadd{LMAm }\DIFaddend and LMAs, respectively) leads
to improved predictions of photosynthetic capacity
(\emph{A}\textsubscript{max}), dark respiration rate
(\emph{R}\textsubscript{dark}), and leaf lifespan (LL), as well as clear
relationships with traits used for independent model evaluation
(nitrogen, phosphorus, and cellulose concentrations). The model with
\DIFdelbegin \DIFdel{LMAp }\DIFdelend \DIFaddbegin \DIFadd{LMAm }\DIFaddend and LMAs showed better predictive accuracy than model with LMA
alone for both intraspecific variation in relation to light and
interspecific variation. Below, we elaborate on the insights gained from
our analysis and the implications of our results for the representation
of leaf functional diversity in global ecosystem models.

Decomposing LMA into \DIFdelbegin \DIFdel{LMAp }\DIFdelend \DIFaddbegin \DIFadd{LMAm }\DIFaddend and LMAs provides insights into why
interspecific variation in leaf traits related to photosynthesis and
metabolism are primarily area-dependent (i.e., primarily independent of
LMA when expressed per-unit area rather than mass-dependent
(\protect\DIFdelbegin %DIFDELCMD < \hyperlink{ref-Osnas2013}{Osnas et al. 2013}%%%
\DIFdelend \DIFaddbegin \hyperlink{ref-Osnas2018}{Osnas et al., 2018}\DIFaddend ,
\protect\DIFdelbegin %DIFDELCMD < \hyperlink{ref-Osnas2018}{2018}%%%
\DIFdelend \DIFaddbegin \hyperlink{ref-Osnas2013}{2013}\DIFaddend )). We expected that leaf
metabolic traits tend to be area-depend when variance in \DIFdelbegin \DIFdel{LMAp }\DIFdelend \DIFaddbegin \DIFadd{LMAm }\DIFaddend is larger
than variance in LMAs. However, our results suggest that the scaling
slopes (\DIFdelbegin \DIFdel{\(\alpha_p\) }\DIFdelend \DIFaddbegin \DIFadd{\(\alpha_m\) }\DIFaddend and \(\alpha_s\)) between \DIFdelbegin \DIFdel{LMAp}\DIFdelend \DIFaddbegin \DIFadd{LMAm}\DIFaddend , LMAs and
\emph{A}\textsubscript{area} also affect trait area proportionality.
Leaf metabolic traits tend to be area-dependent when (i) variance in
\DIFdelbegin \DIFdel{LMAp }\DIFdelend \DIFaddbegin \DIFadd{LMAm }\DIFaddend is small or similar to variance in LMAs, (ii) the effects of \DIFdelbegin \DIFdel{LMAp
}\DIFdelend \DIFaddbegin \DIFadd{LMAm
}\DIFaddend and LMAs on \emph{A}\textsubscript{area} (\DIFdelbegin \DIFdel{\(\alpha_p\) }\DIFdelend \DIFaddbegin \DIFadd{\(\alpha_m\) }\DIFaddend and \(\alpha_s\))
are small, (iii) or combinations of those (Fig. Fig.~\ref{fig-massplt}).
Consistent with this explanation, the assemblage we examined where LMAs
counted for the highest fraction of total LMA variation (\DIFdelbegin \DIFdel{-80}\DIFdelend \DIFaddbegin \DIFadd{93.4}\DIFaddend \% for
Panama shade leaves) is also the assembled with the highest degree of
trait area-dependence (\protect\DIFdelbegin %DIFDELCMD < \hyperlink{ref-Osnas2018}{Osnas et al.
%DIFDELCMD < 2018}%%%
\DIFdelend \DIFaddbegin \hyperlink{ref-Osnas2018}{Osnas et al.,
2018}\DIFaddend ). Our results also suggest that high degree of trait
area-dependence in GLOPNET is due to the negative effect of LMAs on
\emph{A}\textsubscript{area}.

The scaling slope \DIFdelbegin \DIFdel{\(\alpha_p\) }\DIFdelend \DIFaddbegin \DIFadd{\(\alpha_m\) }\DIFaddend less than 1 (Table 2) suggests a
diminishing return on addition \DIFdelbegin \DIFdel{LMAp }\DIFdelend \DIFaddbegin \DIFadd{LMAm }\DIFaddend (i.e., doubling \DIFdelbegin \DIFdel{LMAp }\DIFdelend \DIFaddbegin \DIFadd{LMAm }\DIFaddend does not simply
result in doubling \emph{A}\textsubscript{max}).

\begin{enumerate}
\def\labelenumi{\arabic{enumi}.}
\tightlist
\item
  High \DIFdelbegin \DIFdel{LMAp }\DIFdelend \DIFaddbegin \DIFadd{LMAm }\DIFaddend might cause lower mesophyll conductance
\item
  High \DIFdelbegin \DIFdel{LMAp }\DIFdelend \DIFaddbegin \DIFadd{LMAm }\DIFaddend (thick leaves) might cause chloroplast self-shading within a
  leaf
\item
  \DIFdelbegin \DIFdel{LMAp }\DIFdelend \DIFaddbegin \DIFadd{LMAm }\DIFaddend = (purely) metabolic active leaf mass + something we couldn't
  estimate in our analysis. All the leaf mass needs to be allocated to
  \DIFdelbegin \DIFdel{LMAp }\DIFdelend \DIFaddbegin \DIFadd{LMAm }\DIFaddend or LMAs in our model. However, variation in non-structural
  carbohydrates in leaves, which is not estimated in our model, is
  relatively large
  (\protect\DIFdelbegin %DIFDELCMD < \hyperlink{ref-Martinez-Vilalta2016}{Martínez-Vilalta et al.
%DIFDELCMD <   2016}%%%
\DIFdelend \DIFaddbegin \hyperlink{ref-Martinez-Vilalta2016}{Martínez-Vilalta et al.,
  2016}\DIFaddend ). Simply partitioning leaf mass into two functions might
  underestimate how much \emph{A}\textsubscript{max} depending on the
  metabolic leaf mass .
\end{enumerate}

Decomposing LMA also provides insights as to why intraspecific patterns
of trait variation differ from those observed across species. In
contrast to interspecific LMA variation, our analysis suggests that \DIFdelbegin \DIFdel{LMAp
}\DIFdelend \DIFaddbegin \DIFadd{LMAm
}\DIFaddend contributes half or more of the intraspecific increase in LMA from shade
to sun (Figs. Fig.~\ref{fig-massplt} and SX). The increase in \DIFdelbegin \DIFdel{LMAp }\DIFdelend \DIFaddbegin \DIFadd{LMAm }\DIFaddend from
shade to sun -- which likely reflects an increase in the size and number
of palisade mesophyll cells with increasing light availability
(\protect\DIFdelbegin %DIFDELCMD < \hyperlink{ref-Onoda2008}{Onoda et al. 2008}%%%
\DIFdel{,
}\DIFdelend \DIFaddbegin \hyperlink{ref-Onoda2008}{Onoda et al., 2008}\DIFadd{;
}\DIFaddend \protect\DIFdelbegin %DIFDELCMD < \hyperlink{ref-Terashima2011}{Terashima et al. 2011}%%%
\DIFdelend \DIFaddbegin \hyperlink{ref-Terashima2011}{Terashima et al., 2011}\DIFaddend ) -- is
also associated with an increase in LMAs from shade to sun (Fig.
Fig.~\ref{fig-boxplt_pa}). This positive covariance between \DIFdelbegin \DIFdel{LMAp }\DIFdelend \DIFaddbegin \DIFadd{LMAm }\DIFaddend and
LMAs within species means that per-area values of \DIFdelbegin \DIFdel{LMAp-proportional
}\DIFdelend \DIFaddbegin \DIFadd{LMAm-proportional
}\DIFaddend traits (e.g., \emph{A}\textsubscript{area}) have a strong, positive
relationship with total LMA, which implies trait mass-proportionality
(\protect\DIFdelbegin %DIFDELCMD < \hyperlink{ref-Osnas2018}{Osnas et al. 2018}%%%
\DIFdelend \DIFaddbegin \hyperlink{ref-Osnas2018}{Osnas et al., 2018}\DIFaddend ).

The improved predictions and understanding provided by decomposing LMA
into photosynthetic and structural components challenge the view that
leaf functional diversity can be accurately represented by a single leaf
economics spectrum (LES) axis (\protect\DIFdelbegin %DIFDELCMD < \hyperlink{ref-Wright2004a}{Wright et al. 2004}%%%
\DIFdelend \DIFaddbegin \hyperlink{ref-Wright2004a}{I. J.
Wright et al., 2004}\DIFaddend ). Lloyd et al.
(\protect\hyperlink{ref-Lloyd2013}{2013}) argued that the apparent
dominance of a single LES axis is an artifact of expressing
area-dependent leaf traits on a per-mass basis, and Osnas et al.
(\protect\hyperlink{ref-Osnas2013}{2013}) demonstrated that across the
global flora, traits related to photosynthesis and metabolism are indeed
area-dependent. Intraspecific patterns in trait variation, which
contrast with interspecific patterns, pose additional challenges for a
one-dimensional view of leaf functional diversity. Our analysis shows
that considering two primary axes of leaf trait variation
(photosynthesis and structure) provides improved quantitative
predictions and insights compared to LES of a single dominant axis. Our
results point to a simple two-dimensional framework for representing
leaf functional diversity in global ecosystem models: a \DIFdelbegin \DIFdel{LMAp }\DIFdelend \DIFaddbegin \DIFadd{LMAm }\DIFaddend axis that
determines \emph{A}\textsubscript{max}, accounts for nearly all
\emph{R}\textsubscript{dark} (see rp and rs estimates in Table S1) and a
LMAs axis that determines potential LL through its effects on leaf
toughness (\protect\DIFdelbegin %DIFDELCMD < \hyperlink{ref-Kleyer2012}{Kleyer et al. 2012}%%%
\DIFdelend \DIFaddbegin \hyperlink{ref-Kleyer2012}{Kleyer et al., 2012}\DIFaddend ). In
the datasets we analyzed (the global flora and tropical trees in
Panama), these two axes are only weakly correlated with each other (Fig.
S\DIFdelbegin \DIFdel{\ref{fig-LMAp_LMAs}}\DIFdelend \DIFaddbegin \DIFadd{\ref{fig-LMAm_LMAs}}\DIFaddend ), which suggests that trait-based approaches to
global ecosystem modeling (\protect\DIFdelbegin %DIFDELCMD < \hyperlink{ref-Scheiter2013}{Scheiter
%DIFDELCMD < et al. 2013}%%%
\DIFdel{, }\DIFdelend \DIFaddbegin \hyperlink{ref-Scheiter2013}{Scheiter
et al., 2013}\DIFadd{; }\DIFaddend \protect\DIFdelbegin %DIFDELCMD < \hyperlink{ref-Wullschleger2014}{Wullschleger et
%DIFDELCMD < al. 2014}%%%
\DIFdelend \DIFaddbegin \hyperlink{ref-Wullschleger2014}{Wullschleger et
al., 2014}\DIFaddend ) could consider these as independent axes. For example, to
simulate individual competing trees which from diverse community of
growth strategies (\protect\DIFdelbegin %DIFDELCMD < \hyperlink{ref-Sakschewski2015}{Sakschewski
%DIFDELCMD < et al. 2015}%%%
\DIFdel{, }\DIFdelend \DIFaddbegin \hyperlink{ref-Sakschewski2015}{Sakschewski
et al., 2015}\DIFadd{; }\DIFaddend \protect\DIFdelbegin %DIFDELCMD < \hyperlink{ref-Sakschewski2016}{Sakschewski et al.
%DIFDELCMD < 2016}%%%
\DIFdelend \DIFaddbegin \hyperlink{ref-Sakschewski2016}{Sakschewski et
al., 2016}\DIFaddend ), it would be possible to draw \DIFdelbegin \DIFdel{LMAp }\DIFdelend \DIFaddbegin \DIFadd{LMAm }\DIFaddend and LMAs values from
distributions with realistic ranges instead of a single LMA
distribution.

Our statistical decomposition of LMA into metabolic and structural
components provides important insights, but additional insights and
accuracy could be gained by a more mechanistic modeling approach. For
example, John et al. (\protect\hyperlink{ref-John2017}{2017}) decomposed
interspecific LMA variation into anatomical components such as the size,
number of layers, and mass density of cells in different leaf tissues.
If such detailed information became available for a large number of
leaves, representing both intra- and interspecific variation, it should
be possible to quantify how these anatomical traits scale up to
leaf-level \emph{A}\textsubscript{max} , \emph{R}\textsubscript{dark},
and LL. A simpler alternative would be to modify our model to account
for variation in cell wall thickness (TCW): for a given cell size,
increasing TCW would lead to an increase in lamina density, cellulose
per volume, toughness, and LL
(\protect\DIFdelbegin %DIFDELCMD < \hyperlink{ref-Kitajima2010}{Kitajima and Poorter 2010}%%%
\DIFdelend \DIFaddbegin \hyperlink{ref-Kitajima2016}{Kitajima et al., 2016}\DIFaddend ,
\protect\DIFdelbegin %DIFDELCMD < \hyperlink{ref-Kitajima2012}{Kitajima et al. 2012}%%%
\DIFdel{,
}\DIFdelend \DIFaddbegin \hyperlink{ref-Kitajima2012}{2012}\DIFadd{;
}\DIFaddend \protect\DIFdelbegin %DIFDELCMD < \hyperlink{ref-Kitajima2016}{2016}%%%
\DIFdelend \DIFaddbegin \hyperlink{ref-Kitajima2010}{Kitajima \& Poorter, 2010}\DIFaddend ), and a
decrease in mesophyll conductance and \emph{A}\textsubscript{max}
(\protect\DIFdelbegin %DIFDELCMD < \hyperlink{ref-Evans2009}{Evans et al. 2009}%%%
\DIFdel{,
}\DIFdelend \DIFaddbegin \hyperlink{ref-Evans2009}{Evans et al., 2009}\DIFadd{;
}\DIFaddend \protect\DIFdelbegin %DIFDELCMD < \hyperlink{ref-Terashima2011}{Terashima et al. 2011}%%%
\DIFdel{,
}\DIFdelend \DIFaddbegin \hyperlink{ref-Onoda2017}{Onoda et al., 2017}\DIFadd{;
}\DIFaddend \protect\DIFdelbegin %DIFDELCMD < \hyperlink{ref-Onoda2017}{Onoda et al. 2017}%%%
\DIFdelend \DIFaddbegin \hyperlink{ref-Terashima2011}{Terashima et al., 2011}\DIFaddend ). Indeed,
our model might capture those patterns. The negative effects of LMAs on
\emph{A}\textsubscript{area} in GLOPNET data indicate that high amount
of photosynthetic proteins is not always translated into high
photosynthetic capacity. High LMAs that may be associated with a
particular anatomical components might lead to lower mesophyll
conductance.

\hypertarget{conclusions}{%
\section{Conclusions}\label{conclusions}}

It is widely recognized that LMA variation is associated with multiple
tissues and functions, including metabolically active mesophyll that
largely determines photosynthetic capacity, as well as structural and
chemical components that contribute primarily to leaf toughness and
defense (\protect\DIFdelbegin %DIFDELCMD < \hyperlink{ref-Roderick1999}{Roderick et al. 1999}%%%
\DIFdel{,
}\DIFdelend \DIFaddbegin \hyperlink{ref-Lusk2010}{Lusk et al., 2010}\DIFadd{;
}\DIFaddend \protect\DIFdelbegin %DIFDELCMD < \hyperlink{ref-Shipley2006}{Shipley et al. 2006}%%%
\DIFdel{,
}\DIFdelend \DIFaddbegin \hyperlink{ref-Roderick1999}{Roderick et al., 1999}\DIFadd{;
}\DIFaddend \protect\DIFdelbegin %DIFDELCMD < \hyperlink{ref-Lusk2010}{Lusk et al. 2010}%%%
\DIFdelend \DIFaddbegin \hyperlink{ref-Shipley2006}{Shipley et al., 2006}\DIFaddend ). It should
not be surprising then, that partitioning LMA into metabolic and
structural components yields enhanced predictions and improved
understanding of patterns of leaf trait variation both within and among
species. Yet for over a decade, the vast literature on leaf traits has
been strongly influenced by the view that leaf trait variation can
usefully be represented by a single dominant axis of LES. Our results
provide quantitative evidence that this one-dimensional view of leaf
trait variation is insufficient, and our model provides a biological
explanation for previous statistical analyses that have demonstrated
area-dependence of leaf traits across species
(\protect\DIFdelbegin %DIFDELCMD < \hyperlink{ref-Osnas2013}{Osnas et al. 2013}%%%
\DIFdelend \DIFaddbegin \hyperlink{ref-Osnas2013}{Osnas et al., 2013}\DIFaddend ), while also
explaining mass-dependence within species. Our results suggest that
small scaling slope between metabolic leaf mass and photosynthetic rate,
and similar variance in \DIFdelbegin \DIFdel{LMAp }\DIFdelend \DIFaddbegin \DIFadd{LMAm }\DIFaddend and in LMAs leads to area-proportionality
of interspecific leaf traits. Thus, strong interspecific correlations
between LMA and mass-normalized photosynthetic capacity (and related
traits, such as respiration rate, and nitrogen and phosphorus
concentrations) are likely driven by mass-normalization itself, rather
than any functional dependence of these photosynthetic and metabolic
traits on LMA (\protect\DIFdelbegin %DIFDELCMD < \hyperlink{ref-Lloyd2013}{Lloyd et al. 2013}%%%
\DIFdel{,
}\DIFdelend \DIFaddbegin \hyperlink{ref-Lloyd2013}{Lloyd et al., 2013}\DIFadd{;
}\DIFaddend \protect\DIFdelbegin %DIFDELCMD < \hyperlink{ref-Osnas2013}{Osnas et al. 2013}%%%
\DIFdelend \DIFaddbegin \hyperlink{ref-Osnas2013}{Osnas et al., 2013}\DIFaddend ). In contrast,
intraspecific variation in LMA is driven by coordinated changes in
structural and metabolic mass components, which explains why
mass-normalized photosynthetic and metabolic traits vary little from sun
to shade within species (\protect\DIFdelbegin %DIFDELCMD < \hyperlink{ref-Aranda2004}{Aranda et
%DIFDELCMD < al. 2004}%%%
\DIFdel{, }\DIFdelend \DIFaddbegin \hyperlink{ref-Aranda2004}{Aranda et
al., 2004}\DIFadd{; }\DIFaddend \protect\DIFdelbegin %DIFDELCMD < \hyperlink{ref-Poorter2006b}{Poorter et al. 2006}%%%
\DIFdel{,
}\DIFdelend \DIFaddbegin \hyperlink{ref-Niinemets2015}{Niinemets et al.,
2015}\DIFadd{; }\DIFaddend \protect\DIFdelbegin %DIFDELCMD < \hyperlink{ref-Niinemets2015}{Niinemets et al. 2015}%%%
\DIFdelend \DIFaddbegin \hyperlink{ref-Poorter2006b}{Poorter et al., 2006}\DIFaddend ).

\hypertarget{acknowledgments}{%
\section{Acknowledgments}\label{acknowledgments}}

We thank Jonathan Dushoff for statistical advice and Martijn Slot
for~helpful~comments that improved the paper. Mirna Samaniego and Milton
Garica provided indispensable assistance in data collection. We thank
the Smithsonian Tropical Research Institute (STRI), the Tropical~Canopy
Biology Program at STRI and the Andrew W. Mellon Foundation for
supporting this work. MK was supported by a Postdoctoral~Fellowship
for~Research Abroad from the Japan Society for the Promotion of Science
and the CAS President's International Fellowship for Young Staff.

\hypertarget{author-contributions}{%
\section{Author contributions}\label{author-contributions}}

MK, JWL and JLDO conceived of the study; KK, SJW and SAVB contributed
data; MK devised the analytical approach and performed analyses; MK and
JWL wrote the first draft of the manuscript, and all authors contributed
to revisions.

\hypertarget{references}{%
\section{References}\label{references}}

\hypertarget{refs}{}
\begin{CSLReferences}{1}{0}
\leavevmode\vadjust pre{\hypertarget{ref-Aranda2004}{}}%
Aranda, I., \DIFdelbegin \DIFdel{F. Pardo, L. Gil, and }\DIFdelend \DIFaddbegin \DIFadd{Pardo, F., Gil, L., \& Pardos, }\DIFaddend J. A. \DIFdelbegin \DIFdel{Pardos. 2004.
}\href{https://doi.org/10.1016/j.actao.2004.01.003}{\DIFdel{Anatomical basis of
the change in leaf mass per area and nitrogen investment with relative
irradiance within the canopy of eight temperate tree species}}%DIFAUXCMD
\DIFdel{.
Acta
Oecologica 25:}\DIFdelend \DIFaddbegin \DIFadd{(2004). Anatomical
basis of the change in leaf mass per area and nitrogen investment with
relative irradiance within the canopy of eight temperate tree species.
}\emph{\DIFadd{Acta Oecologica}}\DIFadd{, }\emph{25}(3)\DIFadd{, }\DIFaddend 187--195.
\DIFaddbegin \url{https://doi.org/10.1016/j.actao.2004.01.003}
\DIFaddend 

\leavevmode\vadjust pre{\hypertarget{ref-Betancourt2016}{}}%
Betancourt, M. \DIFdelbegin \DIFdel{2016. }\href{https://arxiv.org/abs/1604.00695}{\DIFdel{Diagnosing
}%DIFDELCMD < {%%%
\DIFdel{Suboptimal Cotangent Disintegrations}%DIFDELCMD < } %%%
\DIFdel{in }%DIFDELCMD < {%%%
\DIFdel{Hamiltonian Monte Carlo}%DIFDELCMD < }%%%
}%DIFAUXCMD
\DIFdel{. }\DIFdelend \DIFaddbegin \DIFadd{(2016). }\emph{\DIFadd{Diagnosing }{\DIFadd{Suboptimal Cotangent
Disintegrations}} \DIFadd{in }{\DIFadd{Hamiltonian Monte Carlo}}}\DIFadd{.
}\url{https://arxiv.org/abs/1604.00695}
\DIFaddend 

\leavevmode\vadjust pre{\hypertarget{ref-Blonder2011}{}}%
Blonder, B., \DIFdelbegin \DIFdel{C. Violle, }\DIFdelend \DIFaddbegin \DIFadd{Violle, C., Bentley, }\DIFaddend L. P.\DIFdelbegin \DIFdel{Bentley, and }\DIFdelend \DIFaddbegin \DIFadd{, \& Enquist, }\DIFaddend B. J. \DIFdelbegin \DIFdel{Enquist.
2011.
}\href{https://doi.org/10.1111/j.1461-0248.2010.01554.x}{\DIFdel{Venation
networks and the origin of the leaf economics spectrum}}%DIFAUXCMD
\DIFdel{.
Ecology Letters
14:}\DIFdelend \DIFaddbegin \DIFadd{(2011).
Venation networks and the origin of the leaf economics spectrum.
}\emph{\DIFadd{Ecology Letters}}\DIFadd{, }\emph{14}(2)\DIFadd{, }\DIFaddend 91--100.
\DIFaddbegin \url{https://doi.org/10.1111/j.1461-0248.2010.01554.x}
\DIFaddend 

\leavevmode\vadjust pre{\hypertarget{ref-Bonan2002}{}}%
Bonan, G. B., \DIFdelbegin \DIFdel{S. Levis, L. Kergoat, and }\DIFdelend \DIFaddbegin \DIFadd{Levis, S., Kergoat, L., \& Oleson, }\DIFaddend K. W. \DIFdelbegin \DIFdel{Oleson.
2002.
}\href{https://doi.org/10.1029/2000GB001360}{\DIFdel{Landscapes as patches of
plant functional types: }%DIFDELCMD < {%%%
\DIFdel{An}%DIFDELCMD < } %%%
\DIFdel{integrating concept for climate and
ecosystem models}}%DIFAUXCMD
\DIFdel{. Global Biogeochemical Cycles 16: }\DIFdelend \DIFaddbegin \DIFadd{(2002).
Landscapes as patches of plant functional types: }{\DIFadd{An}} \DIFadd{integrating
concept for climate and ecosystem models. }\emph{\DIFadd{Global Biogeochemical
Cycles}}\DIFadd{, }\emph{16}(2)\DIFadd{, }\DIFaddend 5-1-5-23.
\DIFaddbegin \url{https://doi.org/10.1029/2000GB001360}
\DIFaddend 

\leavevmode\vadjust pre{\hypertarget{ref-Burnham2002}{}}%
Burnham, K. P., \DIFdelbegin \DIFdel{and }\DIFdelend \DIFaddbegin \DIFadd{\& Anderson, }\DIFaddend D. R. \DIFdelbegin \DIFdel{Anderson. 2002.
}\href{https://doi.org/10.1007/b97636}{\DIFdel{Model }%DIFDELCMD < {%%%
\DIFdel{Selection}%DIFDELCMD < } %%%
\DIFdel{and }%DIFDELCMD < {%%%
\DIFdel{Multimodel
Inference}%DIFDELCMD < }%%%
\DIFdel{: }%DIFDELCMD < {%%%
\DIFdel{A Practical Information-Theoretic Approach}%DIFDELCMD < }%%%
}%DIFAUXCMD
\DIFdel{. Second}\DIFdelend \DIFaddbegin \DIFadd{(2002). }\emph{\DIFadd{Model }{\DIFadd{Selection}} \DIFadd{and
}{\DIFadd{Multimodel Inference}}\DIFadd{: }{\DIFadd{A Practical Information-Theoretic Approach}}}
\DIFadd{(Second)}\DIFaddend . {Springer-Verlag}\DIFdelbegin \DIFdel{, }%DIFDELCMD < {%%%
\DIFdel{New York}%DIFDELCMD < }%%%
\DIFdel{. }\DIFdelend \DIFaddbegin \DIFadd{. }\url{https://doi.org/10.1007/b97636}
\DIFaddend 

\leavevmode\vadjust pre{\hypertarget{ref-Carpenter2017}{}}%
Carpenter, B., \DIFdelbegin \DIFdel{A. Gelman, }\DIFdelend \DIFaddbegin \DIFadd{Gelman, A., Hoffman, }\DIFaddend M. D.\DIFdelbegin \DIFdel{Hoffman, D. Lee, }\DIFdelend \DIFaddbegin \DIFadd{, Lee, D., Goodrich, }\DIFaddend B.\DIFdelbegin \DIFdel{Goodrich}\DIFdelend ,
\DIFdelbegin \DIFdel{M.}\DIFdelend Betancourt, M.\DIFdelbegin \DIFdel{Brubaker, }\DIFdelend \DIFaddbegin \DIFadd{, Brubaker, M., Guo, }\DIFaddend J.\DIFdelbegin \DIFdel{Guo, }\DIFdelend \DIFaddbegin \DIFadd{, Li, }\DIFaddend P.\DIFdelbegin \DIFdel{Li, and A.Riddell}\DIFdelend \DIFaddbegin \DIFadd{, \& Riddell, A. (2017).
Stan : }{\DIFadd{A Probabilistic Programming Language}}\DIFaddend . \DIFdelbegin \DIFdel{2017.
}\href{https://doi.org/10.18637/jss.v076.i01}{\DIFdel{Stan : }%DIFDELCMD < {%%%
\DIFdel{A Probabilistic
Programming Language}%DIFDELCMD < }%%%
}%DIFAUXCMD
\DIFdel{.
Journal of Statistical Software 76: }\DIFdelend \DIFaddbegin \emph{\DIFadd{Journal of
Statistical Software}}\DIFadd{, }\emph{76}(1)\DIFadd{, }\DIFaddend 1--32.
\DIFaddbegin \url{https://doi.org/10.18637/jss.v076.i01}
\DIFaddend 

\leavevmode\vadjust pre{\hypertarget{ref-Evans2009}{}}%
Evans, J. R., \DIFdelbegin \DIFdel{R. Kaldenhoff, B. Genty, and I. Terashima.
2009.
}\href{https://doi.org/10.1093/jxb/erp117}{\DIFdel{Resistances along the }%DIFDELCMD < {%%%
\DIFdel{CO2}%DIFDELCMD < }
%DIFDELCMD < %%%
\DIFdel{diffusion pathway inside leaves}}%DIFAUXCMD
\DIFdel{.
Journal of Experimental Botany
60:}\DIFdelend \DIFaddbegin \DIFadd{Kaldenhoff, R., Genty, B., \& Terashima, I. (2009).
Resistances along the }{\DIFadd{CO2}} \DIFadd{diffusion pathway inside leaves.
}\emph{\DIFadd{Journal of Experimental Botany}}\DIFadd{, }\emph{60}(8)\DIFadd{, }\DIFaddend 2235--2248.
\DIFaddbegin \url{https://doi.org/10.1093/jxb/erp117}
\DIFaddend 

\leavevmode\vadjust pre{\hypertarget{ref-Falster2012}{}}%
Falster, D. S., \DIFaddbegin \DIFadd{Reich, }\DIFaddend P. B.\DIFdelbegin \DIFdel{Reich, }\DIFdelend \DIFaddbegin \DIFadd{, Ellsworth, }\DIFaddend D. S.\DIFdelbegin \DIFdel{Ellsworth, }\DIFdelend \DIFaddbegin \DIFadd{, Wright, }\DIFaddend I. J.\DIFdelbegin \DIFdel{Wright, M. Westoby, J. Oleksyn, and }\DIFdelend \DIFaddbegin \DIFadd{, Westoby,
M., Oleksyn, J., \& Lee, }\DIFaddend T. D. \DIFdelbegin \DIFdel{Lee. 2012.
}\href{https://doi.org/10.1111/j.1469-8137.2011.03940.x}{\DIFdel{Lifetime return
on investment increases with leaf lifespan among 10 }%DIFDELCMD < {%%%
\DIFdel{Australian}%DIFDELCMD < }
%DIFDELCMD < %%%
\DIFdel{woodland species}}%DIFAUXCMD
\DIFdel{.
New Phytologist 193:}\DIFdelend \DIFaddbegin \DIFadd{(2012). Lifetime return on investment
increases with leaf lifespan among 10 }{\DIFadd{Australian}} \DIFadd{woodland species.
}\emph{\DIFadd{New Phytologist}}\DIFadd{, }\emph{193}(2)\DIFadd{, }\DIFaddend 409--419.
\DIFaddbegin \url{https://doi.org/10.1111/j.1469-8137.2011.03940.x}
\DIFaddend 

\leavevmode\vadjust pre{\hypertarget{ref-Gelman2013}{}}%
Gelman, A., \DIFaddbegin \DIFadd{Carlin, }\DIFaddend J. B.\DIFdelbegin \DIFdel{Carlin, }\DIFdelend \DIFaddbegin \DIFadd{, Stern, }\DIFaddend H. S.\DIFdelbegin \DIFdel{Stern, }\DIFdelend \DIFaddbegin \DIFadd{, Dunson, }\DIFaddend D. B.\DIFdelbegin \DIFdel{Dunson, A. Vehtari, and }\DIFdelend \DIFaddbegin \DIFadd{, Vehtari, A., \&
Rubin, }\DIFaddend D. B. \DIFdelbegin \DIFdel{Rubin. 2013. Bayesian }%DIFDELCMD < {%%%
\DIFdel{Data Analysis}%DIFDELCMD < }%%%
\DIFdel{, }%DIFDELCMD < {%%%
\DIFdel{Third Edition}%DIFDELCMD < }%%%
\DIFdelend \DIFaddbegin \DIFadd{(2013). }\emph{\DIFadd{Bayesian }{\DIFadd{Data Analysis}}\DIFadd{, }{\DIFadd{Third Edition}}}\DIFaddend .
{Taylor \& Francis}.

\leavevmode\vadjust pre{\hypertarget{ref-Huntingford2017}{}}%
Huntingford, C., \DIFaddbegin \DIFadd{Atkin, }\DIFaddend O. K.\DIFdelbegin \DIFdel{Atkin, A. }\DIFdelend \DIFaddbegin \DIFadd{, }\DIFaddend Martinez-de la Torre, \DIFaddbegin \DIFadd{A., Mercado, }\DIFaddend L. M.\DIFdelbegin \DIFdel{Mercado,
}\DIFdelend \DIFaddbegin \DIFadd{,
Heskel, }\DIFaddend M. A.\DIFdelbegin \DIFdel{Heskel, }\DIFdelend \DIFaddbegin \DIFadd{, Harper, }\DIFaddend A. B.\DIFdelbegin \DIFdel{Harper, }\DIFdelend \DIFaddbegin \DIFadd{, Bloomfield, }\DIFaddend K. J.\DIFdelbegin \DIFdel{Bloomfield, O}\DIFdelend \DIFaddbegin \DIFadd{, O'Sullivan, O}\DIFaddend . S.\DIFdelbegin \DIFdel{O'Sullivan,
}\DIFdelend \DIFaddbegin \DIFadd{,
Reich, }\DIFaddend P. B.\DIFdelbegin \DIFdel{Reich, }\DIFdelend \DIFaddbegin \DIFadd{, Wythers, }\DIFaddend K. R.\DIFdelbegin \DIFdel{Wythers, }\DIFdelend \DIFaddbegin \DIFadd{, Butler, }\DIFaddend E. E.\DIFdelbegin \DIFdel{Butler, M. Chen, }\DIFdelend \DIFaddbegin \DIFadd{, Chen, M., Griffin, }\DIFaddend K. L.\DIFdelbegin \DIFdel{Griffin,
P. Meir, }\DIFdelend \DIFaddbegin \DIFadd{,
Meir, P., Tjoelker, }\DIFaddend M. G.\DIFdelbegin \DIFdel{Tjoelker, }\DIFdelend \DIFaddbegin \DIFadd{, Turnbull, }\DIFaddend M. H.\DIFdelbegin \DIFdel{Turnbull, S. Sitch, A. Wiltshire, and Y. Malhi. 2017.
}\href{https://doi.org/10.1038/s41467-017-01774-z}{\DIFdel{Implications of
improved representations of plant respiration in a changing climate}}%DIFAUXCMD
\DIFdel{. Nature Communications 8:}\DIFdelend \DIFaddbegin \DIFadd{, Sitch, S., Wiltshire, A., \&
Malhi, Y. (2017). Implications of improved representations of plant
respiration in a changing climate. }\emph{\DIFadd{Nature Communications}}\DIFadd{,
}\emph{8}(1)\DIFadd{, }\DIFaddend 1602. \DIFaddbegin \url{https://doi.org/10.1038/s41467-017-01774-z}
\DIFaddend 

\leavevmode\vadjust pre{\hypertarget{ref-John2017}{}}%
John, G. P., \DIFdelbegin \DIFdel{C. Scoffoni, }\DIFdelend \DIFaddbegin \DIFadd{Scoffoni, C., Buckley, }\DIFaddend T. N.\DIFdelbegin \DIFdel{Buckley, R. Villar, H. Poorter, and L. Sack. 2017. }\href{https://doi.org/10.1111/ele.12739}{\DIFdel{The anatomical and
compositional basis of leaf mass per area}}%DIFAUXCMD
\DIFdel{. Ecology Letters 20:}\DIFdelend \DIFaddbegin \DIFadd{, Villar, R., Poorter, H., \&
Sack, L. (2017). The anatomical and compositional basis of leaf mass per
area. }\emph{\DIFadd{Ecology Letters}}\DIFadd{, }\emph{20}(4)\DIFadd{, }\DIFaddend 412--425.
\DIFaddbegin \url{https://doi.org/10.1111/ele.12739}
\DIFaddend 

\leavevmode\vadjust pre{\hypertarget{ref-Kitajima2013}{}}%
Kitajima, K., \DIFaddbegin \DIFadd{Cordero, }\DIFaddend R. A.\DIFdelbegin \DIFdel{Cordero, and }\DIFdelend \DIFaddbegin \DIFadd{, \& Wright, }\DIFaddend S. J. \DIFdelbegin \DIFdel{Wright. 2013.
}\href{https://doi.org/10.1093/aob/mct036}{\DIFdel{Leaf life span spectrum of
tropical woody seedlings: }%DIFDELCMD < {%%%
\DIFdel{Effects}%DIFDELCMD < } %%%
\DIFdel{of light and ontogeny and
consequences for survival}}%DIFAUXCMD
\DIFdel{. Annals of Botany 112: }\DIFdelend \DIFaddbegin \DIFadd{(2013). Leaf life span
spectrum of tropical woody seedlings: }{\DIFadd{Effects}} \DIFadd{of light and ontogeny
and consequences for survival. }\emph{\DIFadd{Annals of Botany}}\DIFadd{, }\emph{112}(4)\DIFadd{,
}\DIFaddend 685--699. \DIFaddbegin \url{https://doi.org/10.1093/aob/mct036}
\DIFaddend 

\leavevmode\vadjust pre{\hypertarget{ref-Kitajima2012}{}}%
Kitajima, K., \DIFaddbegin \DIFadd{Llorens, }\DIFaddend A. M.\DIFdelbegin \DIFdel{Llorens, C. Stefanescu, }\DIFdelend \DIFaddbegin \DIFadd{, Stefanescu, C., Timchenko, }\DIFaddend M. V.\DIFdelbegin \DIFdel{Timchenko, }\DIFdelend \DIFaddbegin \DIFadd{, Lucas,
}\DIFaddend P. W.\DIFdelbegin \DIFdel{Lucas, and }\DIFdelend \DIFaddbegin \DIFadd{, \& Wright, }\DIFaddend S. J. \DIFdelbegin \DIFdel{Wright. 2012.
}\href{https://doi.org/10.1111/j.1469-8137.2012.04203.x}{\DIFdel{How
cellulose-based leaf toughness and lamina density contribute to long
leaf lifespans of shade-tolerant species}}%DIFAUXCMD
\DIFdel{. New Phytologist 195:}\DIFdelend \DIFaddbegin \DIFadd{(2012). How cellulose-based leaf toughness and
lamina density contribute to long leaf lifespans of shade-tolerant
species. }\emph{\DIFadd{New Phytologist}}\DIFadd{, }\emph{195}(3)\DIFadd{, }\DIFaddend 640--652.
\DIFaddbegin \url{https://doi.org/10.1111/j.1469-8137.2012.04203.x}
\DIFaddend 

\leavevmode\vadjust pre{\hypertarget{ref-Kitajima2010}{}}%
Kitajima, K., \DIFdelbegin \DIFdel{and L. Poorter. 2010.
}\href{https://doi.org/10.1111/j.1469-8137.2010.03212.x}{\DIFdel{Tissue-level
leaf toughness, but not lamina thickness, predicts sapling leaf lifespan
and shade tolerance of tropical tree species}}%DIFAUXCMD
\DIFdel{. New Phytologist
186:}\DIFdelend \DIFaddbegin \DIFadd{\& Poorter, L. (2010). Tissue-level leaf toughness, but
not lamina thickness, predicts sapling leaf lifespan and shade tolerance
of tropical tree species. }\emph{\DIFadd{New Phytologist}}\DIFadd{, }\emph{186}(3)\DIFadd{,
}\DIFaddend 708--721. \DIFaddbegin \url{https://doi.org/10.1111/j.1469-8137.2010.03212.x}
\DIFaddend 

\leavevmode\vadjust pre{\hypertarget{ref-Kitajima2016}{}}%
Kitajima, K., \DIFaddbegin \DIFadd{Wright, }\DIFaddend S. J.\DIFdelbegin \DIFdel{Wright, and }\DIFdelend \DIFaddbegin \DIFadd{, \& Westbrook, }\DIFaddend J. W. \DIFdelbegin \DIFdel{Westbrook. 2016.
}\href{https://doi.org/10.1098/rsfs.2015.0100}{\DIFdel{Leaf cellulose density as
the key determinant of inter- and intra-specific variation in leaf
fracture toughness in a species-rich tropical forest}}%DIFAUXCMD
\DIFdel{.
Interface Focus
6:}\DIFdelend \DIFaddbegin \DIFadd{(2016). Leaf cellulose
density as the key determinant of inter- and intra-specific variation in
leaf fracture toughness in a species-rich tropical forest.
}\emph{\DIFadd{Interface Focus}}\DIFadd{, }\emph{6}(3)\DIFadd{, }\DIFaddend 20150100.
\DIFaddbegin \url{https://doi.org/10.1098/rsfs.2015.0100}
\DIFaddend 

\leavevmode\vadjust pre{\hypertarget{ref-Kleyer2012}{}}%
Kleyer, M., \DIFdelbegin \DIFdel{S. Dray, F. Bello, J. Lepš, }\DIFdelend \DIFaddbegin \DIFadd{Dray, S., Bello, F., Lepš, J., Pakeman, }\DIFaddend R. J.\DIFdelbegin \DIFdel{Pakeman, B. Strauss,
W.
Thuiller, and S. Lavorel. 2012.
}\href{https://doi.org/10.1111/j.1654-1103.2012.01402.x}{\DIFdel{Assessing
species and community functional responses to environmental gradients:
}%DIFDELCMD < {%%%
\DIFdel{Which}%DIFDELCMD < } %%%
\DIFdel{multivariate methods?}} %DIFAUXCMD
\DIFdel{Journal of Vegetation Science
23: }\DIFdelend \DIFaddbegin \DIFadd{, Strauss, B.,
Thuiller, W., \& Lavorel, S. (2012). Assessing species and community
functional responses to environmental gradients: }{\DIFadd{Which}} \DIFadd{multivariate
methods? }\emph{\DIFadd{Journal of Vegetation Science}}\DIFadd{, }\emph{23}(5)\DIFadd{, }\DIFaddend 805--821.
\DIFaddbegin \url{https://doi.org/10.1111/j.1654-1103.2012.01402.x}
\DIFaddend 

\leavevmode\vadjust pre{\hypertarget{ref-Lemoine2019}{}}%
Lemoine, N. P. \DIFdelbegin \DIFdel{2019. }\href{https://doi.org/10.1111/oik.05985}{\DIFdel{Moving
beyond noninformative priors: Why and how to choose weakly informative
priors in }%DIFDELCMD < {%%%
\DIFdel{Bayesian}%DIFDELCMD < } %%%
\DIFdel{analyses}}%DIFAUXCMD
\DIFdel{. Oikos 128: }\DIFdelend \DIFaddbegin \DIFadd{(2019). Moving beyond noninformative priors: Why and how
to choose weakly informative priors in }{\DIFadd{Bayesian}} \DIFadd{analyses.
}\emph{\DIFadd{Oikos}}\DIFadd{, }\emph{128}(7)\DIFadd{, }\DIFaddend 912--928.
\DIFaddbegin \url{https://doi.org/10.1111/oik.05985}
\DIFaddend 

\leavevmode\vadjust pre{\hypertarget{ref-Lloyd2013}{}}%
Lloyd, J., \DIFdelbegin \DIFdel{K. Bloomfield, }\DIFdelend \DIFaddbegin \DIFadd{Bloomfield, K., Domingues, }\DIFaddend T. F.\DIFdelbegin \DIFdel{Domingues, and }\DIFdelend \DIFaddbegin \DIFadd{, \& Farquhar, }\DIFaddend G. D. \DIFdelbegin \DIFdel{Farquhar.
2013.
}\href{https://doi.org/10.1111/nph.12281}{\DIFdel{Photosynthetically relevant
foliar traits correlating better on a mass vs an area basis: }%DIFDELCMD < {%%%
\DIFdel{Of}%DIFDELCMD < }
%DIFDELCMD < %%%
\DIFdel{ecophysiological relevance or just a case of mathematical imperatives
and statistical quicksand?}} %DIFAUXCMD
\DIFdel{New Phytologist 199: }\DIFdelend \DIFaddbegin \DIFadd{(2013).
Photosynthetically relevant foliar traits correlating better on a mass
vs an area basis: }{\DIFadd{Of}} \DIFadd{ecophysiological relevance or just a case of
mathematical imperatives and statistical quicksand? }\emph{\DIFadd{New
Phytologist}}\DIFadd{, }\emph{199}(2)\DIFadd{, }\DIFaddend 311--321.
\DIFaddbegin \url{https://doi.org/10.1111/nph.12281}
\DIFaddend 

\leavevmode\vadjust pre{\hypertarget{ref-Lusk2010}{}}%
Lusk, C. H., \DIFdelbegin \DIFdel{Y. Onoda, R. Kooyman, and A. Gutiérrez-Girón.
2010.
}\href{https://doi.org/10.1111/j.1469-8137.2010.03202.x}{\DIFdel{Reconciling
species-level vs plastic responses of evergreen leaf structure to light
gradients: }%DIFDELCMD < {%%%
\DIFdel{Shade}%DIFDELCMD < } %%%
\DIFdel{leaves punch above their weight}}%DIFAUXCMD
\DIFdel{. New Phytologist
186: }\DIFdelend \DIFaddbegin \DIFadd{Onoda, Y., Kooyman, R., \& Gutiérrez-Girón, A. (2010).
Reconciling species-level vs plastic responses of evergreen leaf
structure to light gradients: }{\DIFadd{Shade}} \DIFadd{leaves punch above their weight.
}\emph{\DIFadd{New Phytologist}}\DIFadd{, }\emph{186}(2)\DIFadd{, }\DIFaddend 429--438.
\DIFaddbegin \url{https://doi.org/10.1111/j.1469-8137.2010.03202.x}
\DIFaddend 

\leavevmode\vadjust pre{\hypertarget{ref-Lusk2008}{}}%
Lusk, C. H., \DIFaddbegin \DIFadd{Reich, }\DIFaddend P. B.\DIFdelbegin \DIFdel{Reich, }\DIFdelend \DIFaddbegin \DIFadd{, Montgomery, }\DIFaddend R. A.\DIFdelbegin \DIFdel{Montgomery, }\DIFdelend \DIFaddbegin \DIFadd{, Ackerly, }\DIFaddend D. D.\DIFdelbegin \DIFdel{Ackerly, and J. Cavender-Bares. 2008.
}\href{https://doi.org/10.1016/j.tree.2008.02.006}{\DIFdel{Why are evergreen
leaves so contrary about shade?}} %DIFAUXCMD
\DIFdel{Trends in Ecology and Evolution
23:}\DIFdelend \DIFaddbegin \DIFadd{, \&
Cavender-Bares, J. (2008). Why are evergreen leaves so contrary about
shade? }\emph{\DIFadd{Trends in Ecology and Evolution}}\DIFadd{, }\emph{23}(6)\DIFadd{, }\DIFaddend 299--303.
\DIFaddbegin \url{https://doi.org/10.1016/j.tree.2008.02.006}
\DIFaddend 

\leavevmode\vadjust pre{\hypertarget{ref-Martinez-Vilalta2016}{}}%
Martínez-Vilalta, J., \DIFdelbegin \DIFdel{A. Sala, D. Asensio, L. Galiano, G. Hoch,
S.
Palacio, }\DIFdelend \DIFaddbegin \DIFadd{Sala, A., Asensio, D., Galiano, L., Hoch, G.,
Palacio, S., Piper, }\DIFaddend F. I.\DIFdelbegin \DIFdel{Piper, and F. Lloret. 2016.
}\href{https://doi.org/10.1002/ecm.1231}{\DIFdel{Dynamics of non-structural
carbohydrates in terrestrial plants: A global synthesis}}%DIFAUXCMD
\DIFdel{. Ecological
Monographs 86: }\DIFdelend \DIFaddbegin \DIFadd{, \& Lloret, F. (2016). Dynamics of
non-structural carbohydrates in terrestrial plants: A global synthesis.
}\emph{\DIFadd{Ecological Monographs}}\DIFadd{, }\emph{86}(4)\DIFadd{, }\DIFaddend 495--516.
\DIFaddbegin \url{https://doi.org/10.1002/ecm.1231}
\DIFaddend 

\leavevmode\vadjust pre{\hypertarget{ref-Niinemets2015}{}}%
Niinemets, Ü., \DIFaddbegin \DIFadd{Keenan, }\DIFaddend T. F.\DIFdelbegin \DIFdel{Keenan, and L. Hallik. 2015.
}\href{https://doi.org/10.1111/nph.13096}{\DIFdel{A worldwide analysis of
within-canopy variations in leaf structural, chemical and physiological
traits across plant functional types}}%DIFAUXCMD
\DIFdel{. New Phytologist 205:}\DIFdelend \DIFaddbegin \DIFadd{, \& Hallik, L. (2015). A worldwide analysis
of within-canopy variations in leaf structural, chemical and
physiological traits across plant functional types. }\emph{\DIFadd{New
Phytologist}}\DIFadd{, }\emph{205}(3)\DIFadd{, }\DIFaddend 973--993.
\DIFaddbegin \url{https://doi.org/10.1111/nph.13096}
\DIFaddend 

\leavevmode\vadjust pre{\DIFdelbegin %DIFDELCMD < \hypertarget{ref-Onoda2008}{}%%%
\DIFdelend \DIFaddbegin \hypertarget{ref-Onoda2015}{}\DIFaddend }%
Onoda, Y., \DIFdelbegin \DIFdel{F. Schieving, and }\DIFdelend \DIFaddbegin \DIFadd{Schieving, F., \& Anten, }\DIFaddend N. P. R. \DIFdelbegin \DIFdel{Anten. 2008.
}\href{https://doi.org/10.1093/aob/mcn013}{\DIFdel{Effects of light and nutrient
availability on leaf mechanical properties of }%DIFDELCMD < {%%%
\DIFdel{Plantago}%DIFDELCMD < } %%%
\DIFdel{major: }%DIFDELCMD < {%%%
\DIFdel{A}%DIFDELCMD < }
%DIFDELCMD < %%%
\DIFdel{conceptual approach}}%DIFAUXCMD
\DIFdel{. Annals of
Botany 101:727--736. }\DIFdelend \DIFaddbegin \DIFadd{(2015). A novel method of
measuring leaf epidermis and mesophyll stiffness shows the ubiquitous
nature of the sandwich structure of leaf laminas in broad-leaved
angiosperm species. }\emph{\DIFadd{Journal of Experimental Botany}}\DIFadd{, }\emph{66}(9)\DIFadd{,
2487--2499. }\url{https://doi.org/10.1093/jxb/erv024}
\DIFaddend 

\leavevmode\vadjust pre{\DIFdelbegin %DIFDELCMD < \hypertarget{ref-Onoda2015}{}%%%
\DIFdelend \DIFaddbegin \hypertarget{ref-Onoda2008}{}\DIFaddend }%
Onoda, Y., \DIFdelbegin \DIFdel{F. Schieving, and }\DIFdelend \DIFaddbegin \DIFadd{Schieving, F., \& Anten, }\DIFaddend N. P. R. \DIFdelbegin \DIFdel{Anten. 2015.
}\href{https://doi.org/10.1093/jxb/erv024}{\DIFdel{A novel method of measuring
leaf epidermis and mesophyll stiffness shows the ubiquitous nature of
the sandwich structure of leaf laminas in broad-leaved angiosperm
species}}%DIFAUXCMD
\DIFdel{. Journal of Experimental Botany 66: 2487--2499. }\DIFdelend \DIFaddbegin \DIFadd{(2008). Effects of light
and nutrient availability on leaf mechanical properties of }{\DIFadd{Plantago}}
\DIFadd{major: }{\DIFadd{A}} \DIFadd{conceptual approach. }\emph{\DIFadd{Annals of Botany}}\DIFadd{, }\emph{101}(5)\DIFadd{,
727--736. }\url{https://doi.org/10.1093/aob/mcn013}
\DIFaddend 

\leavevmode\vadjust pre{\hypertarget{ref-Onoda2017}{}}%
Onoda, Y., \DIFaddbegin \DIFadd{Wright, }\DIFaddend I. J.\DIFdelbegin \DIFdel{Wright, }\DIFdelend \DIFaddbegin \DIFadd{, Evans, }\DIFaddend J. R.\DIFdelbegin \DIFdel{Evans, K. }\DIFdelend \DIFaddbegin \DIFadd{, }\DIFaddend Hikosaka, K.\DIFdelbegin \DIFdel{Kitajima, }\DIFdelend \DIFaddbegin \DIFadd{, Kitajima, K.,
Niinemets, }\DIFaddend Ü.\DIFdelbegin \DIFdel{Niinemets, }\DIFdelend \DIFaddbegin \DIFadd{, Poorter, }\DIFaddend H.\DIFdelbegin \DIFdel{Poorter, }\DIFdelend \DIFaddbegin \DIFadd{, Tosens, }\DIFaddend T.\DIFdelbegin \DIFdel{Tosens}\DIFdelend \DIFaddbegin \DIFadd{, \& Westoby, M. (2017).
Physiological and structural tradeoffs underlying the leaf economics
spectrum. }\emph{\DIFadd{New Phytologist}}\DIFaddend , \DIFdelbegin \DIFdel{and M.Westoby. 2017.
}\href{https://doi.org/10.1111/nph.14496}{\DIFdel{Physiological and structural
tradeoffs underlying the leaf economics spectrum}}%DIFAUXCMD
\DIFdel{.
New Phytologist
214:}\DIFdelend \DIFaddbegin \emph{214}(4)\DIFadd{, }\DIFaddend 1447--1463.
\DIFaddbegin \url{https://doi.org/10.1111/nph.14496}
\DIFaddend 

\leavevmode\vadjust pre{\hypertarget{ref-Osada2001}{}}%
Osada, N., \DIFdelbegin \DIFdel{H. Takeda, A. Furukawa, and M. Awang. 2001.
}\href{https://doi.org/10.1046/j.0022-0477.2001.00590.x}{\DIFdel{Leaf dynamics
and maintenance of tree crowns in a malaysian rain forest stand}}%DIFAUXCMD
\DIFdel{.
Journal of Ecology 89:}\DIFdelend \DIFaddbegin \DIFadd{Takeda, H., Furukawa, A., \& Awang, M. (2001). Leaf dynamics
and maintenance of tree crowns in a malaysian rain forest stand.
}\emph{\DIFadd{Journal of Ecology}}\DIFadd{, }\emph{89}(5)\DIFadd{, }\DIFaddend 774--782.
\DIFaddbegin \url{https://doi.org/10.1046/j.0022-0477.2001.00590.x}
\DIFaddend 

\leavevmode\vadjust pre{\hypertarget{ref-Osnas2018}{}}%
Osnas, J. L. D., \DIFdelbegin \DIFdel{M. Katabuchi, K. Kitajima, }\DIFdelend \DIFaddbegin \DIFadd{Katabuchi, M., Kitajima, K., Wright, }\DIFaddend S. J.\DIFdelbegin \DIFdel{Wright, }\DIFdelend \DIFaddbegin \DIFadd{, Reich, }\DIFaddend P.
B.\DIFdelbegin \DIFdel{Reich, }\DIFdelend \DIFaddbegin \DIFadd{, Van Bael, }\DIFaddend S. A.\DIFdelbegin \DIFdel{Van Bael, }\DIFdelend \DIFaddbegin \DIFadd{, Kraft, }\DIFaddend N. J. B.\DIFdelbegin \DIFdel{Kraft, }\DIFdelend \DIFaddbegin \DIFadd{, Samaniego, }\DIFaddend M. J.\DIFdelbegin \DIFdel{Samaniego, }\DIFdelend \DIFaddbegin \DIFadd{, Pacala, }\DIFaddend S. W.\DIFdelbegin \DIFdel{Pacala,
and }\DIFdelend \DIFaddbegin \DIFadd{,
\& Lichstein, }\DIFaddend J. W. \DIFdelbegin \DIFdel{Lichstein. 2018.
}\href{https://doi.org/10.1073/pnas.1803989115}{\DIFdel{Divergent drivers of leaf
trait variation within species, among species, and among functional
groups.}} %DIFAUXCMD
\DIFdel{Proceedings of the National Academy of Sciences of the United
States of America 115:}\DIFdelend \DIFaddbegin \DIFadd{(2018). Divergent drivers of leaf trait variation
within species, among species, and among functional groups.
}\emph{\DIFadd{Proceedings of the National Academy of Sciences of the United
States of America}}\DIFadd{, }\emph{115}(21)\DIFadd{, }\DIFaddend 5480--5485.
\DIFaddbegin \url{https://doi.org/10.1073/pnas.1803989115}
\DIFaddend 

\leavevmode\vadjust pre{\hypertarget{ref-Osnas2013}{}}%
Osnas, J. L. D., \DIFaddbegin \DIFadd{Lichstein, }\DIFaddend J. W.\DIFdelbegin \DIFdel{Lichstein, }\DIFdelend \DIFaddbegin \DIFadd{, Reich, }\DIFaddend P. B.\DIFdelbegin \DIFdel{Reich, and }\DIFdelend \DIFaddbegin \DIFadd{, \& Pacala, }\DIFaddend S. W.
\DIFdelbegin \DIFdel{Pacala. 2013.
}\href{https://doi.org/10.1126/science.1231574}{\DIFdel{Global leaf trait
relationships: }%DIFDELCMD < {%%%
\DIFdel{Mass}%DIFDELCMD < }%%%
\DIFdel{, area, and the leaf economics spectrum}}%DIFAUXCMD
\DIFdel{. Science
340: }\DIFdelend \DIFaddbegin \DIFadd{(2013). Global leaf trait relationships: }{\DIFadd{Mass}}\DIFadd{, area, and the leaf
economics spectrum. }\emph{\DIFadd{Science}}\DIFadd{, }\emph{340}(6133)\DIFadd{, }\DIFaddend 741--744.
\DIFaddbegin \url{https://doi.org/10.1126/science.1231574}
\DIFaddend 

\leavevmode\vadjust pre{\hypertarget{ref-Poorter2009}{}}%
Poorter, H., \DIFaddbegin \DIFadd{Niinemets, }\DIFaddend Ü.\DIFdelbegin \DIFdel{Niinemets, L. Poorter, }\DIFdelend \DIFaddbegin \DIFadd{, Poorter, L., Wright, }\DIFaddend I. J.\DIFdelbegin \DIFdel{Wright, and R.
Villar. 2009. }\href{https://doi.org/10.1111/j.1469-8137.2009.02830.x}{\DIFdel{Causes and
consequences of variation in leaf mass per area (}%DIFDELCMD < {%%%
\DIFdel{LMA}%DIFDELCMD < }%%%
\DIFdel{): }%DIFDELCMD < {%%%
\DIFdel{A}%DIFDELCMD < }
%DIFDELCMD < %%%
\DIFdel{meta-analysis}}%DIFAUXCMD
\DIFdel{. New Phytologist 182: }\DIFdelend \DIFaddbegin \DIFadd{, \& Villar, R.
(2009). Causes and consequences of variation in leaf mass per area
(}{\DIFadd{LMA}}\DIFadd{): }{\DIFadd{A}} \DIFadd{meta-analysis. }\emph{\DIFadd{New Phytologist}}\DIFadd{, }\emph{182}(3)\DIFadd{,
}\DIFaddend 565--588. \DIFaddbegin \url{https://doi.org/10.1111/j.1469-8137.2009.02830.x}
\DIFaddend 

\leavevmode\vadjust pre{\hypertarget{ref-Poorter2006b}{}}%
Poorter, H., \DIFdelbegin \DIFdel{S. Pepin, }\DIFdelend \DIFaddbegin \DIFadd{Pepin, S., Rijkers, }\DIFaddend T.\DIFdelbegin \DIFdel{Rijkers}\DIFdelend , \DIFdelbegin \DIFdel{Y.}\DIFdelend De Jong, \DIFaddbegin \DIFadd{Y., Evans, }\DIFaddend J. R.\DIFdelbegin \DIFdel{Evans, and C. Körner. 2006. }\href{https://doi.org/10.1093/jxb/erj002}{\DIFdel{Construction
costs, chemical composition and payback time of high- and low-irradiance
leaves}}%DIFAUXCMD
\DIFdel{. Journal of Experimental Botany 57:}\DIFdelend \DIFaddbegin \DIFadd{, \&
Körner, C. (2006). Construction costs, chemical composition and payback
time of high- and low-irradiance leaves. }\emph{\DIFadd{Journal of Experimental
Botany}}\DIFadd{, }\emph{\DIFadd{57}}\DIFadd{(2 SPEC. ISS.), }\DIFaddend 355--371.
\DIFaddbegin \url{https://doi.org/10.1093/jxb/erj002}
\DIFaddend 

\leavevmode\vadjust pre{\hypertarget{ref-Reich2014}{}}%
Reich, P. B. \DIFdelbegin \DIFdel{2014. }\href{https://doi.org/10.1111/1365-2745.12211}{\DIFdel{The
world-wide 'fast-slow' plant economics spectrum: }%DIFDELCMD < {%%%
\DIFdel{A}%DIFDELCMD < } %%%
\DIFdel{traits manifesto}}%DIFAUXCMD
\DIFdel{. Journal of Ecology 102: }\DIFdelend \DIFaddbegin \DIFadd{(2014). The world-wide 'fast-slow' plant economics
spectrum: }{\DIFadd{A}} \DIFadd{traits manifesto. }\emph{\DIFadd{Journal of Ecology}}\DIFadd{,
}\emph{102}(2)\DIFadd{, }\DIFaddend 275--301. \DIFaddbegin \url{https://doi.org/10.1111/1365-2745.12211}
\DIFaddend 

\leavevmode\vadjust pre{\hypertarget{ref-Roderick1999}{}}%
Roderick, M. L., \DIFaddbegin \DIFadd{Berry, }\DIFaddend S. L.\DIFdelbegin \DIFdel{Berry, }\DIFdelend \DIFaddbegin \DIFadd{, Noble, }\DIFaddend I. R.\DIFdelbegin \DIFdel{Noble, and }\DIFdelend \DIFaddbegin \DIFadd{, \& Farquhar, }\DIFaddend G. D. \DIFdelbegin \DIFdel{Farquhar.
1999.
}\href{https://doi.org/10.1046/j.1365-2435.1999.00368.x}{\DIFdel{A theoretical
approach to linking the composition and morphology with the function of
leaves}}%DIFAUXCMD
\DIFdel{. Functional Ecology 13:}\DIFdelend \DIFaddbegin \DIFadd{(1999).
A theoretical approach to linking the composition and morphology with
the function of leaves. }\emph{\DIFadd{Functional Ecology}}\DIFadd{, }\emph{13}(5)\DIFadd{,
}\DIFaddend 683--695. \DIFaddbegin \url{https://doi.org/10.1046/j.1365-2435.1999.00368.x}
\DIFaddend 

\leavevmode\vadjust pre{\hypertarget{ref-Russo2016}{}}%
Russo, S. E., \DIFdelbegin \DIFdel{and K. Kitajima. 2016.
}\href{https://doi.org/10.1007/978-3-319-27422-5_17}{\DIFdel{The }%DIFDELCMD < {%%%
\DIFdel{Ecophysiology}%DIFDELCMD < }
%DIFDELCMD < %%%
\DIFdel{of }%DIFDELCMD < {%%%
\DIFdel{Leaf Lifespan}%DIFDELCMD < } %%%
\DIFdel{in }%DIFDELCMD < {%%%
\DIFdel{Tropical Forests}%DIFDELCMD < }%%%
\DIFdel{: }%DIFDELCMD < {%%%
\DIFdel{Adaptive}%DIFDELCMD < } %%%
\DIFdel{and }%DIFDELCMD < {%%%
\DIFdel{Plastic
Responses}%DIFDELCMD < } %%%
\DIFdel{to }%DIFDELCMD < {%%%
\DIFdel{Environmental Heterogeneity}%DIFDELCMD < }%%%
}%DIFAUXCMD
\DIFdel{. Pages 357--383 }\emph{\DIFdel{in}}
%DIFAUXCMD
\DIFdelend \DIFaddbegin \DIFadd{\& Kitajima, K. (2016). The }{\DIFadd{Ecophysiology}} \DIFadd{of }{\DIFadd{Leaf
Lifespan}} \DIFadd{in }{\DIFadd{Tropical Forests}}\DIFadd{: }{\DIFadd{Adaptive}} \DIFadd{and }{\DIFadd{Plastic Responses}} \DIFadd{to
}{\DIFadd{Environmental Heterogeneity}}\DIFadd{. In }\DIFaddend G. Goldstein \DIFdelbegin \DIFdel{and }\DIFdelend \DIFaddbegin \DIFadd{\& }\DIFaddend S. L. Santiago \DIFdelbegin \DIFdel{, editors. Tropical }%DIFDELCMD < {%%%
\DIFdel{Tree Physiology}%DIFDELCMD < }%%%
\DIFdelend \DIFaddbegin \DIFadd{(Eds.),
}\emph{\DIFadd{Tropical }{\DIFadd{Tree Physiology}}} \DIFadd{(pp. 357--383)}\DIFaddend . {Springer
International Publishing}\DIFdelbegin \DIFdel{, }%DIFDELCMD < {%%%
\DIFdel{Cham}%DIFDELCMD < }%%%
\DIFdel{.
}\DIFdelend \DIFaddbegin \DIFadd{.
}\url{https://doi.org/10.1007/978-3-319-27422-5_17}
\DIFaddend 

\leavevmode\vadjust pre{\hypertarget{ref-Sakschewski2016}{}}%
Sakschewski, B., \DIFdelbegin \DIFdel{W. }\DIFdelend von Bloh, \DIFaddbegin \DIFadd{W., Boit, }\DIFaddend A.\DIFdelbegin \DIFdel{Boit, }\DIFdelend \DIFaddbegin \DIFadd{, Poorter, }\DIFaddend L.\DIFdelbegin \DIFdel{Poorter}\DIFdelend , \DIFdelbegin \DIFdel{M.}\DIFdelend Peña-Claros, \DIFdelbegin \DIFdel{J.}\DIFdelend \DIFaddbegin \DIFadd{M.,
}\DIFaddend Heinke, J.\DIFdelbegin \DIFdel{Joshi}\DIFdelend \DIFaddbegin \DIFadd{, Joshi, J., \& Thonicke, K. (2016). Resilience of }{\DIFadd{Amazon}}
\DIFadd{forests emerges from plant trait diversity. }\emph{\DIFadd{Nature Climate
Change}}\DIFaddend , \DIFdelbegin \DIFdel{and K.Thonicke. 2016.
}\href{https://doi.org/10.1038/nclimate3109}{\DIFdel{Resilience of }%DIFDELCMD < {%%%
\DIFdel{Amazon}%DIFDELCMD < }
%DIFDELCMD < %%%
\DIFdel{forests emerges from plant trait diversity}}%DIFAUXCMD
\DIFdel{. Nature Climate Change
6:}\DIFdelend \DIFaddbegin \emph{6}(11)\DIFadd{, }\DIFaddend 1032--1036.
\DIFaddbegin \url{https://doi.org/10.1038/nclimate3109}
\DIFaddend 

\leavevmode\vadjust pre{\hypertarget{ref-Sakschewski2015}{}}%
Sakschewski, B., \DIFdelbegin \DIFdel{W. }\DIFdelend von Bloh, \DIFdelbegin \DIFdel{A.}\DIFdelend \DIFaddbegin \DIFadd{W., }\DIFaddend Boit, A.\DIFdelbegin \DIFdel{Rammig, }\DIFdelend \DIFaddbegin \DIFadd{, Rammig, A., Kattge, }\DIFaddend J.\DIFdelbegin \DIFdel{Kattge, }\DIFdelend \DIFaddbegin \DIFadd{,
Poorter, }\DIFaddend L.\DIFdelbegin \DIFdel{Poorter}\DIFdelend , \DIFdelbegin \DIFdel{J.}\DIFdelend Peñuelas, \DIFdelbegin \DIFdel{and K.Thonicke. 2015.
}\href{https://doi.org/10.1111/gcb.12870}{\DIFdel{Leaf and stem economics spectra
drive diversity of functional plant traits in a dynamic global
vegetation model}}%DIFAUXCMD
\DIFdel{. Global Change Biology 21:}\DIFdelend \DIFaddbegin \DIFadd{J., \& Thonicke, K. (2015). Leaf and stem
economics spectra drive diversity of functional plant traits in a
dynamic global vegetation model. }\emph{\DIFadd{Global Change Biology}}\DIFadd{,
}\emph{21}(7)\DIFadd{, }\DIFaddend 2711--2725. \DIFaddbegin \url{https://doi.org/10.1111/gcb.12870}
\DIFaddend 

\leavevmode\vadjust pre{\hypertarget{ref-Scheiter2013}{}}%
Scheiter, S., \DIFdelbegin \DIFdel{L. Langan, and }\DIFdelend \DIFaddbegin \DIFadd{Langan, L., \& Higgins, }\DIFaddend S. I. \DIFdelbegin \DIFdel{Higgins. 2013.
}\href{https://doi.org/10.1111/nph.12210}{\DIFdel{Next-generation dynamic global
vegetation models: }%DIFDELCMD < {%%%
\DIFdel{Learning}%DIFDELCMD < } %%%
\DIFdel{from community ecology}}%DIFAUXCMD
\DIFdel{. New Phytologist
198: }\DIFdelend \DIFaddbegin \DIFadd{(2013). Next-generation
dynamic global vegetation models: }{\DIFadd{Learning}} \DIFadd{from community ecology.
}\emph{\DIFadd{New Phytologist}}\DIFadd{, }\emph{198}(3)\DIFadd{, }\DIFaddend 957--969.
\DIFaddbegin \url{https://doi.org/10.1111/nph.12210}
\DIFaddend 

\leavevmode\vadjust pre{\hypertarget{ref-Shipley2006}{}}%
Shipley, B., \DIFaddbegin \DIFadd{Lechowicz, }\DIFaddend M. J.\DIFdelbegin \DIFdel{Lechowicz, I. Wright, and }\DIFdelend \DIFaddbegin \DIFadd{, Wright, I., \& Reich, }\DIFaddend P. B. \DIFdelbegin \DIFdel{Reich.
2006.
}\href{https://doi.org/10.1890/05-1051}{\DIFdel{Fundamental trade-offs generating
the worldwide leaf economics spectrum}}%DIFAUXCMD
\DIFdel{.
Ecology 87:}\DIFdelend \DIFaddbegin \DIFadd{(2006).
Fundamental trade-offs generating the worldwide leaf economics spectrum.
}\emph{\DIFadd{Ecology}}\DIFadd{, }\emph{87}(3)\DIFadd{, }\DIFaddend 535--541.
\DIFaddbegin \url{https://doi.org/10.1890/05-1051}
\DIFaddend 

\leavevmode\vadjust pre{\hypertarget{ref-Tcherkez2017}{}}%
Tcherkez, G., \DIFdelbegin \DIFdel{P. Gauthier, }\DIFdelend \DIFaddbegin \DIFadd{Gauthier, P., Buckley, }\DIFaddend T. N.\DIFdelbegin \DIFdel{Buckley, }\DIFdelend \DIFaddbegin \DIFadd{, Busch, }\DIFaddend F. A.\DIFdelbegin \DIFdel{Busch, }\DIFdelend \DIFaddbegin \DIFadd{, Barbour, }\DIFaddend M.
M.\DIFdelbegin \DIFdel{Barbour, D.
Bruhn, }\DIFdelend \DIFaddbegin \DIFadd{, Bruhn, D., Heskel, }\DIFaddend M. A.\DIFdelbegin \DIFdel{Heskel, }\DIFdelend \DIFaddbegin \DIFadd{, Gong, }\DIFaddend X. Y.\DIFdelbegin \DIFdel{Gong, }\DIFdelend \DIFaddbegin \DIFadd{, Crous, }\DIFaddend K. Y.\DIFdelbegin \DIFdel{Crous, K. Griffin,
D. Way, }\DIFdelend \DIFaddbegin \DIFadd{, Griffin, K.,
Way, D., Turnbull, M., Adams, }\DIFaddend M. \DIFdelbegin \DIFdel{Turnbull, M. }\DIFdelend A.\DIFdelbegin \DIFdel{Adams, }\DIFdelend \DIFaddbegin \DIFadd{, Atkin, }\DIFaddend O. K.\DIFdelbegin \DIFdel{Atkin, }\DIFdelend \DIFaddbegin \DIFadd{, Farquhar, }\DIFaddend G. D.\DIFdelbegin \DIFdel{Farquhar, and G. Cornic. 2017.
}\href{https://doi.org/10.1111/nph.14816}{\DIFdel{Leaf day respiration: Low }%DIFDELCMD < {%%%
\DIFdel{CO2}%DIFDELCMD < }
%DIFDELCMD < %%%
\DIFdel{flux but high significance for metabolism and carbon balance}}%DIFAUXCMD
\DIFdel{. New
Phytologist 216: }\DIFdelend \DIFaddbegin \DIFadd{, \&
Cornic, G. (2017). Leaf day respiration: Low }{\DIFadd{CO2}} \DIFadd{flux but high
significance for metabolism and carbon balance. }\emph{\DIFadd{New Phytologist}}\DIFadd{,
}\emph{216}(4)\DIFadd{, }\DIFaddend 986--1001. \DIFaddbegin \url{https://doi.org/10.1111/nph.14816}
\DIFaddend 

\leavevmode\vadjust pre{\hypertarget{ref-Terashima2011}{}}%
Terashima, I., \DIFaddbegin \DIFadd{Hanba, }\DIFaddend Y. T.\DIFdelbegin \DIFdel{Hanba, D. Tholen, and U. Niinemets. 2011.
}\href{https://doi.org/10.1104/pp.110.165472}{\DIFdel{Leaf }%DIFDELCMD < {%%%
\DIFdel{Functional Anatomy}%DIFDELCMD < }
%DIFDELCMD < %%%
\DIFdel{in }%DIFDELCMD < {%%%
\DIFdel{Relation}%DIFDELCMD < } %%%
\DIFdel{to }%DIFDELCMD < {%%%
\DIFdel{Photosynthesis}%DIFDELCMD < }%%%
}%DIFAUXCMD
\DIFdel{. Plant Physiology 155:}\DIFdelend \DIFaddbegin \DIFadd{, Tholen, D., \& Niinemets, U. (2011). Leaf
}{\DIFadd{Functional Anatomy}} \DIFadd{in }{\DIFadd{Relation}} \DIFadd{to }{\DIFadd{Photosynthesis}}\DIFadd{. }\emph{\DIFadd{Plant
Physiology}}\DIFadd{, }\emph{155}(1)\DIFadd{, }\DIFaddend 108--116.
\DIFaddbegin \url{https://doi.org/10.1104/pp.110.165472}
\DIFaddend 

\leavevmode\vadjust pre{\hypertarget{ref-Vehtari2017}{}}%
Vehtari, A., \DIFdelbegin \DIFdel{A}\DIFdelend \DIFaddbegin \DIFadd{Gelman, A., \& Gabry, J. (2017). Practical }{\DIFadd{Bayesian}} \DIFadd{model
evaluation using leave-one-out cross-validation and }{\DIFadd{WAIC}}\DIFaddend .
\DIFdelbegin \DIFdel{Gelman, and J. Gabry. 2017.
}\href{https://doi.org/10.1007/s11222-016-9696-4}{\DIFdel{Practical }%DIFDELCMD < {%%%
\DIFdel{Bayesian}%DIFDELCMD < }
%DIFDELCMD < %%%
\DIFdel{model evaluation using leave-one-out cross-validation and }%DIFDELCMD < {%%%
\DIFdel{WAIC}%DIFDELCMD < }%%%
}%DIFAUXCMD
\DIFdel{.
Statistics and Computing:}\DIFdelend \DIFaddbegin \emph{\DIFadd{Statistics and Computing}}\DIFadd{, }\emph{\DIFadd{27}}\DIFadd{, }\DIFaddend 1413--1432.
\DIFaddbegin \url{https://doi.org/10.1007/s11222-016-9696-4}
\DIFaddend 

\leavevmode\vadjust pre{\hypertarget{ref-Vehtari2014}{}}%
Vehtari, A., \DIFdelbegin \DIFdel{T. Mononen, V. Tolvanen, T. Sivula, and O.
Winther. 2014.
}\href{https://arxiv.org/abs/1408.4050v2}{\DIFdel{Bayesian leave-one-out
cross-validation approximations for }%DIFDELCMD < {%%%
\DIFdel{Gaussian}%DIFDELCMD < } %%%
\DIFdel{latent variable models}}%DIFAUXCMD
\DIFdel{. arXiv preprint arXiv:1408.4050v2. }\DIFdelend \DIFaddbegin \DIFadd{Mononen, T., Tolvanen, V., Sivula, T., \& Winther, O.
(2014). Bayesian leave-one-out cross-validation approximations for
}{\DIFadd{Gaussian}} \DIFadd{latent variable models. }\emph{\DIFadd{arXiv Preprint
arXiv:1408.4050v2}}\DIFadd{. }\url{https://arxiv.org/abs/1408.4050v2}
\DIFaddend 

\leavevmode\vadjust pre{\hypertarget{ref-Villar2001}{}}%
Villar, R., \DIFdelbegin \DIFdel{and J. Merino. 2001.
}\href{https://doi.org/10.1046/j.1469-8137.2001.00147.x}{\DIFdel{Comparison of
leaf construction costs in woody species with differing leaf life-spans
in contrasting ecosystems}}%DIFAUXCMD
\DIFdel{. New Phytologist 151:}\DIFdelend \DIFaddbegin \DIFadd{\& Merino, J. (2001). Comparison of leaf construction costs
in woody species with differing leaf life-spans in contrasting
ecosystems. }\emph{\DIFadd{New Phytologist}}\DIFadd{, }\emph{151}(1)\DIFadd{, }\DIFaddend 213--226.
\DIFaddbegin \url{https://doi.org/10.1046/j.1469-8137.2001.00147.x}
\DIFaddend 

\leavevmode\vadjust pre{\hypertarget{ref-Westoby2013}{}}%
Westoby, M., \DIFaddbegin \DIFadd{Reich, }\DIFaddend P. B.\DIFdelbegin \DIFdel{Reich, and }\DIFdelend \DIFaddbegin \DIFadd{, \& Wright, }\DIFaddend I. J. \DIFdelbegin \DIFdel{Wright. 2013.
}\href{https://doi.org/10.1111/nph.12345}{\DIFdel{Understanding ecological
variation across species: }%DIFDELCMD < {%%%
\DIFdel{Area-based}%DIFDELCMD < } %%%
\DIFdel{vs mass-based expression of leaf
traits}}%DIFAUXCMD
\DIFdel{. New Phytologist 199: }\DIFdelend \DIFaddbegin \DIFadd{(2013). Understanding
ecological variation across species: }{\DIFadd{Area-based}} \DIFadd{vs mass-based
expression of leaf traits. }\emph{\DIFadd{New Phytologist}}\DIFadd{, }\emph{199}(2)\DIFadd{,
}\DIFaddend 322--323. \DIFaddbegin \url{https://doi.org/10.1111/nph.12345}
\DIFaddend 

\leavevmode\vadjust pre{\hypertarget{ref-Westoby2000}{}}%
Westoby, M., \DIFdelbegin \DIFdel{D. Warton, and }\DIFdelend \DIFaddbegin \DIFadd{Warton, D., \& Reich, }\DIFaddend P. B. \DIFdelbegin \DIFdel{Reich. 2000.
}\href{https://doi.org/10.1086/303346}{\DIFdel{The }%DIFDELCMD < {%%%
\DIFdel{Time Value}%DIFDELCMD < } %%%
\DIFdel{of }%DIFDELCMD < {%%%
\DIFdel{Leaf Area}%DIFDELCMD < }%%%
}%DIFAUXCMD
\DIFdel{.
The American Naturalist 155:}\DIFdelend \DIFaddbegin \DIFadd{(2000). The }{\DIFadd{Time Value}} \DIFadd{of
}{\DIFadd{Leaf Area}}\DIFadd{. }\emph{\DIFadd{The American Naturalist}}\DIFadd{, }\emph{155}(5)\DIFadd{, }\DIFaddend 649--656.
\DIFaddbegin \url{https://doi.org/10.1086/303346}
\DIFaddend 

\leavevmode\vadjust pre{\hypertarget{ref-Williams1989}{}}%
Williams, K., \DIFaddbegin \DIFadd{Field, }\DIFaddend C. B.\DIFdelbegin \DIFdel{Field, and }\DIFdelend \DIFaddbegin \DIFadd{, \& Mooney, }\DIFaddend H. A. \DIFdelbegin \DIFdel{Mooney. 1989.
}\href{https://doi.org/10.1086/284910}{\DIFdel{Relationships }%DIFDELCMD < {%%%
\DIFdel{Among Leaf
Construction Cost}%DIFDELCMD < }%%%
\DIFdel{, }%DIFDELCMD < {%%%
\DIFdel{Leaf Longevity}%DIFDELCMD < }%%%
\DIFdel{, and }%DIFDELCMD < {%%%
\DIFdel{Light Environment}%DIFDELCMD < } %%%
\DIFdel{in
}%DIFDELCMD < {%%%
\DIFdel{Rain-Forest Plants}%DIFDELCMD < } %%%
\DIFdel{of the }%DIFDELCMD < {%%%
\DIFdel{Genus Piper}%DIFDELCMD < }%%%
}%DIFAUXCMD
\DIFdel{. The American Naturalist
133:}\DIFdelend \DIFaddbegin \DIFadd{(1989). Relationships
}{\DIFadd{Among Leaf Construction Cost}}\DIFadd{, }{\DIFadd{Leaf Longevity}}\DIFadd{, and }{\DIFadd{Light
Environment}} \DIFadd{in }{\DIFadd{Rain-Forest Plants}} \DIFadd{of the }{\DIFadd{Genus Piper}}\DIFadd{. }\emph{\DIFadd{The
American Naturalist}}\DIFadd{, }\emph{133}(2)\DIFadd{, }\DIFaddend 198--211.
\DIFaddbegin \url{https://doi.org/10.1086/284910}
\DIFaddend 

\leavevmode\vadjust pre{\hypertarget{ref-Wright2005}{}}%
Wright, I. J., \DIFaddbegin \DIFadd{Reich, }\DIFaddend P. B.\DIFdelbegin \DIFdel{Reich, }\DIFdelend \DIFaddbegin \DIFadd{, Cornelissen, }\DIFaddend J. H. C.\DIFdelbegin \DIFdel{Cornelissen, }\DIFdelend \DIFaddbegin \DIFadd{, Falster, }\DIFaddend D. S.\DIFdelbegin \DIFdel{Falster,
}\DIFdelend \DIFaddbegin \DIFadd{,
Groom, }\DIFaddend P. K.\DIFdelbegin \DIFdel{Groom, K. Hikosaka, W. Lee, }\DIFdelend \DIFaddbegin \DIFadd{, Hikosaka, K., Lee, W., Lusk, }\DIFaddend C. H.\DIFdelbegin \DIFdel{Lusk, }\DIFdelend \DIFaddbegin \DIFadd{, Niinemets, }\DIFaddend Ü.\DIFdelbegin \DIFdel{Niinemets,
J. Oleksyn, N.
Osada, H. Poorter, }\DIFdelend \DIFaddbegin \DIFadd{,
Oleksyn, J., Osada, N., Poorter, H., Warton, }\DIFaddend D. I.\DIFdelbegin \DIFdel{Warton, and M.
Westoby. 2005.
}\href{https://doi.org/10.1111/j.1466-822x.2005.00172.x}{\DIFdel{Modulation of
leaf economic traits and trait relationships by climate}}%DIFAUXCMD
\DIFdel{. Global Ecology
and Biogeography 14:}\DIFdelend \DIFaddbegin \DIFadd{, \& Westoby, M.
(2005). Modulation of leaf economic traits and trait relationships by
climate. }\emph{\DIFadd{Global Ecology and Biogeography}}\DIFadd{, }\emph{14}(5)\DIFadd{, }\DIFaddend 411--421.
\DIFaddbegin \url{https://doi.org/10.1111/j.1466-822x.2005.00172.x}
\DIFaddend 

\leavevmode\vadjust pre{\hypertarget{ref-Wright2004a}{}}%
Wright, I. J., \DIFaddbegin \DIFadd{Reich, }\DIFaddend P. B.\DIFdelbegin \DIFdel{Reich, M. Westoby, }\DIFdelend \DIFaddbegin \DIFadd{, Westoby, M., Ackerly, }\DIFaddend D. D.\DIFdelbegin \DIFdel{Ackerly, Z. Baruch,
F.
Bongers, }\DIFdelend \DIFaddbegin \DIFadd{, Baruch, Z.,
Bongers, F., Cavender-Bares, J., Chapin, T., Cornellssen, }\DIFaddend J. \DIFdelbegin \DIFdel{Cavender-Bares, T. Chapin, J. }\DIFdelend H. C.\DIFdelbegin \DIFdel{Cornellssen,
M. Diemer, J. Flexas, E. Garnier, }\DIFdelend \DIFaddbegin \DIFadd{,
Diemer, M., Flexas, J., Garnier, E., Groom, }\DIFaddend P. K.\DIFdelbegin \DIFdel{Groom, J. Gulias, K. Hikosaka, }\DIFdelend \DIFaddbegin \DIFadd{, Gulias, J., Hikosaka,
K., Lamont, }\DIFaddend B. B.\DIFdelbegin \DIFdel{Lamont, T. }\DIFdelend \DIFaddbegin \DIFadd{, Lee, T., }\DIFaddend Lee, W.\DIFdelbegin \DIFdel{Lee, C. Lusk, J. J. Midgley, M. L. Navas, Ü.
Niinemets, J. Oleksyn, H. Osada, H. Poorter, P. Pool, L. Prior, V. I.
Pyankov, C.Roumet, S. C. Thomas, M. G. Tjoelker, E. J. Veneklaas, and
R.
Villar. 2004. }\href{https://doi.org/10.1038/nature02403}{\DIFdel{The
worldwide leaf economics spectrum}}%DIFAUXCMD
\DIFdel{. Nature 428:}\DIFdelend \DIFaddbegin \DIFadd{, Lusk, C., \ldots{} Villar, R.
(2004). The worldwide leaf economics spectrum. }\emph{\DIFadd{Nature}}\DIFadd{,
}\emph{428}(6985)\DIFadd{, }\DIFaddend 821--827. \DIFaddbegin \url{https://doi.org/10.1038/nature02403}
\DIFaddend 

\leavevmode\vadjust pre{\hypertarget{ref-Wright2003}{}}%
Wright, S. J., \DIFdelbegin \DIFdel{V. Horlyck, Y.Basset}\DIFdelend \DIFaddbegin \DIFadd{Horlyck}\DIFaddend , \DIFaddbegin \DIFadd{V., Basset, Y., Barrios, }\DIFaddend H.\DIFdelbegin \DIFdel{Barrios, }\DIFdelend \DIFaddbegin \DIFadd{, Bethancourt, }\DIFaddend A.\DIFdelbegin \DIFdel{Bethancourt, }\DIFdelend \DIFaddbegin \DIFadd{,
Bohlman, }\DIFaddend S.\DIFdelbegin \DIFdel{Bohlman}\DIFdelend , \DIFdelbegin \DIFdel{G.}\DIFdelend Gilbert, G.\DIFdelbegin \DIFdel{Goldstein, }\DIFdelend \DIFaddbegin \DIFadd{, Goldstein, G., Graham, }\DIFaddend E. A.\DIFdelbegin \DIFdel{Graham, K. Kitajima,
}\DIFdelend \DIFaddbegin \DIFadd{, Kitajima, K.,
Lerdau, }\DIFaddend M. T.\DIFdelbegin \DIFdel{Lerdau, }\DIFdelend \DIFaddbegin \DIFadd{, Meinzer, }\DIFaddend F. C.\DIFdelbegin \DIFdel{Meinzer, F. Ødegaard, }\DIFdelend \DIFaddbegin \DIFadd{, Ødegaard, F., Reynolds, }\DIFaddend D. R.\DIFdelbegin \DIFdel{Reynolds, }\DIFdelend \DIFaddbegin \DIFadd{, Roubik, }\DIFaddend D.
W.\DIFdelbegin \DIFdel{Roubik, S.
Sakai, M. Samaniego, }\DIFdelend \DIFaddbegin \DIFadd{, Sakai, S., Samaniego, M., Sparks, }\DIFaddend J. P.\DIFdelbegin \DIFdel{Sparks, S. }\DIFdelend \DIFaddbegin \DIFadd{, }\DIFaddend Van Bael, \DIFdelbegin \DIFdel{K.Winter, and G. Zotz. 2003. }\DIFdelend \DIFaddbegin \DIFadd{S., \ldots{}
Zotz, G. (2003). }\DIFaddend Tropical {Canopy Biology Program}, {Republic} of
{Panama}. \DIFdelbegin \DIFdel{Pages
137--155 }\emph{\DIFdel{in}} %DIFAUXCMD
\DIFdelend \DIFaddbegin \DIFadd{In }\DIFaddend Y. Basset, V. Horlyck, \DIFdelbegin \DIFdel{and }\DIFdelend \DIFaddbegin \DIFadd{\& }\DIFaddend S. J. Wright \DIFdelbegin \DIFdel{, editors. Studying }%DIFDELCMD < {%%%
\DIFdel{Forest Canopies}%DIFDELCMD < } %%%
\DIFdel{from }%DIFDELCMD < {%%%
\DIFdel{Above}%DIFDELCMD < }%%%
\DIFdel{: }%DIFDELCMD < {%%%
\DIFdel{The International Canopy Crane
Network}%DIFDELCMD < }%%%
\DIFdel{.
}%DIFDELCMD < {%%%
\DIFdel{Panama}%DIFDELCMD < }%%%
\DIFdel{.
}\DIFdelend \DIFaddbegin \DIFadd{(Eds.),
}\emph{\DIFadd{Studying }{\DIFadd{Forest Canopies}} \DIFadd{from }{\DIFadd{Above}}\DIFadd{: }{\DIFadd{The International Canopy
Crane Network}}} \DIFadd{(pp. 137--155).
}\DIFaddend 

\leavevmode\vadjust pre{\hypertarget{ref-Wullschleger2014}{}}%
Wullschleger, S. D., \DIFaddbegin \DIFadd{Epstein, }\DIFaddend H. E.\DIFdelbegin \DIFdel{Epstein, }\DIFdelend \DIFaddbegin \DIFadd{, Box, }\DIFaddend E. O.\DIFdelbegin \DIFdel{Box, }\DIFdelend \DIFaddbegin \DIFadd{, Euskirchen, }\DIFaddend E. S.\DIFdelbegin \DIFdel{Euskirchen,
S.
Goswami, }\DIFdelend \DIFaddbegin \DIFadd{,
Goswami, S., Iversen, }\DIFaddend C. M.\DIFdelbegin \DIFdel{Iversen, J. Kattge, }\DIFdelend \DIFaddbegin \DIFadd{, Kattge, J., Norby, }\DIFaddend R. J.\DIFdelbegin \DIFdel{Norby, }\DIFdelend \DIFaddbegin \DIFadd{, Van Bodegom, }\DIFaddend P.
M.\DIFdelbegin \DIFdel{Van Bodegom, and
X. Xu. 2014. }\href{https://doi.org/10.1093/aob/mcu077}{\DIFdel{Plant functional
types in }%DIFDELCMD < {%%%
\DIFdel{Earth}%DIFDELCMD < } %%%
\DIFdel{system models: }%DIFDELCMD < {%%%
\DIFdel{Past}%DIFDELCMD < } %%%
\DIFdel{experiences and future directions
for application of dynamic vegetation models in high-latitude
ecosystems}}%DIFAUXCMD
\DIFdel{. Annals of Botany 114:
}\DIFdelend \DIFaddbegin \DIFadd{, \& Xu, X. (2014). Plant functional types in }{\DIFadd{Earth}} \DIFadd{system models:
}{\DIFadd{Past}} \DIFadd{experiences and future directions for application of dynamic
vegetation models in high-latitude ecosystems. }\emph{\DIFadd{Annals of Botany}}\DIFadd{,
}\emph{114}(1)\DIFadd{, }\DIFaddend 1--16. \DIFaddbegin \url{https://doi.org/10.1093/aob/mcu077}
\DIFaddend 

\leavevmode\vadjust pre{\hypertarget{ref-Xu2017}{}}%
Xu, X., \DIFdelbegin \DIFdel{D. Medvigy, S.}\DIFdelend \DIFaddbegin \DIFadd{Medvigy, D., }\DIFaddend Joseph Wright, \DIFaddbegin \DIFadd{S., Kitajima, }\DIFaddend K.\DIFdelbegin \DIFdel{Kitajima, J.Wu, }\DIFdelend \DIFaddbegin \DIFadd{, Wu, J., Albert, }\DIFaddend L.
P.\DIFdelbegin \DIFdel{Albert, }\DIFdelend \DIFaddbegin \DIFadd{, Martins, }\DIFaddend G. A.\DIFdelbegin \DIFdel{Martins, }\DIFdelend \DIFaddbegin \DIFadd{, Saleska, }\DIFaddend S. R.\DIFdelbegin \DIFdel{Saleska, and }\DIFdelend \DIFaddbegin \DIFadd{, \& Pacala, }\DIFaddend S. W. \DIFdelbegin \DIFdel{Pacala. 2017.
}\href{https://doi.org/10.1111/ele.12804}{\DIFdel{Variations of leaf longevity in
tropical moist forests predicted by a trait-driven carbon optimality
model}}%DIFAUXCMD
\DIFdel{. Ecology Letters 20:}\DIFdelend \DIFaddbegin \DIFadd{(2017). Variations
of leaf longevity in tropical moist forests predicted by a trait-driven
carbon optimality model. }\emph{\DIFadd{Ecology Letters}}\DIFadd{, }\emph{20}(9)\DIFadd{,
}\DIFaddend 1097--1106. \DIFaddbegin \url{https://doi.org/10.1111/ele.12804}
\DIFaddend 

\end{CSLReferences}

\newpage

\hypertarget{tables}{%
\section{Tables}\label{tables}}

Table 1. Summary of the model comparison. Lower approximate
leave-one-out cross-validation information criterion (LOOIC) indicates
better predictive accuracy. The models here are the five moodel forms
representing LL: (1) LMA, (2) \DIFdelbegin \DIFdel{LMAp }\DIFdelend \DIFaddbegin \DIFadd{LMAm }\DIFaddend and LMAs, (3) LMA and light, (4)
\DIFdelbegin \DIFdel{LMAp}\DIFdelend \DIFaddbegin \DIFadd{LMAm}\DIFaddend , LMAs and light, and (5) \DIFdelbegin \DIFdel{LMAp}\DIFdelend \DIFaddbegin \DIFadd{LMAm}\DIFaddend , LMAs/LT and light, with four types
of parameter restrictions: (a) \(\alpha_s = 0\) and \DIFdelbegin \DIFdel{\(\beta_p = 0\)}\DIFdelend \DIFaddbegin \DIFadd{\(\beta_m = 0\)}\DIFaddend , (b)
\DIFdelbegin \DIFdel{\(\beta_p = 0\)}\DIFdelend \DIFaddbegin \DIFadd{\(\beta_m = 0\)}\DIFaddend , (c) \(\alpha_s = 0\), and (d) \DIFdelbegin \DIFdel{\(\alpha_p > \alpha_s\)
and \(\beta_s > \beta_p\)}\DIFdelend \DIFaddbegin \DIFadd{\(\alpha_m > \alpha_s\)
and \(\beta_s > \beta_m\)}\DIFaddend .

\begin{table}
\centering
\begin{tabular}{lrr}
\toprule
Model & N & LOOIC\\
\midrule
\addlinespace[0.3em]
\multicolumn{3}{l}{\textbf{GLOPNET}}\\
\hspace{1em}2b & 198 & 721.2\\
\hspace{1em}2a & 198 & 759.8\\
\hspace{1em}2d & 198 & 761.8\\
\hspace{1em}2c & 198 & 764.2\\
\hspace{1em}1 & 198 & 766.4\\
\addlinespace[0.3em]
\multicolumn{3}{l}{\textbf{Panama}}\\
\hspace{1em}4a & 106 & 461.0\\
\hspace{1em}5a & 106 & 472.3\\
\hspace{1em}2c & 106 & 478.2\\
\hspace{1em}4c & 106 & 487.2\\
\hspace{1em}2a & 106 & 500.3\\
\hspace{1em}5c & 106 & 502.1\\
\hspace{1em}4b & 106 & 508.2\\
\hspace{1em}4d & 106 & 508.6\\
\hspace{1em}3 & 106 & 510.1\\
\hspace{1em}2b & 106 & 526.7\\
\hspace{1em}2d & 106 & 526.9\\
\hspace{1em}1 & 106 & 536.8\\
\bottomrule
\end{tabular}
\end{table}

\newpage

Table 2. Posterior means {[}95\% credible interval{]} of parameters for
the best models. Model 1b (\DIFdelbegin \DIFdel{LMAp }\DIFdelend \DIFaddbegin \DIFadd{LMAm }\DIFaddend and LMAs) and Model 4a (\DIFdelbegin \DIFdel{LMAp}\DIFdelend \DIFaddbegin \DIFadd{LMAm}\DIFaddend , LMAs and
light) are shown for GLOPNET and Panama data, respectively. Bold values
are significantly different from zero based on the 95\% credible
interval.

\begin{longtable}[]{@{}
  >{\raggedright\arraybackslash}p{(\columnwidth - 4\tabcolsep) * \real{0.4271}}
  >{\raggedright\arraybackslash}p{(\columnwidth - 4\tabcolsep) * \real{0.2917}}
  >{\raggedright\arraybackslash}p{(\columnwidth - 4\tabcolsep) * \real{0.2812}}@{}}
\toprule()
\begin{minipage}[b]{\linewidth}\raggedright
Parameters
\end{minipage} & \begin{minipage}[b]{\linewidth}\raggedright
GLOPNET
\end{minipage} & \begin{minipage}[b]{\linewidth}\raggedright
Panama
\end{minipage} \\
\midrule()
\endhead
Effect of \DIFdelbegin \DIFdel{LMAp }\DIFdelend \DIFaddbegin \DIFadd{LMAm }\DIFaddend on \emph{A}\textsubscript{area} (\(\alpha_p\)) &
\textbf{0.282 {[}0.204, 0.36{]}} & \textbf{0.578 {[}0.429, 0.723{]}} \\
Effect of LMAs on \emph{A}\textsubscript{area} (\(\alpha_s\)) &
\textbf{-0.131 {[}-0.201, -0.058{]}} & - \\
Effect of LMAs on LL (\(\beta_s\)) & \textbf{0.888 {[}0.761, 1.018{]}} &
\textbf{0.352 {[}0.17, 0.531{]}} \\
Effect of \DIFdelbegin \DIFdel{LMAp }\DIFdelend \DIFaddbegin \DIFadd{LMAm }\DIFaddend on \emph{R}\textsubscript{area} (\(\gamma_p\)) &
\textbf{0.289 {[}0.202, 0.38{]}} & \textbf{0.612 {[}0.388, 0.823{]}} \\
Effect of LMAs on \emph{R}\textsubscript{area} (\(\gamma_s\)) & 0.035
{[}-0.045, 0.115{]} & 0.038 {[}-0.25, 0.351{]} \\
Effect of light on LL (\(\theta\)) & - & \textbf{-0.83 {[}-1.116,
-0.538{]}} \\
\bottomrule()
\end{longtable}

\newpage

\DIFdelbegin %DIFDELCMD < \hypertarget{figures}{%
%DIFDELCMD < \section{Figures}\label{figures}}
%DIFDELCMD < 

%DIFDELCMD < %%%
\DIFdelend \begin{figure}

{\centering \DIFdelbeginFL %DIFDELCMD < \includegraphics{/home/mattocci/LMAms/figs/hypo.png}
%DIFDELCMD < %%%
\DIFdelendFL \DIFaddbeginFL \includegraphics{/Users/mattocci/Dropbox/MS/LMAms/figs/hypo.png}
\DIFaddendFL 

}

\caption{\label{fig-Hplt}Example of a two-dimensional functional space
that can result in either a two- or one-dimensional trait space,
depending on how metabolic traits are normalized. (A) Hypothetical
independent variation in two leaf mass per area (LMA) components:
metabolic leaf mass per area (\DIFdelbeginFL \DIFdelFL{LMAp}\DIFdelendFL \DIFaddbeginFL \DIFaddFL{LMAm}\DIFaddendFL , which largely determines per-area
values of photosynthesis, respiration, and nutrient concentrations) and
structural leaf mass per area (LMAs, which determines leaf toughness but
has little effect on metabolic traits). (B) Two-dimensional relationship
between photosynthetic capacity (\emph{A}\textsubscript{max}) per-unit
leaf area and total LMA (equal to \DIFdelbeginFL \DIFdelFL{LMAp }\DIFdelendFL \DIFaddbeginFL \DIFaddFL{LMAm }\DIFaddendFL + LMAs). (C) One-dimensional
relationship between \emph{A}\textsubscript{max} per-unit leaf mass and
total LMA. \DIFdelbeginFL \DIFdelFL{LMAp }\DIFdelendFL \DIFaddbeginFL \DIFaddFL{LMAm }\DIFaddendFL and LMAs are simulated from a hypothetical scenario of
lognormal distributions with medians of 80 and standard deviations of
exp(0.8) and exp(0.7), and zero covariance. Variation in
\emph{A}\textsubscript{max} values are derived from our analysis of the
GLOPNET database.
\DIFdelbeginFL \DIFdelFL{\(A_{\mathrm{area} \, i}=1.81 \times \mathrm{LMAp}_i^{0.28}\mathrm{LMAs}^{-0.13}\epsilon_i\)}\DIFdelendFL \DIFaddbeginFL \DIFaddFL{\(A_{\mathrm{area} \, i}=1.77 \times \mathrm{LMAm}_i^{0.28}\mathrm{LMAs}^{-0.13}\epsilon_i\)}\DIFaddendFL ,
where \(\epsilon_i\) is log-normally distributed with log-mean with 0,
log-scale parameter with 0.31 (see Methods and Results).}

\end{figure}

\begin{figure}

{\centering \DIFdelbeginFL %DIFDELCMD < \includegraphics{/home/mattocci/LMAms/figs/gl_point.png}
%DIFDELCMD < %%%
\DIFdelendFL \DIFaddbeginFL \includegraphics{/Users/mattocci/Dropbox/MS/LMAms/figs/gl_point.png}
\DIFaddendFL 

}

\caption{\label{fig-GLplt}Observed and estimated leaf-trait
relationships in the global GLOPNET dataset. Estimates are from Model 2b
(Table 1). Leaf life span (LL), net photosynthetic rate per unit leaf
area (\emph{A}\textsubscript{area}), and dark respiration rate per unit
leaf area (\emph{R}\textsubscript{area}) are plotted against observed
LMA, posterior means of \DIFdelbeginFL \DIFdelFL{LMAp }\DIFdelendFL \DIFaddbeginFL \DIFaddFL{LMAm }\DIFaddendFL and LMAs. Pearson correlation coefficients
(\emph{r}) for LMA (left column) and posterior means of Pearson
correlation coefficients (\(\bar{r}\)) or partial correlation
coefficients (\(\bar{\rho}\)) of \DIFdelbeginFL \DIFdelFL{LMAp }\DIFdelendFL \DIFaddbeginFL \DIFaddFL{LMAm }\DIFaddendFL (middle column) and LMAs (right
column) are shown. Note that LL was modeled by LMAs alone.}

\end{figure}

\newpage

\begin{figure}

{\centering \DIFdelbeginFL %DIFDELCMD < \includegraphics{/home/mattocci/LMAms/figs/pa_point.png}
%DIFDELCMD < %%%
\DIFdelendFL \DIFaddbeginFL \includegraphics{/Users/mattocci/Dropbox/MS/LMAms/figs/pa_point.png}
\DIFaddendFL 

}

\caption{\label{fig-PAplt}Observed and estimated leaf-trait
relationships in the Panama dataset. Estimates are from Model 4a (Table
1). Details as \DIFdelbeginFL \DIFdelFL{in }\DIFdelendFL \DIFaddbeginFL \DIFaddFL{for }\DIFaddendFL Fig.~\ref{fig-GLplt}. Results for other LL models are
summarized in Table SX. Note that \emph{A}\textsubscript{area} and LL
were modeled by \DIFdelbeginFL \DIFdelFL{LMAp }\DIFdelendFL \DIFaddbeginFL \DIFaddFL{LMAm }\DIFaddendFL and LMAs alone, respectively. \DIFaddbeginFL \DIFaddFL{The results shown
here include all leaves in the Panama dataset.}\DIFaddendFL }

\end{figure}

\newpage

\begin{figure}

{\centering \DIFdelbeginFL %DIFDELCMD < \includegraphics{/home/mattocci/LMAms/figs/pa_point_par_ll.png}
%DIFDELCMD < %%%
\DIFdelendFL \DIFaddbeginFL \includegraphics{/Users/mattocci/Dropbox/MS/LMAms/figs/pa_point_par_ll.png}
\DIFaddendFL 

}

\caption{\label{fig-LLplt}Relationship between residuals of leaf
lifespan (LL) against light and residuals of LMAs against light \DIFaddbeginFL \DIFaddFL{in the
Panama dataset}\DIFaddendFL . The dashed line indicates the 1:1 relationship. The
posterior means of the partial correlation coefficient (\(\bar{\rho}\))
is shown. \DIFaddbeginFL \DIFaddFL{The results shown here include all leaves in the Panama
dataset.}\DIFaddendFL }

\end{figure}

\newpage

\begin{figure}

{\centering \DIFdelbeginFL %DIFDELCMD < \includegraphics{/home/mattocci/LMAms/figs/vpart_intra.png}
%DIFDELCMD < %%%
\DIFdelendFL \DIFaddbeginFL \includegraphics{/Users/mattocci/Dropbox/MS/LMAms/figs/vpart_intra.png}
\DIFaddendFL 

}

\caption{\label{fig-vpart}Variance partitioning on LMA components
between and within leaf habits (evergreen vs.~deciuous) for the GLOPNET
dataset and the Panama dataset, and between and within sites (wet
vs.~dry) and light (sun vs.~shade) for the Panama dataset. \DIFaddbeginFL \DIFaddFL{To isolate
the effects of intraspecific variation (i.e., plastic responses to
light), the Panama results shown here only include species for which
both sun and shade leaves were available.}\DIFaddendFL }

\end{figure}

\newpage

\begin{figure}

{\centering \DIFdelbeginFL %DIFDELCMD < \includegraphics{/home/mattocci/LMAms/figs/box_de.png}
%DIFDELCMD < %%%
\DIFdelendFL \DIFaddbeginFL \includegraphics{/Users/mattocci/Dropbox/MS/LMAms/figs/box_de.png}
\DIFaddendFL 

}

\caption{\label{fig-boxplt_de}Boxplots comparing leaf mass per area
(LMA) and posterior medians of photosynthetic and structural LMA
components (\DIFdelbeginFL \DIFdelFL{LMAp }\DIFdelendFL \DIFaddbeginFL \DIFaddFL{LMAm }\DIFaddendFL and LMAs, respectively) across deciduous (D) and
evergreen (E) leaves in the GLOPNET dataset (a) and in the Panama
dataset (b). The center line in each box indicates the median, upper and
lower box edges indicate the interquartile range, whiskers show 1.5
times the interquartile range, and points are outliers. Groups sharing
the same letters are not significantly different (P \textgreater{} 0.05;
t-tests). \DIFdelbeginFL \DIFdelFL{To isolate the effects of intraspecific variation (i.e.,
plastic responses to light), the }\DIFdelendFL \DIFaddbeginFL \DIFaddFL{The }\DIFaddendFL Panama results \DIFdelbeginFL \DIFdelFL{shown here }\DIFdelendFL only include species for which both sun and
shade leaves were available\DIFdelbeginFL \DIFdelFL{(}\textbf{\DIFdelFL{revise this text}}%DIFAUXCMD
\DIFdelFL{)}\DIFdelendFL . Qualitatively similar results were obtained
when all Panama species were included (Fig. S\ref{fig-box_inter}). Note
that the vertical axis is on a log\textsubscript{10} scale.}

\end{figure}

\newpage

\begin{figure}

{\centering \DIFdelbeginFL %DIFDELCMD < \includegraphics{/home/mattocci/LMAms/figs/box_pa.png}
%DIFDELCMD < %%%
\DIFdelendFL \DIFaddbeginFL \includegraphics{/Users/mattocci/Dropbox/MS/LMAms/figs/box_pa.png}
\DIFaddendFL 

}

\caption{\label{fig-boxplt_pa}Boxplots comparing leaf mass per area
(LMA) and posterior medians of photosynthetic and structural LMA
components (\DIFdelbeginFL \DIFdelFL{LMAp }\DIFdelendFL \DIFaddbeginFL \DIFaddFL{LMAm }\DIFaddendFL and LMAs, respectively across sites (wet and dry) and
canopy strata (sun and shade) in the Panama dataset (bottom). \DIFdelbeginFL \DIFdelFL{To isolate
the effects of intraspecific variation (i.e., plastic responses to
light), the }\DIFdelendFL \DIFaddbeginFL \DIFaddFL{The }\DIFaddendFL Panama
results \DIFdelbeginFL \DIFdelFL{shown here }\DIFdelendFL only include species for which both sun and shade leaves were
available. Qualitatively similar results were obtained when all Panama
species were included (Fig. S\ref{fig-box_inter}). Details as for
Fig.~\ref{fig-boxplt_de}\DIFaddbeginFL \DIFaddFL{.}\DIFaddendFL }

\end{figure}

\newpage

\begin{figure}

{\centering \DIFdelbeginFL %DIFDELCMD < \includegraphics{/home/mattocci/LMAms/figs/mass_prop_mv.png}
%DIFDELCMD < %%%
\DIFdelendFL \DIFaddbeginFL \includegraphics{/Users/mattocci/Dropbox/MS/LMAms/figs/mass_prop_mv.png}
\DIFaddendFL 

}

\caption{\label{fig-massplt}The relationships between mass dependency
(\emph{b} in Eq.~\ref{eq-var}) and relative variance in LMAs to LMA for
the global GLOPNET dataset, sun leaves in Panama, and shade leaves in
Panama. Solid lines indicate simulated means and shaded regions indicate
95\% CI. Each symbol indicates the estimated values from the empirical
data. The photosynthetic rate (\emph{A}\textsubscript{max}) is primarily
mass-dependent (\emph{b} \textgreater{} 0.5), primarily area-dependent
(0.5 \textgreater{} \emph{b} \textgreater{} 0) and purely area-dependent
(\emph{b} = 0) (\protect\DIFdelbeginFL %DIFDELCMD < \hyperlink{ref-Osnas2018}{Osnas et al. 2018}%%%
\DIFdelendFL \DIFaddbeginFL \hyperlink{ref-Osnas2018}{Osnas et al., 2018}\DIFaddendFL ).
Note that when \emph{b} \textgreater{} 1, \emph{A}\textsubscript{area}
dramatically increases with LMA, which is not realistic.}

\end{figure}

\newpage

\begin{figure}

{\centering \DIFdelbeginFL %DIFDELCMD < \includegraphics{/home/mattocci/LMAms/figs/pa_point_npc.png}
%DIFDELCMD < %%%
\DIFdelendFL \DIFaddbeginFL \includegraphics{/Users/mattocci/Dropbox/MS/LMAms/figs/pa_point_npc.png}
\DIFaddendFL 

}

\caption{\label{fig-PA-NPC}Measured traits related to photosynthesis and
metabolism traits (nitrogen and phosphorus per-unit leaf area;
\emph{N}\textsubscript{area} and \emph{P}\textsubscript{area}) are
strongly correlated with estimates (posterior medians) of the
photosynthetic LMA component (\DIFdelbeginFL \DIFdelFL{LMAp}\DIFdelendFL \DIFaddbeginFL \DIFaddFL{LMAm}\DIFaddendFL ), and a measured structural trait
(cellulose per-unit leaf area; \emph{CL}\textsubscript{area}) is
strongly correlated with estimates of the structural LMA component
(LMAs) for the Panama dataset. Note that sun and shade leaves align
along a single relationship for \emph{CL}\textsubscript{area} vs.~LMAs,
but not for \emph{CL}\textsubscript{area} vs.~LMA or \DIFdelbeginFL \DIFdelFL{LMAp}\DIFdelendFL \DIFaddbeginFL \DIFaddFL{LMAm}\DIFaddendFL .
\emph{N}\textsubscript{area}, \emph{P}\textsubscript{area}, and
\emph{CL}\textsubscript{area} data were not used to fit the models, and
are presented here as independent support for the model results. Pearson
correlation coefficients (\emph{r}) for LMA (left column) and partial
correlation coefficients (\(\rho\)) of \DIFdelbeginFL \DIFdelFL{LMAp }\DIFdelendFL \DIFaddbeginFL \DIFaddFL{LMAm }\DIFaddendFL (middle column) and LMAs
(right column) are shown. Analogous results were obtained for
\emph{N}\textsubscript{area} and \emph{P}\textsubscript{area} for
GLOPNET (Fig. S\ref{fig-glnp}). \DIFdelbeginFL \DIFdelFL{Results for other LL models are reported
in Table SX. Analogous }\DIFdelendFL \DIFaddbeginFL \DIFaddFL{The }\DIFaddendFL results \DIFdelbeginFL \DIFdelFL{were obtained }\DIFdelendFL \DIFaddbeginFL \DIFaddFL{shown here include all
leaves }\DIFaddendFL for \DIFaddbeginFL \DIFaddFL{which }\DIFaddendFL \emph{N}\textsubscript{area}\DIFdelbeginFL \DIFdelFL{and }\DIFdelendFL \DIFaddbeginFL \DIFaddFL{,
}\DIFaddendFL \emph{P}\textsubscript{area} \DIFdelbeginFL \DIFdelFL{for
GLOPNET (Fig. S\ref{fig-glnp})}\DIFdelendFL \DIFaddbeginFL \DIFaddFL{and }\emph{\DIFaddFL{CL}}\DIFaddFL{\textsubscript{area} are
available}\DIFaddendFL .}

\end{figure}

\end{document}

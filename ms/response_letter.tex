% Options for packages loaded elsewhere
\PassOptionsToPackage{unicode}{hyperref}
\PassOptionsToPackage{hyphens}{url}
\PassOptionsToPackage{dvipsnames,svgnames,x11names}{xcolor}
%
\documentclass[
  12pt,
  letterpaper,
  DIV=11,
  numbers=noendperiod]{scrartcl}

\usepackage{amsmath,amssymb}
\usepackage{iftex}
\ifPDFTeX
  \usepackage[T1]{fontenc}
  \usepackage[utf8]{inputenc}
  \usepackage{textcomp} % provide euro and other symbols
\else % if luatex or xetex
  \usepackage{unicode-math}
  \defaultfontfeatures{Scale=MatchLowercase}
  \defaultfontfeatures[\rmfamily]{Ligatures=TeX,Scale=1}
\fi
\usepackage{lmodern}
\ifPDFTeX\else  
    % xetex/luatex font selection
\fi
% Use upquote if available, for straight quotes in verbatim environments
\IfFileExists{upquote.sty}{\usepackage{upquote}}{}
\IfFileExists{microtype.sty}{% use microtype if available
  \usepackage[]{microtype}
  \UseMicrotypeSet[protrusion]{basicmath} % disable protrusion for tt fonts
}{}
\makeatletter
\@ifundefined{KOMAClassName}{% if non-KOMA class
  \IfFileExists{parskip.sty}{%
    \usepackage{parskip}
  }{% else
    \setlength{\parindent}{0pt}
    \setlength{\parskip}{6pt plus 2pt minus 1pt}}
}{% if KOMA class
  \KOMAoptions{parskip=half}}
\makeatother
\usepackage{xcolor}
\setlength{\emergencystretch}{3em} % prevent overfull lines
\setcounter{secnumdepth}{-\maxdimen} % remove section numbering
% Make \paragraph and \subparagraph free-standing
\makeatletter
\ifx\paragraph\undefined\else
  \let\oldparagraph\paragraph
  \renewcommand{\paragraph}{
    \@ifstar
      \xxxParagraphStar
      \xxxParagraphNoStar
  }
  \newcommand{\xxxParagraphStar}[1]{\oldparagraph*{#1}\mbox{}}
  \newcommand{\xxxParagraphNoStar}[1]{\oldparagraph{#1}\mbox{}}
\fi
\ifx\subparagraph\undefined\else
  \let\oldsubparagraph\subparagraph
  \renewcommand{\subparagraph}{
    \@ifstar
      \xxxSubParagraphStar
      \xxxSubParagraphNoStar
  }
  \newcommand{\xxxSubParagraphStar}[1]{\oldsubparagraph*{#1}\mbox{}}
  \newcommand{\xxxSubParagraphNoStar}[1]{\oldsubparagraph{#1}\mbox{}}
\fi
\makeatother


\providecommand{\tightlist}{%
  \setlength{\itemsep}{0pt}\setlength{\parskip}{0pt}}\usepackage{longtable,booktabs,array}
\usepackage{calc} % for calculating minipage widths
% Correct order of tables after \paragraph or \subparagraph
\usepackage{etoolbox}
\makeatletter
\patchcmd\longtable{\par}{\if@noskipsec\mbox{}\fi\par}{}{}
\makeatother
% Allow footnotes in longtable head/foot
\IfFileExists{footnotehyper.sty}{\usepackage{footnotehyper}}{\usepackage{footnote}}
\makesavenoteenv{longtable}
\usepackage{graphicx}
\makeatletter
\def\maxwidth{\ifdim\Gin@nat@width>\linewidth\linewidth\else\Gin@nat@width\fi}
\def\maxheight{\ifdim\Gin@nat@height>\textheight\textheight\else\Gin@nat@height\fi}
\makeatother
% Scale images if necessary, so that they will not overflow the page
% margins by default, and it is still possible to overwrite the defaults
% using explicit options in \includegraphics[width, height, ...]{}
\setkeys{Gin}{width=\maxwidth,height=\maxheight,keepaspectratio}
% Set default figure placement to htbp
\makeatletter
\def\fps@figure{htbp}
\makeatother

\usepackage[default]{sourcesanspro}
\usepackage{sourcecodepro}
\usepackage{xcolor}
\usepackage{mdframed}
% Define blockquote colors
\definecolor{blockquote-border}{RGB}{221,221,221}
\definecolor{blockquote-text}{RGB}{119,119,119}
% Define custom blockquote environment
\newmdenv[
  rightline=false,
  bottomline=false,
  topline=false,
  linewidth=3pt,
  linecolor=blockquote-border,
  skipabove=\parskip,
  innerleftmargin=10pt,
  innerrightmargin=0pt,
  frametitlebackgroundcolor=gray!20
]{customblockquote}
\renewenvironment{quote}
  {\begin{customblockquote}\color{blockquote-text}\ignorespaces}
  {\end{customblockquote}}
\KOMAoption{captions}{tableheading}
\makeatletter
\@ifpackageloaded{caption}{}{\usepackage{caption}}
\AtBeginDocument{%
\ifdefined\contentsname
  \renewcommand*\contentsname{Table of contents}
\else
  \newcommand\contentsname{Table of contents}
\fi
\ifdefined\listfigurename
  \renewcommand*\listfigurename{List of Figures}
\else
  \newcommand\listfigurename{List of Figures}
\fi
\ifdefined\listtablename
  \renewcommand*\listtablename{List of Tables}
\else
  \newcommand\listtablename{List of Tables}
\fi
\ifdefined\figurename
  \renewcommand*\figurename{Figure}
\else
  \newcommand\figurename{Figure}
\fi
\ifdefined\tablename
  \renewcommand*\tablename{Table}
\else
  \newcommand\tablename{Table}
\fi
}
\@ifpackageloaded{float}{}{\usepackage{float}}
\floatstyle{ruled}
\@ifundefined{c@chapter}{\newfloat{codelisting}{h}{lop}}{\newfloat{codelisting}{h}{lop}[chapter]}
\floatname{codelisting}{Listing}
\newcommand*\listoflistings{\listof{codelisting}{List of Listings}}
\makeatother
\makeatletter
\makeatother
\makeatletter
\@ifpackageloaded{caption}{}{\usepackage{caption}}
\@ifpackageloaded{subcaption}{}{\usepackage{subcaption}}
\makeatother

\ifLuaTeX
  \usepackage{selnolig}  % disable illegal ligatures
\fi
\usepackage{bookmark}

\IfFileExists{xurl.sty}{\usepackage{xurl}}{} % add URL line breaks if available
\urlstyle{same} % disable monospaced font for URLs
\hypersetup{
  pdfauthor={Masatoshi Katabuchi},
  colorlinks=true,
  linkcolor={blue},
  filecolor={Maroon},
  citecolor={Blue},
  urlcolor={Blue},
  pdfcreator={LaTeX via pandoc}}


\author{Masatoshi Katabuchi}
\date{2025-01-22}

\begin{document}


Dear Dr.~Niinemets,

We hereby resubmit the manuscript (OECO-D-24-00601), ``Decomposing leaf
mass into metabolic and structural components explains divergent
patterns of trait variation within and among plant species''.

We thank the associate editor and the reviewers for their careful
attention to detail in our manuscript and for providing comments that
have greatly improved the study.

Based on the comments, we have updated the manuscript to reflect the
following:

\begin{enumerate}
\def\labelenumi{\arabic{enumi}.}
\tightlist
\item
\item
\item
\end{enumerate}

The programming code and the data supporting the findings of this study
will be deposited at Zenodo (https://zenodo.org) and Github
(https://github.com/mattocci27/leaf-disc). At the moment, we have
attached the stan code for our Bayesian model.

Below we provide detailed responses. The editing history is recorded in
a separate pdf file.

We appreciate the opportunity to revise and improve our manuscript, and
we thank the editors and reviewers for their time and consideration.

Sincerely,

Masatoshi Katabuchi

On behalf of the authors: Masatoshi Katabuchi, Kaoru Kitajima, S. Joseph
Wright, Sunshine A. Van Bael, Jeanne L. D. Osnas, and Jeremy W.
Lichstein

Corresponding author and contact information: Key Laboratory of Tropical
Forest Ecology, Xishuangbanna Tropical Botanical Garden, Chinese Academy
of Sciences, Mengla, Yunnan 666303, China

Email: katabuchi@xtbg.ac.cn; mattocci27@gmail.com

\newpage

\section{Responses}\label{responses}

\begin{quote}
Handling Editor: The referees were generally positive regarding
OECO-D-24-00601 by Katabuchi et al., both recommending the paper could
be acceptable with minor revisions. The revisions recommended center
around 1) revising the paper (mostly discussion) to broaden the appeal
to an audience beyond the trait-based reader; 2) more effectively
presenting the main, ``take-home'' message and its implications, and 3)
providing clarity on the physiological dimensions and their
implications. Novelty was listed as a concern, but both referees felt
the study should lead to good discussions and thus will be well received
by the trait-based audience. However, the authors could improve the
impact of the discussion by emphasizing better the strength of the
findings and their contributions to our understanding, while worrying
less about how this work confirms prior work. if the work is deemed
confirmatory, it could have difficulties getting priority for
publication. We look forward to seeing the revisions.
\end{quote}

\begin{quote}
Reviewer \#1: The manuscript develops a statistical method to separate
leaf mass per area (LMA) into structural (LMAs) and metabolic (LMAm)
components with the assumption that the metabolic component should be
more correlated with other metabolic traits such as area-based Amax and
Rdark and that the structural component should be more correlated with
leaf longevity. Such analysis was conducted for two data sets - GLOPNET
that only has sun-lit leaf traits and the BCI data that has both sun-lit
and shade leaf traits. Overall, evergreen species have lower LMAm
fraction than deciduous species and shade leaves have lower LMAm
fraction than conspecific sunlit leaves. The analysis argues for going
beyond the single-axis mass-based leaf economics spectrum and the need
to consider at least the structural and metabolic dimensions in
applications such as Earth System Modeling.
\end{quote}

Thank you so much for your positive comments and constructive feedback.

\begin{quote}
Overall, I enjoyed reading the manuscript, which attempts to decompose
LMA that is long known to be a very `complex' trait. My main concern is
that the derivation of LMAs and LMAm, albeit informative, is largely
statistical. What is the key evidence that the decomposition makes
physiological sense? In my understanding, the analysis with Narea,
Parea, CLarea (Fig. S7-S9) is somewhat relevant but those three traits
are also complex. And Fig. S8 shows LMA rather than LMAs or LMAm are
better correlated with Narea, Parea, and CLarea. I was wondering whether
it would be helpful to only include Aarea or Rarea in your statistical
fitting and see whether the results change\ldots{}
\end{quote}

Thank you for your thoughtful suggestion to test the model by separately
including Aarea or Rarea in the statistical fitting. We appreciate your
input and agree that such an analysis could provide additional insights.

However, since the estimated coefficient for the metabolic component of
LMAs (β\_m) was effectively zero for both the GLOPNET and Panama
datasets, we believe that separately including Aarea or Rarea would not
alter the current patterns. Our simultaneous analysis approach allows us
to account for the interdependencies between these traits, which we
believe is biologically meaningful.

Given the minor revision scope and the 10-day revision timeline, we have
focused on addressing key aspects raised by the reviewers, such as
improving the clarity of our discussion and broadening the appeal of our
findings. We hope that the current approach is acceptable within these
constraints.

\begin{quote}
Some minor comments throughout the text Line 108-111 This statement is
surely true. Adding degrees of freedom will always make the statistical
fit better\ldots.
\end{quote}

We used the term `predictive power' rather than `statistical fit' to
emphasize our focus on model performance and to account for potential
overfitting (Line XXX).

\begin{quote}
Line 220 typo in `abovµe'
\end{quote}

We have corrected the typo (Line XXX).

\begin{quote}
Line 771 Fig. 2, It is interesting that the `Unclassified' species take
the lower end of the LMAs distribution. Are they mostly grasses or
aquatic plants? It might be a worthy discussion point and an indirect
evidence of the physiological basis of derived LMAs
\end{quote}

The ``Unclassified'' group consists of 32 grass/herb species and 18
tree/shrub species. We have added a discussion on the potential
physiological basis of the derived LMAs in the manuscript (Line XXX).''

\begin{quote}
Reviewer\# 2: OECO-D-24-00601 is largely well written with few editorial
mistakes and points of poor clarity. The science is sound, but novelty
does not appear high as currently presented. The focus is to dissect the
Leaf mass per area metrics into sub-component contributions of
physiology and structure, noting focusing on just LMA as a predictor is
too simple. Most people would probably agree with the simplicity of
using LMA, but this is also its beauty and a core rationale for the
trait-based approach which is trying to get around the complexity and
potential morass of specific details that contribute to fitness but may
be difficult to measure and model on large populations. Overall, the
exercise presented should be interesting to the trait based audience,
which is sizable, though probably not a larger audience as it appears to
be concentrating on certain details of a focused nature. The overall
conclusion that a one dimensional leaf function is multidimensional and
one dimensional treatments are unsatisfying, and may be underselling the
overall trait-based approach which does consider multiple function
beyond LMA. To argue that models should consider multiple axes of leaf
functional diversity and link to nutrients states the obvious, and in
the end is not satisfying, especially since this study did not
incorporate nutrients such as N, P, K. I dont recommend that the core of
the study be altered, but do suggest the discussion be refocused
slightly to be more appealing to a broader audience and to finish with
conclusions that go beyond the obvious to perhaps address how the
multi-functional approach might specifically improve the predictability
and utility of models.
\end{quote}

\begin{quote}
Specific comments to help guide revisions: 1) For a study such as this,
the first paragraph of the discussion is the central element that can
capture or turn-off a broader audience. If it effectively summarizes the
take home message and its broader significance, then the study can hold
the readers attention, if it does not, it loses the readers. In this
regard, the opening of the discussion is weak, and the take-home message
is not clear, giving the paper the sense it is mired in the weeds.
Consider refocusing the start of the introduction to emphasize the
important contribution of this paper and its larger significance.
\end{quote}

\begin{quote}
\begin{enumerate}
\def\labelenumi{\arabic{enumi})}
\setcounter{enumi}{1}
\tightlist
\item
  A lot of results are reiterated in the discussion, possibly at the
  expense of interpretation and placing the finding in the context of
  the literature. Please consider less results presentation in favor of
  more explanation of the results and their interpretation. For example,
  on line 384-393, why are these points important, and why is it worth
  noting median LMA's of functional groups? Usually, placing
  supplemental data such as this as a key point of a paragraph on the
  second page of the discussion makes the reader wonder why the data is
  not presented in the main text, and why a paragraph in this position
  is not emphasizing findings in the main text.
\end{enumerate}
\end{quote}

\begin{quote}
\begin{enumerate}
\def\labelenumi{\arabic{enumi})}
\setcounter{enumi}{2}
\tightlist
\item
  Line 409-411: States the obvious.We know not all LMAs are equal.
\end{enumerate}
\end{quote}

\begin{quote}
\begin{enumerate}
\def\labelenumi{\arabic{enumi})}
\setcounter{enumi}{3}
\tightlist
\item
  line 414: ``are broadly consistent'' - not the best way to get
  accepted in a journal that requires novelty for acceptance. Perhaps
  better to start the paragraph with the second sentence, and then
  discussing how this division into physiological and structrural LMA's
  can move understanding forward. Could they be further subdivided to
  gain even more predictability, building off a the general logic of the
  paper? How might this be done?
\end{enumerate}
\end{quote}

\begin{quote}
\begin{enumerate}
\def\labelenumi{\arabic{enumi})}
\setcounter{enumi}{4}
\tightlist
\item
  I got a sense the paper is repetitious and could be proofed to remove
  redundancy.
\end{enumerate}
\end{quote}

\begin{quote}
\begin{enumerate}
\def\labelenumi{\arabic{enumi})}
\setcounter{enumi}{5}
\tightlist
\item
  Not clear about Figure 1. If it is theoretical data, then why not
  express as continuous responses rather than specific data points?
  Please be more explicit in the legend regarding how the data were
  generated. Summarize this in the legend for Fig 1 and specifically
  note it is a simulated data set generated from xxx, as explained in
  Section S5.
\end{enumerate}
\end{quote}




\end{document}

% Options for packages loaded elsewhere
\PassOptionsToPackage{unicode}{hyperref}
\PassOptionsToPackage{hyphens}{url}
\PassOptionsToPackage{dvipsnames,svgnames,x11names}{xcolor}
%
\documentclass[
  12pt,
  letterpaper,
  DIV=11,
  numbers=noendperiod]{scrartcl}

\usepackage{amsmath,amssymb}
\usepackage{iftex}
\ifPDFTeX
  \usepackage[T1]{fontenc}
  \usepackage[utf8]{inputenc}
  \usepackage{textcomp} % provide euro and other symbols
\else % if luatex or xetex
  \usepackage{unicode-math}
  \defaultfontfeatures{Scale=MatchLowercase}
  \defaultfontfeatures[\rmfamily]{Ligatures=TeX,Scale=1}
\fi
\usepackage{lmodern}
\ifPDFTeX\else  
    % xetex/luatex font selection
\fi
% Use upquote if available, for straight quotes in verbatim environments
\IfFileExists{upquote.sty}{\usepackage{upquote}}{}
\IfFileExists{microtype.sty}{% use microtype if available
  \usepackage[]{microtype}
  \UseMicrotypeSet[protrusion]{basicmath} % disable protrusion for tt fonts
}{}
\makeatletter
\@ifundefined{KOMAClassName}{% if non-KOMA class
  \IfFileExists{parskip.sty}{%
    \usepackage{parskip}
  }{% else
    \setlength{\parindent}{0pt}
    \setlength{\parskip}{6pt plus 2pt minus 1pt}}
}{% if KOMA class
  \KOMAoptions{parskip=half}}
\makeatother
\usepackage{xcolor}
\setlength{\emergencystretch}{3em} % prevent overfull lines
\setcounter{secnumdepth}{-\maxdimen} % remove section numbering
% Make \paragraph and \subparagraph free-standing
\makeatletter
\ifx\paragraph\undefined\else
  \let\oldparagraph\paragraph
  \renewcommand{\paragraph}{
    \@ifstar
      \xxxParagraphStar
      \xxxParagraphNoStar
  }
  \newcommand{\xxxParagraphStar}[1]{\oldparagraph*{#1}\mbox{}}
  \newcommand{\xxxParagraphNoStar}[1]{\oldparagraph{#1}\mbox{}}
\fi
\ifx\subparagraph\undefined\else
  \let\oldsubparagraph\subparagraph
  \renewcommand{\subparagraph}{
    \@ifstar
      \xxxSubParagraphStar
      \xxxSubParagraphNoStar
  }
  \newcommand{\xxxSubParagraphStar}[1]{\oldsubparagraph*{#1}\mbox{}}
  \newcommand{\xxxSubParagraphNoStar}[1]{\oldsubparagraph{#1}\mbox{}}
\fi
\makeatother


\providecommand{\tightlist}{%
  \setlength{\itemsep}{0pt}\setlength{\parskip}{0pt}}\usepackage{longtable,booktabs,array}
\usepackage{calc} % for calculating minipage widths
% Correct order of tables after \paragraph or \subparagraph
\usepackage{etoolbox}
\makeatletter
\patchcmd\longtable{\par}{\if@noskipsec\mbox{}\fi\par}{}{}
\makeatother
% Allow footnotes in longtable head/foot
\IfFileExists{footnotehyper.sty}{\usepackage{footnotehyper}}{\usepackage{footnote}}
\makesavenoteenv{longtable}
\usepackage{graphicx}
\makeatletter
\def\maxwidth{\ifdim\Gin@nat@width>\linewidth\linewidth\else\Gin@nat@width\fi}
\def\maxheight{\ifdim\Gin@nat@height>\textheight\textheight\else\Gin@nat@height\fi}
\makeatother
% Scale images if necessary, so that they will not overflow the page
% margins by default, and it is still possible to overwrite the defaults
% using explicit options in \includegraphics[width, height, ...]{}
\setkeys{Gin}{width=\maxwidth,height=\maxheight,keepaspectratio}
% Set default figure placement to htbp
\makeatletter
\def\fps@figure{htbp}
\makeatother

\usepackage[default]{sourcesanspro}
\usepackage{sourcecodepro}
\usepackage{xcolor}
\usepackage{mdframed}
% Define blockquote colors
\definecolor{blockquote-border}{RGB}{221,221,221}
\definecolor{blockquote-text}{RGB}{119,119,119}
% Define custom blockquote environment
\newmdenv[
  rightline=false,
  bottomline=false,
  topline=false,
  linewidth=3pt,
  linecolor=blockquote-border,
  skipabove=\parskip,
  innerleftmargin=10pt,
  innerrightmargin=0pt,
  frametitlebackgroundcolor=gray!20
]{customblockquote}
\renewenvironment{quote}
  {\begin{customblockquote}\color{blockquote-text}\ignorespaces}
  {\end{customblockquote}}
\KOMAoption{captions}{tableheading}
\makeatletter
\@ifpackageloaded{caption}{}{\usepackage{caption}}
\AtBeginDocument{%
\ifdefined\contentsname
  \renewcommand*\contentsname{Table of contents}
\else
  \newcommand\contentsname{Table of contents}
\fi
\ifdefined\listfigurename
  \renewcommand*\listfigurename{List of Figures}
\else
  \newcommand\listfigurename{List of Figures}
\fi
\ifdefined\listtablename
  \renewcommand*\listtablename{List of Tables}
\else
  \newcommand\listtablename{List of Tables}
\fi
\ifdefined\figurename
  \renewcommand*\figurename{Figure}
\else
  \newcommand\figurename{Figure}
\fi
\ifdefined\tablename
  \renewcommand*\tablename{Table}
\else
  \newcommand\tablename{Table}
\fi
}
\@ifpackageloaded{float}{}{\usepackage{float}}
\floatstyle{ruled}
\@ifundefined{c@chapter}{\newfloat{codelisting}{h}{lop}}{\newfloat{codelisting}{h}{lop}[chapter]}
\floatname{codelisting}{Listing}
\newcommand*\listoflistings{\listof{codelisting}{List of Listings}}
\makeatother
\makeatletter
\makeatother
\makeatletter
\@ifpackageloaded{caption}{}{\usepackage{caption}}
\@ifpackageloaded{subcaption}{}{\usepackage{subcaption}}
\makeatother

\ifLuaTeX
  \usepackage{selnolig}  % disable illegal ligatures
\fi
\usepackage{bookmark}

\IfFileExists{xurl.sty}{\usepackage{xurl}}{} % add URL line breaks if available
\urlstyle{same} % disable monospaced font for URLs
\hypersetup{
  pdfauthor={Masatoshi Katabuchi},
  colorlinks=true,
  linkcolor={blue},
  filecolor={Maroon},
  citecolor={Blue},
  urlcolor={Blue},
  pdfcreator={LaTeX via pandoc}}


\author{Masatoshi Katabuchi}
\date{2025-02-13}

\begin{document}


Dear Dr.~Niinemets,

We hereby resubmit the manuscript (OECO-D-24-00601), ``Decomposing leaf
mass into metabolic and structural components explains divergent
patterns of trait variation within and among plant species''. We thank
the handling editor and the reviewers for their careful attention to
detail in our manuscript and for providing comments that have greatly
improved the study.

Based on the comments, we have updated the manuscript to reflect the
following:

\begin{enumerate}
\def\labelenumi{\arabic{enumi}.}
\item
  \textbf{Broader significance:} We now emphasize that decomposing LMA
  into metabolic (LMAm) and structural (LMAs) components has
  applications well beyond trait‐based ecology, offering a better
  framework for modeling and global change research.
\item
  \textbf{Clear take‐home message:} The opening paragraph of the
  Discussion has been reorganized to highlight our main
  conclusions---namely how a two‐component LMA model helps explain
  divergent patterns of leaf traits within vs.~among species, and how
  this refined perspective improves trait‐based models.
\item
  \textbf{Physiological context:} We refined our explanation of how LMAs
  vs.~LMAm differ functionally (toughness/longevity vs.~photosynthetic
  capacity/respiration). We also moved the former Fig. S10 to Fig. S1 to
  better highlight the two axes in the Discussion.
\end{enumerate}

The programming code and the data supporting the findings of this study
will be deposited at Zenodo (https://zenodo.org). The corresponding
Github (https://github.com/mattocci27/LMAms) is already publicly
available.

We believe these revisions strengthen the manuscript and clarify both
the novelty and broader applications of our findings.

Below we provide detailed responses. The editing history is recorded in
a separate pdf file.

Sincerely,

Masatoshi Katabuchi

On behalf of the authors: Masatoshi Katabuchi, Kaoru Kitajima, S. Joseph
Wright, Sunshine A. Van Bael, Jeanne L. D. Osnas, and Jeremy W.
Lichstein

Corresponding author and contact information: Key Laboratory of Tropical
Forest Ecology, Xishuangbanna Tropical Botanical Garden, Chinese Academy
of Sciences, Mengla, Yunnan 666303, China\\
Email: katabuchi@xtbg.ac.cn; mattocci27@gmail.com

\newpage

\section{Responses}\label{responses}

Please note that line numbers refer to LMAms\_main\_diff.pdf that
contains editing history. The word file has slightly different line
numbers.

\begin{quote}
Handling Editor: The referees were generally positive regarding
OECO-D-24-00601 by Katabuchi et al., both recommending the paper could
be acceptable with minor revisions. The revisions recommended center
around 1) revising the paper (mostly discussion) to broaden the appeal
to an audience beyond the trait-based reader; 2) more effectively
presenting the main, ``take-home'' message and its implications, and 3)
providing clarity on the physiological dimensions and their
implications. Novelty was listed as a concern, but both referees felt
the study should lead to good discussions and thus will be well received
by the trait-based audience. However, the authors could improve the
impact of the discussion by emphasizing better the strength of the
findings and their contributions to our understanding, while worrying
less about how this work confirms prior work. if the work is deemed
confirmatory, it could have difficulties getting priority for
publication. We look forward to seeing the revisions.
\end{quote}

We have substantially revised the Discussion (particularly the first
Discussion paragraph and the newly added last Discussion paragraph) to
emphasize how decomposing LMA into LMAm and LMAs can be applied in
broader ecological and modeling contexts. In particular, we highlight
how the framework clarifies divergent within‐ vs.~among‐species trait
patterns and how Earth system models can benefit from more nuanced
treatment of LMA components. These changes underscore the novelty and
global significance of partitioning LMA---beyond simply confirming prior
work.

\begin{quote}
Reviewer \#1: The manuscript develops a statistical method to separate
leaf mass per area (LMA) into structural (LMAs) and metabolic (LMAm)
components with the assumption that the metabolic component should be
more correlated with other metabolic traits such as area-based Amax and
Rdark and that the structural component should be more correlated with
leaf longevity. Such analysis was conducted for two data sets - GLOPNET
that only has sun-lit leaf traits and the BCI data that has both sun-lit
and shade leaf traits. Overall, evergreen species have lower LMAm
fraction than deciduous species and shade leaves have lower LMAm
fraction than conspecific sunlit leaves. The analysis argues for going
beyond the single-axis mass-based leaf economics spectrum and the need
to consider at least the structural and metabolic dimensions in
applications such as Earth System Modeling.
\end{quote}

Thank you so much for your positive comments and constructive feedback.

\begin{quote}
Overall, I enjoyed reading the manuscript, which attempts to decompose
LMA that is long known to be a very `complex' trait. My main concern is
that the derivation of LMAs and LMAm, albeit informative, is largely
statistical. What is the key evidence that the decomposition makes
physiological sense? In my understanding, the analysis with Narea,
Parea, CLarea (Fig. S7-S9) is somewhat relevant but those three traits
are also complex. And Fig. S8 shows LMA rather than LMAs or LMAm are
better correlated with Narea, Parea, and CLarea. I was wondering whether
it would be helpful to only include Aarea or Rarea in your statistical
fitting and see whether the results change\ldots{}
\end{quote}

We agree it is crucial to establish that LMAm vs.~LMAs have
physiological underpinnings. In the revised text (lines 365-380 in the
tracked‐changes PDF), we highlighted how variation in
\emph{R}\textsubscript{area} can be explained by the balance of LMAm and
LMAs. This underscores that LMAm captures metabolically active tissues
(e.g., mesophyll) directly involved in photosynthesis and respiration,
while LMAs reflects structural components (e.g., fibers, cell walls)
tied to leaf toughness and longevity, yet contributes little to
maintenance respiration.

Regarding the suggestion to fit the model exclusively to
\emph{A}\textsubscript{area} or \emph{R}\textsubscript{area}: To
clarify, the models were fit to \emph{A}\textsubscript{area},
\emph{R}\textsubscript{area}, LMA, and LL (not
\emph{N}\textsubscript{area}, \emph{P}\textsubscript{area}, or
CL\textsubscript{area}). This is explained on lines 131-134, and we have
now added an additional clarification of this point on lines 227-228 Our
results suggest that LMAs does not strongly influence
\emph{A}\textsubscript{area} or \emph{R}\textsubscript{area}. Therefore,
restricting the analysis to \emph{A}\textsubscript{area} or
\emph{R}\textsubscript{area} would not substantially constrain LMAs. It
is possible that the decomposition of LMA into LMAm and LMAs would be
robust to removing LL from the model-fitting, but we believe the most
direct way to address our main questions is to include in the model the
measured traits that we expect to be strongly affected by LMAm and LMAs;
this includes LL, which we expect to be strongly affected by LMAs.

\begin{quote}
Some minor comments throughout the text Line 108-111 This statement is
surely true. Adding degrees of freedom will always make the statistical
fit better\ldots.
\end{quote}

Good point. We have changed `improved statistical fits' to `improved
predictions and insights' (line 104). Similarly, in the next sentence,
we changed `explain variation in other traits' to `predict and
understand variation in other traits' (line 107). The new wording
emphasizes predictive ability (which does not necessarily increase with
the number of model parameters) as well as the insights gained from
treating LMA as a composite trait.

\begin{quote}
Line 220 typo in `abovµe'
\end{quote}

We have corrected the typo (Line 200).

\begin{quote}
Line 771 Fig. 2, It is interesting that the `Unclassified' species take
the lower end of the LMAs distribution. Are they mostly grasses or
aquatic plants? It might be a worthy discussion point and an indirect
evidence of the physiological basis of derived LMAs.
\end{quote}

Thanks for the suggestion. The unclassified species at the lower end of
LMAs are mainly herbaceous species (65\% are grasses or forbs), which
generally have lower leaf tissue density. We now discuss how this
supports a physiological basis for low LMAs in the revised text (Line
401-419).

\begin{quote}
Reviewer\# 2: OECO-D-24-00601 is largely well written with few editorial
mistakes and points of poor clarity. The science is sound, but novelty
does not appear high as currently presented. The focus is to dissect the
Leaf mass per area metrics into sub-component contributions of
physiology and structure, noting focusing on just LMA as a predictor is
too simple. Most people would probably agree with the simplicity of
using LMA, but this is also its beauty and a core rationale for the
trait-based approach which is trying to get around the complexity and
potential morass of specific details that contribute to fitness but may
be difficult to measure and model on large populations. Overall, the
exercise presented should be interesting to the trait based audience,
which is sizable, though probably not a larger audience as it appears to
be concentrating on certain details of a focused nature. The overall
conclusion that a one dimensional leaf function is multidimensional and
one dimensional treatments are unsatisfying, and may be underselling the
overall trait-based approach which does consider multiple function
beyond LMA. To argue that models should consider multiple axes of leaf
functional diversity and link to nutrients states the obvious, and in
the end is not satisfying, especially since this study did not
incorporate nutrients such as N, P, K. I dont recommend that the core of
the study be altered, but do suggest the discussion be refocused
slightly to be more appealing to a broader audience and to finish with
conclusions that go beyond the obvious to perhaps address how the
multi-functional approach might specifically improve the predictability
and utility of models.
\end{quote}

We appreciate the suggestion to emphasize the broader significance of
our work and to more explicitly explain the implications for ecosystem
models. These points are now addressed in Discussion, where we added new
text to highlight the broader significance (lines 330-353) and specific
implications for ecosystem models (lines 516-528; e.g., biased
representation of photosynthesis and leaf respiration in single-axis
approaches).

\begin{quote}
Specific comments to help guide revisions: 1) For a study such as this,
the first paragraph of the discussion is the central element that can
capture or turn-off a broader audience. If it effectively summarizes the
take home message and its broader significance, then the study can hold
the readers attention, if it does not, it loses the readers. In this
regard, the opening of the discussion is weak, and the take-home message
is not clear, giving the paper the sense it is mired in the weeds.
Consider refocusing the start of the introduction to emphasize the
important contribution of this paper and its larger significance.
\end{quote}

We have revised the beginning of the Discussion to focus it on a clear,
simple take-home message: Decomposing LMA into structural and metabolic
components explains divergent inter- and intraspecific patterns in leaf
trait variation that cannot be explained by a single trait axis.
Therefore, understanding leaf functional variation and accurately
representing this variation in ecosystem models will require at least
two leaf trait dimensions.

\begin{quote}
\begin{enumerate}
\def\labelenumi{\arabic{enumi})}
\setcounter{enumi}{1}
\tightlist
\item
  A lot of results are reiterated in the discussion, possibly at the
  expense of interpretation and placing the finding in the context of
  the literature. Please consider less results presentation in favor of
  more explanation of the results and their interpretation. For example,
  on line 384-393, why are these points important, and why is it worth
  noting median LMA's of functional groups? Usually, placing
  supplemental data such as this as a key point of a paragraph on the
  second page of the discussion makes the reader wonder why the data is
  not presented in the main text, and why a paragraph in this position
  is not emphasizing findings in the main text.
\end{enumerate}
\end{quote}

In accordance with your suggestion, we streamlined the Discussion by
reducing repeated numeric summaries and providing more interpretive
context in reference to existing literature. For instance, instead of
listing median LMA values for multiple subgroups, we now emphasize why
differences in metabolic vs.~structural investment matter for leaf
longevity or respiratory costs (lines 365-380).

\begin{quote}
\begin{enumerate}
\def\labelenumi{\arabic{enumi})}
\setcounter{enumi}{2}
\tightlist
\item
  Line 409-411: States the obvious.We know not all LMAs are equal.
\end{enumerate}
\end{quote}

We have removed this statement.

\begin{quote}
\begin{enumerate}
\def\labelenumi{\arabic{enumi})}
\setcounter{enumi}{3}
\tightlist
\item
  line 414: ``are broadly consistent'' - not the best way to get
  accepted in a journal that requires novelty for acceptance. Perhaps
  better to start the paragraph with the second sentence, and then
  discussing how this division into physiological and structrural LMA's
  can move understanding forward. Could they be further subdivided to
  gain even more predictability, building off a the general logic of the
  paper? How might this be done?
\end{enumerate}
\end{quote}

Thank you for the suggestion. We removed phrases like ``are broadly
consistent,'' which might suggest minimal novelty. The paragraph is now
focused on the physiological basis of low LMAs in unclassified species,
which are primarily grasses and herbaceous forbs (lines 392-419).

\begin{quote}
\begin{enumerate}
\def\labelenumi{\arabic{enumi})}
\setcounter{enumi}{4}
\tightlist
\item
  I got a sense the paper is repetitious and could be proofed to remove
  redundancy.
\end{enumerate}
\end{quote}

We removed several redundant statements and reorganized portions of the
text to improve readability (e.g., lines 354-419 in ``Modeling Amax,
Rdark, and LL\ldots{}''). We also revised the sections ``Implications
for understanding LL variation'' and ``Implications for understanding
relationships between photosynthetic capacity and LMA'', making them
more concise and avoiding repetitive statements.

\begin{quote}
\begin{enumerate}
\def\labelenumi{\arabic{enumi})}
\setcounter{enumi}{5}
\tightlist
\item
  Not clear about Figure 1. If it is theoretical data, then why not
  express as continuous responses rather than specific data points?
  Please be more explicit in the legend regarding how the data were
  generated. Summarize this in the legend for Fig 1 and specifically
  note it is a simulated data set generated from xxx, as explained in
  Section S5.
\end{enumerate}
\end{quote}

We have clarified in the figure caption that Fig. 1 illustrates
simulated data drawn from normal distributions, displayed as points to
mimic empirical sampling (now explained on lines 721-722). We agree that
it would be possible to represent these relationships as continuous
surfaces, but this would require a more complicated three-dimensional
figure (e.g., Fig. 1a would show LMAm and LMAs on the x- and y-axes, and
the joint probability density on the z-axis). We think it is more
helpful to show the simulation results with a sample of points,
analogous to an empirical trait dataset (we have now clarified this in
the figure caption on lines 728-729).




\end{document}

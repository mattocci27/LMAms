\documentclass[12pt,]{article}
\usepackage{lmodern}
\usepackage{amssymb,amsmath}
\usepackage{ifxetex,ifluatex}
\usepackage{fixltx2e} % provides \textsubscript
\ifnum 0\ifxetex 1\fi\ifluatex 1\fi=0 % if pdftex
  \usepackage[T1]{fontenc}
  \usepackage[utf8]{inputenc}
\else % if luatex or xelatex
  \ifxetex
    \usepackage{mathspec}
  \else
    \usepackage{fontspec}
  \fi
  \defaultfontfeatures{Ligatures=TeX,Scale=MatchLowercase}
\fi
% use upquote if available, for straight quotes in verbatim environments
\IfFileExists{upquote.sty}{\usepackage{upquote}}{}
% use microtype if available
\IfFileExists{microtype.sty}{%
\usepackage{microtype}
\UseMicrotypeSet[protrusion]{basicmath} % disable protrusion for tt fonts
}{}
\usepackage[margin=1in]{geometry}
\usepackage{hyperref}
\hypersetup{unicode=true,
            pdfborder={0 0 0},
            breaklinks=true}
\urlstyle{same}  % don't use monospace font for urls
\usepackage{longtable,booktabs}
\usepackage{graphicx,grffile}
\makeatletter
\def\maxwidth{\ifdim\Gin@nat@width>\linewidth\linewidth\else\Gin@nat@width\fi}
\def\maxheight{\ifdim\Gin@nat@height>\textheight\textheight\else\Gin@nat@height\fi}
\makeatother
% Scale images if necessary, so that they will not overflow the page
% margins by default, and it is still possible to overwrite the defaults
% using explicit options in \includegraphics[width, height, ...]{}
\setkeys{Gin}{width=\maxwidth,height=\maxheight,keepaspectratio}
\IfFileExists{parskip.sty}{%
\usepackage{parskip}
}{% else
\setlength{\parindent}{0pt}
\setlength{\parskip}{6pt plus 2pt minus 1pt}
}
\setlength{\emergencystretch}{3em}  % prevent overfull lines
\providecommand{\tightlist}{%
  \setlength{\itemsep}{0pt}\setlength{\parskip}{0pt}}
\setcounter{secnumdepth}{0}
% Redefines (sub)paragraphs to behave more like sections
\ifx\paragraph\undefined\else
\let\oldparagraph\paragraph
\renewcommand{\paragraph}[1]{\oldparagraph{#1}\mbox{}}
\fi
\ifx\subparagraph\undefined\else
\let\oldsubparagraph\subparagraph
\renewcommand{\subparagraph}[1]{\oldsubparagraph{#1}\mbox{}}
\fi

%%% Use protect on footnotes to avoid problems with footnotes in titles
\let\rmarkdownfootnote\footnote%
\def\footnote{\protect\rmarkdownfootnote}

%%% Change title format to be more compact
\usepackage{titling}

% Create subtitle command for use in maketitle
\newcommand{\subtitle}[1]{
  \posttitle{
    \begin{center}\large#1\end{center}
    }
}

\setlength{\droptitle}{-2em}

  \title{}
    \pretitle{\vspace{\droptitle}}
  \posttitle{}
    \author{}
    \preauthor{}\postauthor{}
    \date{}
    \predate{}\postdate{}
  
\usepackage{times}
\usepackage{setspace}
\doublespacing
\usepackage{lineno}
\linenumbers
\usepackage{amsmath}
% \usepackage{array}
\usepackage{booktabs}
\usepackage{longtable}
\usepackage{array}
\usepackage{multirow}
\usepackage[table]{xcolor}
\usepackage{wrapfig}
\usepackage{float}
\usepackage{colortbl}
\usepackage{pdflscape}
\usepackage{tabu}
\usepackage{threeparttable}
\usepackage{threeparttablex}
\usepackage[normalem]{ulem}
\usepackage{makecell}

\usepackage{amsthm}
\newtheorem{theorem}{Theorem}
\newtheorem{lemma}{Lemma}
\theoremstyle{definition}
\newtheorem{definition}{Definition}
\newtheorem{corollary}{Corollary}
\newtheorem{proposition}{Proposition}
\theoremstyle{definition}
\newtheorem{example}{Example}
\theoremstyle{definition}
\newtheorem{exercise}{Exercise}
\theoremstyle{remark}
\newtheorem*{remark}{Remark}
\newtheorem*{solution}{Solution}
\begin{document}

\textbf{Running title}: Metabolic and structural leaf mass

\textbf{Decomposing leaf mass into metabolic and structural components
explains divergent patterns of trait variation within and among plant
species}

Masatoshi Katabuchi\textsuperscript{1,2,6}, Kaoru
Kitajima\textsuperscript{1,3,4}, S. Joseph Wright\textsuperscript{4},
Sunshine A. Van Bael\textsuperscript{4,5}, Jeanne L. D.
Osnas\textsuperscript{1} and Jeremy W. Lichstein\textsuperscript{1}

\textsuperscript{1} Department of Biology, University of Florida,
Gainesville, FL 32611, USA

\textsuperscript{2} Kellogg Biological Station, Michigan State
University, Hickory Corners, MI 49060, USA

\textsuperscript{3} Graduate School of Agriculture, Kyoto University,
Kitashirakawa Oiwake-Cho, Kyoto 606-8502 Japan

\textsuperscript{4} Smithsonian Tropical Research Institute, 9100 Panama
City Pl., Washington, DC 20521

\textsuperscript{5} Department of Ecology and Evolutionary Biology,
Tulane University, New Orleans, LA 70118 USA

\textsuperscript{6}\textbf{Corresponding Author}: E-mail:
\href{mailto:mattocci27@gmail.com}{\nolinkurl{mattocci27@gmail.com}}

\newpage

\hypertarget{abstract}{%
\section{Abstract}\label{abstract}}

\begin{itemize}
\item
  Across the global flora, photosynthetic and metabolic rates depend
  more strongly on leaf area than leaf mass. In contrast, intraspecific
  variation in these rates is strongly mass-dependent. These contrasting
  patterns suggest that the causes of variation in leaf mass per area
  (LMA) may be fundamentally different within vs.~among species.
\item
  We used statistical methods to decompose LMA into two conceptual
  components -- `metabolic LMAm (which determines photosynthetic
  capacity and metabolic rates, and also affects optimal leaf lifespan)
  and `structural' LMAs (which determines leaf toughness and potential
  leaf lifespan) using leaf trait data from tropical forest sites in
  Panama and a global leaf-trait database.
\item
  Statistically decomposing LMA into LMAm and LMAs provides improved
  predictions of trait variation (photosynthesis, respiration, and
  lifespan) across the global flora, and within and among tropical plant
  species in Panama. Our analysis shows that small scaling slope between
  metabolic leaf mass and photosynthetic rate, and similar variance in
  LMAm and in LMAs leads to area-proportionality of interspecific leaf
  traits. In contrast, intraspecific LMA variation is due to changes in
  LMAm, which explains why photosynthetic and metabolic traits are
  mass-dependent within species
\item
  Our results suggest that leaf trait variation is multi-dimensional and
  is not well-represented by the one-dimensional leaf economics
  spectrum.
\end{itemize}

\hypertarget{introduction}{%
\section{Introduction}\label{introduction}}

The LES captures tight relationships among leaf mass per area (LMA),
leaf lifespan (LL), and mass-normalized leaf traits related to carbon
fixation and nutrient use ranging from short-lived leaves with high
photosynthetic potential and fast returns on investment to long-lived
leaves with low photosynthetic potential and slow returns (Reich,
\protect\hyperlink{ref-Reich2014}{2014}; Westoby \& Wright,
\protect\hyperlink{ref-Westoby2006a}{2006}; Wright et al.,
\protect\hyperlink{ref-Wright2004a}{2004}\protect\hyperlink{ref-Wright2004a}{a}).
The strong correlations suggest the presence of a single dominant axis
of leaf functional variation (Wright et al.,
\protect\hyperlink{ref-Wright2004a}{2004}\protect\hyperlink{ref-Wright2004a}{a}).
However, recent analyses suggest that the strong correlations are
largely the result of high interspecific variation in LMA combined with
mass-normalization of area-dependent traits (Lloyd, Bloomfield,
Domingues, \& Farquhar, \protect\hyperlink{ref-Lloyd2013}{2013}; Osnas,
Lichstein, Reich, \& Pacala, \protect\hyperlink{ref-Osnas2013}{2013}).
Thus, although mass-normalization and the LES can be justified based on
economic principles (Westoby, Reich, \& Wright,
\protect\hyperlink{ref-Westoby2013}{2013}), the evidence for a single
dominant axis is questionable.

Furthermore, different leaf assemblages exhibit different patterns of
trait variation with respect to leaf mass and leaf area. For example,
across global species, whole-leaf values of traits related to
photosynthesis and metabolism (e.g., the photosynthetic capacity of an
entire leaf, units = moles CO\textsubscript{2} fixed per-unit time; or
the total amount of N or P in an entire leaf, units = grams of nitrogen
or phosphorus) tend to be roughly proportional to leaf area; whereas
across intraspecific light gradients these same whole-leaf trait values
tend to be roughly proportional to leaf mass. Functional groups (e.g.,
deciduous vs.~evergreen angiosperms) also differ from each other in
terms of how interspecific trait variation (within a given functional
group) depends on leaf mass vs.~area. These divergent patterns in leaf
trait variation suggest the presence of multiple drivers of trait
variation, which may be difficult to capture with a single axis.

One promising approach for understanding cause and consequence of leaf
trait variation is to conceptualize LMA as the sum of photosynthetic and
structural components (Osnas et al.,
\protect\hyperlink{ref-Osnas2018}{2018}). Although this two-dimensional
model of trait variation is simplistic (e.g., there is no explicit
treatment of leaf hydraulics), it provides qualitative insights into the
causes of the above divergent trait patterns, because variation in LMAm
with high metabolic rate leads to mass-dependence of photosynthetic and
metabolic traits, whereas variation in LMAs or LMAm with low metabolic
rate leads to area-dependence of these same traits (Osnas et al.,
\protect\hyperlink{ref-Osnas2018}{2018}). For example, large difference
in photosynthetic rate between canopy and understory leaves (REF?) would
imply large variance in LMAm, which would lead mass-dependence in
intraspecific variation across vertical canopy (light) gradients. On the
other hand, large variation in LL in the global flora (Wright et al.,
\protect\hyperlink{ref-Wright2004a}{2004}\protect\hyperlink{ref-Wright2004a}{a})
would suggest large variation in LMAs, which would lead area-dependence
of leaf traits.

However, the above conceptual model has not been translated into a
quantitative model that predicts trait values, which limits our capacity
to test and apply the model. One approach to developing a quantitative
framework would be to directly measure LMAm and LMAs, and then to
estimate relationships between these traits and other traits of interest
(e.g., \emph{A}\textsubscript{max}, \emph{R}\textsubscript{dark}, and
LL). However, although certain leaf mass components can be neatly
partitioned to either LMAm or LMAs (e.g., chloroplasts contribute to
photosynthesis, but not structure), other leaf mass components cannot.
For example, thick cell walls contribute to structural toughness, but at
least some cell wall mass is required for the biomechanical support that
enables photosynthesis. An alternative approach to implementing a
quantitative form of the LMAm-LMAs model, and the one we explore in this
paper, is to specify hypotheses for how LMAm and LMAs relate to measured
traits (e.g., \emph{A}\textsubscript{max}, \emph{R}\textsubscript{dark},
and LL) in the form of a quantitative model, and to evaluate the model
using statistical methods.

In this paper, we show quantitative version of the above conceptual
model using leaf trait data from two tropical forest sites (sun and
shade leaves from wet and dry sites in Panama) and the GLOPNET global
leaf traits database (Wright et al.,
\protect\hyperlink{ref-Wright2004a}{2004}\protect\hyperlink{ref-Wright2004a}{a}).
The goal of our analysis is to evaluate if the inferred LMAm and LMAs
values can explain divergent patterns in leaf trait data, and if so, to
use the model to elucidate the causes of these divergent patterns.
First, we describe a statistical modeling framework to estimate LMAm and
LMAs those show better correlations with metabolic rates and leaf
lifespan than LMA does. Then, we ask the following questions: (1) How
does variation in LMA components related with mass and area
proportionality of leaf traits? (2) Do LMAm and LMAs differ between
evergreen and deciduous species, and between sun and shade leaves? and
(3) How are measurable leaf photosynthetic and structural traits (e.g.,
concentrations of nitrogen and cellulose) related to LMAm and LMAs?

\hypertarget{material-and-methods}{%
\section{Material and Methods}\label{material-and-methods}}

\hypertarget{model-overview}{%
\subsection{Model overview}\label{model-overview}}

We developed a statistical modeling framework to partition LMA into
additive LMAs and LMAm components (see below and Supplement S3). For
sample \emph{i} (where `sample' refers to a species, or a species ×
canopy position combination; see Datasets below), we partition
LMA\textsubscript{\emph{i}} into LMAm\textsubscript{\emph{i}} =
f\textsubscript{\emph{i}} × LMA\textsubscript{\emph{i}} and
LMAs\textsubscript{\emph{i}} = (1 -- \emph{f\textsubscript{i}}) ×
LMA\textsubscript{\emph{i}} by estimating a latent variable
\emph{f\textsubscript{i}}. The latent variables
\emph{f\textsubscript{i}} are not directly observed, but they can be
constrained by available data using Bayesian methods (Bishop,
\protect\hyperlink{ref-Bishop2006}{2006}; Gelman \& Hill,
\protect\hyperlink{ref-Gelman2006}{2006}). For example, posterior
distributions for LMAm\textsubscript{\emph{i}} should tend to converge
on high values for leaves with high \emph{A}\textsubscript{area}, and
posterior distributions for LMAs\textsubscript{\emph{i}} should tend to
converge on high values for leaves with high LL. Given the large number
of free parameters (i.e., one latent variable per leaf sample), it is
possible for the model to over-fit the data, which could lead to
spurious inferences. Therefore, we performed tests with randomized data
(see below) and with simulated data to evaluate model performance under
a range of conditions (Supplement S4). Tests with simulated data suggest
that our modeling approach is robust and not prone to producing
artefactual results.

\hypertarget{modeling-leaf-lifespan-photosynthetic-capacity-and-dark-respiration-in-relation-to-photosynthetic-and-structural-leaf-mass-components}{%
\subsection{Modeling leaf lifespan, photosynthetic capacity, and dark
respiration in relation to photosynthetic and structural leaf mass
components}\label{modeling-leaf-lifespan-photosynthetic-capacity-and-dark-respiration-in-relation-to-photosynthetic-and-structural-leaf-mass-components}}

We assume that the sum of high metabolic leaf mass per area (LMAm) and
structural leaf mass per area (LMAs) is equal to total observed LMA for
leaf sample \emph{i} (where a `sample' is a species in the GLOPNET
dataset, or a species × canopy position combination in the Panama
dataset):

\begin{align}
  &\mathrm{LMA}_{i} =\mathrm{LMAm}_{i} + \mathrm{LMAs}_{i} \label{eq:LMA}\\
  &\mathrm{LMAm}_{i} = f_{i} \mathrm{LMA}_{i} \label{eq:LMAm}\\
  &\mathrm{LMAs}_{i} = (1 - f_{i})  \mathrm{LMA}_{i}\label{eq:LMAs}
\end{align}

where, \emph{f\textsubscript{i}} is the fraction of
LMA\textsubscript{\emph{i}} that is comprised of
LMAm\textsubscript{\emph{i}}. The \emph{f\textsubscript{i}} terms are
not directly observed but can be estimated as latent variables in a
Bayesian modeling framework (see details below). In our model, net
photosynthetic capacity (\emph{A}\textsubscript{area}) is determined by
LMAm, LL is determined by LMAs, and total leaf dark respiration is
determined by both metabolic and structural tissues:

\begin{align}
& \mathrm{E}[A_{\mathrm{area} \, i}]
= \alpha_0\mathrm{LMAm}_{i}^{\alpha_m}\mathrm{LMAs}_i^{\alpha_s}  =  \alpha_0 (f_i \mathrm{LMA}_{i})^{\alpha_m} \bigl\{(1-f_i) \mathrm{LMA}_{i}\bigr\}^{\alpha_s} \label{eq:E-A} \\
& \mathrm{E}[\mathrm{LL}_i] = \beta_0\mathrm{LMAm}_{i}^{\beta_m} \mathrm{LMAs}_{i}^{\beta_s}   = \beta_0 (f_i \mathrm{LMA}_{i})^{\beta_m} \bigl\{(1-f_i) \mathrm{LMA}_{i}\bigr\}^{\beta_s} \label{eq:E-LL} \\
& \mathrm{E}[R_{\mathrm{area} \, i}]
= \gamma_0\mathrm{LMAm}_{i}^{\gamma_m}ı \mathrm{LMAs}_{i}^{\gamma_s}
= \gamma_0 (f_i \mathrm{LMA}_{i})^{\gamma_m}  \bigl\{(1-f_i)\mathrm{LMA}_{i}\bigr\}^{\gamma_s} \label{eq:E-R}
\end{align}

where, \emph{A}\textsubscript{area} \textsubscript{\emph{i}},
\emph{R}\textsubscript{area} \textsubscript{\emph{i}}, and
LL\textsubscript{\emph{i}}, are, respectively, the net photosynthetic
rate (\emph{A}\textsubscript{max}) per unit area, dark respiration rate
(\emph{R}\textsubscript{dark}) per unit area, and leaf life span of leaf
\emph{i}; \(x_0\) is fitted constant; and \(x_m\) and \(x_s\) quantify
how traits depend on metabolic and structural tissues, respectively; and
\(\epsilon_{1-3i}\) are sample specific random variables. The logarithms
of net photosynthesis, LL, and \emph{R}\textsubscript{area} are assumed
to have a multivariate normal distribution (MVN) (see supplement) .

Leaf density is also known as a good predictor for LL (Kitajima,
Cordero, \& Wright, \protect\hyperlink{ref-Kitajima2013}{2013}; Kitajima
et al., \protect\hyperlink{ref-Kitajima2012}{2012}). LMAs in Eq.
\eqref{eq:E-LL} could be replaced by leaf structural density (LMAs/LT)
where LT is leaf thickness. Model with leaf density yield similar
results but less predictive according to cross-validation (see below),
so we only present analyses using LMAs.

According to the optimal LL theory (Kikuzawa,
\protect\hyperlink{ref-Kikuzawa1991}{1991}; Kikuzawa, Shirakawa, Suzuki,
\& Umeki, \protect\hyperlink{ref-Kikuzawa2004}{2004}), leaves under
strong light tend to be replaced quickly to maximize a leaf's lifetime
carbon gain per-unit time. In addition to Eq. \eqref{eq:E-LL}, we
considered the effect of light on LL:

\begin{align}
\mathrm{E[LL}_i] = \beta_0\mathrm{LMAm}_{i}^{\beta_m} \mathrm{LMAs}_{i}^{\beta_s} exp(\theta Light_i) \label{eq:E-LL2}
\end{align}

where \(\theta\) is the effect of light for sun leaves and the dummy
variable \(Light_i\) is set to 1 for sun leaves and 0 for shade leaves.
We included the light effect for the Panama data (see Methods/Datasets
in main text), which includes both sun and shade leaves for XX out of
130 total species.

\hypertarget{datasets}{%
\subsection{Datasets}\label{datasets}}

To fit the models described above, we used LMA (g
m\textsuperscript{-2}), net photosynthetic capacity per unit leaf area
(\emph{A}\textsubscript{area}; mol s\textsuperscript{-1}
m\textsuperscript{-2}), dark respiration rate per unit leaf area
(\emph{R}\textsubscript{area}; mol s\textsuperscript{-1}
m\textsuperscript{-2}) and LL (months) from the GLOPNET global leaf
traits database (Wright et al.,
\protect\hyperlink{ref-Wright2004a}{2004}\protect\hyperlink{ref-Wright2004a}{a})
and from two tropical forest sites in Panama: Monumental Natural
Metropolitano (MNM, ``dry site'') and Bosque Protector San Lorenzo (SL,
``wet site''). The Panama data include leaves sampled at two canopy
positions (``sun'': full sun at the top of the canopy; and ``shade'':
well shaded, sampled within 2 m of the forest floor) from trees within
reach of a canopy crane at each site. The dry MNM site is a
semi-deciduous coastal Pacific forest with a 5-month dry season from
December-April and 1740 mm of annual rainfall (Wright et al.,
\protect\hyperlink{ref-Wright2003}{2003}). The MNM crane is 40 m tall
with a 51 m long boom. The wet SL site is an evergreen Caribbean coastal
forest with 3100 mm of annual rainfall (Wright et al.,
\protect\hyperlink{ref-Wright2003}{2003}). The SL crane is 52 m tall
with a 54 m long boom.

After deleting leaf samples (i.e., database records, which typically
average over multiple individual leaves) that lacked one of the four
traits (LMA, \emph{A}\textsubscript{area}, \emph{R}\textsubscript{area},
or LL), 198 samples for 198 unique species were available for GLOPNET,
and 130 samples for 104 unique species were available for Panama (dry
and wet sites combined; 26 species sampled in both sun and shade; no
species with all four traits available at both sites). Both datasets
include additional traits that we used to interpret model results, but
which were not used to fit models. These traits include nitrogen and
phosphorus content per leaf unit area (\emph{N}\textsubscript{area} and
\emph{P}\textsubscript{area}; g m\textsuperscript{-2}) in both datasets,
leaf habit in GLOPNET (deciduous or evergreen), and cellulose content
per unit area (\emph{CL}\textsubscript{area}; g m\textsuperscript{-2})
in Panama.

\hypertarget{model-estimation}{%
\subsection{Model estimation}\label{model-estimation}}

We fit the model without light effects to GLOPNET data and the model
with light effect to the Panama data because GLOPNET primarily
represents interspecific variation among sun leaves, whereas the Optimal
LL Model was motivated by the negative intraspecific LL-LMA relationship
observed in Panama (Osnas et al.,
\protect\hyperlink{ref-Osnas2018}{2018}; Xu et al.,
\protect\hyperlink{ref-Xu2017}{2017}) and elsewhere (Lusk, Reich,
Montgomery, Ackerly, \& Cavender-Bares,
\protect\hyperlink{ref-Lusk2008}{2008}).

Posterior distributions of all parameters were estimated using the
Hamiltonian Monte Carlo algorithm (HMC) implemented in Stan (Carpenter
et al., \protect\hyperlink{ref-Carpenter2017}{2017}). See supplement X
for more detail. The Stan code use to fit models are also available from
Github at: \url{https://github.com/mattocci27/LMAmLMAs}. Some parameters
are exchangeable in our model. For example,
\((f_i, \ \alpha_m, \ \alpha_s)\) = (0.2, 0.5, 1.0) and (0.8, 1.0, 0.5)
yield the identical likelihood, which results in bimodal distributions
of the posterior distributions of parameters (See X). To avoid this
non-identifiability, i) we simply switched labels so that \(\alpha_m\)
\textgreater{} \(\alpha_s\) (LMAm is associated with metabolic rates by
our definition) or ii) we constrained \(\alpha_m\) \textgreater{}
\(\alpha_s\). Both methods yield qualitatively similar results and we
report the second method for the main text. Convergence of the posterior
distribution was assessed with the Gelman-Rubin statistic with a
convergence threshold of 1.1 for all diagnostics (Gelman, Hwang, \&
Vehtari, \protect\hyperlink{ref-Gelman2014}{2014}).

Because our statistical approach includes one latent variable
\emph{f\textsubscript{i}} to partition LMA into LMAm and LMAs for each
leaf sample, one might expect a good match between predictions and
observations simply due to the large number of free parameters, whether
or not the model captured important biological mechanisms. However, the
same models applied to randomly shuffled observe dataset did not produce
any significant patterns (Supplement X). Basically, the posterior
distributions of \(f_i\) were changed little from the prior (i.e., all
the average \(f_i\) tend to be 0.5 for all the samples). Thus, we assume
that estimates of LMAm and LMAs reflect meaningful pattens in the data.

\hypertarget{model-selection}{%
\subsection{Model selection}\label{model-selection}}

We fit five types of models representing the combinations of: (1) only
LMA (i.e., without the latent variables \emph{f\textsubscript{i}}), (2)
LMA and light effects, (3) LMAm and LMAs, (4) LMAm, LMAs and light
effects, and (5) LMAm, LMAs and LT. We applied 1 and 3 for GLOPNET and
1-5 for the Panama data which have LT data (n = 106). We applied 10-fold
cross validation to determine the best model. Because our model does not
estimate the prior distributions of the latent variable
\emph{f\textsubscript{i}} in the train datasets, it requires additional
calculation to estimate the latent variable \emph{f\textsubscript{i}} in
the test datasets. Thus, less computationally intensive methods such as
widely applicable information criterion (WAIC; (Watanabe,
\protect\hyperlink{ref-Watanabe2010}{2010})), and Pareto-smoothed
importance sampling leave-one-out cross-validation (PSIS-LOO; (Vehtari,
Mononen, Tolvanen, Sivula, \& Winther,
\protect\hyperlink{ref-Vehtari2014}{2014})) are not appropriate. In
order to estimate the latent variable \emph{f\textsubscript{i}}, we
assigned the prior distributions to the latent variable
\emph{f\textsubscript{i}} in the test datasets based on the mean and
variance of \emph{f\textsubscript{i}} for each functional group in the
train datasets. Then, we calculated the expected log point-wise
predictive density (ELPD) for each fold. We select the best model based
on the ELPD.

\hypertarget{variance-partitioning}{%
\subsection{Variance partitioning}\label{variance-partitioning}}

We used the following identity to estimate the relative contributions of
LMAp and LMAs to LMA variance, where again LMA = LMAp + LMAs:

\begin{align}
\mathrm{Var}(Y = X1 + X2) = \mathrm{Cov}(Y, X1+X2) = \mathrm{Cov}(Y,X1) + \mathrm{Cov}(Y,X2) \label{eq:var}
\end{align}

Thus, the fractions of total LMA variance due to variance in LMAp and
LMAs were determined by the covariances Cov(LMA, LMAp) and Cov(LMA,
LMAs), respectively, taken as proportions of the total variance
Var(LMA).

\hypertarget{results}{%
\section{Results}\label{results}}

\textbf{1. Nearly all leaf dark respiration is associated with metabolic
leaf tissue mass.} Metabolic leaf mass (LMAm) accounted for nearly all
leaf dark respiration; i.e., estimated dark respiration rate per-unit
structural mass (\(\gamma_s\)) was close to zero in analyses of both
GLOPNET and Panama data (Table 1). Thus, although building costs are
likely similar for different leaf chemical components and tissues
(Villar \& Merino, \protect\hyperlink{ref-Villar2001}{2001}; Williams,
Field, \& Mooney, \protect\hyperlink{ref-Williams1989}{1989}), our
results suggest that leaf mass associated with photosynthetic function
accounts for nearly all leaf maintenance respiration.

\textbf{2. Decomposing LMA into metabolic and structural components
leads to improved predictions of \emph{A}\textsubscript{area},
\emph{R}\textsubscript{area} and LL.}

For both the GLOPNET global datasets and the Panama datasets, model
including LMAm and LMAs showed better predictions than model including
LMA based on cross-validation (Table 1), suggesting that decompose LMA
into metabolic and structural components leads improved predictions of
\emph{A}\textsubscript{area}, \emph{R}\textsubscript{area} and LL.

For the GLOPNET global dataset, \emph{A}\textsubscript{area} had a
strong positive correlation with LMAm, a weak negative correlation with
LMAs, and a non-significant correlation with total LMA (Figs. 1a-c;
Table 2). \emph{R}\textsubscript{area} in GLOPNET also had a strong
positive correlation with LMAm, which was stronger than the correlation
between \emph{R}\textsubscript{area} and either LMA or LMAs (Figs.
1d-f). Finally, LL in GLOPNET had a strong positive correlation with
LMAs, which was stronger than the correlation between LL and either LMA
or LMAm (Figs. 1g-i).

For the Panama dataset, we evaluated multiple models due to the
availability of both sun and shade leaves. The model including the light
effects fit the data better than the model without the light effects
according to cross-validation (Table 1). All Panama results we report
are for the model including the light effects unless stated otherwise.
\emph{A}\textsubscript{area} and \emph{R}\textsubscript{area} had
stronger and more positive correlations with LMAm than with LMA or LMAs
(Figs. 2a-f and Table S1). LL was not significantly correlated with LMA
when all leaves were combined, but was strongly correlated with LMA for
shade leaves at the dry site (Fig. 2g). There was an apparent negative
correlation between LMAm and LL (Fig. 2h) but the effect of LMAm on LL
was not significant (Table 2), implying that a weak covariance between
LMAm and LMAs produced this negative correlation. There was a positive
correlation between LL and LMAs (Fig. 2i and Table 2). Because the LL
vs.~LMAs relationship ignores important factors (i.e., effects of light
on realized photosynthetic rates), the explained variance of LL from the
model was higher (r\textsuperscript{2} = 0.51) than expected from the
correlation between LMAs and LL.

\textbf{3. Interspecific mass proportionality of
\emph{A}\textsubscript{max} depended on both scaling slope and variance
in LMA components.}

\textbf{4. Evergreen leaves have greater LMAs than deciduous leaves, and
sun leaves have both greater LMAm and LMAs than shade leaves.} In the
GLOPNET dataset, evergreen leaves had significantly higher LMAs than
deciduous leaves, but the two groups had similar LMAm (Fig. 5). Thus,
the higher total LMA in evergreen leaves in GLOPNET was primarily due to
differences in LMAs (which comprised a greater fraction of LMA in
evergreen than in deciduous leaves; Fig. S1a). Similar results were
obtained for the Panama dataset (evergreen leaves have higher LMAs but
similar LMAm compared to deciduous leaves; Fig. S1b). In the Panama
dataset, LMAm was significantly higher in sun leaves than in shade
leaves (Fig. 5). In Panama wet site, LMAs in sun leaves were higher than
in shade leaves, but mean values of LMAm and LMAs were similar in the
dry site (Fig. 5). Thus, in contrast to interspecific variation in LMA
(which is driven by variation in both LMAm and LMAs; Fig. 4),
intraspecific variation in LMA reflects changes in LMAm.

\textbf{5. Nitrogen and phosphorus per-unit leaf area are strongly
correlated with LMAm, and cellulose per-unit leaf area is strongly
correlated with LMAs.} In the GLOPNET dataset,
\emph{N}\textsubscript{area} and \emph{P}\textsubscript{area} had strong
positive correlations with LMAm, but only weak correlations with LMAs
(Fig. S3). Similarly, in the Panama dataset,
\emph{N}\textsubscript{area} and \emph{P}\textsubscript{area} had strong
positive correlations with LMAm, but were not correlated with LMAs (Fig.
6). In contrast, cellulose per-unit leaf area
(\emph{CL}\textsubscript{area}), which was available for the Panama
dataset but not for GLOPNET, had a strong positive correlation with
LMAs, and a weak positive correlation with LMAm (Fig. 6).
\emph{CL}\textsubscript{area} was more strongly correlated with LMA than
with LMAm or LMAs, but sun and shade leaves aligned along a common
\emph{CL}\textsubscript{area}-LMAs relationship, as opposed to being
offset for LMA and LMAm (Figs. 6g-i).

\hypertarget{discussion}{%
\section{Discussion}\label{discussion}}

Our analyses demonstrate that decomposing LMA variation into separate
metabolic and structural components (LMAm and LMAs, respectively) leads
to improved predictions of photosynthetic capacity
(\emph{A}\textsubscript{max}), dark respiration rate
(\emph{R}\textsubscript{dark}), and leaf lifespan (LL), as well as clear
relationships with traits used for independent model evaluation
(nitrogen, phosphorus, and cellulose concentrations). Tests with
simulated data suggest that our results are robust and reflect
meaningful (Supplement X), previously unreported patterns in leaf trait
data. Below, we elaborate on the insights gained from our analysis and
the implications of our results for the representation of leaf
functional diversity in global ecosystem models.

Decomposing LMA into LMAm and LMAs provides insights into why
interspecific variation in leaf traits related to photosynthesis and
metabolism are primarily area-dependent (i.e., primarily independent of
LMA when expressed per-unit area rather than mass-dependent (Osnas et
al., \protect\hyperlink{ref-Osnas2018}{2018},
\protect\hyperlink{ref-Osnas2013}{2013})). We expected that leaf
metabolic traits tend to be area-depend when variance in LMAm is larger
than variance in LMAs. However, our results suggest that the scaling
slope (\(\alpha_m\)) between LMAm and \emph{A}\textsubscript{area} also
affect trait area proportionality. Leaf metabolic traits tend to be
area-dependent when scaling slope (\(\alpha_m\)) between LMAm is small
and/or variance in LMAm is small or similar to variance in LMAs (Fig.
4). Consistent with this explanation, our results suggest that
interspecific LMA variation across the global flora and across tropical
tree species in Panama is due to both LMAm and LMAs, and the scaling
slope \(\alpha_m\) between LMAm and \emph{A}\textsubscript{area} are
small. The scaling slope \(\alpha_m\) will asymptotically reach its
maximum (below 1). This is probably because mesophyll thinkness in dense
leaves (REF) and with-in leaf shading (REF) will reduce
\emph{A}\textsubscript{max}.

Decomposing LMA also provides insights as to why intraspecific patterns
of trait variation differ from those observed across species. In
contrast to interspecific LMA variation, our analysis suggests that LMAp
contributes half or more of the intraspecific increase in LMA from shade
to sun (Figs. 5 and SX). The increase in LMAm from shade to sun -- which
likely reflects an increase in the size and number of palisade mesophyll
cells with increasing light availability (Onoda, Schieving, \& Anten,
\protect\hyperlink{ref-Onoda2008}{2008}; Terashima, Hanba, Tholen, \&
Niinemets, \protect\hyperlink{ref-Terashima2011}{2011}) -- is also
associated with an increase in LMAs from shade to sun (Fig. 5). This
positive covariance between LMAm and LMAs within species means that
per-area values of LMAm-proportional traits (e.g.,
\emph{A}\textsubscript{area}) have a strong, positive relationship with
total LMA, which implies trait mass-proportionality (Supplement S1-2).

The improved predictions and understanding provided by decomposing LMA
into photosynthetic and structural components challenge the view that
leaf functional diversity can be accurately represented by a single leaf
economics spectrum (LES) axis (Wright et al.,
\protect\hyperlink{ref-Wright2004}{2004}\protect\hyperlink{ref-Wright2004}{b}).
Lloyd et al. (\protect\hyperlink{ref-Lloyd2013}{2013}) argued that the
apparent dominance of a single LES axis is an artifact of expressing
area-dependent leaf traits on a per-mass basis, and Osnas et al.
(\protect\hyperlink{ref-Osnas2013}{2013}) demonstrated that across the
global flora, traits related to photosynthesis and metabolism are indeed
area-dependent. Intraspecific patterns in trait variation, which
contrast with interspecific patterns, pose additional challenges for a
one-dimensional view of leaf functional diversity. Our analysis shows
that considering two primary axes of leaf trait variation
(photosynthesis and structure) provides improved quantitative
predictions and insights compared to LES of a single dominant axis. Our
results point to a simple two-dimensional framework for representing
leaf functional diversity in global ecosystem models: a LMAm axis that
determines \emph{A}\textsubscript{max}, accounts for nearly all
\emph{R}\textsubscript{dark} (see rp and rs estimates in Table S1) and a
LMAs axis that determines potential LL through its effects on leaf
toughness (Kleyer et al., \protect\hyperlink{ref-Kleyer2012}{2012}). In
the datasets we analyzed (the global flora and tropical trees in
Panama), these two axes are only weakly correlated with each other (Fig.
S5), which suggests that trait-based approaches to global ecosystem
modeling (Scheiter, Langan, \& Higgins,
\protect\hyperlink{ref-Scheiter2013}{2013}; Wullschleger et al.,
\protect\hyperlink{ref-Wullschleger2014}{2014}) could consider these as
independent axes. For example, to simulate individual competing trees
which from diverse community of growth strategies (Sakschewski et al.,
\protect\hyperlink{ref-Sakschewski2015}{2015},
\protect\hyperlink{ref-Sakschewski2016}{2016}), it would be possible to
draw LMAm and LMAs values from distributions with realistic ranges
instead of a single LMA distribution.

Our statistical decomposition of LMA into metabolic and structural
components provides important insights, but additional insights and
accuracy could be gained by a more mechanistic modeling approach. For
example, John et al. (\protect\hyperlink{ref-John2017}{2017}) decomposed
interspecific LMA variation into anatomical components such as the size,
number of layers, and mass density of cells in different leaf tissues.
If such detailed information became available for a large number of
leaves, representing both intra- and interspecific variation, it should
be possible to quantify how these anatomical traits scale up to
leaf-level \emph{A}\textsubscript{max} , \emph{R}\textsubscript{dark},
and LL. A simpler alternative would be to modify our model to account
for variation in cell wall thickness (TCW): for a given cell size,
increasing TCW would lead to an increase in lamina density, cellulose
per volume, toughness, and LL (Kitajima et al.,
\protect\hyperlink{ref-Kitajima2012}{2012}; Kitajima \& Poorter,
\protect\hyperlink{ref-Kitajima2010}{2010}; Kitajima, Wright, \&
Westbrook, \protect\hyperlink{ref-Kitajima2016}{2016}), and a decrease
in mesophyll conductance and \emph{A}\textsubscript{max} (Evans,
Kaldenhoff, Genty, \& Terashima,
\protect\hyperlink{ref-Evans2009}{2009}; Onoda et al.,
\protect\hyperlink{ref-Onoda2017}{2017}; Terashima et al.,
\protect\hyperlink{ref-Terashima2011}{2011}). Indeed, our model might
capture those patterns. The negative effects of LMAs on
\emph{A}\textsubscript{area} in GLOPNET data indicate that high amount
of photosynthetic proteins is not always translated into high
photosynthetic capacity. High LMAs that may be associated with a
particular anatomical components might lead to lower mesophyll
conductance.

\hypertarget{conclusions}{%
\section{Conclusions}\label{conclusions}}

It is widely recognized that LMA variation is associated with multiple
tissues and functions, including metabolically active mesophyll that
largely determines photosynthetic capacity, as well as structural and
chemical components that contribute primarily to leaf toughness and
defense (Lusk, Onoda, Kooyman, \& Gutiérrez-Girón,
\protect\hyperlink{ref-Lusk2010}{2010}; Roderick, Berry, Noble, \&
Farquhar, \protect\hyperlink{ref-Roderick1999}{1999}; Shipley,
Lechowicz, Wright, \& Reich, \protect\hyperlink{ref-Shipley2006}{2006}).
It should not be surprising then, that partitioning LMA into metabolic
and structural components yields enhanced predictions and improved
understanding of patterns of leaf trait variation both within and among
species. Yet for over a decade, the vast literature on leaf traits has
been strongly influenced by the view that leaf trait variation can
usefully be represented by a single dominant axis of LES. Our results
provide quantitative evidence that this one-dimensional view of leaf
trait variation is insufficient, and our model provides a biological
explanation for previous statistical analyses that have demonstrated
area-dependence of leaf traits across species (Osnas et al.,
\protect\hyperlink{ref-Osnas2013}{2013}), while also explaining
mass-dependence within species. Our results suggest that small scaling
slope between metabolic leaf mass and photosynthetic rate, and similar
variance in LMAm and in LMAs leads to area-proportionality of
interspecific leaf traits. Thus, strong interspecific correlations
between LMA and mass-normalized photosynthetic capacity (and related
traits, such as respiration rate, and nitrogen and phosphorus
concentrations) are likely driven by mass-normalization itself, rather
than any functional dependence of these photosynthetic and metabolic
traits on LMA (Lloyd et al., \protect\hyperlink{ref-Lloyd2013}{2013};
Osnas et al., \protect\hyperlink{ref-Osnas2013}{2013}). In contrast,
intraspecific variation in LMA is driven by coordinated changes in
structural and metabolic mass components, which explains why
mass-normalized photosynthetic and metabolic traits vary little from sun
to shade within species (Aranda, Pardo, Gil, \& Pardos,
\protect\hyperlink{ref-Aranda2004}{2004}; Niinemets, Keenan, \& Hallik,
\protect\hyperlink{ref-Niinemets2015}{2015}; Poorter et al.,
\protect\hyperlink{ref-Poorter2006b}{2006}).

\hypertarget{acknowledgments}{%
\section{Acknowledgments}\label{acknowledgments}}

We thank Jonathan Dushoff for statistical advice and Martijn Slot
for~helpful~comments that improved the paper. Mirna Samaniego and Milton
Garica provided indispensable assistance in data collection. We thank
the Smithsonian Tropical Research Institute (STRI), the Tropical~Canopy
Biology Program at STRI and the Andrew W. Mellon Foundation for
supporting this work. MK was supported by a Postdoctoral~Fellowship
for~Research Abroad from the Japan Society for the Promotion of Science.

\hypertarget{author-contributions}{%
\section{Author contributions}\label{author-contributions}}

MK, JWL and JLDO conceived of the study; KK, SJW and SAVB contributed
data; MK devised the analytical approach and performed analyses; MK and
JWL wrote the first draft of the manuscript, and all authors contributed
to revisions.

\hypertarget{references}{%
\section{References}\label{references}}

\hypertarget{refs}{}
\leavevmode\hypertarget{ref-Aranda2004}{}%
Aranda, I., Pardo, F., Gil, L., \& Pardos, J. A. (2004). Anatomical
basis of the change in leaf mass per area and nitrogen investment with
relative irradiance within the canopy of eight temperate tree species.
\emph{Acta Oecologica}, \emph{25}(3), 187--195.
doi:\href{https://doi.org/10.1016/j.actao.2004.01.003}{10.1016/j.actao.2004.01.003}

\leavevmode\hypertarget{ref-Bishop2006}{}%
Bishop, C. M. (2006). \emph{Pattern Recognition and Machine Learning}.
New York: Springer.

\leavevmode\hypertarget{ref-Carpenter2017}{}%
Carpenter, B., Gelman, A., Hoffman, M. D., Lee, D., Goodrich, B.,
Betancourt, M., \ldots{} Riddell, A. (2017). Stan : A Probabilistic
Programming Language. \emph{Journal of Statistical Software},
\emph{76}(1), 1--32.
doi:\href{https://doi.org/10.18637/jss.v076.i01}{10.18637/jss.v076.i01}

\leavevmode\hypertarget{ref-Evans2009}{}%
Evans, J. R., Kaldenhoff, R., Genty, B., \& Terashima, I. (2009).
Resistances along the CO2 diffusion pathway inside leaves. \emph{Journal
of Experimental Botany}, \emph{60}(8), 2235--2248.
doi:\href{https://doi.org/10.1093/jxb/erp117}{10.1093/jxb/erp117}

\leavevmode\hypertarget{ref-Gelman2006}{}%
Gelman, A., \& Hill, J. (2006). \emph{Multilevel regression} (pp.
235--236).
doi:\href{https://doi.org/10.1017/CBO9780511790942.014}{10.1017/CBO9780511790942.014}

\leavevmode\hypertarget{ref-Gelman2014}{}%
Gelman, A., Hwang, J., \& Vehtari, A. (2014). Understanding predictive
information criteria for Bayesian models. \emph{Statistics and
Computing}, \emph{24}(6), 997--1016.
doi:\href{https://doi.org/10.1007/s11222-013-9416-2}{10.1007/s11222-013-9416-2}

\leavevmode\hypertarget{ref-John2017}{}%
John, G. P., Scoffoni, C., Buckley, T. N., Villar, R., Poorter, H., \&
Sack, L. (2017). The anatomical and compositional basis of leaf mass per
area. \emph{Ecology Letters}, \emph{20}(4), 412--425.
doi:\href{https://doi.org/10.1111/ele.12739}{10.1111/ele.12739}

\leavevmode\hypertarget{ref-Kikuzawa1991}{}%
Kikuzawa, K. (1991). A Cost-Benefit Analysis of Leaf Habit and Leaf
Longevity of Trees and Their Geographical. \emph{The American
Naturalist}, \emph{138}(5), 1250--1263.
doi:\href{https://doi.org/10.2307/2462519}{10.2307/2462519}

\leavevmode\hypertarget{ref-Kikuzawa2004}{}%
Kikuzawa, K., Shirakawa, H., Suzuki, M., \& Umeki, K. (2004). Mean labor
time of a leaf. \emph{Ecological Research}, \emph{19}(4), 365--374.
doi:\href{https://doi.org/10.1111/j.1440-1703.2004.00657.x}{10.1111/j.1440-1703.2004.00657.x}

\leavevmode\hypertarget{ref-Kitajima2013}{}%
Kitajima, K., Cordero, R. A., \& Wright, S. J. (2013). Leaf life span
spectrum of tropical woody seedlings: Effects of light and ontogeny and
consequences for survival. \emph{Annals of Botany}, \emph{112}(4),
685--699.
doi:\href{https://doi.org/10.1093/aob/mct036}{10.1093/aob/mct036}

\leavevmode\hypertarget{ref-Kitajima2012}{}%
Kitajima, K., Llorens, A. M., Stefanescu, C., Timchenko, M. V., Lucas,
P. W., \& Wright, S. J. (2012). How cellulose-based leaf toughness and
lamina density contribute to long leaf lifespans of shade-tolerant
species. \emph{New Phytologist}, \emph{195}(3), 640--652.
doi:\href{https://doi.org/10.1111/j.1469-8137.2012.04203.x}{10.1111/j.1469-8137.2012.04203.x}

\leavevmode\hypertarget{ref-Kitajima2010}{}%
Kitajima, K., \& Poorter, L. (2010). Tissue-level leaf toughness, but
not lamina thickness, predicts sapling leaf lifespan and shade tolerance
of tropical tree species. \emph{New Phytologist}, \emph{186}(3),
708--721.
doi:\href{https://doi.org/10.1111/j.1469-8137.2010.03212.x}{10.1111/j.1469-8137.2010.03212.x}

\leavevmode\hypertarget{ref-Kitajima2016}{}%
Kitajima, K., Wright, S. J., \& Westbrook, J. W. (2016). Leaf cellulose
density as the key determinant of inter- and intra-specific variation in
leaf fracture toughness in a species-rich tropical forest.
\emph{Interface Focus}, \emph{6}(3), 20150100.
doi:\href{https://doi.org/10.1098/rsfs.2015.0100}{10.1098/rsfs.2015.0100}

\leavevmode\hypertarget{ref-Kleyer2012}{}%
Kleyer, M., Dray, S., Bello, F., Lepš, J., Pakeman, R. J., Strauss, B.,
\ldots{} Lavorel, S. (2012). Assessing species and community functional
responses to environmental gradients: Which multivariate methods?
\emph{Journal of Vegetation Science}, \emph{23}(5), 805--821.
doi:\href{https://doi.org/10.1111/j.1654-1103.2012.01402.x}{10.1111/j.1654-1103.2012.01402.x}

\leavevmode\hypertarget{ref-Lloyd2013}{}%
Lloyd, J., Bloomfield, K., Domingues, T. F., \& Farquhar, G. D. (2013).
Photosynthetically relevant foliar traits correlating better on a mass
vs an area basis: Of ecophysiological relevance or just a case of
mathematical imperatives and statistical quicksand? \emph{New
Phytologist}, \emph{199}(2), 311--321.
doi:\href{https://doi.org/10.1111/nph.12281}{10.1111/nph.12281}

\leavevmode\hypertarget{ref-Lusk2010}{}%
Lusk, C. H., Onoda, Y., Kooyman, R., \& Gutiérrez-Girón, A. (2010).
Reconciling species-level vs plastic responses of evergreen leaf
structure to light gradients: Shade leaves punch above their weight.
\emph{New Phytologist}, \emph{186}(2), 429--438.
doi:\href{https://doi.org/10.1111/j.1469-8137.2010.03202.x}{10.1111/j.1469-8137.2010.03202.x}

\leavevmode\hypertarget{ref-Lusk2008}{}%
Lusk, C. H., Reich, P. B., Montgomery, R. A., Ackerly, D. D., \&
Cavender-Bares, J. (2008). Why are evergreen leaves so contrary about
shade? \emph{Trends in Ecology and Evolution}, \emph{23}(6), 299--303.
doi:\href{https://doi.org/10.1016/j.tree.2008.02.006}{10.1016/j.tree.2008.02.006}

\leavevmode\hypertarget{ref-Niinemets2015}{}%
Niinemets, Ü., Keenan, T. F., \& Hallik, L. (2015). A worldwide analysis
of within-canopy variations in leaf structural, chemical and
physiological traits across plant functional types. \emph{New
Phytologist}, \emph{205}(3), 973--993.
doi:\href{https://doi.org/10.1111/nph.13096}{10.1111/nph.13096}

\leavevmode\hypertarget{ref-Onoda2008}{}%
Onoda, Y., Schieving, F., \& Anten, N. P. (2008). Effects of light and
nutrient availability on leaf mechanical properties of Plantago major: A
conceptual approach. \emph{Annals of Botany}, \emph{101}(5), 727--736.
doi:\href{https://doi.org/10.1093/aob/mcn013}{10.1093/aob/mcn013}

\leavevmode\hypertarget{ref-Onoda2017}{}%
Onoda, Y., Wright, I. J., Evans, J. R., Hikosaka, K., Kitajima, K.,
Niinemets, Ü., \ldots{} Westoby, M. (2017). Physiological and structural
tradeoffs underlying the leaf economics spectrum. \emph{New
Phytologist}, \emph{214}(4), 1447--1463.
doi:\href{https://doi.org/10.1111/nph.14496}{10.1111/nph.14496}

\leavevmode\hypertarget{ref-Osnas2018}{}%
Osnas, J. L. D., Katabuchi, M., Kitajima, K., Wright, S. J., Reich, P.
B., Van Bael, S. A., \ldots{} Lichstein, J. W. (2018). Divergent drivers
of leaf trait variation within species, among species, and among
functional groups. \emph{Proceedings of the National Academy of Sciences
of the United States of America}, 201803989.
doi:\href{https://doi.org/10.1073/pnas.1803989115}{10.1073/pnas.1803989115}

\leavevmode\hypertarget{ref-Osnas2013}{}%
Osnas, J. L. D., Lichstein, J. W., Reich, P. B., \& Pacala, S. W.
(2013). Global leaf trait relationships: Mass, area, and the leaf
economics spectrum. \emph{Science}, \emph{340}(6133), 741--744.
doi:\href{https://doi.org/10.1126/science.1231574}{10.1126/science.1231574}

\leavevmode\hypertarget{ref-Poorter2006b}{}%
Poorter, H., Pepin, S., Rijkers, T., De Jong, Y., Evans, J. R., \&
Körner, C. (2006). Construction costs, chemical composition and payback
time of high- and low-irradiance leaves. \emph{Journal of Experimental
Botany}, \emph{57}(2 SPEC. ISS.), 355--371.
doi:\href{https://doi.org/10.1093/jxb/erj002}{10.1093/jxb/erj002}

\leavevmode\hypertarget{ref-Reich2014}{}%
Reich, P. B. (2014). The world-wide 'fast-slow' plant economics
spectrum: A traits manifesto. \emph{Journal of Ecology}, \emph{102}(2),
275--301.
doi:\href{https://doi.org/10.1111/1365-2745.12211}{10.1111/1365-2745.12211}

\leavevmode\hypertarget{ref-Roderick1999}{}%
Roderick, M. L., Berry, S. L., Noble, I. R., \& Farquhar, G. D. (1999).
A theoretical approach to linking the composition and morphology with
the function of leaves. \emph{Functional Ecology}, \emph{13}(5),
683--695.
doi:\href{https://doi.org/10.1046/j.1365-2435.1999.00368.x}{10.1046/j.1365-2435.1999.00368.x}

\leavevmode\hypertarget{ref-Sakschewski2015}{}%
Sakschewski, B., Bloh, W. von, Boit, A., Rammig, A., Kattge, J.,
Poorter, L., \ldots{} Thonicke, K. (2015). Leaf and stem economics
spectra drive diversity of functional plant traits in a dynamic global
vegetation model. \emph{Global Change Biology}, \emph{21}(7),
2711--2725.
doi:\href{https://doi.org/10.1111/gcb.12870}{10.1111/gcb.12870}

\leavevmode\hypertarget{ref-Sakschewski2016}{}%
Sakschewski, B., Von Bloh, W., Boit, A., Poorter, L., Peña-Claros, M.,
Heinke, J., \ldots{} Thonicke, K. (2016). Resilience of Amazon forests
emerges from plant trait diversity. \emph{Nature Climate Change},
\emph{6}(11), 1032--1036.
doi:\href{https://doi.org/10.1038/nclimate3109}{10.1038/nclimate3109}

\leavevmode\hypertarget{ref-Scheiter2013}{}%
Scheiter, S., Langan, L., \& Higgins, S. I. (2013). Next-generation
dynamic global vegetation models: Learning from community ecology.
\emph{New Phytologist}, \emph{198}(3), 957--969.
doi:\href{https://doi.org/10.1111/nph.12210}{10.1111/nph.12210}

\leavevmode\hypertarget{ref-Shipley2006}{}%
Shipley, B., Lechowicz, M. J., Wright, I., \& Reich, P. B. (2006).
Fundamental trade-offs generating the worldwide leaf economics spectrum.
\emph{Ecology}, \emph{87}(3), 535--541.
doi:\href{https://doi.org/10.1890/05-1051}{10.1890/05-1051}

\leavevmode\hypertarget{ref-Terashima2011}{}%
Terashima, I., Hanba, Y. T., Tholen, D., \& Niinemets, U. (2011). Leaf
Functional Anatomy in Relation to Photosynthesis. \emph{Plant
Physiology}, \emph{155}(1), 108--116.
doi:\href{https://doi.org/10.1104/pp.110.165472}{10.1104/pp.110.165472}

\leavevmode\hypertarget{ref-Vehtari2014}{}%
Vehtari, A., Mononen, T., Tolvanen, V., Sivula, T., \& Winther, O.
(2014). Bayesian leave-one-out cross-validation approximations for
Gaussian latent variable models. \emph{arXiv Preprint
arXiv:1408.4050v2}. Retrieved from \url{http://arxiv.org/abs/1412.7461}

\leavevmode\hypertarget{ref-Villar2001}{}%
Villar, R., \& Merino, J. (2001). Comparison of leaf construction costs
in woody species with differing leaf-spans in contrasting ecosystems.
\emph{New Phytologist}, \emph{151}, 213--226.
doi:\href{https://doi.org/10.1046/j.1469-8137.2001.00147.x/pdf}{10.1046/j.1469-8137.2001.00147.x/pdf}

\leavevmode\hypertarget{ref-Watanabe2010}{}%
Watanabe, S. (2010). Asymptotic Equivalence of Bayes Cross Validation
and Widely Applicable Information Criterion in Singular Learning Theory.
\emph{The Journal of Machine Learning Research}, \emph{11}, 3571--3594.
Retrieved from \url{http://arxiv.org/abs/1004.2316}

\leavevmode\hypertarget{ref-Westoby2013}{}%
Westoby, M., Reich, P. B., \& Wright, I. J. (2013). Understanding
ecological variation across species: Area-based vs mass-based expression
of leaf traits. \emph{New Phytologist}, \emph{199}(2), 322--323.
doi:\href{https://doi.org/10.1111/nph.12345}{10.1111/nph.12345}

\leavevmode\hypertarget{ref-Westoby2006a}{}%
Westoby, M., \& Wright, I. J. (2006). Land-plant ecology on the basis of
functional traits. \emph{Trends in Ecology and Evolution}, \emph{21}(5),
261--268.
doi:\href{https://doi.org/10.1016/j.tree.2006.02.004}{10.1016/j.tree.2006.02.004}

\leavevmode\hypertarget{ref-Williams1989}{}%
Williams, K., Field, C. B., \& Mooney, H. A. (1989). Relationships Among
Leaf Construction Cost, Leaf Longevity, and Light Environment in
Rain-Forest Plants of the Genus Piper. \emph{The American Naturalist},
\emph{133}(2), 198--211.
doi:\href{https://doi.org/10.1086/284910}{10.1086/284910}

\leavevmode\hypertarget{ref-Wright2004a}{}%
Wright, I. J., Reich, P. B., Westoby, M., Ackerly, D. D., Baruch, Z.,
Bongers, F., \ldots{} Villar, R. (2004a). The worldwide leaf economics
spectrum. \emph{Nature}, \emph{428}(6985), 821--827.
doi:\href{https://doi.org/10.1038/nature02403}{10.1038/nature02403}

\leavevmode\hypertarget{ref-Wright2004}{}%
Wright, S. J., Calderón, O., Hernandéz, A., \& Paton, S. (2004b). Are
lianas increasing in importance in tropical forests? A 17-year record
from panama. \emph{Ecology}, \emph{85}(2), 484--489.
doi:\href{https://doi.org/10.1890/02-0757}{10.1890/02-0757}

\leavevmode\hypertarget{ref-Wright2003}{}%
Wright, S. J., Horlyck, V., Basset, Y., Barrios, H., Bethancourt, A.,
Bohlman, S., \ldots{} Zotz, G. (2003). Tropical Canopy Biology Program,
Republic of Panama. In Y. Basset, V. Horlyck, \& S. J. Wright (Eds.),
\emph{Studying forest canopies from above: The international canopy
crane network} (pp. 137--155). Panama.

\leavevmode\hypertarget{ref-Wullschleger2014}{}%
Wullschleger, S. D., Epstein, H. E., Box, E. O., Euskirchen, E. S.,
Goswami, S., Iversen, C. M., \ldots{} Xu, X. (2014). Plant functional
types in Earth system models: Past experiences and future directions for
application of dynamic vegetation models in high-latitude ecosystems.
\emph{Annals of Botany}, \emph{114}(1), 1--16.
doi:\href{https://doi.org/10.1093/aob/mcu077}{10.1093/aob/mcu077}

\leavevmode\hypertarget{ref-Xu2017}{}%
Xu, X., Medvigy, D., Joseph Wright, S., Kitajima, K., Wu, J., Albert, L.
P., \ldots{} Pacala, S. W. (2017). Variations of leaf longevity in
tropical moist forests predicted by a trait-driven carbon optimality
model. \emph{Ecology Letters}, \emph{20}(9), 1097--1106.
doi:\href{https://doi.org/10.1111/ele.12804}{10.1111/ele.12804}

\newpage

\hypertarget{tables}{%
\section{Tables}\label{tables}}

\textbf{Table 1}: Summary of the model comparison based on exact 10-fold
cross-validation. Higher expected log point-wise predictive density
(ELPD) indicates better predictive accuracy. N; number of samples.

\begin{tabular}{l|l|l}
\hline
 & ELPD & N\\
\hline
**GLOPNET** &  & \\
\hline
LMAm and LMAs & -357 & 198\\
\hline
LMA & -385 & 198\\
\hline
**Panama** &  & \\
\hline
LMAm, LMAs and light & -240 & 106\\
\hline
LMAm and LMAs & -242 & 106\\
\hline
LMAm, LMAs, LT and light & -247 & 106\\
\hline
LMA and light & -253 & 106\\
\hline
LMA & -264 & 106\\
\hline
\end{tabular}

\newpage

\hypertarget{section}{%
\section{}\label{section}}

\textbf{Table 2}: Posterior medians {[}95\% credible interval{]} of
parameters for the best models. Bold values are significantly different
from zero based on the 95\% credible interval.

\begin{tabular}{l|l|l}
\hline
Parameters & GLOPNET & Panama\\
\hline
Effect of LMAm on *A*\textasciitilde{}area\textasciitilde{} (\$\textbackslash{}alpha\_m\$) & **0.365 [0.19, 0.474]** & **0.632 [0.459, 0.82]**\\
\hline
Effect of LMAs on *A*\textasciitilde{}area\textasciitilde{} (\$\textbackslash{}alpha\_s\$) & **-0.248 [-0.347, -0.104]** & 0.143 [-0.065, 0.314]\\
\hline
Effect of LMAm on LL (\$\textbackslash{}beta\_m\$) & 0.05 [-0.176, 0.379] & 0.067 [-0.201, 0.34]\\
\hline
Effect of LMAs on LL (\$\textbackslash{}beta\_s\$) & **0.322 [0.154, 0.461]** & **0.531 [0.293, 0.738]**\\
\hline
Effect of LMAm on *R*\textasciitilde{}area\textasciitilde{} (\$\textbackslash{}gamma\_m\$) & **1.032 [0.772, 1.211]** & **0.791 [0.408, 1.101]**\\
\hline
Effect of LMAs on *R*\textasciitilde{}area\textasciitilde{} (\$\textbackslash{}gamma\_s\$) & -0.029 [-0.15, 0.111] & 0.034 [-0.324, 0.446]\\
\hline
Effect of light on LL (\$\textbackslash{}theta\$) & - & **-1.123 [-1.516, -0.726]**\\
\hline
\end{tabular}

\newpage

\hypertarget{figures}{%
\section{Figures}\label{figures}}

\includegraphics{LMAps_main_re_files/figure-latex/GLplt-1.pdf}

\textbf{Figure 1}: Observed and estimated leaf-trait relationships in
the global GLOPNET dataset. Leaf life span (LL), net photosynthetic rate
per unit leaf area (\emph{A}\textsubscript{area}), and dark respiration
rate per unit leaf area (\emph{R}\textsubscript{area}) are plotted
against observed LMA and estimates (posterior means) of photosynthetic
and structural LMA components (LMAm and LMAs, respectively). Pearson
correlation coefficients are for LMA (left column) and posterior
distributions of LMAm (middle column) and LMAs (right column).

\newpage

\hypertarget{section-1}{%
\section{}\label{section-1}}

\includegraphics{LMAps_main_re_files/figure-latex/PAplt-1.pdf}

\textbf{Figure 2}: Observed and estimated leaf-trait relationships in
the Panama dataset. Estimates are from the Optimal LL Model (Eq. 8).
Details as in Fig. 1. Results for other LL models are summarized in
Table SX.

\newpage

\hypertarget{section-2}{%
\section{}\label{section-2}}

\includegraphics{LMAps_main_re_files/figure-latex/LLplt-1.pdf}

\textbf{Figure 3}: Observed vs.~predicted leaf lifespan (LL) in the
Optimal LL Model (Eq. 8). Predicted LL values are posterior medians. The
dashed line indicates the 1:1 relationship. The
\emph{r}\textsuperscript{2} value is the posterior median of Bayesian
\emph{r}\textsuperscript{2} (Gelman et al.~2017).

\newpage

\hypertarget{section-3}{%
\section{}\label{section-3}}

\includegraphics{LMAps_main_re_files/figure-latex/massplt-1.pdf}

\textbf{Figure 4}: The relationships among mass proportionality
(\emph{b\textsubscript{i}} in Eq. X), variance in LMAm and LMAs, and
scaling slopes between LMAm and \emph{A}\textsubscript{area}
(\(lpha_m\)) and LMAs and \emph{A}\textsubscript{area} (\(lpha_s\)).
Blue colors indicate the photosynthetic rate is primarily mass
proportional (\emph{b\textsubscript{i}} is large) and red colors
indicate the photosynthetic rate is primarily area proportional
(\emph{b\textsubscript{i}} is close to 0).

\newpage

\hypertarget{section-4}{%
\section{}\label{section-4}}

\includegraphics{LMAps_main_re_files/figure-latex/boxplt-1.pdf}

\textbf{Figure 5}: Boxplots comparing leaf mass per area (LMA) and
posterior means of photosynthetic and structural LMA components (LMAm
and LMAs, respectively) across deciduous (D) and evergreen (E) leaves in
the GLOPNET dataset (top), and across sites (wet and dry) and canopy
strata (sun and shade) in the Panama dataset (bottom). The center line
in each box indicates the median, upper and lower box edges indicate the
interquartile range, whiskers show 1.5 times the interquartile range,
and points are outliers. Groups sharing the same letters are not
significantly different (P \textgreater{} 0.05; t-tests). To isolate the
effects of intraspecific variation (i.e., plastic responses to light),
the Panama results shown here only include species for which both sun
and shade leaves were available. Qualitatively similar results were
obtained when all Panama species were included (Fig. S2). Note that the
vertical axis is on a log\textsubscript{10} scale.

\newpage

\hypertarget{section-5}{%
\section{}\label{section-5}}

\includegraphics{LMAps_main_re_files/figure-latex/PA-NPC-1.pdf}

\textbf{Figure 6}: Measured traits related to photosynthesis and
metabolism traits (nitrogen and phosphorus per-unit leaf area;
\emph{N}\textsubscript{area} and \emph{P}\textsubscript{area}) are
strongly correlated with estimates (posterior means) of the
photosynthetic LMA component (LMAm), and a measured structural trait
(cellulose per-unit leaf area; \emph{CL}\textsubscript{area}) is
strongly correlated with estimates of the structural LMA component
(LMAs) for the Panama dataset. Note that sun and shade leaves align
along a single relationship for \emph{CL}\textsubscript{area} vs.~LMAs,
but not for \emph{CL}\textsubscript{area} vs.~LMA or LMAm.
\emph{N}\textsubscript{area}, \emph{P}\textsubscript{area}, and
\emph{CL}\textsubscript{area} data were not used to fit the models, and
are presented here as independent support for the model results.
Analogous results were obtained for \emph{N}\textsubscript{area} and
\emph{P}\textsubscript{area} vs.~LMAm for GLOPNET (Fig. S3). Others
details as in Fig. GL. Results for other LL models are reported in Table
SX.


\end{document}
